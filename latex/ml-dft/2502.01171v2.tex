\documentclass[11pt,a4paper]{article}
\usepackage[utf8]{inputenc}
\usepackage[T1]{fontenc}
\usepackage{amsmath}
\usepackage{amsfonts}
\usepackage{amssymb}
\usepackage{amsthm}
\usepackage{graphicx}
\usepackage{hyperref}
\usepackage{cite}
\usepackage{natbib}
\usepackage{geometry}
\usepackage{booktabs}
\usepackage{multirow}
\usepackage{array}
\usepackage{caption}
\usepackage{subcaption}
\usepackage{float}
\usepackage{url}
\geometry{margin=1in}

\title{Efficient and Scalable Density Functional Theory Hamiltonian Prediction through Adaptive Sparsity}
\author{Erpai Luo, Xinran Wei, Lin Huang, Yunyang Li, Han Yang, Zaishuo Xia, Zun Wang, Chang Liu, Bin Shao, Jia Zhang}
\date{\today}

\begin{document}
\maketitle

\begin{abstract}
Hamiltonian matrix prediction is pivotal in com-
putational chemistry, serving as the foundation
for determining a wide range of molecular prop-
erties. While SE(3) equivariant graph neu-
ral networks have achieved remarkable success
in this domain, their substantial computational
cost—driven by high-order tensor product (TP)
operations—restricts their scalability to large
molecular systems with extensive basis sets. To
address this challenge, we introduce SPHNet, an
efficient and scalable equivariant network, that in-
corporates adaptive SParsity into Hamiltonian pre-
diction. SPHNet employs two innovative sparse
gates to selectively constrain non-critical inter-
action combinations, significantly reducing ten-
sor product computations while maintaining ac-
curacy. To optimize the sparse representation, we
develop a Three-phase Sparsity Scheduler, ensur-
ing stable convergence and achieving high perfor-
mance at sparsity rates of up to 70\%. Extensive
evaluations on QH9 and PubchemQH datasets
demonstrate that SPHNet achieves state-of-the-
art accuracy while providing up to a 7x speedup
over existing models. Beyond Hamiltonian pre-
diction, the proposed sparsification techniques
also hold significant potential for improving the
efficiency and scalability of other SE(3) equiv-
ariant networks, further broadening their appli-
cability and impact. Our code can be found at
https://github.com/microsoft/SPHNet.
*Equal contribution1Department of Automation, Tsinghua
University, Beijing, China2Microsoft Research AI for Science,
Beijing, China3Department of Computer Science, Yale Univer-
sity, CT, USA4Department of Computer Science, University of
California, CA, USA. Correspondence to: Xinran Wei <weixin-
ran@microsoft.com >, Lin Huang <huang.lin@microsoft.com >,
Jia Zhang <jia.zhang@microsoft.com >.
Proceedings of the 42ndInternational Conference on Machine
Learning , Vancouver, Canada. PMLR 267, 2025. Copyright 2025
by the author(s).1. Introduction
The Kohn-Sham Hamiltonian matrix is a fundamental quan-
tity in computational chemistry, as it enables the prediction
of all molecular properties obtainable through traditional
Density Functional Theory (DFT) calculations (Hohenberg
\& Kohn, 1964; te Vrugt et al., 2020). Accurately predicting
the Hamiltonian matrix allows for the rapid determination
of system energy, HOMO-LUMO gaps, electron density,
and other important properties (Fang et al., 2022; Zang et al.,
2023; Batzner et al., 2022; Batatia et al., 2022; Wang et al.,
2022; Li et al., 2024; Chen et al., 2024). Consequently,
recent efforts have focused on using machine learning tech-
niques to predict the Hamiltonian matrix (Sch ¨utt et al., 2017;
Unke et al., 2021; Yu et al., 2023; Gong et al., 2023; Zhou
et al., 2022; Zhong et al., 2023). Among these, methods
based on SE(3) equivariant graph neural networks (Unke
et al., 2021; Yu et al., 2023; Gong et al., 2023; Yu et al.,
2023; Li et al., 2025b) have achieved superior accuracy due
to their consistency with the equivariance of the Hamilto-
nian matrix. These methods typically project molecular
features into spherical harmonics space and use tensor prod-
uct operations to interact features of different orders, which
are crucial for maintaining equivariance.
Figure 1. (A)The number of tensor products grows quadratically
with the number of atoms N, as the Hamiltonian includes features
for all possible atomic pair combinations. (B)The time cost of
tensor products grows with the sixth power of their order L, where
the increase in order corresponds to the expansion in the number of
orbital types in the DFT basis set. For example, the def2-SVP basis
set requires a maximum order of 4, while def2-TZVP demands an
order of 6.
However, the efficiency issues introduced by tensor product
operations have become a significant bottleneck. On the one
hand, as the number of atoms in the system increases, the
1arXiv:2502.01171v2 [cs.LG] 22 May 2025
\end{abstract}

Efficient and Scalable Density Functional Theory Hamiltonian Prediction through Adaptive Sparsity
number of tensor product operations grows quadratically,
significantly increasing computational cost and memory re-
quirements, as illustrated in Fig.1(A). On the other hand,
since each order in the output representation corresponds to
different orbitals in the DFT basis set, the order of tensor
product operations must increase as the DFT basis set ex-
pands. As shown in Fig. 1(B), the computational complexity
of tensor products scales with the sixth power of the order,
causing the time cost of a single tensor product operation to
grow rapidly with larger basis sets. Therefore, optimizing
both the number and efficiency of tensor product operations
is critical for extending Hamiltonian prediction to larger
systems and larger basis sets.
The recent advancements in leveraging sparsity in numerical
DFT solvers (Burow \& Sierka, 2011; Laqua et al., 2022) and
message-passing networks (Liu et al., 2023; Peng \& Zhang,
2022; Passaro \& Zitnick, 2023) inspire our exploration of
incorporating adaptive sparsity into Hamiltonian prediction
models. To address the aforementioned challenges, we
developed SPHNet, and the main contribution of this work
can be summarized as follows:
•We propose a scalable and efficient network for Hamil-
tonian prediction, termed SPHNet , which introduce
adaptive sparsity into equivariant networks by intro-
ducing two sparse gates. Specifically, we employ the
Sparse Pair Gate to filter out unimportant node pairs,
reducing the number of tensor product computations,
and the Sparse TP Gate to prune less significant in-
teractions across different orders in tensor product,
thereby improving the efficiency of tensor operations.
•To optimize the sparse representation, we develop a
Three-phase Sparsity Scheduler , which ensures effi-
cient weight updates for all combinations through three
stages: random, adaptive, and fixed. This approach fa-
cilitates stable convergence to the optimal set while
maintaining high accuracy at sparsity rates up to 70\%.
•We demonstrate the enhancements of SPHNet across
multiple datasets, showing that it outperforms existing
models in accuracy on QH9 and PubchemQH, while
achieving up to a 7x speedup and reducing memory
usage by up to 75\%. It also demonstrates compara-
ble performance on small molecular trajectories on
the MD17 dataset. Moreover, the proposed adaptive
sparsification techniques exhibit strong promise in im-
proving the computational efficiency and scalability of
other SE(3) equivariant networks, paving the way for
broader applications across diverse tasks.

\section{Related Works}

SE(3) Equivariant Neural Network. The SE(3) equiv-
ariant neural network is widely used in AI for chemistrydue to its unique advantage in predicting quantum tensors
(Fuchs et al., 2020; Du et al., 2022; Musaelian et al., 2023;
Liao \& Smidt, 2022; Liao et al., 2023; Batzner et al., 2022;
2023). Key models include SE(3)-Transformer (Fuchs et al.,
2020), which introduced a robust self-attention mechanism
for 3D point clouds and graphs, and Equiformer (Liao \&
Smidt, 2022), which predicted molecular properties using
SE(3) Transformer architecture. EquiformerV2 (Liao et al.,
2023) improved on this by employing efficient eSCN convo-
lutions (Passaro \& Zitnick, 2023), outperforming traditional
networks like GemNet (Gasteiger et al., 2022) and Torchmd-
Net (Th ¨olke \& De Fabritiis, 2022). Allegro (Musaelian et al.,
2023) used a local, equivariant model without atom-centered
message passing, showing excellent generalization.
Hamiltonian Matrix Prediction. Hamiltonian predic-
tion has advanced with neural networks in recent years.
SchNOrb (Sch ¨utt et al., 2019) extended SchNet (Sch ¨utt
et al., 2017) for high-accuracy molecular orbital predictions,
while PhiSNet (Unke et al., 2021) successfully introduced
SE(3) networks, significantly improving accuracy while fac-
ing inefficiency due to tensor product operations. QHNet
(Yu et al., 2023) improved efficiency by reducing the num-
ber of tensor products and (Li et al., 2025b) further extends
the scalability and applicability of Hamiltonian matrix pre-
diction on large molecular systems by introducing a novel
loss function. At the same time, methods such as DeepH
(Li et al., 2022) are focused on the Hamiltonian prediction
of the periodic system.
Network Sparsification. Network sparsification enhances
computational efficiency by removing redundant neural net-
work components. Early methods like Optimal Brain Dam-
age (LeCun et al., 1990) and Optimal Brain Surgeon (Has-
sibi \& Stork, 1993) pruned weights based on importance,
followed by retraining to restore performance. Iterative
pruning and retraining (Han et al., 2015) became widely
used for reducing complexity while preserving accuracy.
The Lottery Ticket Hypothesis (LTH) (Frankle \& Carbin,
2019) later showed that sparse subnetworks in dense models
could independently match full model performance. Spar-
sification has since expanded to target weights, nodes, and
edges, proving effective in molecular property prediction
(Liu et al., 2023; Peng \& Zhang, 2022). Recent work has
also extended to pruning weights of the Clebsch-Gordan
tensor product (Wang et al., 2023), demonstrating potential
despite unproven efficiency.

\section{Preliminary}

DFT Hamiltonian . Density Functional Theory (DFT), as
formulated by Hohenberg and Kohn (Hohenberg \& Kohn,
1964) and further developed through the Kohn-Sham equa-
tions by Kohn and Sham (Kohn \& Sham, 1965), represents
a key computational quantum mechanical methodology for

\section{Efficient and Scalable Density Functional Theory Hamiltonian Prediction through Adaptive Sparsity}

studying the electronic structure of many-body systems,
including atoms, molecules, and solids (Jain et al., 2016;
te Vrugt et al., 2020). The core principle of DFT lies in
solving the matrix Cvia the Kohn-Sham (KS) equations,
summarized briefly as (Szabo \& Ostlund, 2012):
HC=SCϵ, (1)
where Crepresenting the coefficients of molecular orbitals,
Hsignifies the Hamiltonian matrix, Srefers to the over-
lap matrix, and ϵis the diagonal matrix encapsulating or-
bital energies. Note that C∈Rn×n0,H,S∈Rn×n, and
ϵ∈Rno×no, where ndenotes the number of basis func-
tions and nodenotes the number of atomic orbitals used in
DFT calculations. Primarily, an iterative self-consistent field
(SCF) methodology (Payne et al., 1992; Cances \& Le Bris,
2000; Wang \& Baerends, 2022; Kudin et al., 2002) is uti-
lized to solve the Kohn-Sham (KS) equations and to obtain
the Hamiltonian matrix, requiring a temporal complexity of
O(n3).
SE(3) Equivariant and Tensor Product. The coordi-
nate system in 3D Euclidean space is the most commonly
used description for molecular systems. In this space, the
group of 3D translations and rotations forms the SE(3)
group. A function f(·)is SE(3) equivariant if it satisfies
f(D(g)x) =D(g)f(x), where D(g)is a representation of
SE(3) acting on x. Therefore, a neural network in which
every operation satisfies the definition of f(·)is referred
to as an SE(3) equivariant neural network. Among these
operations, the tensor product operation is the most crucial.
Typically, higher-order irreducible representations (irreps)
are constructed using spherical harmonics. The tensor prod-
uct operation combines two irreps, xandy, with rotational
orders ℓ1andℓ2respectively, using the Clebsch-Gordan
(CG) coefficients (Griffiths \& Schroeter, 2018), resulting in
a new irreducible representation of order ℓ3as:
(xℓ1⊗yℓ2)ℓ3
m3=ℓ1X
m1=−ℓ1ℓ2X
m2=−ℓ2C(ℓ3,m3)
(ℓ1,m1),(ℓ2,m2)xℓ1
m1yℓ2
m2,
(2)
where mdenotes the m-th element in the irreducible repre-
sentation and satisfies −ℓ≤m≤ℓ. The output order ℓ3is
restricted by |ℓ1−ℓ2| ≤ℓ3≤ |ℓ1+ℓ2|. Since each order
has to interact with every other order, the computational
complexity of the full tensor product using irreps up to or-
der L have a computational complexity of O(L6), which
constrains its applicability for systems with higher degrees.

\section{Methodology}

4.1. Three-phase Sparsity Scheduler
To ensure stable selection of combinations in the Sparse
Tensor Product Gate and Sparse Pair Gate, we propose theThree-phase Sparsity Scheduler, a three-stage selection func-
tion that always selects retained elements from the set based
on a learnable weight matrix W. Thus, for any unsparsified
setU, the scheduler can be defined as:
TSS(W, k) =

RANDOM (W,1−k),if epoch < t,
TOP(W,1−k), if epoch =t,
TOP(Wp2′,1−k), if epoch > t,
(3)
where krepresents the sparsity rate of the tensor product, tis
the round of the training epoch, and TSS(·)always returns a
subset UTSSofU, containing (1−k)|U|elements selected
based on W.
Specifically, in the first phase, Wis initialized as an all-ones
vector, and RANDOM (·)denotes a random function that
selects elements from the set independently of Wwith a
given probability 1−k. This ensures that all parameters
are unbiasedly selected and updated. In the second phase,
TOP(·)is used to select the elements whose weights are
within the top 1−kpercent of all elements, ensuring that
the optimal combinations are chosen globally. In the third
phase, Wp2′represents a fixed vector without gradients,
which is frozen after the final backward update of the second
phase. From this point onward, the selected combinations
remain unchanged. This design prioritizes efficiency, as
static connections typically result in faster computation.
4.2. Sparse Tenor Product Gate
The Sparse TP Gate is designed to accelerate individual
tensor product computations by filtering out unnecessary
cross-order combinations in traditional tensor product opera-
tions, as illustrated in Fig.2(B). By doing so, tensor product
operations implemented with e3nn (Geiger et al., 2022) can
achieve near-linear acceleration, as demonstrated in the Ap-
pendix B.3. Moreover, this approach can be easily adapted
to various tensor product implementations by simply modi-
fying the instructions that declare cross-order combinations
in the tensor product operation.
Generally, the classical tensor product operation includes
the interaction of every element min every irrep order ℓ.
The aim of Sparse Tensor Product Gate is to learn a set of
the most valuable combinations UTSS
c from the complete
setUc=\{(ℓ1, ℓ2, ℓ3)|ℓ3∈[|ℓ1−ℓ2|, ℓ1+ℓ2]\}in the
tensor product and remove the redundant combinations ac-
cordingly. UTSS
cis defined by a one-dimensional learnable
value score Wℓ1,ℓ2,ℓ3c of length |Uc|as:
UTSS
c=\{(ℓ1, ℓ2, ℓ3)|Wℓ1,ℓ2,ℓ3
c ∈TSS(Wc, k)\},(4)
where TSS(·)is the selection scheduler defined in Equation
3. Accordingly, we can obtain the updated combination
weight from the original weights cdefined in classical tensor

\section{Efficient and Scalable Density Functional Theory Hamiltonian Prediction through Adaptive Sparsity}

Figure 2. (A)The overall architecture of SPHNet. Atomic numbers and atomic coordinates are first passed through the Vectorial Node
Interaction Blocks to obtain atomic features xℓ
i. Subsequently, the Sparse Pair Gate selects the key pair set (i, j)for the Spherical Node
Interaction Blocks, where the irreps xℓ
iare elevated from the ℓ= 1 toLmaxduring the interaction process. Next, the Sparse Tensor
Product Gate in the construction block identifies the key cross-order combinations (ℓ1, ℓ2, ℓ3)for the diagonal blocks, yielding diagonal
pair features fii. For non-diagonal blocks, both the Pair Gate and Tensor Product Gate are applied to select the critical pairs and tensor
product combinations, producing non-diagonal pair features fij. Finally, these features are fed into the expansion block to construct the
predicted Hamiltonian matrix. (B)Sparse Pair Gate: It takes pairwise features as input, computes weights for each pair (i, j), and selects
a optimal subset using the sparsity scheduler. (C)Sparse Tensor Product Gate: Similarly, it utilizes the sparsity scheduler to identify an
optimal subset of cross-order combinations (ℓ1, ℓ2, ℓ3)based on learnable weights. (D)Three-Phase Sparsity Scheduler: Designed for the
sparse gates, it operates in three phases: random, adaptive, and fixed.
product by:
c′=\{cℓ1,ℓ2,ℓ3×Wℓ1,ℓ2,ℓ3
c |(ℓ1, ℓ2, ℓ3)∈UTSS
c\}.(5)
In practical experiments, the sparsity rate kis set based on
the details in Section 5.4.
4.3. Sparse Pair Gate
Since the output of the Hamiltonian matrix encompasses
interactions between all pairs of atoms, previous models
typically define inter-atomic relationships as fully con-
nected. This results in tensor product computations that
scale quadratically with the number of nodes. However, we
observed that not every block requires such dense connec-
tivity. To reduce the number of tensor product operations
in the network, we developed the Sparse Pair Gate, which
adaptively selects the most important edges.
Similarly with the tensor product gate, the Sparse Pair Gateselects a subset from all the node pair Up=\{(i, j)|i̸=j\},
and uses a linear layer Fp(·)to learn the weight of each pair
Wij
pin this fully connected graph. This can be formulated
as below:
Iij= (x0
i∥x0
j∥⟨xi,xj⟩1:), (6)
Wij
p= Sigmoid ( Fp(Iij)), (7)
where x0
iandx0
jare the zero-order features of the irreps,
||is concatenation operation in the last feature dimension
and⟨·,·⟩stands for the inner product operation. Thus, the
sparse pair gate can be formulated as:
UTSS
p=\{(i, j)|Wij
p∈TSS(Wp, k)\}, (8)
where TSS(·)is the same as Equation 3. Accordingly, we
can get the pair weight wijofxi,xjused in tensor product:
wij=Fr(RBF( ⃗ rij))×Fs(Wij
p×Iij), (9)

\section{Efficient and Scalable Density Functional Theory Hamiltonian Prediction through Adaptive Sparsity}

where Fr(·)andFs(·)are linear layers, RBF(·)is the radial
basis function. The extra computational cost introduced by
above sparse gates and the scheduler is minimal and has
negligible impact on the overall time, which is discussed in
detailed in Appendix D.7.
4.4. SPHNet
With the above two sparse gates, SPHNet can be formu-
lated as four modules. The model takes atomic numbers Z
and 3D coordinates of molecular systems as input features,
initializes the node representation xiwith a linear func-
tion as xi=Fx(Zi), processes them through the Vectorial
Node Interaction Block, and sequentially passes through
the Spherical Node Interaction Block, Pair Construction
Module, and Expansion Module to output the predicted
Hamiltonian matrix H, as illustrated in Fig.2.
Node Interaction Block . The Node Interaction Block ag-
gregates information from neighboring nodes through a
message-passing mechanism, thereby extracting irreducible
representations of nodes. Specifically, the irreducible rep-
resentation xican be updated by aggregating the message
mijas,
ˆxℓ
i=Fm(xℓ
i+X
jmℓ
ij), (10)
where Fm(·)is a linear layer.
To achieve sufficiently interactive high-order information
while minimizing the associated computational cost, we
propose a mechanism that gradually increases the maximum
order of node representations. Initially, four Vectorial Node
Interaction Blocks are applied to obtain vectorial represen-
tations mℓ
ij, where ℓ≤1, as follows:
mℓ
ij=xℓ
j⊙(wij⊙⃗ rij),s.t. l≤1, (11)
where wijis the weight defined in Equation 9, and ⃗ rij
is the distance vector between atoms. This representation
does not involve interactions between high-order tensors,
thereby eliminating tensor product operations, as detailed in
Appendix D.3.
Next, we apply two Spherical Node Interaction Blocks to
capture interactions for high-order irreducible representa-
tions, increasing the highest order of the irreducible repre-
sentation to match the maximum required order determined
by the basis set. Specifically, these blocks project distance
information ⃗ rijinto the high-order spherical space using
spherical harmonics function Yℓ
m(·). Subsequently, mℓ
ijin
this block is computed under the constraints imposed by
Sparse Pair Gate as:
mℓ3
ij=X
ℓ1,ℓ2xℓ2
j⊗ℓ3
ℓ1,ℓ2wℓ1,ℓ2,ℓ3
ij Yℓ2
m(⃗ rij),
s.t. (i, j)∈UTSS
p, ℓ 1, ℓ2, ℓ3≤Lmax,(12)where Lmaxrepresents the allowed max order of irreps in
the network and wijis the weight defined in Equation 9.
Pair Construction Block . Consists of the Non-Diagonal
block and the Diagonal block. The Non-Diagonal block uses
these irreducible representations to compute the interaction
from node jto node i, generating the pair-wise features
fijfor the Non-Diagonal blocks of the Hamiltonian matrix.
The Diagonal block calculates the self-interaction of node i
to produce the node-wise feature fiifor the diagonal blocks
of the Hamiltonian matrix.
As shown in Fig.2(B), we use Sparse Tensor Product Gate
before both the Diagonal and Non-Diagonal block, and use
Sparse Pair Gate before Non-Diagonal block, therefore the
fii,fijcan be reformulated as:
fℓ3
ii=X
\{ℓ1,ℓ2,ℓ3\}∈UTSScW(ℓ1,ℓ2,ℓ3)
ii ×(ˆxℓ1
i⊗ℓ3
ℓ1,ℓ2ˆxℓ2
i),(13)
fℓ3
ij=X
\{ℓ1,ℓ2,ℓ3\}∈UTSScc′(ℓ1,ℓ2,ℓ3)×(ˆxℓ1
i⊗ℓ3
ℓ1,ℓ2wℓ1,ℓ2,ℓ3
ij ˆxℓ2
j),
s.t.(i, j)∈UTSS
p, ℓ 1, ℓ2, ℓ3≤Lmax,
(14)
where Wiiis learnable parameters for each combination
inUTSS
c,c′is an updated combination weight defined in
Equation 5 and wijis the weight defined in Equation 9.
It’s noteworthy that the Sparse Pair Gate is only employed
from the second Spherical Node Interaction Block and the
Non-Diagonal Pair Construction block to ensure complete
information flow between nodes, and more details could be
found in Appendix D.5.
Expansion Block . Finally, the expansion block utilizes
the tensor expansion operation to obtain the non-diagonal
Hamiltonian block hijand diagonal Hamiltonian block hii.
The detailed formulation can be found in Appendix D.6.
Considering the symmetry of the Hamiltonian matrix, where
for any sub-block hij, there exists a corresponding sub-
block hjisuch that hji=hT
ij. So we can further enhance
the model’s efficiency by constructing the node pair features
fijfrom the upper triangular part of the Hamiltonian ma-
trix, i.e., \{fij|i < j\}. Leveraging this symmetry allows
us to halve the number of tensor products required when
predicting the node pair irreps fij, significantly reducing the
computational burden associated with constructing hij.

\section{Result
We conducted a series of experiments to compare the overall}

performance of SPHNet with the previous state-of-the-art
model, including SchNOrb (Sch ¨utt et al., 2019), PhiSNet
(Unke et al., 2021), QHNet (Yu et al., 2023), and WANet (Li
et al., 2025b), and the results listed in the table are sourced
from their papers. Note that the results for PhiSNet and
SchNOrb can only be conducted on the MD17 dataset, as
these models are specifically designed for trajectory datasets

\section{Efficient and Scalable Density Functional Theory Hamiltonian Prediction through Adaptive Sparsity}

Table 1. Experimental results on QH9 dataset.
Dataset ModelH↓ ϵ↓ ψ↑ Memory ↓ Speed ↑ Speedup ↑
[10−6Eh] [10−6Eh] [10−2] [GB/Sample] [Sample/Sec] Ratio
QH9-stable iidQHNet 76.31 798.51 95.85 0.70 19.2 1.0x
WANet 80.00 833.62 96.86 N/A N/A N/A
SPHNet 45.48 334.28 97.75 0.23 76.80 4.0x
QH9-stable oodQHNet 72.11 644.17 93.68 0.70 21.12 1.0x
SPHNet 43.33 186.40 98.16 0.23 78.72 3.7x
QH9-dynamic geoQHNet 70.03 408.31 97.13 0.70 24.68 1.0x
WANet 74.74 416.57 99.68 N/A N/A N/A
SPHNet 52.18 100.88 99.12 0.23 82.56 3.3x
QH9-dynamic molQHNet 120.10 2182.06 89.63 0.70 23.04 1.0x
SPHNet 108.19 1724.10 91.49 0.23 80.64 3.5x
in conformational space. Additionally, since WANet has
not been open-sourced, we only report results on a subset
of datasets and metrics based on the information provided
in their paper. Detailed experimental settings are described
in Appendix A.
Datasets. Three datasets were used in the experiments:
The MD17 dataset includes Hamiltonian matrices for the
trajectory data of four molecules: water, ethanol, malon-
dialdehyde, and uracil, containing 4,900, 30,000, 26,978,
and 30,000 structures, respectively. The QH9 dataset (Yu
et al., 2024), consisting of Hamiltonian matrices for 134k
small molecules with no more than 20 atoms, and the Pub-
ChemQH dataset (Li et al., 2025b), containing Hamiltonian
matrices for 50k molecules with atom counts ranging from
40 to 100. The QH9 and PubChemQH use the B3LYP
exchange-correlation function, and the MD17 uses the PBE
exchange-correlation function. The orbital basis Def2-SVP
is used for MD17 and QH9, while the Def2-TZVP is used
for PubChemQH. Compared to the MD17 and QH9 datasets,
the maximum number of orbitals in the Hamiltonian of the
PubChemQH dataset increases by about 15 times, result-
ing in a 200x difference in the number of elements in the
Hamiltonian matrix.
Evaluation metrics. The evaluation metrics include Hamil-
tonian MAE (mean absolute error between predicted and
Hamiltonian matrices calculated by DFT), occupied energy
MAE ϵ(mean absolute error of energies of occupied molecu-
lar orbitals calculated from predicted Hamiltonian matrices),
Csimilarity(cosine similarity of coefficients for occupied
molecular orbitals), training GPU memory (GB per sample),
training speed (training samples per second) and speedup
ratio compared to QHNet.
5.1. Performance on QH9 Dataset
We first evaluated the performance of SPHNet on the QH9
dataset. Note that the QH9 dataset has two subsets and
four different splits, including the stable-iid, stable-ood,
dynamic-geo, and dynamic-mol. We trained SPHNet onfour different split sets and compared the results with the
baseline models QHNet and WANet. As shown in Table
1, for two stable split sets, SPHNet attained a speedup of
3.7x to 4x over QHNet while improving the accuracy of the
Hamiltonian Mean Absolute Error (MAE) and the occupied
energy MAE. Furthermore, SPHNet significantly reduced
GPU memory usage, requiring only 30\% memory usage of
the baseline model. For two dynamic split sets, SPHNet
maintained a speedup of 3.3x to 3.5x, simultaneously en-
hancing prediction accuracy and decreasing GPU memory
usage by over 70\%. These results underscore SPHNet’s effi-
ciency, achieving substantial speedups and resource savings
without sacrificing precision, indicating that SPHNet can
effectively learn molecular features and the Hamiltonian
matrix within the small-scale dataset, establishing a solid
foundation for calculations in larger-scale systems.
5.2. Performance on PubChemQH Dataset
We further validated our approach on the larger-scale molec-
ular system and trained SPHNet on the PubChemQH dataset.
It was observed that SPHNet trains over 7.1x times faster
than the baseline QHNet while achieving a better Hamilto-
nian prediction accuracy, as shown in Table 2, which has a
higher speedup ratio compared to that in the smaller-scale
datasets. Moreover, GPU memory consumption of SPH-
Net was significantly lower, at only 25\% and 37\% of the
baselines.
Table 2. Experimental results on PubChemQH dataset.
Model QHNet WANet SPHNet
H[10−6Eh]↓ 123.74 99.98 97.31
ϵ[Eh]↓ 3.33 1.17 2.16
ψ[10−2]↑ 2.32 3.13 2.97
Memory [GB/Sample] ↓ 22.5 15.0 5.62
Speed [Sample/Sec] ↑ 0.44 1.09 3.12
Speedup Ratio ↑ 1.0x 2.4x 7.1x
We observed that SPHNet achieves a higher speedup on

\section{Efficient and Scalable Density Functional Theory Hamiltonian Prediction through Adaptive Sparsity}

the PubChemQH dataset. As described in the dataset sec-
tion, the number of orbitals in the Hamiltonians of Pub-
ChemQH is 15 times greater than in QH9. Additionally,
since PubChemQH was computed using the Def2-TZVP or-
bital basis, the network requires a maximum output angular
momentum order of Lmax= 6, whereas datasets computed
with the Def2-SVP basis only require Lmax= 4. These
make the number and time consumption of tensor products
much higher than that in small-scale datasets. Thus, we can
reasonably infer that the higher speedup achieved on the
PubChemQH dataset is due to the markedly greater com-
putational complexity of its molecules compared to those
in the MD17 and QH9 datasets. This increased complexity
introduces higher information redundancy, making it more
amenable to adaptive sparsification. Our sparsity ablation
experiments across different datasets support this inference.
As molecular size and maximum angular momentum or-
der increase, the rising sparsity rates of sparse pair gates
and sparse tensor product gates generally result in smaller
accuracy losses. Therefore, higher sparsity rates can be
applied to SPHNet, further accelerating its performance on
large-scale datasets. For more details on this ablation study,
please refer to Section 5.4.
Table 3. Experimental results on MD17 dataset.
Dataset Model H[10−6Eh]↓ϵ[10−6Eh]↓ψ[10−2]↑
WaterSchNOrb 165.4 279.3 100.00
PhiSNet 15.67 85.53 100.00
QHNet 10.79 33.76 99.99
SPHNet 23.18 182.29 100.00
EthanolSchNOrb 187.4 334.4 100.00
PhiSNet 20.09 102.04 99.81
QHNet 20.91 81.03 99.99
SPHNet 21.02 82.30 100.00
Malon-
aldehydeSchNOrb 191.1 400.6 99.00
PhiSNet 21.31 100.6 99.89
QHNet 21.52 82.12 99.92
SPHNet 20.67 95.77 99.99
UracilSchNOrb 227.8 1760 90.00
PhiSNet 18.65 143.36 99.86
QHNet 20.12 113.44 99.89
SPHNet 19.36 118.21 99.99
5.3. Performance on MD17 Dataset
We also evaluated the performance of SPHNet on the
smaller-scale MD17 dataset. As presented in Table 3, we
compared SPHNet’s performance with four baseline models
across four MD17 molecules—water, ethanol, malondialde-
hyde, and uracil—comprising 3 to 12 atoms. The results
demonstrate that SPHNet achieves accuracy comparable to
other models, indicating its suitability for small molecular
trajectory datasets.
It is worth noting that predictions on the MD17 dataset rep-
resent a relatively simple task compared to other datasets, as
it focuses solely on small systems and their conformationalspaces. On the one hand, baseline models generally already
perform well across all these datasets, leaving limited room
for SPHNet to achieve significant improvements in either
accuracy or speed. On the other hand, taking the water
molecule with only three atoms as an example, the num-
ber of possible interaction combinations within the system
is inherently small, which limits the potential benefits of
adaptive sparsification.
5.4. Ablation Study on Sparsity Rate
The key idea of SPHNet is the implementation of adaptive
sparsity to reduce computational demands in SE(3) equiv-
ariant networks. The most critical hyperparameter in this
adaptive sparse system is the sparsity rate. To determine the
optimal sparsity rate and assess its impact across various
molecular systems, we conducted a series of experiments.
We varied the sparsity rate from 0\% to 90\% in 10\% intervals
and trained SPHNet on three datasets representing different
molecular scales (Here we chose Ethanol to represent MD17
dataset.). As illustrated in Fig.3, the predicted Hamiltonian
MAE across all three datasets remained relatively stable at
lower sparsity rates but increased significantly when the rate
reached a specific threshold. This finding suggests that an
appropriate range of sparsity has minimal impact on model
accuracy while simultaneously enhancing computational
speed.
Figure 3. The effect of different sparsity rates on the model per-
formance on the datasets with different scale of molecules. See
the detailed computational cost scaling in Appendix B.2 of Pub-
ChemQH Dataset.
Additionally, we observed that the threshold at which the
MAE begins to rise significantly becomes larger as the com-
plexity of the molecular system increases. For the smallest
MD17 system, the critical turning point is at 30\%, while for
the QH9 system, it is 40\%, and for the largest PubChemQH
system, it rises to 70\%. In the experiments conducted in this
study, we adopted the above critical turning point as the spar-
sity rate for training and testing for different datasets. This
observation aligns with our earlier conclusion that the adap-
tive sparsity strategy offers greater potential in large-scale
systems, as there are more calculations of lower importance
that can be optimized. Consequently, the system size can

\section{Efficient and Scalable Density Functional Theory Hamiltonian Prediction through Adaptive Sparsity
serve as a rough estimate for determining the applicable}

sparsity rate, thereby reducing the need for extensive param-
eter searches.
5.5. Analysis of Optimal Selected Set
Experiments demonstrated that the two adaptive sparse gates
effectively select the most important tensor products and
their optimal combinations. Here, we examined the selected
pairs and combinations to understand the learning outcomes
of the sparse gates.
For the optimal pair set, we analyzed the length(atomic dis-
tance) of the selected pairs, as illustrated in Fig.4(A). The
results indicate that sparse gating differs from the commonly
used RBF-based cutoff strategy, as it evaluates the impor-
tance of a pair based solely on its contribution to the output
irreps, rather than its distance. Interestingly, as the pair
length increases, the probability of a pair being selected also
rises. This is likely because the features of long-range pairs
are more challenging to learn, requiring as many samples as
possible to achieve accurate representations. Notably, the
selection probability increases nearly linearly in the distance
range of 16 ˚A to 25 ˚A. This range corresponds to the influ-
ence of electrostatic interactions and may also include weak
van der Waals forces. These findings highlight the impor-
tance of long-range interactions in Hamiltonian prediction
and demonstrate the effectiveness of sparse pair gating in
dynamically integrating both long-range and short-range in-
teractions, ultimately enabling the model to achieve superior
performance.
Figure 4. (A)The distribution of distance between selected node
pairs and the selected proportions of different pair distances in the
pair gate. (B)The proportions of each output order being selected
in the first Non-Diagonal block’s tensor product gate. See the full
cases of selected pairs in Appendix B.4.Then, for the selected optimal combination set, we ana-
lyze the order of output irreps of each selected cross-order
combination in tensor product operations. For a classical
tensor product operation, each irreps of the output xℓ3is con-
tributed by O(ℓ32)cross-channel combinations. As shown
in Fig.4(B) and Fig.7, in the QH9 dataset, the proportion
of each order that is selected decreases monotonically as
the order increases in both Sparse Combination Gates. In
the PubChemQH, the trend is the same except the order
zero has a relatively lower selected ratio. These results sug-
gest that although higher orders’ output irreps contain more
combinations, the importance of a single combination is
weakened as the number of combinations increases, making
them easier to filter out.
5.6. Scaling with Increasing Size of Hamiltonian
To further validate the limits of Hamiltonian prediction scale
achievable by different models under restricted GPU mem-
ory, as well as the scaling of speed and memory consump-
tion with increasing system size, we randomly selected 6
molecules of varying sizes from the PubChemQC PM6
dataset. We computed the labels for these molecules with
the setting of PubChemQH, and conducted tests with a sin-
gle A6000 card based on the outcomes. The results are
illustrated in Fig.5 and Fig.8.
Our method demonstrates better scaling than the baseline
model. As the Hamiltonian size (number of atomic orbitals)
increases, the speedup ratio improves while memory con-
sumption decreases. Thanks to significant savings in both
memory and time, our model can train on larger systems
(around 3000 atomic orbitals) within the same memory con-
sumption, whereas the baseline model is limited to approxi-
mately 1800 atomic orbitals.
Figure 5. Comparison of the training speed with the increasing size
of Hamiltonian.
5.7. Additional Ablation study on SPHNet
To evaluate the effectiveness of the Three-phase Sparsity
scheduler and the Sparse Gate on the model performance,
we carried out a series of ablation studies. The results
showed that these strategies have significant improvements
in model efficiency and accuracy. The details are as follows.

\section{Efficient and Scalable Density Functional Theory Hamiltonian Prediction through Adaptive Sparsity}

Three-phase Sparsity Scheduler. We evaluated the effect
of the three-phase sparsity scheduler on model performance
by training SPHNet under various ablation settings: ex-
cluding the random stage, the adaptive stage, or the fixed
stage. Each configuration was trained five times to assess
performance stability and accuracy. As shown in Table 4,
models trained without the random stage often converged to
local optima, resulting in suboptimal performance. Models
trained without the adaptive stage exhibited greater variance
in outcomes and significantly lower accuracy compared to
SPHNet with the full scheduler. Although omitting the fixed
stage did not substantially affect accuracy, it reduced the
speedup ratio to 5.45 due to the additional computational
overhead from the sparse gate and the non-static graph im-
plementation of tensor products following the Sparse Pair
Gate, in contrast to the optimized implementations in the
e3nn library. These findings highlight the importance of the
three-phase sparsity scheduler in enhancing model perfor-
mance and efficiency.
Table 4. Ablation Study on Three-phase Sparsity Scheduler.
Random Stage ✓ ✗ ✓ ✓
Adaptive Stage ✓ ✓ ✗ ✓
Fixed Stage ✓ ✓ ✓ ✗
H[10−6Eh]↓97.31 112 .68 122 .79 97 .11
±0.52±10.75±19.02 1 .33
Memory [GB/Sample] ↓ 5.62 5.62 5.62 5.62
Speed [Sample/Sec] ↑ 3.12 3.12 3.12 3.12
Speedup Ratio ↑ 7.09× 7.09× 7.09× 5.45×
Sparse Gates. To further evaluate the impact of the Sparse
Pair Gate and the Sparse Tensor Product Gate on overall
performance, we trained SPHNet under three configurations:
without the Sparse Pair Gate, without the Sparse Tensor
Product Gate, and without both gates. As shown in Table
5, the Sparse Pair Gate achieved a 78\% speedup and a 30\%
reduction in GPU memory usage, with minimal impact on
model accuracy. The Sparse Tensor Product Gate, which
removes 70\% of the combinations in the tensor product,
yielded a 160\% speedup with an acceptable loss in accuracy.
These results underscore the contribution of the Sparse Gate
to improving model efficiency. Users may refer to Section
5.4 for guidance on choosing a sparsity ratio that balances
speed and accuracy based on system size.
Table 5. Ablation Study on Sparse Gates.
Sparse Pair Gate ✓ ✗ ✓ ✗
Sparse TP Gate ✓ ✓ ✗ ✗
H[10−6Eh]↓ 97.31 94.31 87.70 86.35
Memory [GB/Sample] ↓ 5.62 8.04 6.98 10.91
Speed [Sample/Sec] ↑ 3.12 1.75 1.20 0.76
Speedup Ratio ↑ 7.09× 3.98× 2.73× 1.73×Applying Sparse Gate to QHNet model. To further as-
sess the effectiveness of the Sparse Gates, we conducted
additional experiments by integrating them into the QH-
Net model. Specifically, the Sparse Pair Gate was applied
to the second non-diagonal pair block, while the Sparse
Tensor Product Gate was applied to the node-wise interac-
tion blocks as well as both diagonal and non-diagonal pair
blocks. As shown in the table below, the sparse gates sig-
nificantly improved the training speed of the QHNet model
and reduced computational resource usage. These results
demonstrate the generalizability of sparse gating mecha-
nisms across different SE(3)-equivariant networks.
Table 6. The effect of sparse gates on QHNet model.
Sparse Pair Gate ✗ ✗ ✓ ✓
Sparse TP Gate ✗ ✓ ✗ ✓
H[10−6Eh]↓ 123.74 128.16 126.27 128.89
Memory [GB/Sample] ↓ 22.50 12.68 10.07 8.46
Speed [Sample/Sec] ↑ 0.44 0.90 0.73 1.45
Speedup Ratio ↑ 1.00× 2.04× 1.66× 3.30×

\section{Conclusion and Future Work}

In this work, we introduced SPHNet, an efficient and scal-
able SE(3) equivariant neural network for Hamiltonian pre-
diction. By incorporating two adaptive sparse gates and
a corresponding three-phase learning scheduler, SPHNet
optimized both the number of tensor product operations
and the efficiency of individual tensor computations. As
a result, SPHNet achieved exceptional efficiency and per-
formance in predicting Hamiltonians for larger molecular
systems. Moreover, the proposed sparsification techniques
demonstrated significant potential for extension to other
SE(3) equivariant networks and broader prediction tasks.
This study opens several promising avenues for future work.
First, the adaptive sparsity technique shows considerable
potential for generalization to a wider range of tasks. Specif-
ically, the sparse tensor product gates can be readily ex-
tended to any SE(3)-equivariant network architecture based
on Clebsch-Gordan tensor products, while the sparse pair
gates could be adapted to more types of pairwise interac-
tions, such as those found in MACE (Batatia et al., 2022)
for multi-body interactions. Second, integrating advanced
loss functions, such as those introduced in (Li et al., 2025b),
could further improve the accuracy of downstream proper-
ties of the Hamiltonian matrix, particularly in enhancing the
applicability for large molecular systems. These directions
will guide future efforts with further experimental valida-
tion, expanding the impact of adaptive sparsification and
optimizing the performance of SE(3)-equivariant networks.

\section{Efficient and Scalable Density Functional Theory Hamiltonian Prediction through Adaptive Sparsity
Impact Statement
This paper presents work whose goal is to speed up the
Hamiltonian matrix prediction process and advance the}

SE(3) equivariant network. There are many potential soci-
etal consequences of our work, none of which we feel must
be specifically highlighted here.
References
Atz, K., Grisoni, F., and Schneider, G. Geometric deep
learning on molecular representations. Nature Machine
Intelligence , 3(12):1023–1032, 2021.
Batatia, I., Kovacs, D. P., Simm, G., Ortner, C., and Cs ´anyi,
G. Mace: Higher order equivariant message passing
neural networks for fast and accurate force fields. Ad-
vances in Neural Information Processing Systems , 35:
11423–11436, 2022.
Batzner, S., Musaelian, A., Sun, L., Geiger, M., Mailoa,
J. P., Kornbluth, M., Molinari, N., Smidt, T. E., and
Kozinsky, B. E (3)-equivariant graph neural networks for
data-efficient and accurate interatomic potentials. Nature
communications , 13(1):2453, 2022.
Batzner, S., Musaelian, A., and Kozinsky, B. Advancing
molecular simulation with equivariant interatomic poten-
tials. Nature Reviews Physics , 5(8):437–438, 2023.
Burow, A. M. and Sierka, M. Linear scaling hierarchi-
cal integration scheme for the exchange-correlation term
in molecular and periodic systems. Journal of Chemi-
cal Theory and Computation , 7(10):3097–3104, 2011.
doi: 10.1021/ct200412r. URL https://doi.org/
10.1021/ct200412r . PMID: 26598153.
Cances, E. and Le Bris, C. On the convergence of scf algo-
rithms for the hartree-fock equations. ESAIM: Mathemat-
ical Modelling and Numerical Analysis , 34(4):749–774,
2000.
Chen, T., Luo, S., He, D., Zheng, S., Liu, T.-Y ., and Wang,
L. Geomformer: A general architecture for geomet-
ric molecular representation learning. arXiv preprint
arXiv:2406.16853 , 2024.
Du, W., Zhang, H., Du, Y ., Meng, Q., Chen, W., Zheng,
N., Shao, B., and Liu, T.-Y . Se (3) equivariant graph
neural networks with complete local frames. In Interna-
tional Conference on Machine Learning , pp. 5583–5608.
PMLR, 2022.
Fang, X., Liu, L., Lei, J., He, D., Zhang, S., Zhou, J., Wang,
F., Wu, H., and Wang, H. Geometry-enhanced molecular
representation learning for property prediction. Nature
Machine Intelligence , 4(2):127–134, 2022.Frankle, J. and Carbin, M. The lottery ticket hypothesis:
Finding sparse, trainable neural networks. In Interna-
tional Conference on Learning Representations (ICLR) ,
2019.
Fuchs, F., Worrall, D., Fischer, V ., and Welling, M. Se
(3)-transformers: 3d roto-translation equivariant attention
networks. Advances in neural information processing
systems , 33:1970–1981, 2020.
Gasteiger, J., Shuaibi, M., Sriram, A., G ¨unnemann, S.,
Ulissi, Z., Zitnick, C. L., and Das, A. Gemnet-oc: devel-
oping graph neural networks for large and diverse molecu-
lar simulation datasets. arXiv preprint arXiv:2204.02782 ,
2022.
Geiger, M., Smidt, T., M., A., Miller, B. K., Boomsma,
W., Dice, B., Lapchevskyi, K., Weiler, M., Tyszkiewicz,
M., Batzner, S., Madisetti, D., Uhrin, M., Frellsen,
J., Jung, N., Sanborn, S., Wen, M., Rackers, J., Rød,
M., and Bailey, M. Euclidean neural networks: e3nn,
April 2022. URL https://doi.org/10.5281/
zenodo.6459381 .
Gong, X., Li, H., Zou, N., Xu, R., Duan, W., and Xu, Y .
General framework for e (3)-equivariant neural network
representation of density functional theory hamiltonian.
Nature Communications , 14(1):2848, 2023.
Griffiths, D. J. and Schroeter, D. F. Introduction to quantum
mechanics . Cambridge university press, 2018.
Han, S., Pool, J., Tran, J., and Dally, W. Learning both
weights and connections for efficient neural network.
Advances in neural information processing systems , 28,
2015.
Hassibi, B. and Stork, D. G. Second order derivatives for
network pruning: Optimal brain surgeon. Advances in
Neural Information Processing Systems (NeurIPS) , 1993.
Hohenberg, P. and Kohn, W. Inhomogeneous electron gas.
Physical review , 136(3B):B864, 1964.
Jain, A., Shin, Y ., and Persson, K. A. Computational predic-
tions of energy materials using density functional theory.
Nature Reviews Materials , 1(1):1–13, 2016.
Kohn, W. and Sham, L. J. Self-consistent equations includ-
ing exchange and correlation effects. Physical review ,
140(4A):A1133, 1965.
Kudin, K. N., Scuseria, G. E., and Cances, E. A black-
box self-consistent field convergence algorithm: One step
closer. The Journal of chemical physics , 116(19):8255–
8261, 2002.

\section{Efficient and Scalable Density Functional Theory Hamiltonian Prediction through Adaptive Sparsity}

Laqua, H., Dietschreit, J. C. B., Kussmann, J., and Ochsen-
feld, C. Accelerating hybrid density functional theory
molecular dynamics simulations by seminumerical in-
tegration, resolution-of-the-identity approximation, and
graphics processing units. Journal of Chemical The-
ory and Computation , 18(10):6010–6020, 2022. doi:
10.1021/acs.jctc.2c00509. URL https://doi.org/
10.1021/acs.jctc.2c00509 . PMID: 36136665.
LeCun, Y ., Denker, J. S., and Solla, S. A. Optimal brain
damage. In Advances in Neural Information Processing
Systems (NeurIPS) , pp. 598–605, 1990.
Li, H., Wang, Z., Zou, N., Ye, M., Xu, R., Gong, X., Duan,
W., and Xu, Y . Deep-learning density functional theory
hamiltonian for efficient ab initio electronic-structure cal-
culation. Nature Computational Science , 2(6):367–377,
2022.
Li, Y ., Wang, Y ., Huang, L., Yang, H., Wei, X., Zhang, J.,
Wang, T., Wang, Z., Shao, B., and Liu, T.-Y . Long-short-
range message-passing: A physics-informed framework
to capture non-local interaction for scalable molecular
dynamics simulation. In International Conference on
Learning Representations , 2024.
Li, Y ., Huang, L., Ding, Z., Wang, C., Wei, X., Yang, H.,
Wang, Z., Liu, C., Shi, Y ., Jin, P., et al. E2former: A
linear-time efficient and equivariant transformer for scal-
able molecular modeling. arXiv e-prints , pp. arXiv–2501,
2025a.
Li, Y ., Xia, Z., Huang, L., Wei, X., Harshe, S., Yang, H.,
Luo, E., Wang, Z., Zhang, J., Liu, C., Shao, B., and
Gerstein, M. Enhancing the scalability and applicability
of kohn-sham hamiltonian for molecular systems. In
International Conference on Learning Representations ,
2025b.
Liao, Y .-L. and Smidt, T. Equiformer: Equivariant graph
attention transformer for 3d atomistic graphs. arXiv
preprint arXiv:2206.11990 , 2022.
Liao, Y .-L., Wood, B., Das, A., and Smidt, T. Equiformerv2:
Improved equivariant transformer for scaling to higher-
degree representations. arXiv preprint arXiv:2306.12059 ,
2023.
Liu, C., Ma, X., Zhan, Y ., Ding, L., Tao, D., Du, B., Hu, W.,
and Mandic, D. P. Comprehensive graph gradual prun-
ing for sparse training in graph neural networks. IEEE
Transactions on Neural Networks and Learning Systems ,
2023.
Musaelian, A., Batzner, S., Johansson, A., Sun, L., Owen,
C. J., Kornbluth, M., and Kozinsky, B. Learning local
equivariant representations for large-scale atomistic dy-
namics. Nature Communications , 14(1):579, 2023.Passaro, S. and Zitnick, C. L. Reducing so (3) convolu-
tions to so (2) for efficient equivariant gnns. In Inter-
national Conference on Machine Learning , pp. 27420–
27438. PMLR, 2023.
Payne, M. C., Teter, M. P., Allan, D. C., Arias, T., and
Joannopoulos, a. J. Iterative minimization techniques for
ab initio total-energy calculations: molecular dynamics
and conjugate gradients. Reviews of modern physics , 64
(4):1045, 1992.
Peng, Y . and Zhang, R. Towards effective sparsification for
molecular graph property prediction. Journal of Chemical
Information and Modeling , 2022.
Sch¨utt, K., Kindermans, P.-J., Sauceda Felix, H. E., Chmiela,
S., Tkatchenko, A., and M ¨uller, K.-R. Schnet: A
continuous-filter convolutional neural network for model-
ing quantum interactions. Advances in neural information
processing systems , 30, 2017.
Sch¨utt, K. T., Gastegger, M., Tkatchenko, A., M ¨uller, K.-
R., and Maurer, R. J. Unifying machine learning and
quantum chemistry with a deep neural network for molec-
ular wavefunctions. Nature communications , 10(1):5024,
2019.
Sun, Q., Liu, J., Zhang, X., and et al. Recent developments
in the pyscf program package. The Journal of Chemical
Physics , 153(2):024109, 2020.
Szabo, A. and Ostlund, N. S. Modern quantum chem-
istry: introduction to advanced electronic structure the-
ory. Courier Corporation, 2012.
te Vrugt, M., L ¨owen, H., and Wittkowski, R. Classical
dynamical density functional theory: from fundamentals
to applications. Advances in Physics , 69(2):121–247,
2020.
Th¨olke, P. and De Fabritiis, G. Torchmd-net: Equivariant
transformers for neural network based molecular poten-
tials. arXiv preprint arXiv:2202.02541 , 2022.
Unke, O., Bogojeski, M., Gastegger, M., Geiger, M., Smidt,
T., and M ¨uller, K.-R. Se (3)-equivariant prediction of
molecular wavefunctions and electronic densities. Ad-
vances in Neural Information Processing Systems , 34:
14434–14447, 2021.
Wang, J. and Baerends, E. J. Self-consistent-field method
for correlated many-electron systems with an entropic
cumulant energy. Physical review letters , 128(1):013001,
2022.
Wang, N., Lin, C., Bronstein, M., and Torr, P. Towards
flexible, efficient, and effective tensor product networks.
InNeurIPS 2023 Workshop: New Frontiers in Graph
Learning , 2023.

\section{Efficient and Scalable Density Functional Theory Hamiltonian Prediction through Adaptive Sparsity}

Wang, Y ., Li, S., He, X., Li, M., Wang, Z., Zheng, N.,
Shao, B., Liu, T.-Y ., and Wang, T. Visnet: an equivariant
geometry-enhanced graph neural network with vector-
scalar interactive message passing for molecules. arXiv
preprint arXiv:2210.16518 , 2022.
Wang, Z., Liu, C., Zou, N., Zhang, H., Wei, X., Huang,
L., Wu, L., and Shao, B. Infusing self-consistency into
density functional theory hamiltonian prediction via deep
equilibrium models. arXiv preprint arXiv:2406.03794 ,
2024.
Yu, H., Xu, Z., Qian, X., Qian, X., and Ji, S. Efficient
and equivariant graph networks for predicting quantum
hamiltonian. In International Conference on Machine
Learning , pp. 40412–40424. PMLR, 2023.
Yu, H., Liu, M., Luo, Y ., Strasser, A., Qian, X., Qian, X., and
Ji, S. Qh9: A quantum hamiltonian prediction benchmark
for qm9 molecules. Advances in Neural Information
Processing Systems , 36, 2024.
Zang, X., Zhao, X., and Tang, B. Hierarchical molecular
graph self-supervised learning for property prediction.
Communications Chemistry , 6(1):34, 2023.
Zhang, H., Liu, C., Wang, Z., Wei, X., Liu, S., Zheng, N.,
Shao, B., and Liu, T.-Y . Self-consistency training for
density-functional-theory hamiltonian prediction. In In-
ternational Conference on Machine Learning , pp. 59329–
59357. PMLR, 2024.
Zhong, Y ., Yu, H., Su, M., Gong, X., and Xiang, H. Transfer-
able equivariant graph neural networks for the hamiltoni-
ans of molecules and solids. npj Computational Materials ,
9(1):182, 2023.
Zhou, G., Lubbers, N., Barros, K., Tretiak, S., and Neb-
gen, B. Deep learning of dynamically responsive chemi-
cal hamiltonians with semiempirical quantum mechanics.
Proceedings of the National Academy of Sciences , 119
(27):e2120333119, 2022.

\section{Efficient and Scalable Density Functional Theory Hamiltonian Prediction through Adaptive Sparsity}

A. Additional Experimental Settings
A.1. Training Setup.
The baseline model QHNet was using its default setting. The training setting for SPHNet is shown in Table 7, most training
settings were set to align with the QHNet’s experiment to make a fair comparison. The batch size for training is 10 on the
MD17 dataset (except Uracil which was set to 5), 32 on the QH9 dataset, and 8 on the PubChemQH dataset. The training
step was 260,000 for the QH9 dataset, 200,000 for the MD17 dataset, and 300,000 for the PubChemQH dataset. There
was a 1,000-step warmup with the polynomial schedule. The maximum learning rate was 1e-3 for the QH9 dataset and
PubChemQH dataset and was 5e-4 for the MD17 dataset. The training and test sets in the QH9 dataset were split in the
same manner as their official implementation, including four different split sets. The MD17 dataset was randomly split into
the train, validation, and test sets of the same size as the QHNet experiment. The PubChemQH datasets were split randomly
into train, validation, and test sets by 80\%, 10\%, and 10\%. The batch size for speed and GPU memory test was set to the
maximum number that the GPU can hold to maximize its capability for fair comparison.
Table 7. Model Training Settings for Different Datasets.
Dataset Batch Size Training Steps Warmup Steps Learning Rate Lmax Sparsity Rate TSS Epoch tTrain/Val/Test Split Method
MD17 Water 10 200,000 1,000 5e-4 4 0.1 3 Random (500/500/3900)
MD17 Ethanol 10 200,000 1,000 5e-4 4 0.3 3 Random (25000/500/4500)
MD17 Malondialdehyde 10 200,000 1,000 5e-4 4 0.3 3 Random (25000/500/1478)
MD17 Uracil 5 200,000 1,000 5e-4 4 0.3 3 Random (25000/500/4500)
QH9 32 260,000 1,000 1e-3 4 0.4 3 Official (4 splits)
PubChemQH 8 300,000 1,000 1e-3 6 0.7 3 Random (40257/5032/5032)
A.2. Hardware and Software.
Our experiments are implemented based on PyTorch 2.1.0, PyTorch Geometric 2.5.0, and e3nn (Geiger et al., 2022) 0.5.1.
In our experiments, the speed and GPU memory metrics are tested on a single NVIDIA RTX A6000 46GB GPU. The
complete training of the models is carried on 4 ×80GB Nvidia A100 GPU. Our code is available in supplementary material.
A.3. Problem Formulation.
The loss function of SPHNet is the MAE plus MSE between predicted matrix Hpred and ground truth matrix Href.
loss=MAE (Href,Hpred) +MSE (Href,Hpred). (15)
SPHNet predicts the gap between the Hamiltonian matrix and the initial guess. The predicted target can be written as:
∆H=Href−Hinit, (16)
where Hinit is the initial guess of Hamiltonian matrix get with the pyscf (Sun et al., 2020) by function
initguess byminao (·). On one hand, we observed that the scale and variance of ∆Hare approximately an order
of magnitude smaller than those of Hrefon large datasets such as PubchemQH, which facilitates more stable learning across
different datasets. On the other hand, the computational cost for calculating the initial guess is very cheap, it would not take
long to obtain all the initial guesses of molecules in the dataset. Thus, we switched the training target to the ∆Hand added
theHinitback to the predicted ∆Hto get the full Hamiltonian matrix.
A.4. Evaluation Metric
Mean Absolute Error (MAE) of Hamiltonian Matrix H: This metric quantifies the Mean Absolute Error in relation
to ground truth data obtained from Density Functional Theory (DFT), accounting for both diagonal and Non-Diagonal
elements that reflect intra- and inter-atomic interactions, respectively. This can be presented as:
H=mean (|Hpred−Hgt|), (17)
where Hpredis the predicted Hamiltonian matrix and Hgtis the reference Hamiltonian matrix .
Mean Absolute Error (MAE) of Occupied Orbital Energies ϵ: This metric evaluates the MAE of the occupied orbital
energies, specifically the Highest Occupied Molecular Orbital (HOMO) and Lowest Unoccupied Molecular Orbital (LUMO)

\section{Efficient and Scalable Density Functional Theory Hamiltonian Prediction through Adaptive Sparsity}

levels. The accuracy of these critical properties is assessed by comparing the energies derived from the predicted Hamiltonian
matrix against those obtained from the reference DFT calculations. The MAE can be expressed as:
ϵ=1
MMX
k=1|ϵpred
k−ϵref
k|, (18)
where Mrepresents the number of occupied orbitals, ϵpred
kandϵref
kare the predicted and reference energy of the k-th occupied
orbital.
Cosine Similarity of Orbital Coefficients ψ: To evaluate the similarity between the predicted and reference electronic
wavefunctions, we calculate the cosine similarity of the coefficients corresponding to the occupied molecular orbitals. This
similarity metric is pivotal for understanding and predicting the chemical properties of the system. The cosine similarity S
can be defined as:
S(ψpred, ψref) =P
iψpred
iψref
i
∥ψpred∥∥ψref∥, (19)
where ∥ψ∥denotes the norm of the vector ψ,ψref
iandψrefare the Coefficient of the i-th occupied molecular orbital in the
predicted and reference wavefunction
B. Additional Experimental Results
B.1. Ablation Study for blocks of SPHNet
We conducted an ablation study to evaluate the effect of different modules in the SPHNet architecture. Specifically,
the standard SPHNet model has 4 Vectorial Node Interaction blocks, 2 Spherical Node Interaction blocks, and 2 Pair
Construction blocks. We removed all the sparse gates and reduced the number of these three kinds of modules to 1,
respectively, and observed the model performance. As shown in the table below, we found that both the Vectorial Node
Interaction block and the Spherical Node Interaction block significantly affect the model performance, indicating that
the design of architectures with progressively increased irreps orders has an important positive impact on the models.
Interestingly, we found that removing one Pair Construction block would not strongly affect the model accuracy, suggesting
that there is actually room to further speed up the model. We will explore this further in our future work.
Table 8. The effect of blocks’ numbers in SPHNet on the PubChemQH dataset.
ModelSparse Pair Sparse TP Vectorial Node Spherical Node Pair Construction H↓
Gate Gate Interaction block Interaction block block [ 10−6Eh]
SPHNet ✗ ✗ 4 2 2 86.35
SPHNet ✗ ✗ 1 2 2 96.01
SPHNet ✗ ✗ 4 1 2 97.35
SPHNet ✗ ✗ 4 2 1 89.17

\section{Efficient and Scalable Density Functional Theory Hamiltonian Prediction through Adaptive Sparsity}

B.2. The Effect of Sparity Rate on the Computational Cost
We analyze the effect of sparsity rate on the model performance in Section 5.4. Here we put the complete results of training
speed and GPU memory in the PubChemQH dataset under every sparse rate. The results showed that the computational cost
reduced linearly as the sparsity increased, which was in line with our expectations.
Table 9. The effect of sparsity rate v.s. computational cost on PubChemQH dataset
Sparsity H [10−6Eh]Speed [Sample/Sec] Memory [GB/Sample]
0 78.48 0.78 17.02
10 79.90 0.96 12.49
20 81.29 1.05 11.21
30 85.33 1.24 10.93
40 86.65 1.50 6.84
50 90.52 1.86 6.22
60 91.90 2.31 5.12
70 97.31 3.12 5.62
80 158.70 3.92 4.68
90 183.52 5.13 3.38
B.3. Single Tensor Product Time Scaling with Sparsity Rate
We test the time consumption for a single tensor product under different sparsity rates. As shown in Fig.6, the time
consumption decreases linearly with increasing sparsity, in line with our expectations. This experiment suggested that the
sparsity strategy can accelerate the tensor product speed on a linear scale.
Figure 6. The time consumption for a single tensor product with scaling sparsity rates.
B.4. Selected Combination Set Details
Here we presented the details of the selected combination sets in all Sparse TP Gate in the SPHNet. We first analysis the
output order of the selected combinations. As shown in Fig.7, the selected combinations’ output order in the Non-Diagonal
block shows a monotonically decreasing trend as analyzed in section 5.5, suggesting the weakening of a single combination
as the number of combinations within the same order increase. In the Diagonal block, we didn’t observe any apparent
pattern, this might be because all the combinations have similar effects on the model performance. However, this requires
further analysis in the future.

\section{Efficient and Scalable Density Functional Theory Hamiltonian Prediction through Adaptive Sparsity}

Figure 7. The proportions of each output order being selected in different tensor product gates.
We then check the combination weights for each of the selected combinations. The details are shown in Table 10, Table
11 and Table 12. These results are from the Sparse Tensor Product Gate before the first Non-Diagonal block. For the
PubchemQH dataset, there are a total of 175 combinations in the tensor product (maximum irreps order 6), and the Sparse
Pair Gate selects 30\% of these combinations. For the QH9 dataset and MD17 dataset, there are a total of 65 combinations in
the tensor product (maximum irreps order 4), and the Sparse Pair Gate selects 60\% of these combinations for QH9, 70\% of
these combinations for MD17.
Table 10. Weight of selected combinations in PubChemQH
Rank Comb Weight Rank Comb Weight Rank Comb Weight
1 (4, 2, 3) 1.1422 19 (2, 5, 5) 1.0736 37 (4, 4, 2) 1.0488
2 (5, 4, 2) 1.1398 20 (3, 0, 3) 1.0725 38 (3, 1, 2) 1.0530
3 (0, 4, 4) 1.1160 21 (4, 3, 5) 1.0710 39 (4, 4, 2) 1.0488
4 (0, 3, 3) 1.1093 22 (0, 5, 5) 1.0710 40 (5, 2, 3) 1.0486
5 (3, 5, 2) 1.1081 23 (2, 3, 3) 1.0682 41 (2, 3, 1) 1.0485
6 (6, 6, 6) 1.1078 24 (4, 6, 2) 1.0664 42 (6, 5, 3) 1.0482
7 (5, 2, 5) 1.1031 25 (4, 6, 4) 1.0651 43 (3, 3, 4) 1.0478
8 (6, 3, 5) 1.1028 26 (6, 3, 3) 1.0647 44 (4, 5, 1) 1.0464
9 (4, 0, 4) 1.1015 27 (6, 5, 1) 1.0607 45 (5, 4, 1) 1.0452
10 (1, 4, 3) 1.0978 28 (3, 5, 4) 1.0600 46 (5, 0, 5) 1.0417
11 (2, 5, 3) 1.0917 29 (6, 4, 2) 1.0596 47 (3, 4, 1) 1.0377
12 (5, 3, 2) 1.0903 30 (2, 0, 2) 1.0579 48 (4, 3, 1) 1.0300
13 (2, 1, 3) 1.0856 31 (3, 2, 1) 1.0566 49 (1, 0, 1) 1.0283
14 (5, 5, 4) 1.0854 32 (5, 4, 4) 1.0545 50 (6, 6, 4) 1.0281
15 (4, 4, 4) 1.0842 33 (5, 6, 1) 1.0537 51 (5, 3, 4) 1.0274
16 (2, 4, 2) 1.0835 34 (4, 3, 3) 1.0535 52 (5, 5, 5) 1.0248
17 (3, 6, 3) 1.0829 35 (4, 5, 3) 1.0534 53 (2, 2, 2) 0.9987
18 (1, 2, 1) 1.0799 36 (3, 1, 2) 1.0530

\section{Efficient and Scalable Density Functional Theory Hamiltonian Prediction through Adaptive Sparsity}

Table 11. Weight of selected combinations in QH9
Rank Comb Weight Rank Comb Weight Rank Comb Weight
1 (1, 3, 2) 1.1748 14 (3, 1, 2) 1.0758 27 (3, 0, 3) 1.0154
2 (2, 4, 2) 1.1722 15 (2, 0, 2) 1.0715 28 (2, 3, 4) 1.0150
3 (3, 2, 1) 1.1661 16 (2, 2, 0) 1.0696 29 (4, 4, 1) 1.0145
4 (3, 4, 1) 1.1433 17 (3, 2, 3) 1.0577 30 (1, 0, 1) 1.0145
5 (3, 4, 3) 1.1284 18 (2, 3, 1) 1.0530 31 (0, 2, 2) 1.0133
6 (2, 1, 3) 1.1240 19 (0, 0, 0) 1.0406 32 (3, 4, 2) 1.0117
7 (3, 3, 2) 1.1233 20 (4, 4, 0) 1.0383 33 (1, 3, 3) 1.0108
8 (2, 2, 2) 1.1163 21 (4, 3, 1) 1.0336 34 (2, 2, 1) 1.0099
9 (1, 1, 0) 1.1157 22 (1, 2, 1) 1.0319 35 (4, 2, 3) 1.0097
10 (4, 4, 2) 1.1085 23 (1, 1, 1) 1.0276 36 (2, 3, 2) 1.0090
11 (0, 3, 3) 1.0972 24 (3, 4, 4) 1.0213 37 (3, 2, 2) 1.0059
12 (3, 3, 1) 1.0853 25 (0, 4, 4) 1.0198 38 (3, 1, 4) 1.0057
13 (0, 1, 1) 1.0788 26 (3, 3, 0) 1.0154 39 (4, 0, 4) 0.9912
Table 12. Weight of selected combinations in MD17 (Ethanol)
Rank Comb Weight Rank Comb Weight Rank Comb Weight
1 (3, 2, 1) 1.0755 16 (3, 3, 1) 1.0213 31 (3, 3, 0) 0.9894
2 (3, 4, 1) 1.0674 17 (4, 4, 2) 1.0206 32 (4, 4, 1) 0.9891
3 (2, 1, 1) 1.0608 18 (2, 2, 1) 1.0141 33 (4, 1, 3) 0.9886
4 (1, 2, 1) 1.0592 19 (3, 1, 2) 1.0119 34 (2, 2, 2) 0.9878
5 (4, 3, 1) 1.0449 20 (2, 0, 2) 1.0090 35 (3, 2, 2) 0.9856
6 (0, 2, 2) 1.0447 21 (4, 4, 0) 1.0037 36 (4, 3, 3) 0.9851
7 (0, 3, 3) 1.0446 22 (2, 1, 3) 1.0016 37 (3, 3, 3) 0.9811
8 (2, 4, 2) 1.0418 23 (3, 3, 2) 1.0003 38 (3, 4, 4) 0.9802
9 (2, 3, 1) 1.0417 24 (3, 4, 2) 0.9995 39 (3, 4, 3) 0.9795
10 (1, 3, 2) 1.0408 25 (0, 4, 4) 0.9991 40 (1, 3, 4) 0.9787
11 (0, 1, 1) 1.0399 26 (3, 0, 3) 0.9978 41 (2, 3, 2) 0.9769
12 (1, 1, 2) 1.0349 27 (4, 3, 2) 0.9972 42 (3, 1, 3) 0.9763
13 (1, 0, 1) 1.0330 28 (3, 2, 3) 0.9928 43 (2, 4, 3) 0.9736
14 (4, 2, 2) 1.0301 29 (2, 3, 4) 0.9920 44 (2, 3, 3) 0.9730
15 (1, 4, 3) 1.0295 30 (0, 0, 0) 0.9911 45 (1, 2, 2) 0.9719
B.5. GPU Memory when Scaling with Increasing Size
We used molecules with different sizes to test the GPU memory consumption of SPHNet and the baseline model QHNet. As
shown in Fig.8, we observed that the GPU memory consumption of SPHNet increased much slower than the baseline model.
When the molecule had more than 1800 atomic orbitals, the baseline model reached the maximum GPU memory, which is 6
times the memory SPHNet needs, while the SPHNet can handle molecules with around 2900 atomic orbitals, making it
possible to train on larger molecule systems.
C. Additional Preliminary
C.1. Spherical Harmonic Function
The spherical harmonic function in SPHNet is used to project the atomic vector ⃗ rijinto spherical space. A real basis for
spherical harmonics Ylm:S2→Rcan be expressed in relation to their complex analogues Yl
m:S2→C. This relationship
is formulated through the following equations:

\section{Efficient and Scalable Density Functional Theory Hamiltonian Prediction through Adaptive Sparsity}

Figure 8. The GPU memory consumption with scaling molecule size.
Yl
m=

i√
2Yl
−|m|−(−1)mYl
|m|, ifm < 0.
Yl
0, ifm= 0.
1√
2
Yl
|m|+ (−1)mYl
−|m|
,ifm > 0.(20)
To ensure both consistency and standardized treatment throughout the analysis, the Condon–Shortley phase convention is
frequently utilized. The inverse relationships that define the complex spherical harmonics Yl
m:S2→Cin terms of the real
spherical harmonics Ylm:S2→Rare outlined as follows:
Yl
m=

1√
2
Y|l|
|m|−iYl,−|m|
, ifm < 0.
Yl,0, ifm= 0.
(−1)m1√w
Yl,|m|+iYl,−|m|
,ifm > 0.(21)
Established theories regarding the analytical solutions of the hydrogen atom indicate that the eigenfunctions corresponding
to the angular component of the wave function are represented as spherical harmonics. Notably, in the absence of magnetic
interactions, the solutions to the non-relativistic Schr ¨odinger equation can also be represented as real functions. This aspect
highlights the common use of real-valued basis functions in electronic structure calculations, as it simplifies software
implementations by eliminating the need for complex algebra. It is crucial to acknowledge that these real-valued functions
occupy a functional space identical to that of their complex counterparts, thus preserving generality and completeness in the
solutions.
C.2. Irreps and Tensor Products
Irreps Representation : SPHNet employs the special orthogonal group SO(3)to capture the essential 3D rotational
symmetries intrinsic to molecular structures. It leverages the irreducible representations (irreps) of SO(3), indexed by
an integer l, which are associated with spherical harmonic functions Ylm. These spherical harmonics impart rotational
characteristics to the feature vectors, thereby ensuring that the model maintains rotational invariance and facilitates consistent
assessment of geometric properties.
Tensor Product : Tensor product is the core operation in SPHNet. It facilitates interactions among irreps linked to distinct
order (angular momentum) land enhances expressiveness within the model. This operation merges two irreps with order
l1andl2to generate a new irrep characterized by order l3. The expansion is accomplished by utilizing Clebsch-Gordan
coefficients, weighted by wm1,m2:
(xℓ1⊗yℓ2)ℓ3
m3=ℓ1X
m1=−ℓ1ℓ2X
m2=−ℓ2wℓ1,ℓ2,ℓ3
m1,m2C(ℓ3,m3)
(ℓ1,m1),(ℓ2,m2)xℓ1
m1yℓ2
m2, (22)

\section{Efficient and Scalable Density Functional Theory Hamiltonian Prediction through Adaptive Sparsity}

This mathematical approach allows for the amalgamation of complex features derived from simpler ones, effectively
capturing the nuanced interactions of angular momentum in the molecular context.
D. Additional Model Details.
Figure 9. The components of SPHNet. (A)The Vectorial Node Interaction Block, which uses a long-short range message-passing
mechanism. (B)The Expansion block. (C)The Spherical Node Interaction Block. (D)The Diagonal block in the Pair Construction block.
(E)The Non-Diagonal block in the Pair Construction block.
D.1. RBF
The RBF here refers to the Radial Basis Functions and is used to construct nonlinear maps of vectors r. We used Exponential
Bernstein Radial Basis Functions here to process the vector of two atoms, and the mathematical formulation is as below:
RBF(r) =Fcut(r,cutoff )×exp (log c+n·(−αr) +v·log(−exp(−αr) + 1)) , (23)
and the cutoff function Fcutis to limit the range of action of the RBF, beyond which inputs will be ignored or reduced in
impact:
Fcut(x,cutoff ) =(
exp
−x2
(cutoff−x)(cutoff+x)
,ifx < cutoff .
0, otherwise .(24)
D.2. Sph Linear
The Sph Linear used in the SPHNet was implemented through the o3.Linear function in the e3nn (Geiger et al., 2022). It is
an O(3) linear operation that takes an irrep as input and outputs the linear combination of input. Specifically, the feature of
each order in the input irreps will be mapped to the feature of a specific order of the output irreps through a learnable matrix.
We didn’t set the instruction attribution in the e3nn ’s (Geiger et al., 2022) function, which resulted in a fully connected
instruction and each order in the input and output irreps will have a connection.

\section{Efficient and Scalable Density Functional Theory Hamiltonian Prediction through Adaptive Sparsity}

D.3. Vectorial Node Interaction Block
As shown in Fig.9(A), the Node Vectorial Node Interaction Block receives the atomic numbers Zand 3D coordinates of the
molecular system ⃗ rijas input and extracts the representations xifor node i. In this block, we adopted the long-short range
message-passing mechanism (Li et al., 2024) to capture the long-range interaction between each atom and obtain more
informative node representations. There are 4 short-range message-passing modules and 4 long-range message-passing
modules. Specifically, for the short-range message-passing module, the module passes the neighboring atoms’ features to
the center atom as defined below:
ˆxℓ
i=Fshort(xℓ
i+X
j∈N(i)mℓ
ij), (25)
where Fshort is the linear layer, mijis the message from atom jto atom i, and the N(i)is the neighboring atoms of atom i.
For long-range message-passing module, the module passes the neighboring groups’ feature to the center atom as defined
below:
ˆxℓ
i=Flong(xℓ
i+X
j∈N(i)mℓ
ij), (26)
where Flongis the linear layer, N(i)is the neighborhood group of atom ion the atom-fragment bipartite graph, and mijis
the message from neighboring group jto atom i.
D.4. Spherical Node Interaction Block
As shown in Fig.9(C), the Spherical Node Interaction Block takes the node feature xiand 3D coordinates of the molecular
system ⃗ rijas input and outputs the new high-order feature through a tensor product operation. Specifically, the input atom
feature xiand its neighboring atom feature are first fed to the Sparse Pair Gate and obtain the selected important pair set
(x′
i,(⃗ rij)′)and their pair weight Wij
p. This weight feature is then multiplied with the RBF of ⃗ rijand gets the final tensor
product weight as defined in Equation 9 and 10. The atom feature xiis finally doing tensor product with the high-order
spherical harmonics projection of the selected (⃗ rij)′, as defined in Equation 12. Last, the ascended node feature ˆxiis
outputted after normalization and sph linear:
xl
i=(
Layernorm (xl
i), ifl= 0.
xl
i×1
LmaxPLmax
l=1
1
2l+1Pl
m=−lxl
i,m2
,ifl >0.(27)
xi=Sphlinear (xi). (28)
There are two Spherical Node Interaction Blocks in the SPHNet. The first block increases the maximum order of input
irreps from zero to the required order without utilizing the pairs selected by the Sparse Pair Gate. The second block inputs
and outputs irreps with the same maximum order, and employs the pairs (x′
i,(⃗ rij)′)selected by the Sparse Pair Gate.
D.5. Pair Construction Block
The Hamiltonian matrix contains the relational information of each pair of atoms in the molecular system. The objective of
the Pair Construction Block is to extend the model’s capacity to consider diagonal node pairs fiiand non-diagonal node pairs
fijin the Hamiltonian matrix. Therefore, there are two subblocks in the Pair Construction block, the Diagonal block for the
diagonal feature and the Non-Diagonal block for the Non-Diagonal feature. There are two Pair Construction blocks in the
SPHNet, to ensure that each subblock hijin the Hamiltonian matrix is covered at least once, the first Pair Construction
block does not use the pair selected by Sparse Pair Gate.
D.5.1. D IAGONAL BLOCK
As shown in Fig.9D, the Diagonal block first uses two separate sph linear layers to get two different representations from the
input node feature, and then uses a self-tensor product operation to obtain fii. Here, the combinations in the tensor product
are selected by the Sparse Tensor Product Gate:
ˆ xi1=Fs1(ˆ xi),ˆ xi2=Fs2(ˆ xi), (29)

\section{Efficient and Scalable Density Functional Theory Hamiltonian Prediction through Adaptive Sparsity}

fℓ3
ii=X
\{ℓ1,ℓ2,ℓ3\}∈UTSScW(ℓ1,ℓ2,ℓ3)
ii ×(ˆ xℓ1
i1⊗ℓ3
ℓ1,ℓ2ˆ xℓ2
i2), (30)
where xiis the atom feature from the Spherical Node Interaction Block, and Fs1andFs2are the sph linear layer. Wiiis the
combination weight defined in Equation 13. Last, the diagonal feature fiiis outputted after normalization and sph linear the
same as in the Spherical Node Interaction block.
D.5.2. N ON-DIAGONAL BLOCK
For the non-diagonal pairs, the module first uses Sparse Pair Gate to select the most valuable pair \{x′i,x′j\}and their tensor
product weight wij, as defined in Equation 9. The tensor product weight is then multiplied with the RBF of⃗r′
ijand gets the
final tensor product weight as defined in Equation 12. Then, the module calculates the tensor product of x′iandx′j, the
process is presented in Fig.9E. Note that the combinations in the tensor product are selected by the Sparse Tensor Product
Gate. The result is regarded as the desired fij, defined in Equation 14. Last, the non-diagonal feature fijis outputted after
normalization and sph linear the same as in the Spherical Node Interaction block.
There are two separate Pair Construction Blocks that receive atom representations from the two SO(3) Convolution Blocks
respectively, and the final node pair feature is the addition of these two Pair Construction Blocks’ output.
D.6. Expansion Block
Once we have gathered the irreps for both diagonal and non-diagonal pairs, the subsequent step involves constructing the
complete Hamiltonian matrix. There are many matrix elements in the Hamiltonian matrix, each matrix element represents
the interaction between two orbitals. Specifically, the block denoted as hijcaptures all interactions between atoms iandj.
Since atoms have varying numbers of orbitals, the shape of the pair block hijis different, and must be determined during
the construction of the Hamiltonian matrix.
Here, same as the previous work (Yu et al., 2023), we introduce an intermediate block Mijcontaining the full orbitals, from
which we can derive hijby extracting the relevant components based on the specific atom types. For instance, in the MD17
dataset, there are four atoms—H, C, N, and O—each with its own set of orbitals. The full orbital set consists of 1s, 2s, 3s, 2p,
3p, 3d orbitals, where M∈R14×14. When dealing with the hydrogen atom (H), only the 1s, 2s, and 2p orbitals are selected
to contribute to the Hamiltonian matrix. By using this strategy, each node’s irrep fijcan be converted into an intermediate
block Mijwith a predetermined structure, irrespective of the atom type. This technique is particularly advantageous as it
can be easily adapted to various molecules.
To construct the intermediate pair blocks with full orbitals using pair irreducible representations, we apply a tensor expansion
operation in conjunction with the filter operation. This expansion is defined by the following relation:

⊗ℓofℓo(mi,mj)
(ℓi,ℓj)=ℓoX
mo=−ℓoC(ℓo,mo)
(ℓi,mi),(ℓj,mj)fℓo
mo, (31)
where Cdenotes the Clebsch-Gordan coefficients, and ⊗symbolizes the tensor expansion which is the converse operation
of the tensor product. Specifically, xℓi⊗yℓjcan be expressed as a sum over tensor expansions:
xℓi⊗yℓj=X
ℓ3Wli,lj,lo⊗fℓo, (32)
subject to the angular momentum coupling constraints |ℓi−ℓj| ≤ℓo≤ℓi+ℓj. The filter then takes the atom types as input,
producing a weight for each path (ℓo1, ℓo2, ℓin):
F(ℓo1,ℓo2,ℓin)
ij =f(Zi, Zj),F(ℓo1,ℓo2,ℓin)
ii =f(Zi), (20)
where Zdenotes the embedding of the atom types. The intermediate blocks Mare generated by the filter, using the node
pair irreducible representations as follows:
M(ℓo1,ℓo2)
ii =X
ℓin,c′F(ℓo1,ℓo2,ℓin)
ii ⊗f(ℓin,c′)
ii , (21)

\section{Efficient and Scalable Density Functional Theory Hamiltonian Prediction through Adaptive Sparsity}

M(ℓo1,ℓo2)
ij =X
ℓin,c′F(ℓo1,ℓo2,ℓin)
ij ⊗f(ℓin,c′)
ij . (33)
Here, c′indicates the channel index in the input irreducible representations, and crepresents the channel index in M. For
instance, there are nine channels with (ℓ1, ℓ2) = (0 ,0)and four channels with (ℓ1, ℓ2) = (1 ,1). A bias term is added for the
node pair representation when ℓin= 0.
Last, we derive hijby extracting the relevant components based on the specific atom types and use these matrix elements to
construct the complete Hamiltonian matrix. As shown in Fig.9(B), for the Non-Diagonal block in the Hamiltonian matrix,
we only construct the node pair features fijfrom the upper triangular part of the Hamiltonian matrix. The lower triangular
part is then obtained by symmetrizing the sub-blocks from the upper triangular part.
D.7. Computational Overhead of the Sparse Gates and Scheduler
Compared to the tensor product operation, Sparse Gate and the three-phase learning schedule only introduce minimal
computational overhead and have little impact on the overall speed. We would like to discuss this part of computational cost
here.
In the three-phase sparsity scheduler, for a given unsparsified set U, the additional computational overhead in the first phase
has a complexity of O(|U|), contributed by the RANDOM( ·) operation. The second phase has a computational overhead of
O(|U|log|U|), arising from the TOP( ·) operation. Since we fix the learnable weight matrix and the selected elements, there
is no additional computational overhead in the third phase. For detailed information, please refer to Equation 3.
For the sparse TP gate, the computational overhead comes from the element-wise multiplication of two weight vectors
(Equation 5), so its complexity is O(|Uc|) =O(L3).
For the sparse pair gate, the additional computational overhead mainly comes from the linear layer Fp(·)in Equation 7, with
its time complexity being the square of its hidden dimension. Other operations, including the inner product (Equation 6) and
the weight calculation (Equation 9), are necessary operations in our framework, even without the sparse pairwise gate.
E. Additional Related works
E.1. SE(3) Equivariant Neural Network
The SE(3) equivariant neural network is one of the most used models in the field of AI for chemistry (Fuchs et al., 2020; Du
et al., 2022; Musaelian et al., 2023; Liao \& Smidt, 2022; Liao et al., 2023; Batzner et al., 2022; 2023). The fundamental
feature of SE(3) equivariant neural network is that all the features and operations in the model are SE(3) equivariant. This
is achieved by using irreducible representations (irreps) and functions of geometry built from spherical harmonics. In the
network, the equivariant features propagate through each layer and interact with other features by equivariant operations,
and finally obtain the desired SE(3) equivariant quantity.
SE(3)-Transformer (Fuchs et al., 2020) proposed a novel self-attention mechanism for 3D point clouds and graphs that
ensures robustness through continuous 3D roto-translation equivariance, which is widely used in the field of quantum
chemistry. The Equiformer (Liao \& Smidt, 2022) introduced a novel graph neural network leveraging SE(3) Transformer
architecture based on irreducible representations to predict molecule property. It demonstrated strong empirical results by
incorporating tensor products and a new attention mechanism called equivariant graph attention. The EquiformerV2 (Liao
et al., 2023) is an improved version of Equiformer, which scales effectively to higher-order representations by replacing
SO(3) convolutions with efficient eSCN convolutions (Passaro \& Zitnick, 2023), and outperforming the traditional network
such as GemNet (Gasteiger et al., 2022) and Torchmd-Net (Th ¨olke \& De Fabritiis, 2022) in molecular energy and force
prediction and other downstream tasks. Allegro (Musaelian et al., 2023) used a strictly local, equivariant model to represent
many-body potential using iterated tensor products of learned representations without atom-centered message passing. It
demonstrates remarkable generalization to out-of-distribution data and accurately recovers structural and kinetic properties
of amorphous electrolytes in agreement with ab initio simulations.
These SE(3) equivariant neural networks greatly improve the performance of Artificial Intelligent in the field of quantum
chemistry. However, the computational complexity of tensor product operation greatly reduces the efficiency of these
SE(3) equivariant models. The eSCN (Passaro \& Zitnick, 2023) presents an efficient method to perform SO(3) equivariant

\section{Efficient and Scalable Density Functional Theory Hamiltonian Prediction through Adaptive Sparsity}

convolutions. It reduces the computational complexity by aligning node embeddings’ primary axis with edge vectors,
transforming the SO(3) convolutions into mathematically equivalent SO(2) convolutions, which decreases complexity
from O(L6)toO(L3). E2Former (Li et al., 2025a) introduces an equivariant and efficient transformer architecture that
incorporates the Wigner 6j convolution (Wigner 6j Conv). By shifting the computational burden from edges to nodes, the
Wigner 6j Conv reduces the complexity from O(|E|)toO(|V|)while preserving both the model’s expressive power and
rotational equivariance.
E.2. Hamiltonian Matrix Prediction
Predicting the Hamiltonian matrix (also known as electronic wavefunctions) is gradually gaining more and more attention
because of the wide range of potential application scenarios of the Hamiltonian matrix. There are more and more works
trying to solve the problem of predicting the Hamiltonian matrix using deep learning techniques.
The SE(3) equivariant neural network is one of the most used models in the field of Hamiltonian matrix prediction (Unke
et al., 2021; Yu et al., 2023; Gong et al., 2023; Atz et al., 2021). The PhiSNet (Unke et al., 2021) is the first method that
primarily focuses on Hamiltonian matrix prediction. It leverages SE(3)-equivariant operations through its whole architecture
to predict molecular wavefunctions and electronic densities, and can reconstruct wavefunctions with high accuracy. The
main problem for PhiSNet is its inefficiency due to the large amount of tensor product operations. To solve this problem,
QHNet (Yu et al., 2023) proposed a model with careful design to greatly reduce the number of tensor products and improve
the efficiency of Hamiltonian prediction. The DeepH-E3 (Gong et al., 2023) is an E(3)-equivariant model that preserves
Euclidean symmetry even with spin-orbit coupling. The method allows for efficient and accurate electronic structure
calculations of large-scale materials by learning from small-sized DFT data, significantly reducing computational costs.
There are also other works that try to enhance the prediction ability from different aspects. DeepH (Li et al., 2022) introduces
a local coordinate system defined for each edge according to its local chemical environment to handle the gauge (or
rotation) covariance of the DFT Hamiltonian matrix. This allows the Hamiltonian matrix blocks to be invariant under
rotation when transformed into the local coordinate system. Self-consistency training (Zhang et al., 2024) leverages the
self-consistency principle of density functional theory (DFT) to train a model without requiring the labeled Hamiltonian
matrice. This enhances the generalization and efficiency and reduces reliance on costly DFT calculations for supervision.
The DEHQ (Wang et al., 2024) integrates Deep Equilibrium Models (DEQs) to predict Hamiltonians. It inherently captures
the self-consistent nature of Hamiltonians, a critical aspect often overlooked by traditional machine learning methods. By
employing DEQs, the model circumvents the need for iterative DFT calculations during training. WANet (Li et al., 2025b)
introduces a scalable deep learning model for Hamiltonian prediction. By leveraging a novel Wavefunction Alignment Loss
(WALoss), the model notably reduces total energy error derived from the predicted Hamiltonian matrix and accelerates SCF
calculations with the predicted Hamiltonian matrix.
23

\end{document}
