\documentclass[11pt,a4paper]{article}
\usepackage[utf8]{inputenc}
\usepackage[T1]{fontenc}
\usepackage{amsmath}
\usepackage{amsfonts}
\usepackage{amssymb}
\usepackage{amsthm}
\usepackage{graphicx}
\usepackage{hyperref}
\usepackage{cite}
\usepackage{natbib}
\usepackage{geometry}
\usepackage{booktabs}
\usepackage{multirow}
\usepackage{array}
\usepackage{caption}
\usepackage{subcaption}
\usepackage{float}
\usepackage{url}
\geometry{margin=1in}

\title{LightPFP: A Lightweight Route to Ab Initio Accuracy at Scale}
\author{Wenwen Li; Nontawat Charoenphakdee; Yong-Bin Zhuang; Ryuhei Okuno; Yuta Tsuboi; So Takamoto; Junichi Ishida; Ju Li}
\date{\today}

\begin{document}
\maketitle

LightPFP: A Lightweight Route to Ab Initio Accuracy at Scale
Wenwen Li,1Nontawat Charoenphakdee,1Yong-Bin Zhuang,1Ryuhei Okuno,1Yuta Tsuboi,1So Takamoto,1
Junichi Ishida,2and Ju Li3,4
1)Preferred Networks Inc., Tokyo, Japan.
2)Matlantis Corporation, Tokyo, Japan.
3)Department of Materials Science and Engineering, Massachusetts Institute of Technology, Cambridge, MA 02139,
USA
4)Department of Nuclear Science and Engineering, Massachusetts Institute of Technology, Cambridge, MA,
USA
(*Electronic mail: nontawat@preferred.jp )
(*Electronic mail: wenwenli@preferred.jp )
(Dated: 7 November 2025)
Atomistic simulation methods have evolved through successive computational levels, each building upon more
fundamental approaches: from quantum mechanics to density functional theory (DFT), and subsequently, to
machine learning interatomic potentials (MLIPs). While universal MLIPs (u-MLIPs) offer broad transfer-
ability, their computational overhead limits large-scale applications. Task-specific MLIPs (ts-MLIPs) achieve
superior efficiency but require prohibitively expensive DFT data generation for each material system. In
this paper, we propose LightPFP, a data-efficient knowledge distillation framework. Instead of using costly
DFT calculations, LightPFP generates a distilled ts-MLIP by leveraging u-MLIP to generate high-quality
training data tailored for specific materials and utilizing a pre-trained light-weight MLIP to further enhance
data efficiency. Across a broad spectrum of materials, including solid-state electrolytes, high-entropy alloys,
and reactive ionic systems, LightPFP delivers three orders of magnitude faster model development than
conventional DFT-based methods, while maintaining accuracy on par with first-principles predictions. More-
over, the distilled ts-MLIPs further sustain the computational efficiency essential for large-scale molecular
dynamics, achieving 1-2 orders of magnitude faster inference than u-MLIPs. The framework further enables
efficient precision transfer learning, where systematic errors from the u-MLIP can be corrected using as few
as 10 high-accuracy DFT data points, as demonstrated for MgO melting point prediction. This u-MLIP-
driven distillation approach enables rapid development of high-fidelity, efficient MLIPs for materials science
applications.
I. INTRODUCTION
The development of accurate and computationally ef-
ficient atomistic energy methods is critical for enabling
large-scale atomistic simulations in materials science,
catalysis, and chemistry. The evolution of these methods
over several decades can be conceptualized as an ecolog-
ical “food chain” (Fig. 1), where each higher level “feeds
on” the computational results of lower levels, gaining effi-
ciency while potentially sacrificing some accuracy in the
process.
At the bottom of the food chain lie the most accu-
rate but computationally intensive quantum mechanical
methods, such as full configuration interaction (FCI) and
quantum Monte Carlo (QMC). Although formally exact,
they do not have great performance in computational
and memory scaling with the number of electrons, and
therefore are limited to systems with on the order of ten
atoms. The second level is occupied by density func-
tional theory (DFT), which “consumes” the results from
lowest-rung, for example, electron gas simulations using
QMC,1to parametrize its exchange-correlation function-
als (e.g. PBE generalized gradient approximation, or
r2SCAN meta-GGA approximation). DFT can handle
a few hundred atoms, which is the reason it is widely
used for crystal structure discovery and property predic-tion. However, it is computationally challenging to simu-
late extended defects directly, or even finite-temperature
sampling.
Moving up the chain, machine learning interatomic
potentials (MLIPs) represent the third level, “feeding
on” large datasets of DFT calculations. Among these,
universal MLIPs (u-MLIPs) have gained significant at-
tention for their broad chemical transferability. They
are trained on chemically diverse structures spanning
many elements and bonding motifs, and they encode
physical symmetries to generalize across the periodic
table, e.g., PFP,2M3GNet,3CHGNet,4MACE-MP-
0.5,6In particular, PFP is noted for being trained
on a highly complex and diverse DFT database, con-
tributing to its superior robustness. Numerous studies
have demonstrated its applicability without fine-tuning
across a wide range of materials, including battery,7–12
metal-organic framework,13,14ceramics,15,16catalyst,17
polymer,18nanotube,19atomic layer deposition,20,21Hy-
drogen storage,22superconductor,23memristor.24De-
spite their universality, computational efficiency remains
a bottleneck in large-scale simulations.
This raises a fundamental question: can we extend
this food chain further to achieve even greater computa-
tional efficiency? Task-specific MLIPs (ts-MLIPs) with
simpler architectures, such as moment tensor potentialarXiv:2510.23064v2 [cond-mat.mtrl-sci] 6 Nov 2025

2
Formally exact methods: 
Quantum Monte Carlo, Full CI, …Density functional theory: 
PBE, r2SCAN, …u-MLIP: 
PFP, MACE-MP-0, …u-MLIP 
distillation:
LightPFP
\~{}100 meV/atom\~{}20 meV/atom\~{}10 meV/atom
Super 
slowVery slowLinear scaling,
but slow;
\~{}104atomsLinear scaling,
fast-to-train,
fast-to-run,
\~{}106atoms
FIG. 1: A standard “food chain” of atomistic calculation
methods.
(MTP),25DeePMD,26and Allegro,27demonstrate that
significant speed improvements are possible, but they
face a critical bottleneck. These methods still “feed” di-
rectly on DFT data—the same food source as universal
MLIPs—requiring extensive and time-consuming DFT
calculations for each new material system. This train-
ing strategy can take weeks or months, severely limiting
their practical deployment despite their superior infer-
ence speed.
To overcome the bottlenecks of both u-MLIPs and ts-
MLIPs, we propose LightPFP, a fast-to-train and fast-to-
run framework for constructing ts-MLIPs through knowl-
edge distillation from a u-MLIP. LightPFP achieves a
favorable balance between computational efficiency and
accuracy while avoiding the prohibitive training costs of
traditional ts-MLIPs that arise from DFT calculations.
To support the assessment of LightPFP’s evolutionary
position toward the “apex predator” in the ecosystem of
atomistic methods, let us consider different sources of er-
ror in a practical atomistic simulation, vis-à-vis the com-
putational cost. For reference, even though Fig. 1does
not show any experimental method, the typical error bars
inexperimental thermochemical measurements of forma-
tion/reaction enthalpies are taken to be 1 kcal/mol, the
so-called “chemical accuracy” as named by John Pople,28
which is 43 meV/atom. The formally exact calculations,
when fully converged in basis sets, etc., should agree with
present-day state-of-the-art experiments to much better
than the chemical accuracy, so much so that these calcu-
lations are sometimes taken to be the ground truth rather
than the experiments. In fact, the largest error comes
from the formally exact →DFT (Fig. 1), due to the intrin-
sic limitations of DFT expressivity. Next, the training of
DFT→PFP takes a long time and a lot of resources,29
but that is already done for each released version of PFP,
and the final DFT →PFP transfer error is small. As will
be shown in the present paper, the PFP →LightPFP ex-
hibits smaller transfer errors and faster training time,
comparing to DFT →PFP (Supplementary Note 1).
One should also consider that many practical simula-
tion tasks incur error beyond the intrinsic level-of-theory
error. For example, in computing the defect formation
energies, if a small calculation supercell with periodicboundary condition (PBC) is used, there will be image
interactions30both electronically and elastically. Thus,
even if DFT is intrinsically more accurate than LightPFP
by 30 meV/atom, LightPFP may end up giving more ac-
curate defect formation energy and other defect reaction
behaviors, by virtue of using a much larger simulation
supercell that greatly reduces the image artifacts. Such
broadly applicable calculations are much, much faster to
run, and potentially more accurate than DFT in prac-
tice, thus, might become really competitive in the atom-
istic simulator ecosystem. The broad applicability of
LightPFP to various systems, from solid electrolyte to
metallurgy, from semiconductor processing to hard ce-
ramics, will be demonstrated in this paper.
Occasionally, developers skip levels on the food chain.
For example, recently a so-called Multi-task Electronic
Hamiltonian Network (MEHnet)31was developed, which
can serve as ts-MLIP (besides other functions) for H, C,
N, O and F elements and organic hydrocarbons. This was
based on CCSD(T) →MEHnet direct transfer. CCSD(T)
is called the “gold standard of quantum chemistry”, close
to the bottom-rung of the food chain, typically achieving
0.1 kcal/mol error with respect to the exact calculations.
DFT and u-MLIP rungs were skipped in the construc-
tion. Although the CCSD(T) →MEHnet transfer error is
very small, typically less than 10 meV/atom. the train-
ing process was very expensive. As a result, no broad
applicability was achieved yet across the whole periodic
table. As another example, the original MTP poten-
tials were trained by DFT →MTP, skipping the u-MLIP
rung. However, the process cannot be done overnight
for a stated chemical space and may take several months
to generate the necessary data. Based on the foregoing
discussion, it is likely that DFT →PFP→LightPFP may
achieve the best universality, practicality, and speed, thus
becoming a potential “apex” on the food chain.
In this paper, we first provide an overview of the
LightPFP knowledge-distillation framework and an as-
sessment of the data efficiency of its distilled, pre-trained
student models. We then demonstrate LightPFP across
four challenging applications that highlight comple-
mentary aspects of the method: (1) Li+diffusion in
the solid electrolyte Li 6PS5Cl and (2) the mechanical
and grain-boundary properties of the high-entropy al-
loy AlCoCrFeNi, both illustrating the trade-off between
model-building/inference speed and predictive accuracy;
(3) the reaction kinetics of SiO 2etching by HF vapor,
showcasing the integration of model distillation with ac-
tive learning for complex reactive simulations; and (4) the
melting point of MgO, demonstrating that when u-MLIP
precision is insufficient, transfer learning with a small,
high-accuracy DFT dataset can substantially improve
performance.

3
Perspective Morrow et al.32Amin et al.33Gardner et al.34Zhang et al.35This work
Teacher is trained from diverse datasets × ✓ ✓ ✓ ✓
Use teacher in data generation ✓ × ✓ ✓ ✓
Use active learning with teacher’s labels × × ✓ ✓ ✓
Use student pretraining × × × × ✓
Does not require teacher’s fine-tuning × × × × ✓
TABLE I: Comparisons of the LightPFP framework with existing works related to distillation across different
perspectives.
Training data 
generation using PFPEvaluation 
using PFPStudent 
fine-tuning
Target structure
Pre-trained student for 
target elements (MTP)
Task-specific 
parameter 
extraction
Pre-trained teacher 
(PFP)
Pre-trained student 
(MTP)Large 
DFT dataset
FIG. 2: Schematic diagram of LightPFP.
II. RESULTS
A. LightPFP Framework Overview
In this section, the overview of LightPFP is presented.
For the teacher model, we employ PFP2based on TeaNet
architecture.36As the student model, we adopt the Mo-
ment Tensor Potential (MTP), proposed by Novikov et
al.25due to its favorable trade-off between accuracy and
efficiency.37The workflow of LightPFP shown in Fig. 2
begins by defining a target structure and generating
training data using PFP, including sampling and label-
ing. Students are pre-trained using Reptile meta-learning
algorithm38on diverse datasets described in reference 2.
Finally, the reduced model is then fine-tuned using PFP-
generated data, followed by an evaluation to assess its
performance. Importantly, pretrained students only need
to prepare once in advance, and can be reused across a
wide range of applications. Moreover, the model sizes of
pretrained students can be reduced by removing MTP
parameters for element pairs that are not present in the
structure when they cover more elements than needed for
a specific material.
B. Data efficiency of pretrained student models
We first demonstrate the enhanced data efficiency of
pretrained student models, using the Ni 3Al alloy39asan example. To this end, a full dataset containing 1529
structures is prepared through the comprehensive sam-
pling involving PFP2in the relevant configuration space.
The sampling methods comprise static and dynamic sam-
pling. The static method samples static structures by
compressing and deforming their lattice, as well as dis-
placing atomic positions. The dynamic sampling uses
MD simulations with initial configurations of both defect-
free and defective bulk structures, as well as surface struc-
tures. The details of sampling parameters are provided in
SI. For testing data efficiency, smaller datasets with sizes
ranging from 100 to 850 are created by two methods,
subsampling from the full dataset and direct sampling
through the decrease of MD steps. Each size dataset
is created five times to obtain the uncertainties of er-
rors. Structures in the datasets obtained by subsampling
tend to be more widely distributed in configuration space,
whereas direct sampling is closer to common user prac-
tice in real situations (i.e. by decreasing MD steps).
We compare the performance of fine-tuned pretrained
and scratch-trained student models on energy and force
errors, as shown in Fig. 3a,b. Across all dataset sizes,
the fine-tuned pretrained student models outperform the
scratch-trained student models. We note that finetuning
pretrained student models on 100 structures performs
almost as well as on 1529 structures. In addition to
the standard energy and force testings, we validate the
performance of student models on different application
tasks, for instance, phonon spectra and surface energies,
as shown in Fig. 3c,d. Comparable to the energy and

4
FIG. 3: Comparison of data efficiency between fine-tuned pretrained and scratch-trained student models.
force testings, the performance of fine-tuned pretrained
models is better than the scratch-trained models. Simi-
lar performance trend can be observed in other properties
(See SI).
Moreover, the performance of fine-tuned pretrained
student models is more robust in application tasks,
whereas scratch-trained models show typical overfitting
behavior. The force errors from the scratch-trained mod-
els on the smaller datasets are lower than on the larger
dataset as shown in Fig. 3(b). However, the errors on
phonon spectra and surface energies are larger as shown
in Fig. 3(c,d). In contrast, although fine-tuned pretrained
student models show a similar trend on force testing,
their performance on application tasks are consistently
reliable across various dataset sizes.
C. Li 6PS5Cl
This example focuses on a common solid-state elec-
trolyte, Li 6PS5Cl, renowned for its high ionic conductiv-
ity, with potential applications in solid-state battery de-
velopment. Extensive experimental and theoretical stud-
ies have been conducted on Li 6PS5Cl. For example, Deng
et al.40usedab initio MD to calculate the diffusion co-
efficient and diffusion activation energy (0.52 eV) of Li
in Li 6PS5Cl crystals. The Li 6PS5Cl system is used as an
example to demonstrate the advantages of the model dis-
tillation method compared to other approaches: (1) di-
rectly using universal potentials, and (2) training MLIPswith traditional DFT datasets. We first validate the ef-
fectiveness of the model distillation approach.
We compare four strategies for using MLIPs to perform
atomistic simulations of Li 6PS5Cl. These are: directly
using the u-MLIP PFP v7.0.0; distilling a compact task-
specific MLIP with the MTP architecture from PFP (as
described above), yielding LightPFP; using another u-
MLIP, MACE-MP-0b3;6and training an MTP model di-
rectly on DFT data (MTP-DFT). For brevity, we refer to
these strategies throughout as PFP, LightPFP, MACE,
and MTP-DFT, respectively. LightPFP and MTP-DFT
share the same MTP architecture, hyperparameters, and
software implementation; consequently, their inference-
time efficiency is essentially identical.
The LightPFP model is obtained by distilling knowl-
edge from the PFP through a two-step data collection
process: (i) sampling Li 6PS5Cl configurations by molec-
ular dynamics and other molecular simulations (such as
lattice stretching, compression, deformation, atomic dis-
placement, etc.) and (ii) labeling the sampled configu-
rations by PFP to obtain their corresponding energies,
forces, and stresses. Dataset acquisition takes 3.5 hours
on a single GPU. The dataset composition is listed in
Table II. Additionally, our commonly used data collec-
tion methods are detailed in supplementary information
I. Subsequently, we perform 1 hour of training, using
weights from a pretrained MTP model for initialization.
Ultimately, using only 4.5 hours, we obtain the LightPFP
model. MTP-DFT shares the same MLIP architecture
as LightPFP, but its dataset is labeled with DFT. To re-
duce cost, we generated trajectories with PFP and then

5
TABLE II: Composition of the training dataset for Li 6PS5Cl
Sampling Number of Number of Comment
method structures atoms
LightPFP Dataset (labeled by PFP)
MD 1600 374400 NPT MD at 300, 500, 1000, 1500K;
1 sample per 100 steps
rattle 10 4160 Random displacement of atoms
compress 22 1144 Compress and stretch lattice
deform 48 2496 Deform lattice
vacancy 100 5100 Create 1\~{}2 vacancy
Total 1780 387300
MTP-DFT Dataset (labeled by DFT)
MD 800 41600 NPT MD at 300, 500, 1000, 1500K
1 sample per 100 steps
rattle 10 520 Random displacement of atoms
compress 22 1144 Compress and stretch lattice
deform 48 2496 Deform lattice
vacancy 100 5100 Create 1\~{}2 vacancy
Total 980 50860
performed post hoc DFT single-point calculations to la-
bel the sampled snapshots, rather than running fully ab
initio MD. Even so, end-to-end data collection required
approximately 100 hours of wall-clock time on our setup.
Using a simple extrapolation—multiplying the number of
MD steps by the average wall time per DFT single-point
used for labeling—we estimate that fully DFT-driven MD
would take on the order of 8,000 hours under compa-
rable settings. Thus, constructing a ts-MLIP in this
traditional DFT-labeled manner is substantially more
time- and compute-intensive than the distilled LightPFP
route. The dataset composition is listed in Table II. Be-
cause Kohn–Sham DFT in plane-wave nominally scales
as O(N3) with system size, we prioritized smaller cells;
consequently, the MTP-DFT dataset is overall smaller
and skewed toward structures with fewer atoms com-
pared to the LightPFP dataset. After data collection,
model training took about one hour.
After preparing the four models, we first test their
computational speed and memory efficiency. Figure 4(a)
shows the MD inference speed varied with different num-
bers of atoms on NVIDIA V100 GPUs with 16 GB GPU
memory. The fastest inference speed of LightPFP/MTP-
DFT (9.7×10−7s/step/atom) is about 50 times faster
than PFP ( 4.9×10−5s/step/atom) and about 160 times
faster than MACE ( 1.6×10−4s/step/atom). In addi-
tion, the maximum size that LightPFP/MTP-DFT can
simulate on a single GPU, i.e., GPU memory efficiency,
far exceeds that of other models. On a GPU with 16 GB
memory, LightPFP/MTP-DFT can simulate up to ap-
proximately 811,200 atoms, which is 14 times that of PFP
(5,616 atoms) and 21 times that of MACE (3,900 atoms).
Note that the inference speed of LightPFP/MTP-DFT is
related to the model’s hyperparameters (e.g., level max,
number of radial basis functions, etc.).
In Fig. 4(b), we plot MD inference speed per atom
against MLIP construction time. As expected fromtheir simpler architectures, LightPFP and MTP-DFT de-
liver 1–2 orders of magnitude higher per-step throughput
than the u-MLIPs (PFP and MACE). The corresponding
trade-off is that u-MLIPs require no task-specific con-
struction, whereas LightPFP and MTP-DFT incur up-
front costs. Notably, LightPFP’s construction is approx-
imately three orders of magnitude faster than the DFT-
based workflow used for MTP-DFT, owing to the much
cheaper data collection via the PFP teacher.
The inset of Fig. 4(b) aggregates construction and run-
time to estimate the total wall-clock time to simulate
a 10,000-atom Li 6PS5Cl system for 10 ns with a 1 fs
timestep ( 107steps). Under this scenario, LightPFP
achieves the shortest total time, completing the task
44–139 ×faster than u-MLIPs, and its advantage grows
with increasing MD length. Conversely, for very short
simulations, LightPFP’s initial construction overhead
can diminish its advantage relative to u-MLIPs. Despite
similar inference speed to LightPFP, MTP-DFT remains
slower overall because its total time is dominated by DFT
data generation.
While LightPFP offers substantially higher overall ef-
ficiency—both in MLIP construction and MD simula-
tion—than existing u-MLIPs and DFT-trained MLIPs,
its attainable precision is constrained by two factors:
(i) reduced model capacity relative to u-MLIPs and
(ii) training on PFP-generated labels rather than DFT,
which can propagate the teacher’s deviations from DFT.
To quantify these effects, we benchmark force predic-
tions against DFT dataset used for MTP-DFT training
(Fig. 5). PFP attains the lowest MAE (0.028 eV/Å).
As expected, LightPFP exhibits a modestly higher MAE
(0.053 eV/Å), reflecting both inherited PFP errors and
architectural simplification. For comparison, the MTP-
DFT trained directly on this dataset achieves 0.044
eV/Å; because portions of the DFT set were used for
training, only the 10\% held-out test split is shown in

6
FIG. 4: (a) Molecular dynamics (MD) computational
speed with Li 6PS5Cl as a function of number of atoms
for three MLIPs: PFP, LightPFP (MTP), and MACE.
(b) Trade-off between the overall time spent on MLIP
building for Li 6PS5Cl, including data collection and
model training, and MD computational speed for PFP,
LightPFP, MACE, and MTP-DFT. Inset: the total
time cost to complete both MLIP building and a 10 ns
MD simulation of a 10,000-atom system With PFP,
LightPFP, MACE, and MTP-DFT.
the parity plot. Crucially, the gap between LightPFP
and PFP/MTP-DFT is small, supporting the feasibility
of distilling a reliable u-MLIP into a lightweight model
with limited loss in precision. Notably, MACE shows the
largest MAE (0.061 eV/Å), underscoring the importance
of teacher quality: a strong universal teacher can yield a
student that, on this benchmark, rivals or even surpasses
more complex models trained directly on DFT.
We next examine how the force-accuracy differences
translate into a transport property by computing Li+dif-
fusion in Li 6PS5Cl. The workflow is: (i) relax a 1 ×1×1
Li6PS5Cl cell (52 atoms), optimizing both atomic posi-
tions and lattice; (ii) run NVT MD for 100 ps at 600, 700,
800, 900, and 1000 K, using the same settings as the ab
initio MD in reference,40with eight independent repli-
cas per temperature for statistics; (iii) extract Li+dif-
fusion coefficients from the mean-squared displacement.
FIG. 5: Parity plot comparing atomic forces predicted
by MLIPs to DFT reference values (a) PFP; (b)
LightPFP; (c) MACE and (d) MTP-DFT
FIG. 6: Arrhenius plot of Li+diffusivity in Li 6PS5Cl
fromab initio MD simulations40and four MLIPs (PFP,
LightPFP, MACE, and MTP-DFT). The diffusion
coefficient are averaged over eight independent
trajectories; the corresponding activation energy Eais
reported in the legend and error bar is derived from the
standard error of fitted slope.
Figure 6compiles diffusion coefficients from the four
MLIPs alongside ab initio MD results from the literature.
Consistent with the force-MAE trends, all four MLIPs
slightly overestimate diffusion relative to the DFT refer-
ence at every temperature. Among them, PFP is closest
to DFT, while MACE shows the largest overestimation,
in line with its higher force MAE. Notably, LightPFP
and MTP-DFT exhibit overestimation magnitudes sim-
ilar to PFP, and the gap between LightPFP and PFP
is small across temperatures despite LightPFP’s some-
what larger force MAE. This suggests that the modest
force-error increase introduced by distillation has only

7
a limited impact on this property. Arrhenius fits yield
activation energies of 0.523 eV (DFT), 0.462 eV (PFP),
0.490 eV (LightPFP), 0.407 eV (MACE) and 0.465 eV
(MTP-DFT). LightPFP’s activation energy is, in fact,
the closest to the DFT value among the MLIPs consid-
ered. While some of this agreement may be incidental
within statistical and methodological uncertainties, it in-
dicates that, at least for this system, errors introduced by
model distillation are not the dominant source of discrep-
ancy in property-level predictions. Instead, differences
arising from simulation setup (thermostatting, sampling
length) and the DFT reference itself can be comparable
to or larger than the residual model error.
D. High entropy alloy
This example focuses on high entropy alloys (HEAs),
specifically the Cantor alloy with a face-centered cubic
(FCC) lattice. The composition is 20\% each of Al, Co,
Cr, Fe, and Ni. HEAs have attracted significant attention
due to their exceptional mechanical properties. However,
their complex multi-element nature poses challenges for
training MLIPs. In the following, we train MLIPs appli-
cable not only to bulk HEA but also to interfaces and
grain boundaries.
As in the previous example, we evaluate the same four
MLIP usage strategies—PFP, MACE, LightPFP, and
MTP-DFT—with the same meanings as defined above.
For LightPFP and MTP-DFT, we construct training
datasets using an identical sampling workflow. Because
equiatomic AlCoCrFeNi high-entropy alloys are substitu-
tional solid solutions without a unique ordered configura-
tion, each lattice site experiences a wide variety of local
chemical environments. To efficiently sample this diver-
sity, we adopt a random-substitution protocol: starting
from an fcc Al host, each lattice site is independently
assigned one of Al, Co, Cr, Fe, Ni with equal probabil-
ity (≈20at.\% per element), and the resulting structures
are sampled using PFP-driven molecular dynamics. This
procedure is repeated across multiple starting cells to di-
versify the dataset. The initial pool includes fcc bulk
crystals, surface slabs with Miller indices less than 4, and
coincidence-site-lattice (CSL) grain boundaries with low
Σ(<10). For LightPFP, PFP-driven MD sampling takes
26 hours to generate 9,638 structures (1,356,616 atoms),
followed by 1 hour of model training (27 hours total).
For the DFT-based baseline (MTP-DFT), we use the
same PFP-driven sampling strategy but label a smaller
set—1,012 configurations (60,360 atoms), including sur-
faces and grain boundaries relevant to the intended ap-
plication—by single-point DFT calculations. Some con-
figurations (e.g., the (3 1 1) slab with at least 144 atoms)
require relatively large cells, making DFT labeling ex-
pensive due to the nominal cubic scaling of Kohn–Sham
DFT. The DFT calculations took 637 hours on a single
GPU; by simple extrapolation, fully ab initio MD sam-
pling would require on the order of 60,000 hours, i.e.,more than three orders of magnitude slower than the
LightPFP route. These results again highlight the ad-
vantage of using a universal potential for rapid, low-cost
data collection.
Runtime benchmarks on an NVIDIA V100 (16 GB)
show that LightPFP and MTP-DFT achieve an infer-
ence speed of 9.8×10−7s/step/atom—66 ×faster than
PFP (6.5×10−5s/step/atom) and 249 ×faster than
MACE ( 2.4×10−4s/step/atom). The maximum sys-
tem size that fits on a single GPU is 716,800 atoms for
LightPFP/MTP-DFT, compared to 13,824 for PFP (52 ×
smaller) and 1,792 for MACE (400 ×smaller). When
construction cost is considered, LightPFP offers the best
overall trade-off: it pairs the fastest inference with a 27-
hour build, which is orders of magnitude cheaper than
the 60,000 hours required for MTP-DFT.
Using DFT forces as ground truth on a held-out test
set, the force MAEs follow the same ordering observed
previously: PFP (0.103 eV/Å) < MTP-DFT (0.123
eV/Å) < LightPFP (0.134 eV/Å) < MACE (0.184 eV/Å).
This again shows that the distilled LightPFP incurs a
modest accuracy penalty relative to its teacher and a
DFT-trained baseline, yet retains substantially higher ef-
ficiency.
We assess the accuracy of the four MLIPs on key prop-
erties of AlCoCrFeNi, using DFT as the reference: the
equation of state (EOS), elastic constants, surface forma-
tion energies, and grain-boundary (GB) formation ener-
gies. Unless otherwise noted, results are averaged over
multiple random elemental arrangements to account for
chemical disorder, and numerical comparisons are sum-
marized in Table III.
We began with the equation of state. Starting from
a relaxed 256-atom bulk cell, we varied the lattice con-
stant by ±5\%, relaxed atomic positions at fixed vol-
ume, and fitted the resulting energy–volume data with a
Birch–Murnaghan EOS to obtain the equilibrium volume
and bulk modulus. PFP and LightPFP closely reproduce
the DFT energy–volume curve. MACE also follows the
DFT curve but exhibits small systematic deviations in
the fitted parameters. By contrast, MTP-DFT underes-
timates the equilibrium volume by approximately 2.5\%,
which may reflect limited coverage of relevant local envi-
ronments in its DFT-labeled training set.
Then, the elastic tensor, bulk, Young’s, and shear mod-
uli are computed with the stress–strain methodology41
using the same bulk structure. PFP provides the clos-
est agreement with DFT with average error of 7.2 GPa.
LightPFP (10.65 GPa) tracks PFP closely. MTP-DFT
(12.55 GPa) generally remains comparable to LightPFP
for these mechanical properties, while MACE shows more
pronounced deviations, 23.35 GPa. Overall, the spread
among PFP, LightPFP, and MTP-DFT is modest for
elasticity, whereas MACE underperforms on this task.
Since the low-index surfaces were included in training
dataset, we evaluated higher-index surfaces with Miller
index > 3 to probe the performance of MLIPs in surface
formation energy calculation. The surface formation en-

8
TABLE III: Comparison of DFT and MLIPs on properties of AlCoCrFeNi high-entropy alloy
Property DFT PFP LightPFP MACE MTP-DFT
Equation of State
Volume (Å3/atom) 11.58 11.51 11.51 11.48 11.29
Bulk modulus (GPa) 165.64 165.66 164.35 159.18 162.27
Mechanical Properties (GPa)
C11 195.2 202.5 196.3 177.2 197.2
C22 211.4 206.9 203.3 183.5 202.7
C33 197.5 206.7 204.3 182.7 203.1
C12 140.9 145.9 151.7 145.9 153.3
C13 142.9 152.9 156.6 148.3 157.6
C23 131.1 137.9 144.9 141.3 148.2
C44 116.5 109.4 106.2 80.2 103.7
C55 124.0 114.2 110.6 84.6 107.1
C66 120.3 112.9 109.9 83.9 106.8
Bulk modulus 159.23 165.45 167.81 157.14 169.02
Shear modulus 69.99 65.79 60.42 45.05 58.54
Young’s modulus 183.14 174.27 161.84 123.36 157.44
Average Error – 7.20 10.65 23.35 12.55
Surface Energy (eV/Å2)
(4, 1, 0) 0.127 0.136 0.133 0.121 0.126
(4, 1, 1) 0.170 0.171 0.165 0.167 0.168
(4, 2, 1) 0.142 0.149 0.148 0.134 0.145
(4, 3, 0) 0.139 0.144 0.143 0.137 0.143
(4, 3, 2) 0.137 0.143 0.145 0.126 0.142
(4, 4, 1) 0.148 0.153 0.153 0.146 0.154
(4, 4, 3) 0.171 0.178 0.175 0.174 0.176
Average Error – 0.0058 0.0053 0.0052 0.0036
Grain Boundary Energy (eV/Å2)
Σ13 22.62/[1 0 0] 0.0559 0.0621 0.0578 0.0424 0.0523
Σ15 48.19/[1 2 0] 0.0794 0.0809 0.0787 0.0602 0.0825
Σ13 147.80/[1 1 1] 0.0378 0.0300 0.0294 0.0206 0.0268
Σ13 67.38/[1 0 0] 0.0584 0.0617 0.0563 0.0332 0.0504
Σ11 129.52/[1 1 0] 0.0955 0.0735 0.0771 0.0670 0.0737
Average Error – 0.0081 0.0063 0.0207 0.0095
ergy was computed as:
γsurf=Esurf−nsurf
nbulkEbulk
2Asurf(1)
whereEsurfis the energy of a slab with two surfaces,
Ebulkis the energy of the bulk HEA, nsurfandnbulkare
the atom counts in the surface and bulk structures, and
Asurfis the surface area. All four MLIPs achieve high ac-
curacy, with average absolute errors below 0.006 eV/Å2
relative to DFT. On this task the inter-model differences
of average error are very small among PFP, LightPFP
and MACE (0.0052-0.0058 eV/Å2); while MTP-DFT
(0.0036 eV/Å2) is marginally closer to DFT.
Several CSL grain boundaries with Σ> 10 are selected
for testing the MLIPs in GB formation energy. The GB
formation energy was computed as:
γGB=EGB−nGB
nbulkEbulk
2AGB(2)
whereEGBandEbulkare the energy of GB and bulk
structures, nGBandnbulkare their atoms counts, and
AGBis the grain boundary area. LightPFP, PFP andMTP-DFT reproduce the GB formation energy with
modest accuracy with an average error < 0.01 eV/Å2,
whereas MACE shows larger deviations.
Across EOS, elasticity, surface energies, and GB en-
ergies, the overall spread among PFP, LightPFP, and
MTP-DFT is small, and no single model dominates all
properties. Importantly, despite its slightly larger force
MAE relative to PFP and MTP-DFT, LightPFP does
not exhibit a clear disadvantage in property-level predic-
tions for this materials. This mirrors the earlier example,
Li6PS5Cl: modest differences in force MAE do not nec-
essarily translate into large discrepancies in computing
materials properties, which can be comparably influenced
by factors such as finite-size effects, and simulation set-
tings. Together with its substantially lower construction
cost and faster inference, these results support model dis-
tillation from a strong universal potential as a practical
and accurate route for property calculations in complex,
chemically disordered materials.

9
E. Dry etching of SiO 2: application of active learning
In this example, we consider a more demanding ap-
plication: dry etching of the SiO 2(100) surface by HF.
Dry etching is a critical step in semiconductor process-
ing, yet atomistic simulations are particularly challeng-
ing. Device-scale simulations require tens to hundreds
of nanometers, while the process itself couples complex
surface reactions with intense atomic interactions under
high-energy bombardment. These demands place strin-
gent requirements on the accuracy and robustness of
MLIPs. Here, we combine model distillation with active
learning to rapidly construct a LightPFP model tailored
to this task, using PFP as the high-fidelity teacher for
data generation and selection. Given the prohibitive cost
of DFT-based active learning in this setting, we do not
construct or compare DFT-labeled MTP models; like-
wise, we focus on the PFP–LightPFP pipeline rather
than benchmarking additional universal models, as our
goal is to demonstrate applicability rather than relative
speed/accuracy.
We briefly outline the active learning workflow. An ini-
tial dataset was collected via PFP-driven sampling, cov-
ering SiO 2bulk and (100) surface, HF gas, and represen-
tative products such as SiF 4and H 2O. Dataset collection
took 4.5 hours and was used to train an initial LightPFP
model. As expected, the initial model was insufficient for
dry-etching simulation, having not yet learned the inter-
actions arising from high-velocity HF impacts on SiO 2.
We then entered an iterative active-learning loop in which
the current LightPFP model drives reactive MD of the
etching process: a HF molecule are inserted above the
SiO2surface with kinetic energies randomly sampled in
the range from 20 eV to 80 eV and directed perpendic-
ular to the surface; trajectories are propagated in the
NVE ensemble for 200 fs with a 0.2 fs timestep to resolve
high-energy collisions, followed by 1,000 fs of NVT dy-
namics (1 fs/step) to cool to 300 K. This insertion cycle
is repeated 100–200 times per iteration. To select in-
formative configurations, we directly compare LightPFP
and PFP predictions and flag frames with large discrep-
ancies; the selected structures are labeled by PFP and
used to re-train LightPFP. To accelerate updates, each
training step is capped at 0.5 hours. This is feasible be-
cause we warm-start from a pretrained student MTP, so
fine-tuning converges rapidly to a satisfactory model for
the next MD round. We perform 15 iterations of data
collection and model update, completing the end-to-end
process within 16 hours. After the active-learning loop,
all collected datasets are pooled and used for a longer
final training run to obtain a more reliable production
LightPFP model. In total, the wall-clock time to build
the LightPFP MLIP for this application is approximately
24 hours.
To validate the reliability of the LightPFP model ob-
tained through active learning, we first examine a repre-
sentative surface reaction. As shown in Fig. 7, an HF
molecule approaches a dangling OH group on a SiO 2
FIG. 7: Reaction pathway from NEB calculation for the
reaction of an HF molecule with a SiO 2surface during
dry etching, computed using PFP and LightPFP.
Atomic structures of the initial state (IS), transition
state (TS), and final state (FS) are shown.
cluster, displaces an H 2O molecule, and forms an Si–F
bond. We computed the reaction pathway and barri-
ers using the nudged elastic band (NEB) method with
both PFP and LightPFP. The initial state (IS), transi-
tion state (TS), and final state (FS) structures—shown as
insets—agree closely between the two models, indicating
a consistent reaction pathway. The forward/backward
barriers are 1.029/1.561 eV for PFP and 0.844/1.560
eV for LightPFP. For reference, literature DFT barri-
ers are 0.929/1.424 eV, while ReaxFF yields 1.848/2.706
eV.42LightPFP’s deviations from DFT are 0.085 eV (for-
ward) and 0.136 eV (backward), comparable to PFP’s
deviations of 0.100 eV and 0.137 eV, and far smaller
than ReaxFF’s errors. Notably, the training data did
not include NEB paths or SiO 2clusters; LightPFP’s
agreement arises from exposure to related configurations
generated during the active-learning MD, demonstrating
useful transferability.
We further assess performance in the MD simulation
of dry etching. Using the same setup, we run simula-
tions with HF incidence energies of 20 eV, 40 eV, and
60 eV. Figure 8shows the number of Si and O atoms
removed during etching. LightPFP and PFP produce
highly consistent etch yields and energy dependences
across all three conditions, indicating that the distilled
model tracks its teacher closely in this complex reactive
MD setting without noticeable behavioral divergence.
To probe scalability and practical applicability, we per-
formed a near feature-scale reactive MD simulation using
LightPFP. The simulation cell measured 10.06 ×10.06
×20.00 nm along the a, b, and c axes, and contained
72,000 Si and O atoms in a SiO 2(100) slab. To emulate
focused dry etching, HF molecules were accelerated to a
kinetic energy of 40 eV and directed toward the surface,
with impact points restricted to a 2 ×2 nm patch. Over

10
FIG. 8: Cumulative number of removed atoms versus time during HF dry etching of a SiO 2surface, from molecular
dynamics simulations using PFP and LightPFP at different incident kinetic energies. (a) Si atoms; (b) O atoms.
FIG. 9: Surface morphology of SiO 2during dry etching at a kinetic energy of 40 eV, obtained from a large-scale
molecular dynamics simulation with LightPFP. (a) Top view after 0.5 ns of etching; the etched region is the central
2×2nm square. Longitudinal cross-sections through the etched region at (b) t = 0.05 ns, (c) t = 0.25 ns, and (d)
t = 0.5 ns.
a total simulation time of 0.5 ns, 1,000 HF molecules
were injected. Figure 9illustrates the evolution of the
surface morphology under these conditions. By approx-
imately 0.05 ns, atoms at the bombarded region begin
to be removed. A recessed pit is clearly visible by 0.25
ns, and by 0.5 ns the crater reaches a depth of about 2
nm. Because the MD timescale is necessarily short, the
HF injection flux used here is higher than in typical ex-
periments; thus absolute etch rates are not directly com-
parable. Nevertheless, the sequence of material removal
and the development of a localized crater, demonstrating
that LightPFP remains stable and predictive in large,
high-flux reactive simulations. These large-scale simu-
lations pave the way for feature-scale studies, including
aspect-ratio effects, lateral etch selectivity, and the inter-
play between energy, dose, and local morphology during
pattern transfer.43F. Melting point of MgO: application of few-shot transfer
learning
When the teacher model (i.e., the universal potential)
exhibits systematic errors in a given system, a distilled
student will generally inherit those deficiencies. To ad-
dress this limitation, we explore a transfer-learning strat-
egy in which a small amount of high-fidelity DFT data is
used to correct the distilled model and enhance its accu-
racy. We validate this idea on the melting point of MgO.
It is well known that DFT with the PBE functional sig-
nificantly underestimates MgO’s melting point relative
to experiment, whereas higher-level functionals such as
r2SCAN yield more accurate predictions. Because both
the PFP model we used and MACE were trained on PBE-
based datasets, they may be less accurate for modeling
MgO melting point.
We first follow our standard distillation workflow to
construct a LightPFP model using PFP-sampled train-
ing data, including crystalline MgO, liquid-phase MgO,
and solid–liquid interface configurations. Data collection

11
and initial model training required 7.5 and 1.0 hours, re-
spectively. We then estimated the melting point with
this original LightPFP. Starting from a solid–liquid co-
existence slab, we performed MD at 2600 K, 2650 K,
2700 K, 2750 K, 2800 K, 2850 K, and 2900 K, and moni-
tored whether the crystalline region advanced or receded.
Progress was quantified by the local octahedral order pa-
rameterqoct.44which approaches 1.0 for Mg/O-centered
octahedra in crystalline MgO. Figure 10(a) shows the
fraction of atoms with qoct> 0.25 versus time. At low
temperatures (e.g., 2600 K), this fraction increases to-
ward 1.0, indicating solidification; at higher temperatures
it decreases, indicating melting. At approximately 2700
K, the fraction remains nearly constant over the trajec-
tory, suggesting solid–liquid equilibrium. As expected
for a PBE-level model, this melting point is substantially
below the experimental range from 3073 to 3250 K and
consistent with prior PBE-trained MLIP studies.45,46
To improve accuracy, we applied few-shot trans-
fer learning from PBE to r2SCAN. Using the original
LightPFP as a starting point, we sampled small MgO
structures (64 atoms each) from MD and selected 10 con-
figurations for r2SCAN single-point calculations. Dur-
ing transfer learning, we froze the LightPFP radial-basis
representation and fine-tuned only the readout network
to minimize energy, force, and stress errors against the
r2SCAN labels. This procedure adapts the model to the
r2SCAN potential energy surface. Freezing the represen-
tation mitigates overfitting and catastrophic forgetting
in the few-shot regime while reducing compute. The
r2SCAN calculations took 1.25 hours, and fine-tuning
required 0.5 hours. Re-evaluating with the few-shot
transfer-learned LightPFP under the same MD protocol,
we obtained an estimated melting point of 3125 K, in ex-
cellent agreement with experiment;45see Fig. 10(b). The
end-to-end wall-clock time was 10.25 hours. By compari-
son, building an r2SCAN-level MLIP in the conventional
way would require r2SCAN labels for thousands of struc-
tures; because r2SCAN is several times slower than PBE,
the speedup of LightPFP at the r2SCAN level is even
more pronounced.
This case illustrates a general recipe for overcoming
teacher limitations. Distill a fast, task-adapted student
from a universal potential for broad coverage and effi-
ciency; then, wherever the teacher is biased or under-
trained, apply few-shot transfer learning using a higher-
fidelity dataset to correct the student. With minimal
additional labeling, the student can surpass the teacher
for the property of interest. This strategy is agnostic to
the source of teacher error and is readily extensible to
other universal models and materials systems.
III. DISCUSSION
We introduced LightPFP, a knowledge distillation
framework that resolves the fundamental trade-off be-
tween transferability and efficiency in Machine learn-ing interatomic potentials (MLIPs). By leveraging high-
fidelity universal MLIPs (u-MLIPs) as computational en-
gines for data generation—rather than relying on expen-
sive DFT calculations—we enable rapid development of
lightweight, task-specific MLIPs (ts-MLIPs) with mini-
mal accuracy loss. Demonstrations using Li 6PS5Cl solid-
state electrolytes and high-entropy alloys show that our
distillation strategy reduces ts-MLIP development time
by three orders of magnitude compared to conventional
DFT-based approaches. The resulting distilled ts-MLIPs
achieve inference speeds one to two orders of magni-
tude faster than u-MLIPs due to their streamlined ar-
chitecture, while maintaining comparable accuracy in
critical calculations such as diffusion activation energies
and surface/grain boundary energies. For more com-
plex systems, we demonstrate the framework’s versatil-
ity through reactive ionic etching of SiO 2surfaces, where
combining model distillation with active learning success-
fully handles intricate chemical processes. The distilled
ts-MLIP accurately reproduces the teacher model’s pre-
dictions for both chemical reactions and etching dynam-
ics in molecular dynamics simulations. Finally, we show
how transfer learning can enhance distilled model accu-
racy when u-MLIP precision is insufficient. Using MgO
melting point prediction as a case study, we improved the
predicted melting temperature from 2700 K to 3125 K
using only 10 additional high-accuracy DFT data points,
achieving excellent agreement with experimental values
of 3100-3200 K. We anticipate that this approach will
generalize to other universal potentials, providing a scal-
able, data-efficient foundation for accurate, production-
scale materials simulations.
For brevity, we present only four representative exam-
ples in the main text. In supplementary information II,
we present 11 additional examples of complex simula-
tions achievable by LightPFP, which might be of interest
to readers
IV. METHODS
A. Density functional theory
Density functional theory (DFT) calculations were em-
ployed for three main purposes: (1) generating training
data for machine learning interatomic potentials (MLIPs)
using conventional approaches; (2) evaluating key ma-
terial properties such as interface energies to bench-
mark the accuracy of various MLIPs, including u-MLIPs
and ts-MLIPs; and (3) producing datasets based on the
r2SCAN exchange–correlation (xc) functional for transfer
learning in LightPFP.
All calculations were performed using spin-polarized
DFT as implemented in the Vienna ab initio Simula-
tion Package (VASP, version 6.4.0) with GPU acceler-
ation. The projector augmented-wave (PAW) method
and a plane-wave basis set with a kinetic-energy cutoff
of 520 eV were employed. For the first two purposes,

12
FIG. 10: Evolution of the fraction of atoms with high local structural order ( qoct> 0.25) during molecular dynamics
simulations at different temperatures. (a) LightPFP, (b) LightPFP after few-shot transfer learning
the Perdew–Burke–Ernzerhof (PBE) generalized gradi-
ent approximation was adopted. The pseudopotentials,
cutoff energies, and k-point meshes followed the settings
of the PFP dataset,2corresponding to a k-point density
of approximately 1000 k-points per reciprocal atom.
For the third purpose, calculations were performed
using the r2SCAN meta-GGA xc functional within the
same VASP framework. The functional was activated
through the Meta-GGA option, with all other compu-
tational parameters—such as the 520 eV kinetic-energy
cutoff—kept consistent with the PBE calculations to en-
sure compatibility. The Brillouin zone was sampled using
a k-point grid generated with a KSPACING parameter
of0.5Å−1, ensuring well-converged total energies.
B. Preferred potential (PFP)
PFP is a commercial universal interatomic potential
available via the Matlantis atomic simulation platform.47
It is trained on a high-quality DFT dataset based on
PBE,48r2SCAN49andωB97X-D50exchange-correlation
(xc) functionals. The highly disordered training struc-
tures, e.g. high temperature MD frames, are included in
the dataset to guarantee its reliability in a wide range of
applications.
C. Moment tensor potential (MTP)

\section{Basis function}

Moment tensor potential (MTP) employs a mathe-
matically rigorous descriptor system based on invari-
ant moment tensors that encode atomic environments.25
In MTP, energy can be calculated by the sum of the
atomic energy functions of each atom iin the structure:E=/summationtext
iVi, where Vi=/summationtext
αξαBα(ni).ξαdenotes a
learnable coefficient of MTP, Bαdenotes a basis func-
tion and nidenotes a set of rij, a relative coordinate
position of atom ito its neighbors. Each basis function
Bαcomprises of matrix contractions of moment descrip-
torsMµ,ν, whereµandνare non-negative integers. The
moment descriptor Mµ,νfor atom iis defined as:
Mµ,ν(ni) =/summationdisplay
jfµ(rij)rij⊗rij⊗···⊗rij/bracehtipupleft/bracehtipdownright/bracehtipdownleft /bracehtipupright
νtimes
whererij=rj−riis the relative position vector to neigh-
bor j within cutoff radius Rcut, “⊗” denotes a tensor outer
product. The function fµdescribed a radial part depend-
ing onµis expressed as
fµ(|rij|,zi,zj) =NQ/summationdisplay
β=1c(β)
µ,zi,zjQβ(|rij|)
wherec(β)
µ,zi,zjis a learnable parameter, zindicates the
atomic type, the radial function Qβ(|rij|)is the combi-
nation of Chebyshev polynomials of the first kind and
cutoff function, and NQis the number of polynomials.
Moment descriptors are contracted to form
rotationally-invariant basis functions Bα(ni)that
preserve SO(3) symmetry, enabling accurate repre-
sentation of complex many-body interactions. The
formulation of MTP achieves high data efficiency—basis
functions span a complete polynomial space while
avoiding explicit angular dependence, enabling accurate
fits with small training sets.25,37With training data,
we fit MTP to learn the parameters ξ=\{ξi,...,ξnB\},
wherenBis the number of basis, and c=\{c(β)
µ,zi,zj\},
where the number of coefficients ncdepends on number
offµ, element pairs including the pair of itself, and NQ:
nc=nfµ×nelem-pair×NQ.

13

\section{Neural network readout}

We extend the standard MTP architecture by replac-
ing its linear energy predictor with a neural network
(NN) employing the multi-layer perceptron architecture
parameterized by θ,Mθ. The modified energy expression
becomes:
ENN-MTP
i=Mθ(\{Bα(ni)\}m
α=1) (3)
whereθdenotes trainable weights, and \{Bα\}are in-
variant descriptors from the preceding tensor layer. This
hybrid architecture introduces controlled nonlinearity to
enhance the capability to capture subtle correlations in
potential energy surfaces (PES) that can be difficult for
linear projections to capture.
D. Pretrained student models
We pretrain the MTP model using the large, di-
verse DFT dataset used for the training of u-MLIP,
PFP. The dataset includes both equilibrium and non-
equilibrium structures, enabling broad transferabil-
ity.Takamoto et al.2Unlike the universal PFP graph
neural network, MTPs have limited capacity and are typ-
ically material-specific, so we do not fit all data jointly.
Instead, we adopt Reptile meta-learning38to obtain an
initialization that adapts rapidly to individual systems:
the dataset is split into 12 tasks by structure type; at
each meta-iteration we sample one task, train for a sin-
gle epoch with Adam51where learning rate is 1e−3,
and batch size is 256), then apply a meta-update with
β= 0.5. We run 100 meta-iterations until energies,
forces, and stresses stabilize across tasks. The pretrained
model has a large parameter set due to the large number
of supported elements. Owing to MTP’s modularity, pa-
rameters can be subset by elements at inference or fine-
tuning. This initialization acts as a strong prior from
diverse chemistry, improving robustness, reducing over-
fitting, and speeding convergence when the target dataset
is small or undersampled. The details of pretrained mod-
els can be found in supplementary information I.
E. Training method
For all LightPFP models trained in the applications de-
scribed in Section II, datasets were split 90\%/10\% into
training and validation. The validation set was used both
for model selection (choosing the checkpoint with the
lowest validation loss) and for reporting validation er-
rors. The training objective combined energy, force, and
stress terms:
L=α·Lenergy+β·Lforce+γ·Lstress (4)whereLenergy is the mean squared error (MSE) of the
energy per atom, Lforceis the MSE of the Cartesian force
components on each atom (x, y, z), and Lstress is the
MSE of the stress tensor components. The coefficients
α,βandγweight the energy, force, and stress losses,
respectively.
Optimization was performed using Adam51with a
batch size of 128, following a three-stage training pro-
cedure. In the first stage, the loss coefficients for energy,
forces, and stress were set to (10−5,10,10−5). In the sec-
ond stage, they were adjusted to (1,0.1,10). In the third
stage, the loss coefficients were automatically determined
to balance the three losses. Specifically, we first com-
puted the total validation loss of the second-stage model
using the stage-two coefficients as loss weight. The coef-
ficient for the energy loss was then calculated as the total
weighted validation loss divided by three and further di-
vided by the energy loss from stage two. The coefficients
for forces and stress were calculated analogously, each us-
ing their respective loss from stage two. The three-stage
procedure yielded faster convergence than conventional
training method without variation of coefficients. A de-
tailed comparison is provided in supplementary informa-
tion I.
A linear warmup learning rate scheduler was applied,
increasing the learning rate from zero to its stage-specific
maximum during the first 20\% of epochs in each stage,
and then linearly decaying it to approach zero by the
final epoch. The learning rates for stage 1, stage 2, and
stage 3 were set to 0.1, 0.01, and 0.01, respectively.
V. DATA AVAILABILITY
The datasets generated and analysed during the
current study are available from public repositories.
The DFT training datasets have been deposited on
the Materials Cloud under the accession number DOI:
XX.XXXXX/materialscloud:XXXXXX (to be assigned).
The PFP-labelled datasets used for model training and
evaluation are available in the accompanying GitHub
repository https://github.com/pfn-attic/light-pfp-paper.
All other data supporting the findings of this study are
available from the corresponding author upon reasonable
request.
VI. CODE AVAILABILITY
All codes used for model training, and atomistic sim-
ulations are available in the LightPFP examples GitHub
repository https://github.com/pfn-attic/light-pfp-paper.
The implementation depends on the PFP and LightPFP
packages, which are components of the Matlantis com-
mercial platform developed by Preferred Networks, Inc.
Access to these packages requires a valid Matlantis li-
cense.

14
VII. ACKNOWLEDGEMENTS
The authors are grateful to Tasuku Onodera, Takashi
Kojima, Masanao Goto, Yuta Tanaka, and Yuji Hakozaki
(ENEOS Holdings, Inc.) for their valuable suggestions
during the development of LightPFP. We also thank
Chikashi Shinagawa (Preferred Networks, Inc.) for in-
sightful discussions regarding the DFT calculation setup.
1D. Ceperley and B. Alder, “Ground-state of the electron-gas by
a stochastic method,” Phys. Rev. Lett. 45, 566–569 (1980).
2S. Takamoto, C. Shinagawa, D. Motoki, K. Nakago, W. Li, I. Ku-
rata, T. Watanabe, Y. Yayama, H. Iriguchi, Y. Asano, et al.,
“Towards universal neural network potential for material discov-
ery applicable to arbitrary combination of 45 elements,” Nature
Communications 13, 2991 (2022).
3C. Chen and S. P. Ong, “A universal graph deep learning inter-
atomic potential for the periodic table,” Nature Computational
Science 2, 718–728 (2022).
4B. Deng, P. Zhong, K. Jun, J. Riebesell, K. Han, C. J. Bartel,
and G. Ceder, “CHGNet as a pretrained universal neural net-
work potential for charge-informed atomistic modelling,” Nature
Machine Intelligence 5, 1031–1041 (2023).
5I. Batatia, D. P. Kovacs, G. Simm, C. Ortner, and G. Csányi,
“MACE: Higher order equivariant message passing neural net-
works for fast and accurate force fields,” Advances in neural in-
formation processing systems 35, 11423–11436 (2022).
6I. Batatia, P. Benner, Y. Chiang, A. M. Elena, D. P. Kovács,
J. Riebesell, X. R. Advincula, M. Asta, M. Avaylon, W. J. Bald-
win,et al., “A foundation model for atomistic materials chem-
istry,” arXiv preprint arXiv:2401.00096 (2023).
7S. Kong, N. Matsui, S. Hori, M. Hirayama, K. Mori, T. Saito,
R. Kanno, and K. Suzuki, “Exploration of lithium-ion conduc-
tors based on local coordination environments using crystallo-
graphic site fingerprints,” Journal of the American Chemical So-
ciety (2025).
8Y. Hinuma and M. Kitta, “Facile formation of two-phase domains
in a single crystalline li7–x ti5o12 particle,” ACS Applied Energy
Materials (2025).
9S. Narumi, H. E. Otal, T. Q. Nguyen, M. Koyama, and N. Zettsu,
“Tailoring the room-temperature miscibility gap in ordered spinel
lini 0.5 mn 1.5 o 4 cathodes by multi-element doping,” Journal
of Materials Chemistry A (2025).
10B. G. Son, C. Kwon, Y. Cho, T. Jang, H. R. Byon, S. Kim, and
E. S. Cho, “Constructing reversible li deposition interfaces by tai-
loring lithiophilic functionalities of a heteroatom-doped graphene
interlayer for highly stable li metal anodes,” ACS Applied Mate-
rials \& Interfaces 16, 32259–32270 (2024).
11O. Kwon, T. Y. Kim, T. Kim, J. Kang, S. Jang, H. Eom,
S. Choi, J. Shin, J. Park, M.-L. Seol, et al., “Intelligent stress-
adaptive binder enabled by shear-thickening property for silicon
electrodes of lithium-ion batteries,” Advanced Energy Materials
14, 2304085 (2024).
12H. Du, Y. Dong, Q.-J. Li, R. Zhao, X. Qi, W.-H. Kan, L. Suo,
L. Qie, J. Li, and Y. Huang, “A new zinc salt chemistry for
aqueous zinc-metal batteries,” Advanced Materials 35, 2210055
(2023).
13T. Shimada, P. M. Usov, Y. Wada, H. Ohtsu, T. Watanabe,
K. Adachi, D. Hashizume, T. Matsumoto, and M. Kawano,
“Long time CO2 storage under ambient conditions in isolated
voids of a porous coordination network facilitated by the “magic
door” mechanism,” Advanced Science 11, 2307417 (2024).
14J. Koh, C. Kwon, H. Kim, E. Lee, A. Machida, Y. Nakahira,
Y. J. Hwang, K. Sakaki, S. Kim, and E. S. Cho, “Defect-driven
evolution of oxo-coordinated cobalt active sites with rapid struc-
tural transformation for efficient water oxidation,” ACS nano 18,
28986–28998 (2024).
15Y. Hinuma, “Neural network potential molecular dynamics simu-
lations of (La, Ce, Pr, Nd) 0.95 (Mg, Zn, Pb, Cd, Ca, Sr, Ba) 0.05F2. 95,” The Journal of Physical Chemistry B 128, 12171–12178
(2024).
16A. Miura, K. Muraoka, K. Maki, S. Kawaguchi, K. Hikima,
H. Muto, A. Matsuda, I. Yamane, T. Shimada, H. Ito, et al.,
“Stress-induced martensitic transformation in Na3YCl6,” Jour-
nal of the American Chemical Society 146, 25263–25269 (2024).
17K. Watanabe, T. Higo, K. Saegusa, S. Matsumoto, H. Sampei,
Y. Isono, A. Shimojuku, H. Furusawa, and Y. Sekine, “Oxidative
dehydrogenation of ethane combined with co2 splitting via chem-
ical looping on in2o3 modified with ni–cu alloy,” ACS Catalysis
15, 5876–5885 (2025).
18T. Honbo, Y. Ono, K. Suetsugu, M. Hara, A. Taborosi, K. Aoki,
S. Nagano, M. Koyama, and Y. Nagao, “Effects of alkyl side
chain length on the structural organization and proton conduc-
tivity of sulfonated polyimide thin films,” ACS Applied Polymer
Materials 6, 13217–13227 (2024).
19K. Hisama, K. V. Bets, N. Gupta, R. Yoshikawa, Y. Zheng,
S. Wang, M. Liu, R. Xiang, K. Otsuka, S. Chiashi, et al., “Molec-
ular dynamics of catalyst-free edge elongation of boron nitride
nanotubes coaxially grown on single-walled carbon nanotubes,”
ACS nano 18, 31586–31595 (2024).
20H. Kim, T. Kim, H. K. Chung, J. Jeon, S.-C. Kim, S. O. Won,
R. Harada, T. Tsugawa, S. Kim, and S. K. Kim, “Sustained
area-selectivity in atomic layer deposition of ir films: Utilization
of dual effects of o3 in deposition and etching,” Small 20, 2402543
(2024).
21S. Jin, C. Kwon, A. Bugaev, B. Karakurt, Y.-C. Lin, L. Savereide,
L. Zhong, V. Boureau, O. Safonova, S. Kim, et al., “Atom-by-
atom design of cu/zro x clusters on mgo for co2 hydrogenation
using liquid-phase atomic layer deposition,” Nature Catalysis 7,
1199–1212 (2024).
22H. Kim, H. Kim, W. Kim, C. Kwon, S.-W. Jin, T. Ha, J.-H. Shim,
S. Park, A. Jamal, S. Kim, et al., “Facile synthesis of nanoporous
mg crystalline structure by organic solvent-based reduction for
solid-state hydrogen storage,” Nature Communications 15, 10800
(2024).
23T. Ishikawa, Y. Tanaka, and S. Tsuneyuki, “Evolutionary search
for superconducting phases in the lanthanum-nitrogen-hydrogen
system with universal neural network potential,” Physical Review
B109, 094106 (2024).
24J. Bae, C. Kwon, S.-O. Park, H. Jeong, T. Park, T. Jang, Y. Cho,
S. Kim, and S. Choi, “Tunable ion energy barrier modulation
through aliovalent halide doping for reliable and dynamic mem-
ristive neuromorphic systems,” Science Advances 10, eadm7221
(2024).
25I. S. Novikov, K. Gubaev, E. V. Podryabinkin, and A. V.
Shapeev, “The mlip package: moment tensor potentials with mpi
and active learning,” Machine Learning: Science and Technology
2, 025002 (2020).
26H. Wang, L. Zhang, J. Han, et al., “Deepmd-kit: A deep learn-
ing package for many-body potential energy representation and
molecular dynamics,” Computer Physics Communications 228,
178–184 (2018).
27A. Musaelian, S. Batzner, A. Johansson, L. Sun, C. J. Owen,
M. Kornbluth, and B. Kozinsky, “Learning local equivariant rep-
resentations for large-scale atomistic dynamics,” Nature Commu-
nications 14, 579 (2023).
28J. A. Pople, “Nobel lecture: Quantum chemical models,” Reviews
of Modern Physics 71, 1267 (1999).
29S. Takamoto, D. Okanohara, Q. Li, and J. Li, “Towards universal
neural network interatomic potential,” J. Materiomics 9, 447–454
(2023).
30J. Li, C. Wang, J. Chang, W. Cai, V. Bulatov, K. Ho, and
S. Yip, “Core energy and peierls stress of a screw dislocation
in bcc molybdenum: A periodic-cell tight-binding study,” Phys.
Rev. B 70, 104113 (2004).
31H. Tang, B. Xiao, W. He, P. Subasic, A. R. Harutyunyan,
Y. Wang, F. Liu, H. Xu, and J. Li, “Approaching coupled-
cluster accuracy for molecular electronic structures with multi-
task learning,” Nature Computational Science 5, 144–154 (2025).

15
32J. D. Morrow and V. L. Deringer, “Indirect learning and phys-
ically guided validation of interatomic potential models,” The
Journal of Chemical Physics 157(2022).
33I. Amin, S. Raja, and A. Krishnapriyan, “Towards fast, special-
ized machine learning force fields: Distilling foundation models
via energy hessians,” arXiv preprint arXiv:2501.09009 (2025).
34J. L. Gardner, D. F. Toit, C. B. Mahmoud, Z. F. Beaulieu,
V. Juraskova, L.-B. Paşca, L. A. Rosset, F. Duarte, F. Martelli,
C. J. Pickard, et al., “Distillation of atomistic foundation mod-
els across architectures and chemical domains,” arXiv preprint
arXiv:2506.10956 (2025).
35D. Zhang, X. Liu, X. Zhang, C. Zhang, C. Cai, H. Bi, Y. Du,
X. Qin, A. Peng, J. Huang, et al., “DPA-2: a large atomic model
as a multi-task learner,” npj Computational Materials 10, 293
(2024).
36S. Takamoto, S. Izumi, and J. Li, “TeaNet: Universal neural
network interatomic potential inspired by iterative electronic re-
laxations,” Computational Materials Science 207, 111280 (2022).
37Y. Zuo, C. Chen, X. Li, Z. Deng, Y. Chen, J. Behler, G. Csányi,
A. V. Shapeev, A. P. Thompson, M. A. Wood, et al., “Perfor-
mance and cost assessment of machine learning interatomic po-
tentials,” The Journal of Physical Chemistry A 124, 731–745
(2020).
38A. Nichol, J. Achiam, and J. Schulman, “On first-order meta-
learning algorithms,” arXiv preprint arXiv:1803.02999 (2018).
39P. Jozwik, W. Polkowski, and Z. Bojar, “Applications of Ni3Al
based intermetallic alloys—current stage and potential percep-
tivities,” Materials 8, 2537–2568 (2015).
40Z. Deng, Z. Zhu, I.-H. Chu, and S. P. Ong, “Data-driven first-
principles methods for the study and design of alkali superionic
conductors,” Chemistry of Materials 29, 281–288 (2017).
41M. De Jong, W. Chen, T. Angsten, A. Jain, R. Notestine,
A. Gamst, M. Sluiter, C. Krishna Ande, S. Van Der Zwaag, J. J.
Plata, et al., “Charting the complete elastic properties of inor-
ganic crystalline compounds,” Scientific data 2, 1–13 (2015).42D. H. Kim, S. J. Kwak, J. H. Jeong, S. Yoo, S. K. Nam, Y. Kim,
and W. B. Lee, “Molecular dynamics simulation of silicon dioxide
etching by hydrogen fluoride using the reactive force field,” ACS
omega 6, 16009–16015 (2021).
43G. S. Oehrlein, S. M. Brandstadter, R. L. Bruce, J. P. Chang,
J. C. DeMott, V. M. Donnelly, R. Dussart, A. Fischer, R. A.
Gottscho, S. Hamaguchi, et al., “Future of plasma etching for
microelectronics: Challenges and opportunities,” Journal of Vac-
uum Science \& Technology B 42(2024).
44N. E. Zimmermann, M. K. Horton, A. Jain, and M. Haranczyk,
“Assessing local structure motifs using order parameters for motif
recognition, interstitial identification, and diffusion path charac-
terization,” Frontiers in Materials 4, 34 (2017).
45S.-M. Liang and R. Schmid-Fetzer, “Complete thermodynamic
description of the Mg-Ca-O phase diagram including the Ca-O,
Mg-O and CaO-MgO subsystems,” Journal of the European Ce-
ramic Society 38, 4768–4785 (2018).
46K. Lee, Y. Park, and S. Han, “Ab initio construction of full phase
diagram of MgO-CaO eutectic system using neural network inter-
atomic potentials,” Physical Review Materials 6, 113802 (2022).
47“Matlantis, software as a service style material discovery tool,”
https://matlantis.com/ .
48J. P. Perdew, K. Burke, and M. Ernzerhof, “Generalized gradient
approximation made simple,” Physical review letters 77, 3865
(1996).
49J. W. Furness, A. D. Kaplan, J. Ning, J. P. Perdew, and J. Sun,
“Accurate and numerically efficient r2SCAN meta-generalized
gradient approximation,” The journal of physical chemistry let-
ters11, 8208–8215 (2020).
50J.-D. Chai and M. Head-Gordon, “Long-range corrected hy-
brid density functionals with damped atom–atom dispersion cor-
rections,” Physical Chemistry Chemical Physics 10, 6615–6620
(2008).
51D. P. Kingma and J. Ba, “Adam: A method for stochastic opti-
mization,” arXiv preprint arXiv:1412.6980 (2014).

Supplementary Information I for: LightPFP: A Lightweight Route to Ab Initio
Accuracy at Scale
Wenwen Li,1,a)Nontawat Charoenphakdee,1,b)Yong-Bin Zhuang,1Ryuhei Okuno,1
Yuta Tsuboi,1So Takamoto,1Junichi Ishida,2and Ju Li3, 4
1)Preferred Networks Inc., Tokyo, Japan.
2)Matlantis Corporation, Tokyo, Japan.
3)Department of Materials Science and Engineering, Massachusetts Institute of
Technology, Cambridge, MA 02139, USA
4)Department of Nuclear Science and Engineering, Massachusetts Institute of
Technology, Cambridge, MA, USA
(Dated: 7 November 2025)
a)Electronic mail: wenwenli@preferred.jp
b)Electronic mail: nontawat@preferred.jp
i

Supplementary Note 1. ERROR TRANSFER IN THE
DFT→PFP→LIGHTPFP PIPELINE
As mentioned in the main text, the dominant error arises from formally exact →DFT due
to DFT’s intrinsic limitations; by contrast, DFT →PFP transfer error is already small (with
the costly training completed), and PFP →LightPFP is even smaller with fast, overnight
training. Considering independent sources of error e1,e2, ...,emoften do not add up linearly
but quadratically (if statistically uncorrelated due to different “physics"):
e=/radicalBig
e2
1+e2
2+...+e2
m, (S1)
if|e1|∼100meV/atom dominates over |e2|,...,|em|, then the leading-order contributions
of|e2|,...,|em|to the total error would be even smaller than it seems, based on Taylor
expansion:
e≈e1+e2
2+...+e2
m
2e1, (S2)
and likely become practically negligible. In other words, if the DFT →PFP and PFP→LightPFP
neural network trainings are done well, LightPFP may represent DFT much better than
how DFT reflects reality.
ii

Supplementary Note 2. DATASET SAMPLING METHODS
A robust training dataset for machine learning interatomic potentials is built by sys-
tematically sampling diverse yet physically meaningful configurations around one or more
initial structures. Sampling methods can be combined and run independently to cover ther-
mal, mechanical, defect, surface, and chemical degrees of freedom. These strategies balance
relevance to the target material with diversity across configuration space, improving both
accuracy and robustness of the potential. The illustration of these sampling methods are
shown at Figure S1
FIG. S1. Illustration of sampling methods used in the LightPFP dataset generation. (a) Uniform
compression/stretch sampling, (b) Displacement sampling, (c) Vacancy sampling, (d) Rattle sam-
pling, (e) Substitution sampling, (f) Surface sampling and (g) Deformation sampling
iii

A. Molecular dynamics sampling
Molecular dynamics (MD) sampling generates structures by propagating the atomic sys-
tem under finite-temperature dynamics, optionally at controlled pressure. By choosing
ensembles such as NVT (constant volume) or NPT (constant pressure), and by varying
temperature, one can explore configurational space from near-equilibrium states to highly
disordered regimes. To prevent collecting redundant configurations, snapshots are taken at
a fixed stride along the trajectory.
B. Uniform compression/stretch sampling
Uniform compression/stretch sampling produces structures by isotropically scaling the
lattice vectors of the input periodic structure, keeping the lattice angles unchanged and
preserving fractional atomic coordinates. This method targets the volume–energy relation-
ship and can be augmented by fixed-cell relaxation or MD runs starting from the scaled
configurations to enrich the dataset at specific densities.
C. Deformation sampling
Deformation (strain) sampling applies prescribed normal and shear strain components
to the unit cell, changing both lattice lengths and angles while maintaining periodicity.
By scanning the six independent components of the strain tensor, one obtains structures
spanning elastic distortions relevant to mechanical properties. Atomic positions may be
further optimized under fixed cell shape to produce relaxed strained configurations, helping
the model learn stress–strain behavior and elastic responses.
D. Displacement sampling
Single-atom displacement sampling perturbs one atom from its equilibrium position along
a Cartesian direction by a controlled amplitude. Such localized perturbations probe the
curvature of the potential energy surface and the force constants around equilibrium, which
are essential for learning vibrational responses. Each displaced configuration is generated
independently from the same starting structure to map local force landscapes efficiently.
iv

E. Rattle sampling
Rattle sampling introduces random displacements to all atoms simultaneously, drawing
each component of the displacement from a specified distribution (e.g., Gaussian). This
global perturbation broadens coverage of non-equilibrium configurations and can reveal fail-
ure modes of the model under larger distortions. Because it may produce unphysical con-
figurations with extreme forces, filtering based on maximum force thresholds and optional
relaxation steps are recommended to maintain data quality. This method is recommended
for the molecular systems since it provided useful information of bond breaking.
F. Vacancy sampling
Vacancy sampling creates point-defect structures by randomly removing one or more
atoms from the initial configuration. These defective structures can be complemented with
fixed-cell relaxations or MD to sample local reconstructions and thermally activated defect
configurations. By including vacancy-containing data, the model gains sensitivity to defect
energetics and local structural changes associated with missing atoms.
G. Surface sampling
Surface sampling constructs slab models by cleaving the periodic bulk along specified
Miller indices and introducing a vacuum layer to isolate the surfaces. Symmetry analysis
can be used to avoid duplicate surfaces generated by equivalent indices in high-symmetry
crystals. Subsequent fixed-cell relaxations and MD on slab geometries enrich the dataset
with surface reconstructions and thermal fluctuations. This approach is intended for periodic
crystalline inputs and targets accurate description of surface energetics and structure.
H. Substitution sampling
Element substitution sampling generates chemically disordered structures by stochasti-
cally replacing atoms in the initial structure with user-specified species at defined probabil-
ities. This method captures configurational variability in multicomponent systems, such as
alloys, by sampling diverse local chemistries. Fixed-cell relaxation and MD can be applied
v

after substitution to explore thermally accessible configurations, improving robustness and
transferability across compositional variations. This method is useful for the solid-solution
and high-entropy alloy.
vi

Supplementary Note 3. HYPERPARAMETERS OF LIGHTPFP MODELS
The hyperparameters of the LightPFP models used in the results section is listed here
Material Cutoff levmax CµCνnqNeural Network Readout
Ni3Al 6.0 8 1 1 16 None
Li6PS5Cl 6.0 8 1 1 16 [16, 16, 1]
HEA 5.0 8 1 1 16 None
MgO 6.0 8 1 1 16 [16, 16, 1]
SiO2-HF 6.0 8 1 1 16 [16, 16, 1]
To specify the complexity of momenta tensor potential, parameter cutoff, levmax, µ,ν,
nqis used. Once such hyperparameters are defined, each admissible basis function must have
the level less than levmax. As explained in the maintext, the basis function Bαcomprises
of matrix contractions of moment descriptors Mµ,ν
Mµ,ν(ni) =/summationdisplay
jfµ(rij)rij⊗rij⊗···⊗rij/bracehtipupleft/bracehtipdownright/bracehtipdownleft /bracehtipupright
νtimes
where the level of basis can be calculated as
lev = 2+ Cµ×µ+Cν×ν (S3)
Thenqis the number of radial basis functions in the polynomial function. 27 basis
functions are admissible when we set levmax=8, Cµ=1,Cν=1:
where “·” is a dot product between vectors and “ :” is a Frobenius product of two matrices.
Intuitively, one can think that the higher the levmax, the more complex the MTP. The
lowerCµandCν, the more complex the MTP. Note that the more complex the MTP, the
more memory and the longer the computation time it requires.
vii

Basis index Moment tensor component Level
B0 M0,0 2
B1 M0,0×M0,0 4
B2 M0,0×M0,0×M0,0 6
B3 M0,0×M0,0×M0,0×M0,08
B4 M1,0 3
B5 M0,0×M1,0 5
B6 M0,0×M0,0×M1,0 7
B7 M1,0×M1,0 6
B8 M0,0×M1,0×M1,0 8
B9 M2,0 4
B10 M0,0×M2,0 6
B11 M0,0×M0,0×M2,0 8
B12 M1,0×M2,0 7
B13 M2,0×M2,0 8
B14 M3,0 5
B15 M0,0×M3,0 7
B16 M1,0×M3,0 8
B17 M4,0 6
B18 M0,0×M4,0 8
B19 M5,0 7
B20 M6,0 8
B21 M0,1·M0,1 6
B22 M0,0×(M0,1·M0,1) 8
B23 M0,1·M1,1 7
B24 M0,1·M2,1 8
B25 M1,1·M1,1 8
B26 M0,2:M0,2 8
TABLE S1. Definitions of Biwith corresponding right-hand side expressions and levels.
viii

Supplementary Note 4. DETAILS FOR HIGH-ENTROPY ALLOY
EXAMPLE
Detailed results for the high-entropy alloy (HEA) case study are presented here. Figure S2
compares the speed of LightPFP with that of other models. Figure S3presents the equation
of state (EOS) of the HEA as computed by different models. Table S2summarizes the
datasets used for the training of LightPFP and MTP-DFT models.
FIG. S2. (a) Molecular dynamics (MD) computational speed with AlCoCrFeNi high-entropy alloy
as a function of number of atoms for three MLIPs: PFP, LightPFP (MTP), and MACE. (b) Trade-
off between the overall time spent on MLIP building for AlCoCrFeNi high-entropy alloy, including
data collection and model training, and MD computational speed for PFP, LightPFP, MACE, and
MTP. Inset: the total time cost to complete both MLIP building and a 10 ns MD simulation of a
10,000-atom system With PFP, LightPFP, MACE, and MTP.
ix

FIG. S3. Equation of states of AlCoCrFeNi high-entropy alloy calculated by DFT, PFP, LightPFP,
MACE and MTP
TABLE S2. Composition of the AlCoCrFeNi high-entropy alloy dataset.
Type of Sampling Number of Number of
structure method structures atoms
LightPFP Dataset (labeled by PFP)
crystal substitution+MD 2040 206040
boundary substitution+MD 6200 1083760
slab substitution+MD 1398 66816
Total 9638 1356616
MTP Dataset (labeled by DFT)
crystal substitution+MD 531 42484
boundary substitution+MD 286 9152
slab substitution+MD 195 8724
Total 1012 60360
x

FIG. S4. Parity plot of DFT forces againes predicted forces by different MLIPs, (a) PFP; (b)
LightPFP; (c) MACE and (d) MTP
xi

Supplementary Note 5. DETAILS IN DATA EFFICIENCY EVALUATION
FIG. S5. Detailed comparison of data efficiency between fine-tuned pretrained and scratch-trained
student models.
xii

Supplementary Note 6. PRETRAINED STUDENT MODEL
a. All-elements pretrained student model We employed a comprehensive dataset to
train PFP, a universal potential-based graph neural networks. This dataset comprises 86
different elements, covering nearly the entire periodic table and encompassing both equilib-
rium structures and numerous disordered structures that deviate from equilibrium states.
The dataset includes not only bulk phases but also complex structures such as surfaces, ad-
sorption configurations, and clusters. This comprehensive coverage is the fundamental rea-
son why PFP exhibits broad applicability across diverse materials simulations. For dataset
details, please refer to Takamoto et al.1.
However, compared to PFP1, moment tensor potentials (MTPs) are compact models with
limited parameters and constrained expressive power, typically applicable only to single ma-
terials systems. Consequently, using MTPs to fit all datasets simultaneously presents sig-
nificant challenges. Therefore, our MTP pretraining strategy aims to optimize the model to
facilitate subsequent fine-tuning for individual tasks, instead of maximizing accuracy across
all datasets. To achieve this objective, we employed the Reptile meta-learning algorithm2.
The Reptile algorithm operates by iteratively sampling tasks from a task distribution
and updating model parameters to enhance the model’s ability to rapidly adapt to new
tasks. In our implementation, we partitioned the complete dataset into 12 specific tasks
based on structural types. During each inner loop iteration, we select a task (i.e., a dataset
containing specific structural types such as single molecules) to train the MTP model. Given
the substantial size of each task’s dataset, we limit training to one epoch per inner loop before
proceeding to the parameter update. The model parameters are then updated according to
the following formula:
δθ=θi−θ,
θ←θ+βδθ,
whereθrepresents the MTP parameters, θidenotes the parameters after the i-th inner
loop, and βis a hyperparameter in the Reptile algorithm that controls the magnitude of the
meta-update step during training. In our implementation, βis set to 0.5. We iteratively
repeat the task sampling and inner-loop/meta-update procedures for 100iterations until
convergence of energy, forces, and stress is observed across all datasets.
We employed the Adam optimization method with a learning rate of 1×10−3. The model
xiii

was trained for 1 epochs with a batch size of 256. Total pretraining time was approximately
100 hours.
For example, pretrained student model with hyperparameter (levmax=8, Cµ=1,Cν=1,
nq=16) contains 86 ×86×4×16 training parameters for the radial function cand additional
27 coefficients for the basis functions ξ. The modular structure of MTP enables selective
parameter extraction during inference or fine-tuning, significantly enhancing computational
efficiency. The extraction procedure is straightforward, depending on elements used for the
task. For example, when handling a material containing only H and O elements, we can ex-
tract the relevant subset of the radial function parameter tensor—specifically a 2 ×2×4×16
matrix corresponding to these elements, while maintaining the coefficients of the basis func-
tion unchanged. Consequently, although the pretrained model may contain numerous pa-
rameters, it automatically reduces to a compact, element-specific model equivalent in size
to those trained from scratch for the particular material system.
b. Specific type pretrained student model In addition to the pretrained LightPFP model
that covers almost all materials, we also tried pretrained LightPFP models for special types
of materials. As an illustrative example, we consider our organic pretrained student model,
which is specifically designed for organic molecular systems. The training process begins
with dataset construction. We randomly sample molecular information, such as SMILES
representations, from PubChem3and generate corresponding three-dimensional conformers.
Several molecules are then randomly placed into a simulation box, ensuring that the overall
density falls within an appropriate range. An optimization algorithm is employed to adjust
atomic positions without breaking chemical bonds, thereby minimizing atomic overlap be-
tween molecules. From these initial configurations, we perform molecular dynamics (MD)
simulations using PFP at temperatures randomly selected between 300 and 3000 K. Each
simulation runs for 1000 steps, and configurations are sampled every 100 steps. This pro-
cedure is repeated many times to obtain a diverse collection of molecular configurations.
The resulting dataset is subsequently used for training a Moment Tensor Potential (MTP)
model, yielding a pretrained MTP tailored for organic systems.
We observe that when the model is restricted to a specific class of materials, the pre-
trained MTP demonstrates a notable capability for direct application without fine-tuning.
As shown in Fig. S6, the pretrained model accurately predicts the densities of various or-
ganic molecules, exhibiting strong agreement with experimental results despite the absence
xiv

of these molecules from the training dataset. This finding suggests a promising new di-
rection: by constraining the material domain, one can develop lightweight machine-learned
interatomic potentials (MLIPs) with reduced generalization compared to universal MLIPs
(uMLIPs), yet capable of fast inference and requiring no additional training. For instance,
pretrained student models can be constructed for specific material classes such as alloys,
oxides, perovskites, and metal–organic frameworks (MOFs).
FIG. S6. Learning curves for different training strategies.
xv

Supplementary Note 7. 3-STAGES TRAINING METHOD
We propose a three-stage training strategy for the LightPFP model. In Stage I, the
optimization focuses on fitting forces; in Stage II, the emphasis shifts to energy and stress;
and in Stage III, the loss terms are balanced so that the energy, force, and stress losses are
of comparable magnitudes. This is achieved by progressively adjusting the coefficients of
the energy ( α), force (β) and stress ( γ) terms in the loss function.
To assess the effectiveness of this strategy, we trained on the Li 6PS5Cl dataset and con-
ducted three fixed-weight baselines. Each baseline uses constant loss weights equal to those
employed in one stage of the three-stage schedule: Baseline 1 ( α= 10−5,β= 10.0,γ= 10−5),
Baseline 2 ( α= 1.0,β= 0.1,γ= 10.0), and Baseline 3 ( α= 26.2,β= 0.034,γ= 1383.1).
The evolution of the losses over epochs is shown in Figure S7, and the final losses are
summarized in Table S3.
FIG. S7. Learning curves for different training strategies.
baseline 1 baseline2 baseline3 3-stages-training
energy1.67×10−45.76×10−64.39×10−64.96×10−6
forces 0.00486 0.00651 0.0110 0.00631
stress3.03×10−71.35×10−71.50×10−79.28×10−8
TABLE S3. Comparison of the final energy, force, and stress losses across training strategies.
Baseline 1 fails to fit the energy accurately, yielding the largest energy loss, while Baseline
3 fails to fit forces accurately. Baseline 2 offers a more balanced trade-off; however, its energy,
force, and stress losses are all larger than those achieved by the three-stage training. Overall,
the proposed three-stage schedule provides the best balance across the three targets, with
xvi

near-optimal energy and stress losses and competitive force accuracy.
xvii

Supplementary Note 8. ACTIVE LEARNING METHOD
Active learning is a powerful approach for developing accurate and efficient interatomic
potentials in molecular dynamics simulations. Here is a brief introduction of the active
learning workflow we used for the "Dry etching of SiO 2" example and several other examples
in the Supplementary Materials II (see Fig. S8):
1. Initial Dataset: A simple initial dataset is necessary for active learning. The initial
dataset does not need to be large and robust.
2. Model Training: The initial LightPFP model is trained with this initial dataset.
3. Exploration: The LightPFP model is then used to drive molecular dynamics (MD)
simulations, exploring new configurations and areas of the potential energy surface.
4. Quality Check: At certain MD steps, check the accuracy of LightPFP. Calculate
the energy, forces, and stress of the MD snapshot using PFP, and compare these with the
LightPFP predictions.
5. Data Selection: Include the MD snapshot in the training dataset if the discrepancy
between PFP and LightPFP is greater than the minimum threshold and less than the max-
imum threshold.
6. Model Update: After several MD simulations are finished or cease based on other
criteria, update the LightPFP model with the dataset collected in current and previous
iterations and the initial dataset.
7. Iteration: Steps 3 to 6 are repeated iteratively. Each iteration improves the potential’s
accuracy and extends its applicability.
This active learning approach allows for the efficient development of accurate potentials by
focusing computational resources on the most informative data points, ultimately resulting
in a potential that can reliably reproduce the behavior of the target system across a wide
range of conditions.
a. Sampling threshold We uses minimum/maximum thresholds to collect high-quality
training data. For more efficient training data collection, it also performs early stopping of
MD simulations or PFP-based sampling upon detecting outliers. Both PFP and LightPFP
are used to calculate the energy, forces, and stress of given MD snapshots. The errors of
LightPFP w.r.t PFP are used to determine if certain MD snapshots should be collected. The
valid error range is defined by both a minimum (lower bound) and a maximum threshold
xviii

FIG. S8. Illustration of active learning workflow
(upper bound). When the error of an MD snapshot is very large, exceeding the upper bound,
it indicates an unreasonable structure that is not beneficial for training. Continuing MD
from this point can lead to more structures that are not valuable. In such cases, the MD
simulation will stop early. In the "Dry etching of SiO 2" example, error checking is performed
every 100 steps, and the selection criterion is: energy error in between 5.0 and 40.0 times
of energy MAE of current using LightPFP model; force error (largest atomic error in the
structure) is in between 1.5 and 50 eV/A.
b. MD early stop As mentioned, the MD simulation will stop early when the discrep-
ancy between PFP and LightPFP is very large. This indicates that the MD has reached
a configuration where the current LightPFP model is unreliable. While structures with
huge errors compared to PFP are not useful for training, the structures leading up to such
structures, typically several MD steps before, are critical. Learning from these preceding
structures helps prevent the MD from evolving into unphysical configurations. Our work-
xix

FIG. S9. Back-tracking mechanism of active learning when MD failed
flow provides two mechanisms for this: (1) Back-tracking Method: When MD stops due to
large prediction errors, the algorithm checks previous MD steps using a binary search until
a training structure with the error in the specified region is found. MD snapshots are cached
to facilitate this process. (2) PFP-based fallback: When MD stops early, the simulation rolls
back to the previous checkpoint and continues using PFP instead of the LightPFP model
for several more additional samples. In the "Dry etching of SiO 2" example, we collect 5
additional training structures based on PFP when MD failed.
c. Model update The model is updated in each iteration with the latest dataset and
all previous datasets. To accelerate active learning, the total number of epochs for model
training is adjusted according to the size of whole datasets, and the training time is kept
roughly constant. This mechanism is designed to handle the gradually increasing dataset
during the active learning iterations. In the "Dry etching of SiO 2" example, we fixed the
time cost for each model update to 0.5 hour.
xx

REFERENCES
1S. Takamoto, C. Shinagawa, D. Motoki, K. Nakago, W. Li, I. Kurata, T. Watanabe,
Y. Yayama, H. Iriguchi, Y. Asano, et al., Nature Communications 13, 2991 (2022).
2A. Nichol, J. Achiam, and J. Schulman, arXiv preprint arXiv:1803.02999 (2018).
3S. Kim, J. Chen, T. Cheng, A. Gindulyte, J. He, S. He, Q. Li, B. A. Shoemaker, P. A.
Thiessen, B. Yu, et al., Nucleic acids research 53, D1516 (2025).
xxi

Supplementary Information II for: LightPFP: A Lightweight Route to Ab Initio
Accuracy at Scale
Wenwen Li,1,a)Nontawat Charoenphakdee,1,b)Yong-Bin Zhuang,1Ryuhei Okuno,1
Yuta Tsuboi,1So Takamoto,1Junichi Ishida,2and Ju Li3, 4
1)Preferred Networks Inc., Tokyo, Japan.
2)Matlantis Corporation, Tokyo, Japan.
3)Department of Materials Science and Engineering, Massachusetts Institute of
Technology, Cambridge, MA 02139, USA
4)Department of Nuclear Science and Engineering, Massachusetts Institute of
Technology, Cambridge, MA, USA
(Dated: 7 November 2025)
a)Electronic mail: wenwenli@preferred.jp
b)Electronic mail: nontawat@preferred.jp
i

CONTENTS
S1. Application 1: Simulation of interfacial structures of Pt
(111)/benzene iv
S1.1. Student fine-tuning iv
S1.2. Evaluation using PFP iv
S2. Application 2: Miscibility of water, benzene and heptane viii
S2.1. Student fine-tuning viii
S2.2. Large-scale MD simulation viii
S3. Application 3: Interface thermal resistance between Ni and
DPO/BP xii
S3.1. Student fine-tuning xii
S3.2. Evaluation using PFP xiii
S4. Application 4: Viscosity of n-decane xvi
S4.1. Student fine-tuning xvii
S4.2. Evaluation using PFP xvii
S4.3. Large-scale MD simulation xviii
S5. Application 5: Crack propagation in graphene nanoribbon xxi
S5.1. Student fine-tuning xxi
S5.2. Evaluation using PFP xxii
S6. Application 6: Friction of Fe 2O3surface with lubricant and fatty
acid surfactant xxv
S6.1. Student fine-tuning xxv
S6.2. Evaluation using PFP xxvi
S7. Application 7: Diffusion behavior in Polymer Ionic Liquid xxviii
S7.1. Student fine-tuning xxviii
S7.2. Evaluation using PFP xxix
S8. Application 8: Mechanical property of SiO 2–P2O5–Al2O3–Na2O
ii

glass xxxii
S8.1. Student fine-tuning xxxii
S8.2. Evaluation using PFP xxxiii
S9. Application 9: Heterogeneous grain boundary between FCC Cu and
BCC Mo xxxvi
S9.1. Student fine-tuning xxxvi
S9.2. Evaluation using PFP xxxvii
S10. Application 10: Micelle simulation xxxix
S10.1. Student fine-tuning xxxix
S10.2. Large-scale MD simulation xl
S11. Application 11: Chemical mechanical polishing of Si surface xliii
S11.1. Student fine-tuning xliii
S11.2. Evaluation using PFP xliv
S11.3. Large-scale MD simulation xliv
References xlvi
iii

S1. APPLICATION 1: SIMULATION OF INTERFACIAL STRUCTURES
OF PT (111)/BENZENE
The main purpose of this example is to demonstrate how to build and validate the
LightPFP model to simulate the solid-liquid interface using the Pt/benzene interface as
a model system. By leveraging LightPFP’s enhanced speed and accuracy, researchers can
gain insights into solid-liquid interface behavior and expand its applications in materials
science, chemistry, and nanotechnology.
S1.1. Student fine-tuning
To train the LightPFP model, we compiled a diverse set of structures with their energies,
forces, and stresses labeled using the PFP. The training dataset encompassed various com-
ponents: bulk Pt, Pt (111) slab, bulk benzene, and Pt (111)/benzene interface structures.
System Methods Num of Num of
structures Atoms
Pt Cell compression/Streching/Deforming
Atom Displacement
Vacancy
Surface and MD (NPT 500–1500 K)2249 111,290
Pt (111) MD (NPT 500–1000 K) 640 43,680
Benzene Cell compression/Streching
MD (NPT 300–400 K; NVT 500–1500 K)984 177,120
Interface MD (NPT 300–800 K; NVT 500–1750 K) 5400 2,632,800
The dataset was randomly split into training and testing datasets, with 90\% comprising
the training dataset and the remaining 10\% constituting the testing dataset.
S1.2. Evaluation using PFP
To validate the performance of the LightPFP models, we performed MD simulations on
a small Pt (111)/benzene interface system and compared the trajectories with PFP.
iv

FIG. S1: Parity plots for energy, forces, and stress on the test set: PFP ground-truth
values (x-axis) vs LightPFP model predictions (y-axis).
The initial simulation box has dimensions of 38.58Å×33.10Å×101.32Å, with cell angles
of90◦,90◦, and120◦. It consists of a total of 9,072 atoms, including 1,512 Pt atoms and
630 benzene molecules. We have chosen a relatively smaller structure in order to compare
the results with PFP.
The initial structure undergoes a 20 ps equilibrium at a temperature of 300.0 K at first,
using the NVT ensemble. Subsequently, NPT MD simulations are conducted for 100 ps at
300.0 K and 1 bar, utilizing the NPT ensemble. MD snapshots is saved for future analysis.
The density of the interface structure is estimated to 5.08 g/cm3from the NPT MD based
on LightPFP, which aligns well with the results from PFP MD trajectory, 5.06 g/cm3. To
calculate the radial distribution function, we used the snapshots of the MD trajectory taken
after 50 ps. The resulting radial distribution function is shown in Fig S2. We discovered
that the results obtained from both the LightPFP and PFP trajectories exhibit a high degree
of agreement.
To characterize the atomic distribution along the z axis (normal to the interface), we
sampled MD trajectory snapshots after 50 ps and generated a z-position density via Gaussian
broadening (width 0.25 Å). The resulting distribution is plotted in Fig. S3.
The density profile in Fig S3shows several peaks in the H and C elements near the Pt
surface (around 20 Å), indicating that the benzene structure is significantly different from
the uniform liquid phase due to adsorption with the Pt surface. As we move further away
from the Pt surface, the interfacial liquid structure gradually transitions towards a uniform
liquid phase. According to the figure, the thickness of the interfacial layer is approximately
v

FIG. S2: Comparison of radial distribution functions from LightPFP and PFP trajectories
15 Å.
The position and intensity of the peaks in the LightPFP model were compared with the
PFP result, and they matched each other across most regions. This alignment demonstrates
that the LightPFP model effectively captures and represents the structural characteristics
of the solid/liquid interface.
Symmetric peaks can be observed around z= 100 Å, indicating the presence of the same
liquid-solid interface due to the periodic boundary condition. However, these peaks exhibit
slight fluctuations in shape due to noise originating from cell shape changes in the NPT MD
simulation.
In Fig S4, we can observe the benzene molecules that have adsorbed onto the Pt sur-
vi

FIG. S3: Spatial distribution of Pt, C and O atoms along the z direction. (left) Whole
simulation box. (right) Zoomed-in view at the interface.
FIG. S4: The benzene adsorption on Pt surface (left) LightPFP (middle) PFP (right) The
bri30 adsorption site of the benzene molecule.
face. Specifically, in the LightPFP MD simulation, 18 benzene molecules were found to be
adsorbed onto the Pt surface within the specified area. In the PFP MD simulation, on the
other hand, there were a total of 21 benzene molecules observed to be adsorbed onto the Pt
surface within the same area. In conclusion, the LightPFP MD simulation reproduced the
surface coverage rate of benzene on the Pt surface well.
Fig.S4(right) illustrates the adsorption structure of a single benzene molecule. The bri30
conformation, in which the center of the benzene molecule is located on the bridge site of
the Pt surface, was found to be the most energetically stable in first-principles calculations1.
Interestingly, we observed that almost all the molecules were adsorbed in the bri30 site in
both the LightPFP and PFP simulations. This result is consistent with the findings of the
first-principle calculations.
vii

S2. APPLICATION 2: MISCIBILITY OF WATER, BENZENE AND
HEPTANE
In this example, we use LightPFP to investigate the miscibility of binary liquid mixtures
among water, benzene and heptane via large-scale molecular dynamics simulations using
LightPFP.
S2.1. Student fine-tuning
We trained LightPFP on a collection of datasets designed to cover both homogeneous and
demixed liquid environments. The initial dataset comprised nine classes of configurations:
(1) bulk water, (2) bulk benzene, (3) bulk heptane, (4) homogeneous water/benzene mix-
tures, (5) homogeneous water/heptane mixtures, (6) homogeneous heptane/benzene mix-
tures, and (7) explicit liquid–liquid interfaces for water/benzene, (8) water/heptane, and (9)
heptane/benzene. To efficiently obtain interfacial training data, we constructed liquid–liquid
interface geometries and sampled them by short MD, rather than relying on spontaneous
demixing from homogeneous starting states. The latter would require prohibitively long
trajectories to capture phase-separated configurations because phase separation proceeds
via slow nucleation, coarsening, and domain growth. By seeding and sampling interfacial
structures, the training set explicitly exposed the model to the distinct local environments
present at liquid–liquid boundaries.
Starting from this initial model, we performed active learning to further improve accuracy
and robustness. For each of the nine system types, we ran 20 ps MD simulations over
280-350 K, monitored model performance, and selectively augmented the training set with
configurations with high disagreement w.r.t. PFP. The resulting LightPFP model was then
used for the large-scale MD simulation.
S2.2. Large-scale MD simulation
Three systems are considered: (1) 17,280 water molecules with 4,320 benzene molecules;
(2) 17,280 water molecules with 2,160 heptane molecules; and (3) 4,320 benzene molecules
with 2,160 heptane molecules. A 1 ns molecular dynamics simulation was performed for each
system. According to experimental data, water and heptane, water and benzene are immis-
viii

FIG. S5: Surface area of liquid-liquid interface.
cible at room temperature, while heptane and benzene are miscible2For the water/benzene
and water/heptane mixtures, spontaneous liquid–liquid phase separation is clearly observed
in the MD trajectories, with the two immiscible components forming distinct phases. In
contrast, no phase separation is observed for the benzene/heptane mixture, as these liq-
uids are mutually miscible. The MD simulation results are all consistent with experimental
observations.
To quantify demixing, we analyzed the MD snapshots and computed the liquid–liquid in-
terfacial area using OVITO’s construct surface mesh modifier with the alpha-shape method3.
For each saved frame, the two species were identified, a triangulated interface was gener-
ated, and the total interfacial area was recorded. The time evolution of this area is plotted
in Fig. S5. For the water/benzene and water/heptane mixtures, the interfacial area drops
rapidly and then approaches a low, nearly steady value, indicating fast coarsening and
macroscopic phase separation. In contrast, for the benzene/heptane mixture, the interfacial
area remains essentially unchanged from the outset over the entire 1 ns window, consistent
with the absence of demixing. In addition, the experimental data show that the solubil-
ity of heptane in water is lower than that of benzene. In MD simulations, we also found
that the interface area decreases more rapidly in the heptane-water system, reflecting faster
nucleation and phase transition dynamics2,4. Figures S6,S7, and S8show representative
interface morphologies at different times, providing a visual corroboration of these trends:
the liquid–liquid interface in water/benzene and water/heptane smooths and recedes over
time, whereas no well-defined interface emerges in benzene/heptane.
ix

FIG. S6: Liquid-liquid interface between water and benzene
FIG. S7: Liquid-liquid interface between water and heptane
x

FIG. S8: Liquid-liquid interface between heptane and benzene
xi

S3. APPLICATION 3: INTERFACE THERMAL RESISTANCE BETWEEN
NI AND DPO/BP
In this example, we use LightPFP to quantify interfacial thermal transport between a Ni
(111) surface and the diphenyl oxide (DPO)/biphenyl (BP) eutectic heat-transfer fluid, a
widely used medium in parabolic trough concentrated solar power systems with a maximum
operating temperature near 400◦C. Our target property is the thermal conductivity across
the metal–fluid interface, which governs heat exchange efficiency in receiver tubes and heat
exchangers5.
S3.1. Student fine-tuning
We trained LightPFP on a dataset tailored to capture both bulk and interfacial envi-
ronments relevant to the Ni–DPO/BP system. The initial configurations encompassed four
classes: bulk Ni, a Ni (111) slab exposing the surface, bulk liquid DPO/BP, and explicit
Ni (111) and DPO/BP interfaces. Each class was sampled via short molecular dynamics
and light “rattle” perturbations to diversify local environments. MD sampling covered NVT
simulations at 500 K, 1000 K, and 1500 K and NPT simulations at 300 K, 400 K, and 500 K.
The high-temperature NVT trajectories were chosen to generate randomized, higher-energy
configurations that improve the robustness and stability of the trained model; in contrast,
the NPT sampling was restricted to lower temperatures to avoid unphysical density fluctu-
ations and cell distortions that can arise at very high temperatures under barostat control.
For bulk liquid DPO/BP and for Ni (111)-DPO/BP interfaces, we also manually created
multiple distinct atomic configurations prior to short MD “structure sampling.” This man-
ual seeding was used to enrich the distribution of molecular orientations in the training set,
because orientational rearrangements in the liquid and at the metal–liquid interface relax
slowly in MD, especially in the 300–500 K range.
Starting from this base model, we performed active learning to improve accuracy and
robustness over the temperature range of interest. For each configuration class, we ran
short MD trajectories spanning 300–500 K under complementary conditions: 5 ps NVT, 10
ps NPT, and 30 ps reverse non-equilibrium MD (rNEMD) to explicitly sample heat-flow
states. We also varied setup details (for example, MD temperature, slab and liquid layer
xii

FIG. S9: Density of Ni and DPO/BP interface structure at 300 K
thickness and the swapping interface rNEMD etc) to enrich coverage of interfacial motifs.
Configurations exhibiting high model disagreement were iteratively labeled and added to
the training set, and the query–retrain loop was repeated until validation metrics stabilized.
S3.2. Evaluation using PFP
We assessed the resulting LightPFP model against a reference PFP potential under
matched conditions. The model reproduces key equilibrium and non-equilibrium observ-
ables: system densities (Fig S9) and radial distribution functions (Fig. S10) for interfacial
systems, as well as steady-state temperature profiles obtained from reverse non-equilibrium
MD across Ni (111)-DPO/BP interfaces (Fig. S11). The close agreement with PFP across
these metrics supports the use of LightPFP for the interfacial thermal conductivity calcula-
tions reported below.
xiii

FIG. S10: Radial distribution function of Ni and DPO/BP interface structure at 300 K
xiv

FIG. S11: Temperature profile across Ni and DPO/BP interface
xv

S4. APPLICATION 4: VISCOSITY OF N-DECANE
In this study, we target the shear viscosity of liquid n-decane via reverse non-equilibrium
molecular dynamics (rNEMD) using the Müller–Plathe momentum-exchange scheme6. To be
more specific, a liquid is equilibrated at the target thermodynamic state and the simulation
box is partitioned into slabs along the gradient direction, with two slabs designated as
momentum source and sink. At fixed intervals, the particle with the largest positive flow-
direction velocity in the source slab and the particle with the most negative component in
the sink slab exchange their flow-direction velocities, imposing a constant momentum flux.
The spatially resolved velocity profile is accumulated and time-averaged; its linear region
away from the exchange slabs provides the velocity gradient. The imposed flux is computed
from the cumulative exchanged momentum divided by cross-sectional area and simulation
time (accounting for periodic shear planes). The shear viscosity is then η=J
dv/dz, where
Jis cumulative exchanged momentum, vis the velocity of atoms and zis atomic position
along z axis.
Important practical considerations accompany rNEMD. Because accessible simulation
times are limited, rNEMD often relies on velocity gradients far larger than those used in
experiments, which can alter molecular orientation (e.g., induce partial alignment along
the flow) and thereby modify the intrinsic shear response. To reduce the artificially im-
posed gradient, one can lower the exchange frequency or elongate the simulation box along
the gradient direction. Lowering the exchange frequency weakens the driving but requires
longer MD sampling to resolve the slope of the velocity profile accurately, particularly for
high-viscosity liquids. Increasing the box length similarly reduces the gradient at a given
momentum flux but raises computational cost. This makes uMLIP require longer time to
compute viscosities for many molecules, whereas the LightPFP method becomes more useful
due to its better computational efficiency. In this example, we selected n-decane as the case
study and trained its LightPFP model. We computed the viscosity of n-decane at high
temperatures under different computational conditions using both PFP and LightPFP as a
validation of LightPFP. Because viscosity is lower at high temperature, PFP also yields good
results. We then used LightPFP to compute the viscosity of n-decane at room temperature,
ultimately obtaining values that closely match experiment.
xvi

S4.1. Student fine-tuning
We trained LightPFP for n-decane with a dataset focused on bulk liquid environments.
Several initial liquid structures were generated with realistic densities to cover differnt
molecule orientations. The initial dataset combined short MD segments and “rattle” per-
turbations to diversify local configurations: NVT trajectories at 500 K, 1000 K, and 1500
K; NPT trajectories at 300 K, 400 K, 500 K, and 600 K; and rattle displacements are
sampled from normal distribution with standard deviation of 0.10 Å and 0.15 Å. As in
our other applications, high-temperature NVT sampling was used to generate randomized,
higher-energy configurations that improve robustness, whereas NPT sampling was focused
at moderate temperatures to avoid unphysical density excursions and cell distortions that
can occur under aggressive barostatting at very high temperatures.
Starting from this base model, we executed an active learning loop tailored to the viscosity
task. For bulk liquid n-decane, we ran short trajectories under complementary ensembles
and non-equilibrium driving: 5 ps NVT, 10–20 ps NPT, and 20–30 ps rNEMD. rNEMD
simulations followed the slab-based momentum-exchange approach, with systematic vari-
ation of setup details (swap interval, swap slab thickness, and cell aspect ratio) to probe
sensitivity and enrich configurational coverage. Configurations exhibiting large model dis-
agreement were labeled and added to the training set; the query–retrain cycle was repeated
until validation metrics stabilized.
S4.2. Evaluation using PFP
After obtaining the LightPFP model, we ran the rNEMD calculation using PFP and
LightPFP. The calculation method is as follows: First, the size of initial simulation box is
17.3x17.3x51.9 angstrom contains 48 n-decane molecules. Then, the MD is performed with
NVT ensemble for 5 ps at 400, 450 and 500 K, and then, NPT ensemble for 20 ps at the same
temperature, 1 bar to achieve equilibrium status. After that, the rNEMD is performed, the
simulation box is divided into 20 slabs. For each temperature, the momenta exchange is
performed for each 100 fs or 500 fs. The rNEMD is performed for 50 ps to achieve a stable
momenta profile across the simulation box.
The accuracy of LightPFP is evaluated by comparision several properties with PFP re-
xvii

TABLE S1: Density of n-decane at different temperatures
Temperature (K) PFP (g/cm3) LightPFP (g/cm3) Error (g/cm3)
400 0.689 0.675 0.014
450 0.647 0.650 0.003
500 0.611 0.617 0.006
sults. As shown in table S1at 400, 450 and 500 K is obtained from the NPT-MD part
of trajectory. The predicted liquid densities agree well with PFP with an average absolute
deviation of 0.0076 g/cm3. The radial distribution functions for n-decane also obtained from
the MD frames took from the NPT part. The result are plotted in Fig S12showing a close
agreement with PFP over the same temperature range.
At last, we compared the viscosity results calculated by LightPFP and PFP (Fig. S13(a)).
First, at 450 K, both PFP and LightPFP yield the same viscosity of 0.2 mPa ·s for exchange
intervals of 100 fs and 500 fs. All results are consistent with experimental data7at 446.75 K.
At 400 K, viscosities computed with a 500 fs momentum-exchange interval were higher than
those with a 100 fs interval for both PFP and LightPFP. Under both intervals, PFP and
LightPFP were in close agreement. The MD-derived viscosities spanned 0.23–0.51 mPa ·s
across the tested conditions, and the experimental value of 0.29 mPa ·s falls within this range.
S4.3. Large-scale MD simulation
After confirm the reliability of the LightPFP model by comparing the high-temperatures
viscosity with PFP and experimental results, we tried to run large-scale MD simulation to
get more accuract viscosity at low temperature. The simulation box size is 34.6 x 34.6 x
103.8 Angstrom, which contains 384 n-decane molecules. We run the MD simulation at 300,
350 and 400 K. The MD protocol is same as the above one, except we prolong the rNEMD
simulation to 1 ns, to get accurate statistic results. The results is shown in Fig. S13(b). The
viscosity is estimated to be 0.85 mPa s at 300 K which is very close to the experiment value,
0.83 mPa s.
xviii

FIG. S12: Radial distribution function of n-decane. (a) 400 K (b) 450 K and (c) 500 K
xix

FIG. S13: Viscosity of n-decane
xx

S5. APPLICATION 5: CRACK PROPAGATION IN GRAPHENE
NANORIBBON
In this example, we assess the capability of LightPFP to describe fracture propagation
in graphene nanoribbon via molecular dynamics8.
S5.1. Student fine-tuning
The training data were constructed to span two-dimensional carbon environments under
mechanical loading, from pristine structures to defect-containing systems. The initial dataset
included defect-free graphene sheets and graphene nanoribbons (GNRs) with armchair (AC)
and zigzag (ZZ) edges, as well as structures with a triangular hole that served as crack
initiators. To sample elastic responses broadly, we (a) performed NVT MD at 500 K, 1000
K, and 1500 K; (b) applied small homogeneous deformations: ±5\% strain to both diagonal
(uniaxial/biaxial) and off-diagonal (simple shear) components of the simulation cell. To
further diversify local environments, atomic positions were rattled with Gaussian noise of
0.1 Å standard deviation. No cracks were included in this initial dataset; thus, the base
model learned from pristine bonding environments across temperatures, strain states, and
edge types without explicit exposure to fracture.
Starting from this model, we carried out active learning to improve robustness in high-
strain regimes. Using the initial structures (graphene sheets and AC/ZZ GNRs), we ran
strain-controlled MD with a “deform extension” protocol that incrementally altered the cell
shape to increase strain every fixed number of MD steps. This procedure drives the sys-
tems into strongly non-linear regimes where bond stretching, bond angle distortions, and
incipient bond breaking occur. During these runs, we monitored model performance and se-
lectively augmented the training set with configurations exhibiting large errors or anomalous
forces/energies under increasing deformation. In addition, we included strained configura-
tions of ribbons with pre-introduced triangular holes to expose the model to local stress
concentration fields characteristic of crack tips and to the chemistry of bond scission in sp2
carbon. The resulting LightPFP model was then used for fracture simulations.
xxi

S5.2. Evaluation using PFP
For evaluation, we simulate crack initiation and growth in AC and ZZ GNRs at 300 K
by introducing a triangular hole on one ribbon edge to serve as a notch. Uniaxial tension
was applied along the ribbon axis (x direction), with the Green-Lagrange strain increased
at a rate of 10−5per femtosecond. Each trajectory was propagated for 50 ps, reaching a
total strain of 0.5. Identical protocols were run with both the original PFP potential and
LightPFP for direct comparison. The MD snapshot during the crack propagation process is
shown in Fig. S14.
Both PFP and Light predict crack initiation at about 6 ps (strain is 0.06) in AC GNR. For
ZZ GNR, the PFP shows crack initiation at about 6 ps while LightPFP is a little bit slower.
LightPFP reproduces the spatial pattern of bond breaking and the subsequent crack path
observed with PFP: bonds fail first near the notch where stress concentrates, and the crack
advances into the ribbon width under continued loading. The straight crack in reproduced
by LightPFP in both AC and ZZ GNR. The predicted morphology and sequence of fracture
events are consistent across edge types, with no spurious branching or unphysical healing
observed in LightPFP.
The stress–strain responses computed from the virial stress is plotted in Fig. S15. The
curve shows closely agreement between PFP and LightPFP over the entire loading history,
including the elastic state where stress and strain change linearly, the crack propagation
state where stress drops rapidly, and the state where stress is 0 after fracture.
Taken together, these results demonstrate that LightPFP, trained without explicit cracks
and refined via strain-driven active learning, accurately captures the initiation and prop-
agation of fractures in graphene nanoribbons. The agreement in crack onset time, crack
morphology, and stress–strain behavior indicates that LightPFP attains PFP-level fidelity
for fracture simulations while retaining its computational efficiency.
xxii

FIG. S14: Illustration of crack propagation in graphene nanoribbons. Each column
displays molecular dynamics snapshots at 5, 6, 7, 8, 9, 10, and 11 ps, corresponding to
strains of 0.05, 0.06, 0.07, 0.08, 0.09, 0.10, and 0.11, respectively. From top to bottom, the
four rows show PFP MD of AC GNR, LightPFP MD of AC GNR, PFP MD of ZZ GNR,
and LightPFP MD of ZZ GNR.
xxiii

FIG. S15: Strain stress curve of AC and ZZ GNR in crack propagation process
xxiv

S6. APPLICATION 6: FRICTION OF FE 2O3SURFACE WITH
LUBRICANT AND FATTY ACID SURFACTANT
In this example, we use LightPFP to investigate lubrication and shear responses at iron
oxide–organic interfaces by simulating friction between two Fe 2O3slabs separated by a mul-
tilayer film of fatty-acid surfactants and a squalane lubricant9. Specifically, the simulation
box comprises a five-layer stack: an Fe 2O3slab, a monolayer of stearic acid or oleic acid, a
squalane layer, a second monolayer of stearic or oleic acid, and a second Fe 2O3slab. Figure
S16shows the atomistic structure of this system. This configuration results in a complex
multilayer structure, making the collection of training data more challenging.
S6.1. Student fine-tuning
We trained LightPFP on a curated dataset comprising (1) bulk Fe 2O3crystals, (2) Fe 2O3
slabs, (3) bulk liquids of stearic acid, oleic acid, and squalane, (4) solid-liquid interfaces
between Fe 2O3and each of stearic acid, oleic acid, and squalane, and (5) liquid-liquid inter-
faces between squalane and stearic acid, and between squalane and oleic acid. We generated
diverse initial structures within each category and used them to sample training configura-
tions. Multiple distinct initial configurations were employed to enhance the diversity of the
training set.
To construct the initial dataset, we combined molecular-dynamics sampling with rattle
perturbations. We performed MD in both NVT and NPT ensembles. NVT sampling at
elevated temperatures (500, 1000, and 1500 K) broadened the range of conformations and
interfacial arrangements, whereas NPT sampling at 300, 400, and 500 K targeted thermody-
namic states relevant to friction simulations and yielded realistic organic-layer densities. To
improve robustness to rare distortions, we applied Gaussian-distributed displacements (stan-
dard deviations 0.10 and 0.15 Å) to the initial structures, generating physically plausible
yet diverse local environments.
Because friction can induce large molecular deformations and uncommon contact geome-
tries that are rare under equilibrium sampling, we augmented the model via active learning
focused on shear. In each iteration, we (i) fixed the bottom part of the system, (ii) imposed
a controlled lateral displacement on the top part to generate steady sliding, and (iii) evolved
xxv

FIG. S16: An atomistic structure of Fe 2O3- lubricant - surfactant system
the remaining atoms with NVT dynamics. Configurations exhibiting large energy/force dis-
crepancies were added to the training set, and the model was retrained. We conducted
10 active-learning rounds: iterations 1–5 used smaller interface cells containing both iron
oxide–molecule solid–liquid interfaces and surfactant–lubricant liquid–liquid interfaces to
rapidly accumulate diverse contact motifs, whereas iterations 6–10 employed larger cells to
capture cooperative rearrangements under shear.
S6.2. Evaluation using PFP
The final LightPFP model was used to simulate the full five-layer stack: Fe 2O3slab –
stearic acid – squalane film – stearic acid – Fe 2O3slab. We validated the model by performing
friction MD with both PFP and LightPFP. The initial system contained 4,604 atoms. Sliding
simulations were run for 300 ps with a top-slab velocity of 30 m/s, while the bottom slab
was fixed. The system temperature was maintained at 300 K using a thermostat. The same
sliding protocol was applied with PFP, and the resulting trajectories and final structures
were compared.
Figure S17shows the displacement of molecules over 300 ps of sliding. As the upper
Fe2O3slab moves, nearby molecules are entrained and translate in the sliding direction.
Both PFP and LightPFP predict that molecules adjacent to the moving Fe 2O3layer travel
by approximately 80 Å due to interfacial friction, with displacements that decay with depth
into the surfactant and lubricant layers. Figure S18depicts the morphology of the moving
Fe2O3slab and adjacent molecules in the final frame. Interfacial shear substantially stretches
xxvi

FIG. S17: Molecular displacements during 300 ps of sliding for the Fe2O3 / fatty acid /
squalane / fatty acid / Fe2O3 stack.
FIG. S18: Final-frame morphology after 300 ps of sliding.
the molecular chains and aligns them along the sliding direction. These results support the
ability of LightPFP to capture coupled solid–organic interfacial mechanics and shear-induced
ordering relevant to boundary lubrication.
xxvii

S7. APPLICATION 7: DIFFUSION BEHAVIOR IN POLYMER IONIC
LIQUID
In this example, we use LightPFP to investigate anion diffusion in a neat polymer ionic
liquid (PIL), taking poly (ethyl vinyl imidazolium) paired with PF 6–as a representative
system10. Poly (ethyl vinyl imidazolium) forms a positively charged polymer network, while
discrete PF 6–anions occupy the interstitial regions of the polymer matrix, collectively giving
rise to the characteristic ion-conducting behavior of PILs.
S7.1. Student fine-tuning
We curated a training set to expose LightPFP to the relevant local environments of the
target PIL across a broad range of densities, temperatures, and chain conformations. To ef-
ficiently cover conformational diversity, we generated multiple initial structures rather than
relying on a single long trajectory: starting from distinct packings accelerates exploration
of chain orientations, local packing motifs, and ion coordination states. Structural assembly
was performed with a dedicated routine that constructs mixed-oligomer boxes comprising
monomers, dimers, trimers, 5-mers, and 7-mers of poly (ethyl vinyl imidazolium), together
with the stoichiometric number of PF 6–anions to satisfy charge neutrality. Molecules were
placed with randomized positions and orientations to maximize initial configurational diver-
sity.
Initial data generation combined short molecular dynamics (MD) sampling and stochas-
tic perturbations. MD trajectories were run in both NVT and NPT ensembles. For NVT,
we used elevated temperatures (500 K, 1000 K, 1500 K) to accelerate configurational de-
correlation and broaden the coverage of local structures. For NPT, we sampled temperatures
(400 K, 500 K, 600 K) representative of the intended application window to capture real-
istic densities and coordination statistics. In addition, a rattle procedure applied Gaussian
displacements (standard deviations of 0.10 and 0.15 Å) to further diversify local atomic
environments and improve robustness. The initial LightPFP model is trained from this
dataset.
Starting from this model, we carried out active learning to enhance accuracy and stability
under production conditions. Each active-learning iteration consisted of multiple short MD
xxviii

TABLE S2: Density of polymer ionic liquid
Temperature (K) PFP (g/cm3) LightPFP (g/cm3)
300 K 1.394 1.389
400 K 1.374 1.372
500 K 1.351 1.356
jobs, each seeded from a newly generated structure via the above mentioned procedure. For
each job, we randomly selected a temperature between 300 and 700 K, ran a 5 ps NVT
trajectory followed by a 50 ps NPT trajectory, and attempted data collection every 100 MD
steps. Configurations identified as poorly described by the current model were labeled and
added to the training set, after which the model was retrained.
S7.2. Evaluation using PFP
To validate LightPFP for the target PIL, we performed MD comparisons against PFP.
First, we generated reference PFP trajectories at 400, 500, and 600 K with NPT ensemble.
Using identical initial configurations, ensemble settings, and integrator parameters, we then
repeated the simulations with LightPFP and saved all trajectories for post hoc analysis.
We computed equilibrium densities from the last 10 ps NPT trajectories. The densities at
300 K, 400 K, and 500 K listed in Table S2. LightPFP densities closely track PFP across this
range. The radial distribution functions (RDFs), which were accumulated over the last 10 ps
of each trajectory at 300 K, 400 K, and 500 K, are shown in Fig. S19. LightPFP reproduces
the positions and heights of the PFP peak, indicating consistent local coordination and
packing.
To probe anion mobility, we monitored phosphorus atoms (proxies for PF 6–) and com-
puted mean squared displacement (MSD), as shown in Fig. S20. MSD curves from LightPFP
agree well with those from PFP, indicating consistent diffusive behavior in the polymer ma-
trix. In addition, the diffusion coefficients D(T) were extracted from the MSD and fitted
to an Arrhenius form to estimate the activation energy for PF 6–transport. The activation
energy from LightPFP is 10.46 kJ/mol, in close agreement with the PFP value of 10.82
kJ/mol. We note that the limited number of temperatures constrains the precision of these
xxix

FIG. S19: Radial distribution function of polymer ionic liquid
xxx

FIG. S20: Diffusion behavior of PF 6–anion (left) means squared displacement of PF 6–
anion at 300 K, 400 K and 500 K (right) Arrhenius plot of PF 6–anion diffusion coefficient
estimates; the results are presented primarily to demonstrate consistency between the two
models.
xxxi

S8. APPLICATION 8: MECHANICAL PROPERTY OF
SIO2–P2O5–AL 2O3–NA 2O GLASS
In this example, we use LightPFP to investigate composition–property relationships and
the mechanical response of multicomponent oxide glasses in the SiO 2–P2O5–Al2O3–Na2O
system, as a stringent test on amorphous materials11. We consider glasses with composi-
tions SiO 2:(79.69-x) mol\%, P 2O5:x mol\%, Al 2O3:13.79 mol\%, and Na 2O:15.52 mol\%, with
x ranging from 0 to 50 mol\%.
S8.1. Student fine-tuning
To build the initial training set, we combined crystalline and disordered configurations.
First, crystal structures of SiO 2, P2O5, Al2O3, and Na 2O were obtained from the Materials
Project to capture characteristic, energetically stable local environments (e.g., tetrahedral
Si). Second, random packings were generated by placing atoms uniformly at random in
periodic simulation boxes subject to a minimum interatomic separation, avoiding atomic
overlap while spanning a broad space of disordered motifs. From these crystalline and
random initial structures, we performed NPT MD at 500 K, 1000 K, and 1500 K, and
sampled uncorrelated snapshots across densities and coordination states to train the initial
LightPFP model.
Active learning was then used to refine accuracy for glassy states. For each composition,
we initiated melt–quench protocols from random packings: (i) NVT MD at 2000 K for
10 ps to fully melt and randomize the network; (ii) linear cooling from 2000 K to 500 K
at 0.1 K/fs to form the glass; and (iii) additional MD at 500 K to relax the structure.
During these runs, we monitored model reliability and selectively augmented the training
set with configurations exhibiting large force/energy discrepancies relative to the reference
PFP, iterating retraining until convergence. This pipeline exposes the model to topological
rearrangements (bond breaking/formation, modifier-induced non-bridging oxygens) and the
broad spectrum of local environments that emerge during melt–quench.
xxxii

FIG. S21: Density of (82-x)SiO 2-xP2O5-16Al 2O3-18Na 2O glass
S8.2. Evaluation using PFP
For evaluation, we generated glass structures at five representative compositions, i.e. (82-
x)SiO 2–xP2O5–16Al 2O3–18Na 2O, where x = 0, 4, 8, 22, 46. Identical melt–quench protocols
were performed with both the PFP and LightPFP. Specifically, each system was equilibrated
in the melt and then quenched at 0.02 K/fs to the 500K, followed by NPT relaxation to
determine the density. The resulting densities are plotted in Fig. S21. In both PFP and
LightPFP simulations, the density increases with SiO 2content. The mean absolute error
(MAE) of the LightPFP-predicted density across the five compositions is 0.014 g/cm3.
In addition, we evaluated the glass structure using the radial distribution function (RDF).
For brevity, Fig. S22shows the RDF for the composition 74SiO 2–16Al 2O3–18Na 2O–8P 2O5.
As shown, LightPFP is in good agreement with PFP across the principal pair correlations.
To assess mechanical response, we computed the elastic stiffness tensor for the repre-
sentative glass 74SiO 2–16Al 2O3–18Na 2O–8P 2O5with both PFP and LightPFP. Results are
listed in Table S3, including major components of elastic tensor and derived bulk, shear,
and Young’s moduli, as well as Poisson’s ratio.
These results indicate that LightPFP can robustly learn and transfer the structural and
mechanical behavior of complex multicomponent oxide glasses from PFP, while accommo-
dating substantial composition variation from silica-rich to phosphate-rich regimes.
xxxiii

FIG. S22: Radial distribution function of 74SiO 2-16Al 2O3-18Na 2O-8P 2O5
xxxiv

TABLE S3: Elastic properties of 74SiO 2-16Al 2O3-18Na 2O-8P 2O5
LightPFP PFP err rel err
C11 53.312 55.138 1.825 0.033
C12 18.077 15.643 2.434 0.156
C13 18.169 15.699 2.470 0.157
C22 58.004 56.435 1.570 0.0278
C23 19.364 16.745 2.619 0.156
C33 58.283 58.618 0.335 0.00571
C44 19.709 20.521 0.812 0.0396
C55 19.283 20.065 0.782 0.0390
C66 18.493 19.606 1.113 0.0568
bulk modulus 31.163 29.580 1.583 0.0535
shear modulus 19.081 20.168 1.087 0.0539
Young’s modulus 47.540 49.299 1.759 0.0357
Poisson ratio 0.246 0.222 0.0235 0.106
xxxv

S9. APPLICATION 9: HETEROGENEOUS GRAIN BOUNDARY
BETWEEN FCC CU AND BCC MO
In this example, we use LightPFP to investigate the energetics of grain boundaries be-
tween FCC Cu and BCC Mo via large-scale atomistic simulations.
S9.1. Student fine-tuning
We constructed initial Cu/Mo bicrystals using the cut-and-concatenate method, starting
from ideal FCC Cu and BCC Mo crystals. The method is shown in Fig. S23. Simply
speaking, the method takes two input crystal structures (i.e., Cu and Mo), cuts out cubic
fragments at arbitrary positions and orientations, and then stitches them together to form
grain-boundary–like structures. Unlike predefined low-energy grain boundaries structures,
the generated structures are highly diverse. They often exhibit large lattice mismatch, and
because the original periodic order is disrupted, numerous defects are introduced. This is not
a drawback; rather, it is advantageous for training MLIPs, as it provides more off-equilibrium
data samples and thereby improves the stability and robustness of the MLIP.
The initial database was assembled via short MD sampling on these crystal and grain
boundary structures. Specifically, we performed NVT trajectories at 500 K, 1000 K, and
1500 K to introduce thermal disorder and local reconstructions, and NPT trajectories at
300 K, 400 K, 500 K, and 600 K to allow relaxation and sampling of strain-accommodated
interfacial structures. The initial LightPFP model is trained on this dataset.
Starting from this initial model, we performed active learning to improve accuracy and
robustness. For one initial Cu/Mo bicrystal structure generated by cut-and-conatenate
method, we ran MD from 300 K to 1000 K using a two-step cycle: (1) 2 ps in the NVT
ensemble to enable short-time reconstructions, followed by (2) 5 ps in the NPT ensemble to
capture stress relaxation and incipient structural transformations. After that, configurations
were relaxed by geometry optimization. Frames identified as low accuracy were added to
the training set and the model was retrained. Iterating this procedure produced a LightPFP
model that faithfully describes bulk phases and interface structure.
xxxvi

FIG. S23: Illustration of cut-and-concatenate method
FIG. S24: Comparison of grain boundary energy of 1000 Cu/Mo interface structures
calculated by PFP and LightPFP
S9.2. Evaluation using PFP
For evaluation, we generated 1000 low-strain CSL Cu/Mo grain-boundary structures
using pymatgen, covering a broad range of misorientations and in-plane shifts. For each
structure, we performed structure optimization first, and then computed the grain-boundary
energy with both the reference PFP model and LightPFP. The grain-boundary energy was
obtained by subtracting appropriate bulk reference energies for FCC Cu and BCC Mo from
the total energy of the bicrystal and normalizing by the interfacial area (accounting for the
two interfaces in periodic slabs). The resulting parity plot is shown in Fig. S24. Across the
1000-member test set, LightPFP achieves good agreement with PFP for the vast majority
of boundaries.
xxxvii

Overall, these results demonstrate that LightPFP can accurately predict Cu/Mo grain-
boundary energies over a broad structural space while matching the reference PFP model
for most configurations, validating its use for high-throughput screening and large-scale
interfacial simulations.
xxxviii

S10. APPLICATION 10: MICELLE SIMULATION
In this example, we investigated micelle formation using LightPFP. As is well known, this
process entails dynamic changes in surfactant structures. The hydrophilic and hydrophobic
groups of each surfactant molecule interact; the hydrophilic groups orient outward toward
the water solvent, while the hydrophobic groups orient inward, shielding themselves from
the water. This mechanism is generally simulated using classical force fields (FFs) because
it typically involves large system sizes. However, the parameterization of classical FFs is
very demanding, and capturing chemical interactions, such as bond formation and disso-
ciation, is difficult. Furthermore, their accuracy is, of course, lower than that of Density
Functional Theory (DFT) calculations. To enable semi-DFT accuracy for large-scale sim-
ulations, which are intractable for DFT or even PFP, we employed LightPFP to perform
dynamical simulations, potentially involving chemical reactions, in larger systems.
To test the capability of LightPFP, we selected a gemini surfactant as a test material.
This surfactant type features dual hydrophobic tails and dual cationic heads. The cationic
heads interact with bromide counter-ions (Br−). It is known that 12-s-12 systems (where ’s’
is the spacer length) tend to form micelles rapidly due to their dual structure (see illustrative
structure in Fig. S25).
S10.1. Student fine-tuning
We prepared the starting structures from the SMILES expression of 12-6-12 system. We
varied the value of s (the spacer length separating the two tails) and generated various
structures with different numbers of surfactant molecules immersed in water solvent using
the LiquidGenerator12function. Using a structure sampling protocol that combined NVT
and NPT MD simulations with rattle sampling, we generated the initial dataset. The PFP
v8.0.0 calculator with the R2SCAN mode was used as the reference.
Next, we performed active learning. We employed the specific-type organic pre-trained
model as the pre-trained model, expecting faster calculations than larger models. The ac-
tive learning consisted of three stages: in the early stage (iterations 0–4), we tested smaller
systems; in the middle stage (iterations 5–8), medium-sized systems; and in the final stage
(iterations 9–12), we used the largest systems as input structures. For each active learn-
xxxix

ing iteration, we defined internal loops, varying the number of surfactant molecules, water
molecules, and the temperature range (300 K to 500 K) to ensure structural diversity. The
MD simulation protocols included NVT (Langevin thermostat) and NPT (Nosé–Hoover
thermostat/barostat) ensembles. After the active learning, we trained the model using the
entire dataset collected for a longer duration to obtain the final LightPFP model for pro-
duction calculations. The entire process took about half a day. We repeated this process
for each surfactant with different spacer lengths s(s= 2,6,10).
S10.2. Large-scale MD simulation
In the production runs, we performed 2 ns NPT simulations at 350 K for s = 2, 6, 10.
Fig.S26shows the number of clusters in the system. In the initial snapshot, this number
equals the total number of surfactant molecules. As they aggregate to form micelles, the
number of clusters decreases. As shown in the figure, the cluster count gradually decreases
over time, and finally, a large cluster is formed in each case. This clearly represents the initial
phase of micelle formation, and LightPFP successfully captures the surfactant aggregation
process. Fig. S27shows the initial and final structures for each system. Consistent with
the cluster analysis, large clusters were formed. Interestingly, the hydrophilic portions are
oriented outward while the hydrophobic tails are oriented inward, as expected from the
electrostatic nature of surfactants.
Although we acknowledge that 2 ns is not sufficient for the complete formation of micelles,
it is valuable to observe the formation process at a semi-DFT level of accuracy. To our
knowledge, this is the first time micelle formation has been simulated using a machine
learning potential (MLIP). Finally, we note that the dataset generated using R2SCAN mode
was crucial for this type of simulation. We also tested the PFP’s PBE+D3 mode, but in
that case, the simulation became unstable, eventually leading to system breakdown. We
attribute this behavior to the inaccurate description of water by the PBE-level functional.
It is known that properties like the radial distribution function (RDF), viscosity, and density
of water are better described by the R2SCAN mode; thus, the superior description of water
likely contributes to the stable MD simulation results.
xl

FIG. S25: Structure of 12- s-12 gemini surfactant with dual hydrophobic tails, dual cationic
heads, and bromide counter-ions.
FIG. S26: Evolution of the number of clusters over time during 2 ns NPT simulations at
350 K for surfactants with spacer lengths s= 2, 6, and 10.
xli

FIG. S27: Snapshots from MD simulations showing initial and final structures for systems
with spacer lengths s= 2, 6, and 10, demonstrating micelle formation.
xlii

S11. APPLICATION 11: CHEMICAL MECHANICAL POLISHING OF SI
SURFACE
To illustrate the practical application of LightPFP, we present an example reproducing
abrasive rolling during the chemical mechanical polishing (CMP) of crystalline silicon by
a silica particle13. The simulation box consists of a Si(100) slab and a spherical SiO 2par-
ticle placed above the slab. The simulation protocol follows the two-stage loading scheme
commonly employed in molecular dynamics (MD) studies of CMP: first, an external normal
(“down”) load is applied to the silica particle to bring it into contact with the silicon surface;
then, a tangential (“driving”) load is applied to the same particle to induce rolling and sliding
motion across the slab.
S11.1. Student fine-tuning
The system construction and initial training dataset are intentionally minimal. Crys-
talline Si and SiO 2structures are generated, from which spherical SiO 2clusters are cut.
Si and SiO 2slabs are then prepared with vacuum layers, and a representative solid/solid
interface is assembled through the simple cut-and-concatenate procedure introduced in Sec-
tionS9. To populate the initial dataset, we sample crystalline structures using MD, rattle,
compression, deformation, and vacancy methods, while non-crystalline and interfacial struc-
tures are sampled using MD and rattle only. This combination of bulk, surface, cluster,
and interface configurations provides chemically diverse yet computationally inexpensive
coverage for the first LightPFP fitting.
The active-learning protocol is used to generate a more robust LightPFP model. Active
learning run the same MD protocol as described above. The structure with large discrepancy
with PFP will be detected and collected for further training. The active cycle is organized
as a simple curriculum across particle sizes (small →medium →large), so that the model
learns contact physics at increasing mechanical intensity. After the active acquisition phase
the accumulated dataset is used for a single, consolidated retraining step to produce the
final LightPFP model.
xliii

S11.2. Evaluation using PFP
For validation, we compare LightPFP with a PFP reference model using a small simula-
tion cell containing 6133 atoms, following the same two-stage loading protocol. Four loading
conditions were tested with down/driving forces of (5/10 eV/Å), (10/10 eV/Å), (5/20 eV/Å),
and (10/20 eV/Å), each simulated for 200 ps. The number of removed atoms was deter-
mined by counting those displaced more than 2 Å during the MD trajectory, as shown in
Figure S28. Both LightPFP and PFP exhibit consistent trends, with the number of removed
atoms increasing in the order: (10/10) < (5/10) < (5/20) < (10/20). However, LightPFP
consistently predicts a higher number of removed atoms than PFP under all loading con-
ditions. This difference likely arises from subtle variations in surface interaction strength.
Despite this quantitative deviation, the qualitative agreement in trend demonstrates that
LightPFP accurately reproduces the underlying contact and material removal physics while
offering improved sensitivity to interfacial dynamics.
S11.3. Large-scale MD simulation
Finally, we demonstrate scalability by simulating the polishing of Si by a 5 nm-diameter
silica particle in a dry environment. This production system comprises 59,266 atoms and is
evolved for 0.6 ns under a 50 eV/Å normal load and a 100 eV/Å driving load. The snapshots
of the MD trajectory is presented in Figure S29, showing the motion of the SiO 2particle
and the corresponding polishing process.
xliv

FIG. S28: Number of removed Si atoms from Si slab during MD simulation of chemical
mechanical polishing. The down/driving force applied on particle is annotated.
FIG. S29: Time evolution of the molecular dynamics simulation illustrating the chemical
mechanical polishing (CMP) of crystalline Si by a 5 nm SiO 2particle. Panels (a)–(d)
correspond to 0, 100, 200, and 300 ps, respectively, showing progressive rolling motion, and
surface atom removal on the Si slab.
xlv

REFERENCES
1C. Morin, D. Simon, and P. Sautet, The Journal of Physical Chemistry B 108, 5653 (2004).
2National Institute of Standards and Technology (NIST) and International Union of Pure
and Applied Chemistry (IUPAC), “Iupac-nist solubility database,” Accessed: 2025-10-13.
3A. Stukowski, Jom 66, 399 (2014).
4National Center for Biotechnology Information, “Pubchem compound summary for hep-
tane,” PubChem, section: Solubility. Accessed 2025-10-13.
5I. Carrillo-Berdugo, J. Navas, and R. Grau-Crespo, The Journal of Chemical Physics 160
(2024).
6B. Hess, The Journal of chemical physics 116, 209 (2002).
7“Nist chemistry webbook,” (2025), entry: n-Decane (CAS RN 124-18-5).
8G. S. Jung, H. Myung, and S. Irle, Machine Learning: Science and Technology 4, 035001
(2023).
9M. Doig, C. P. Warrens, and P. J. Camp, Langmuir 30, 186 (2014).
10X. Luo, H. Liu, and S. J. Paddison, ACS Applied Polymer Materials 3, 141 (2020).
11Y. Qian, B. Song, J. Jin, G. I. Prayogo, K. Utimula, K. Nakano, R. Maezono, K. Hongo,
and G. Zhao, Journal of the American Ceramic Society 105, 6604 (2022).
12Matlantis Contrib package, available at https://github.com/matlantis/
matlantis-contrib .
13L. Si, D. Guo, J. Luo, X. Lu, and G. Xie, Journal of Applied Physics 109(2011).
xlvi

\end{document}
