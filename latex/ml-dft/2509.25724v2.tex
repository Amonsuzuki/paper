\documentclass[11pt,a4paper]{article}
\usepackage[utf8]{inputenc}
\usepackage[T1]{fontenc}
\usepackage{amsmath}
\usepackage{amsfonts}
\usepackage{amssymb}
\usepackage{amsthm}
\usepackage{graphicx}
\usepackage{hyperref}
\usepackage{cite}
\usepackage{natbib}
\usepackage{geometry}
\usepackage{booktabs}
\usepackage{multirow}
\usepackage{array}
\usepackage{caption}
\usepackage{subcaption}
\usepackage{float}
\usepackage{url}
\geometry{margin=1in}

\title{Towards A Universally Transferable Acceleration Method for Density Functional Theory}
\author{Zhe Liu; Yuyan Ni; Zhichen Pu; Qiming Sun; Siyuan Liu; Wen Yan}
\date{\today}

\begin{document}
\maketitle

\begin{abstract}
Recently, sophisticated deep learning-based approaches have been developed for generating efficient
initial guesses to accelerate the convergence of density functional theory (DFT) calculations. While
the actual initial guesses are often density matrices (DM), quantities that can convert into density
matrices also qualify as alternative forms of initial guesses. Hence, existing works mostly rely
on the prediction of the Hamiltonian matrix for obtaining high-quality initial guesses. However,
the Hamiltonian matrix is both numerically difficult to predict and intrinsically non-transferable,
hindering the application of such models in real scenarios. In light of this, we propose a method
that constructs DFT initial guesses by predicting the electron density in a compact auxiliary basis
representation using E(3)-equivariant neural networks. Trained on small molecules with up to 20
atoms, our model is able to achieve an average 33.3\% self-consistent field (SCF) step reduction
on systems up to 60 atoms, substantially outperforming Hamiltonian-centric and DM-centric
models. Critically, this acceleration remains nearly constant with increasing system sizes and
exhibits strong transferring behaviors across orbital basis sets and exchange–correlation (XC)
functionals. To the best of our knowledge, this work represents the first and robust candidate for
a universally transferable DFT acceleration method. We are also releasing the SCFbench dataset
and its accompanying code to facilitate future research in this promising direction.
Correspondence:Qiming Sun at qiming.sun@bytedance.com , Siyuan Liu at siyuan.liu@bytedance.com
Project Page:The code and dataset will be released soon.
10 15 200.51.01.52.02.53.0
in-distributionin-distribution
26 30 40 50 60
out-of-distributionout-of-distribution
Number of AtomsRelative Iteration Count (RIC)RIC=Stepsw/PredictedInitialGuess
Stepsw/DefaultInitialGuess
(lower is better)
Slowdown
Speedup [ours]
H
D
Figure 1Comparison of deep learning-based DFT acceleration methods with different initial guess targets. The main
metric, Relative Iteration Count (RIC), measures the ratio of SCF iterations required with a deep learning initial
guess relative to a baseline. A smaller RIC means fewer SCF iterations required for convergence and is therefore
preferable. While the three models perform similarly on in-distribution (ID) systems, on out-of-distribution (OOD)
systems, our proposed method with electron density ( ρ) as the target performs significantly better than methods based
on Hamiltonian ( H) or density matrix ( D). More crucially, it shows a nearly constant scaling with increasing system
size, which is an ideal property for the task of DFT acceleration.
1arXiv:2509.25724v2 [physics.chem-ph] 15 Oct 2025
\end{abstract}

\section{Introduction
Density Functional Theory (DFT) [ 14,19,28] is a cornerstone of computational chemistry, offering a powerful
framework for predicting the electronic structure and properties of molecules. The most widely applied
algorithm for solving the DFT problem is the self-consistent field (SCF) method, an iterative process that
refines an initial guess for the density matrix until a converged solution is found. However, the iterative nature
of SCF can be computationally expensive, particularly for large systems, creating a significant bottleneck in
chemical discovery.
Machine learning (ML) offers a promising path to accelerate these calculations by providing a high-quality
initial guess for the SCF procedure, as illustrated in Figure 2. A popular approach is to train models to
predict the Hamiltonian matrix [ 25,44,45]. However, this strategy faces critical limitations, particularly for
the large molecules where acceleration is most needed. The poor performance stems from two distinct reasons.
First, even when trained on datasets containing large molecules, the approach is hampered by numerical
instability: small prediction errors in individual Hamiltonian matrix elements can be magnified into large,
physically nonsensical errors for the system as a whole [ 25]. Second, and more critically, the approach fails to
scale to molecules larger than those seen during training.
This lack of transferability is rooted in a funda-
mental limitation of the theory itself: the core
ansatz of Kohn-Sham DFT is that a real system
of interacting electrons can be represented by a
fictitious, non-interacting system that shares the
exact same electron density [ 19]. This makes the
electron density the fundamental physical observ-
able, rather than the Hamiltonian matrix. A key
consequence is that the electron density associ-
atedwithaspecificchemicalenvironmentishighly
transferable. The Hamiltonian matrix, however,
does not share this property; it contains matrix
elements for every pair of atoms in a molecule,
regardless of the distance separating them, mak-
ing its prediction sensitive to the molecule’s entire
global structure. This makes the Hamiltonian a
difficult target for extrapolating to larger, more
complex chemical environments.
Figure 2Top left: The Hamiltonian ( H), density
matrix ( D), and electron density ( ρ) are interdepen-
dent, so any of them can serve as an initial guess. Top
center: An SCF loop iteratively finds the ground state
from the given initial guess. Bottom: An ML model
predicts an initial guess from a molecular structure
to accelerate the SCF loop.
As an alternative, predicting density matrices for generating DFT initial guesses has been proposed [ 7,13,34].
However, this strategy is strongly basis-set dependent. In particular, when diffuse functions are included,
density matrix elements span a much larger numerical range, amplifying numerical uncertainties.
We argue that a more fundamental and transferable target for prediction is the electron density itself. Previous
works have attempted to predict the electron density on real-space grids [ 3]. However, such grid-based
predictions are not directly suitable for constructing an SCF initial guess, as most DFT functionals require
not only the density but also its gradients, which are not readily available from grid-based predictions alone.
Furthermore, even when similar ideas were proposed [ 9,12], a practical method for using the predicted density
to accelerate DFT calculations was never fully realized.
We propose a new paradigm that overcomes these limitations. We train a model to predict the expansion
coefficients of the electron density in a compact auxiliary basis and, crucially, demonstrate how to use
this prediction to construct a transferable initial guess for the SCF process. Our most significant finding
is that the electron density is a highly transferable and scalable property. The model we present can be
extended to molecules that contain significantly more atoms than those in the training set. On benchmark
tests involving all system sizes, our method achieves an substantial improvement over the Hamiltonian- and
density-matrix-based approaches (illustrated in Figure 1).
This electron-density-centric approach offers several additional advantages. First, from a practical standpoint,
2

processing density coefficients is significantly more efficient. The number of coefficients in an auxiliary basis
scales linearly with system size, whereas the Hamiltonian and density matrices scale quadratically. Second,
the electron density can possess lower symmetry ( L) than the Hamiltonian, which is particularly beneficial
for equivariant neural networks where the computational complexity of the tensor product scales as O(L6).
Finally, the local nature of electron density makes our approach highly data-efficient, requiring a smaller
training set to achieve high accuracy.
To facilitate further research in this direction, we introduce a new dataset SCFbench containing the electron
densities of molecules composed of up to seven different elements. We provide benchmark results for two
prominent ML architectures, demonstrating how our electron density prediction task can be seamlessly
integrated into existing models to accelerate quantum chemical calculations.
The main contributions of this work can be summarized as follows:
•We propose a new paradigm for DFT acceleration that targets the electron density—a more fundamental,
local, and data-efficient quantity—to provide a high-quality initial guess for SCF calculations. Specifically,
we take the efforts to implement the procedure for converting the electron density into an initial guess,
the absence of which was the direct cause of the under-development of this principled paradigm.
•We introduce SCFbench, the first public dataset of electron density coefficients specifically designed for
developing and benchmarking DFT acceleration methods.
•We systematically benchmark our electron-density-centric approach for both in-domain and transferring
settings on SCFbench. Results indicate that our approach shows remarkable transferability not only to
larger molecules but also across different exchange-correlation (XC) functionals and orbital basis sets.}

\section{Background}

2.1 Kohn-Sham DFT and the Self-Consistent Field Method
Kohn-Sham (KS) DFT provides a systematic framework to construct the energy functional E[ρ(r)]of a system
based on its electron density [ 28]. The electron density ρ(r)in KS-DFT is constructed from the density
matrixDand a set of basis functions\{ϕ µ(r)\}:
ρ(r) =X
µ,νDµνϕµ(r)ϕ ν(r).(1)
The density matrix is derived from the molecular orbital coefficientsC
Dµν=X
iCµiCνi.(2)
Minimization the total energy with respect to the orbital coefficients leads to a generalized eigenvalue equation:
H[ρ]C=SCϵ.(3)
Here,Sis the overlap matrix for the non-orthogonal basis functions, and ϵis the diagonal matrix of orbital
energies. The Kohn-Sham Hamiltonian matrix His an effective single-particle Hamiltonian. His composed
of three distinct terms:
H=H core+J+V xc.(4)
The core Hamiltonian (H core) is determined solely by the molecular geometry and basis set. The remaining
terms capture the electronic interactions: the Coulomb matrix ( J) for classical electron repulsion and the XC
matrix (V xc) for quantum mechanical effects.
A significant computational challenge arises from the fact that both JandVxcdepend on the density matrix
D, which in turn is constructed from the orbital coefficients C. This interdependence necessitates an iterative
procedure known as the SCF method [ 36]. One common approach to solving the SCF problem is to begin
with an initial guess for the density matrix, D. From D, an initial Hamiltonian His constructed. Solving
3

the eigenvalue problem yields new orbital coefficients C′, which are used to compute an updated density
matrix, D′. This cycle, D→H→C′→D′, is repeated until the density matrix converges and the solution
is deemed self-consistent. A common strategy for this initial guess, such as the default minaomethod in
PySCF [35], is the superposition of atomic densities (SAD) [23, 39].
2.2 Constructing the Kohn-Sham Matrix from Predicted Density
The key insight for our work lies in how the electronic terms are constructed from the predicted electron
density.
Using the density fitting [ 4] approximation, the electron density ρ(r)can be expanded in terms of an auxiliary
basis set\{χ k(r)\}with the expansion coefficientsc k:
ρ(r)≈˜ρ(r) =X
kckχk(r).(5)
These auxiliary basis functions are atom-centered. Typical auxiliary basis sets include def2-universal-jfit [ 42]
and the even-tempered basis (ETB) [ 1], parameterized by β. A smaller βyields a larger ETB basis. The size
of the auxiliary functions is typically three to five times that of the atomic orbital basis functions, which is
significantly smaller than the number of orbital pairs in HandD. In our approach, the auxiliary coefficients
ckare the primary quantities predicted using a machine learning model.
With the auxiliary basis expansion, both the electron density and its gradient can be directly evaluated. This
allows us to efficiently evaluate the XC matrix for generalized gradient approximation (GGA) functionals.
Additionally, the Coulomb matrix J, while formally dependent on the density matrix D, can be computed
efficiently from the coefficients\{c k\}using the density fitting approximation.
This feature makes the GGA framework particularly well-suited for our approach, as a machine learning
prediction of the density coefficients \{ck\}is sufficient to assemble the entire Kohn-Sham Hamiltonian matrix
H. With additional approximations, extensions to more complex functional types are possible. Explicit
formulas for constructing JandVxcfor general XC functionals from the auxiliary density are provided in
Appendix C.
Compared to computing Hdirectly from the full density matrix D, our approach introduces an approximation
toJandVxcvia the fitted density ˜ρ(r). However, The error in this approximation can be systematically
reduced by increasing the number of auxiliary basis functions used in the expansion (see Appendix A for an
illustrative example).
2.3 Equivariant Neural Networks
Physical properties of molecular systems are inherently independent of the choice of coordinate system. Under
spatial transformations such as rotations, translations, or reflections, quantities like energy and electron
density should transform accordingly, preserving their physical meaning. E(3)-equivariant neural networks
are specifically designed to respect these symmetries, where E(3) denotes the Euclidean group of all such
transformations [11, 20].
Formally, an equivariant modelΦsatisfies the following property: when the input atomic coordinates \{ri\}are
transformed by an operation g∈E (3), the output Otransforms according to a corresponding representation
D(g),
Φ(g· \{r i\}) =D(g)Φ(\{r i\})(6)
Here, D(g)is the appropriate representation for the output type. For scalar quantities such as total energy,
D(g)is the identity, reflecting invariance under transformation. For tensorial properties, such as electron
density coefficients in a spherical harmonics basis, D(g)corresponds to the Wigner D-matrix, which encodes the
rotation of these higher-order objects. Incorporating such symmetry constraints into the network architecture
via tensor product operations provides a strong inductive bias, improving generalization and data efficiency
for molecular property prediction.
4

\section{Related Work}

3.1 Hamiltonian Prediction
RecentworkshavedevelopedneuralnetworksfordirectpredictionoftheKohn-ShamHamiltonian. PhiSNet[ 38]
uses SE(3)-equivariant layers to reconstruct molecular wavefunctions and densities. QHNet [ 44] introduces
an efficient SE(3)-equivariant graph network for Hamiltonian prediction with reduced tensor operations.
SPHNet [ 26] incorporates adaptive sparsity into equivariant networks. QHFlow [ 18] employs high-order
equivariant flow matching to generate Hamiltonians conditioned on molecular geometry. The most scalable
Hamiltonian model to date is proposed by Li et al. [25], which introduces the Wavefunction Alignment Loss
(WALoss) to enable Hamiltonian prediction for large molecules and significantly improve the derived energy
compared to previous Hamiltonian models. However, its accuracy for energy prediction remains much lower
than that of direct energy models.
Other related Hamiltonian prediction works include Li et al. [24], Tang et al. [37], Zhang et al. [47].
3.2 Density Matrix Prediction
Recent studies have begun to explore direct prediction of the density matrix. The works of Shao et al. [34]
and Hazra et al. [13]applied kernel-based methods. In contrast, Febrer et al. [7]utilized an equivariant neural
network, focusing on small molecules with a small numerical atomic orbitals basis set. While these approaches
have advanced the field, density matrix prediction still faces challenges with transferability and scalability:
the density matrix elements are highly sensitive to the choice of basis set, which can limit generalization
across chemical systems.
3.3 Electron Density Prediction
ML prediction of electron density has been widely studied [ 3,5,6,8,15,22,31,41]. Early efforts typically
represented the density on real-space grids, which introduced redundancy and high computational cost. In
contrast, representing the density with one-center auxiliary functions provides a more efficient representation
while maintaining good accuracy [ 12]. Fu et al. [9]introduced an accurate and efficient model, SCDP, for
predicting electron density on real-space grids using even-tempered Gaussian functions as auxiliary basis sets,
augmented with off-center virtual orbitals. However, due to the common lack of support of using electron
density as an initial guess in quantum chemistry software, none of these works have explored how the predicted
density could be leveraged to accelerate DFT calculations.
3.4 Public Hamiltonian Datasets
Several publicly available datasets provide Hamiltonian matrices for molecular systems and are closely related
to this work. MD17 [ 33] contains Hamiltonians for thousands of structures of four small molecules, computed
with the def2-SVP basis set and PBE functional. QH9 [ 45] extends the QM9 [ 32] dataset with over 130,000
stable geometries and molecular dynamics trajectories, providing precise Hamiltonians and open-source
benchmarks for model development. The nablaDFT [ 16] and its extension, ∇2DFT [17], offer a large collection
of drug-like molecules with millions of conformations and associated quantum chemistry properties, including
Hamiltonians and geometry optimization trajectories. Other related datasets include QCML [ 10], which uses
numerical atomic orbitals, and PubChemQH [25], which is not yet publicly released.

\section{The SCFbench Dataset}

To support research on scalable and transferable initial guess methods, we introduce the SCFbench dataset.
It was constructed by applying a fragmentation procedure, similar to that of Zheng et al. [48], to drug-like
molecules within the ChEMBL database [ 46]. A subset of fragments containing 20 atoms or less was selected,
covering the elements H, C, N, O, F, P, and S. SCFbench dataset has several features designed to evaluate
transferability and scalability.
In addition to Hamiltonians and density matrices, SCFbench provides electron density expansion coefficients for
three distinct auxiliary basis sets: the computationally efficient def2-universal-jfit [ 42], and two even-tempered
5

Figure 3Statistical analysis of the SCFbench dataset. (a) Distribution of molecule sizes, (b) Proportion of molecules
containing each individual element (H, C, N, O, F, P, S) and element pair present in the dataset, and (c) decomposition
of electron density into irreducible representations (irreps) over the auxiliary basis sets.
basis (ETB) sets built from def2-SVP with β= 2.0and β= 1.5[1]. The resulting dataset of 43,862 molecules
is randomly split into training, validation, and test sets with a ratio of 8:1:1. The test set is used for evaluating
the in-distribution performance of a model, so it is also called as the in-distribution (ID) test set. This dataset
is well-suited for the electron density prediction task and is designed to be lightweight and easily extensible.
A key feature of the dataset is its dedicated out-of-distribution (OOD) test set, designed to address the
challenge of system size transferability. This set comprises 1,050 molecules, consisting of 30 molecules for
each atom count from 26 to 60, allowing for the evaluation of models on systems significantly larger than the
training data. The OOD set also includes the number of SCF cycles required for each molecule.
The data was generated using the PBE functional [ 29], a pure GGA functional. This choice was made because,
as discussed in section 2, the Kohn-Sham Hamiltonian for GGA functionals can be constructed directly from
the electron density coefficients, making it a suitable framework for developing and testing our proposed
method. It is also worth noting that while SCF convergence is often more challenging for GGA functionals
than for hybrids [27, 30], they remain underrepresented in existing Hamiltonian datasets.
For the DFT calculations, we used the def2-SVP basis set [ 43], a [99, 590] atom grid, and an SCF energy
convergence tolerance of1×10−10.

\section{Methods}

5.1 Evaluation
The primary goal of our evaluation is to assess the practical value of using ML-predicted electron density
to accelerate SCF calculations. We evaluate the models on their ability to accelerate SCF convergence for
molecules both within the training distribution (ID) and for molecules significantly larger than those seen
during training (OOD). This directly tests the crucial property of size transferability. Furthermore, we assess
the robustness of the models by testing their transferability across different XC functionals and atomic orbital
basis sets.
Our main metric is the Relative Iteration Count (RIC), defined as the number of SCF cycles required for
convergence using the ML-predicted initial guess, normalized by the number of cycles required using the
standard SAD initialization ( minao). A lower RIC means greater acceleration. In addition, we define a
calculation as converged if it reaches the required PySCF default tolerance within 50 iterations. This allows us
to report a convergence rate, which is a crucial metric for practical usability. For the ID vs. OOD comparison,
we evaluate on a random 1,000-molecule subset of the test set and the full OOD test set, respectively. While
wall-clock time depends on implementation and hardware, the number of SCF cycles provides a deterministic
and hardware-independent measure of an initial guess’s quality. A comprehensive set of additional metrics is
provided in Appendix E.
6

5.2 Model Architectures
Instead of designing a new architecture from the ground up, we adapt two classical models—NequIP [ 2] and
QHNet [44]—by modifying their final prediction heads.
NequIPNequIP [ 2] is an E(3)-equivariant graph neural network that represents atomistic systems as graphs.
Its core operation is a symmetry-preserving convolution where messages, constructed via a tensor product
of neighbor features and a filter made of learnable radial functions and spherical harmonics, are passed
between atoms. This process iteratively refines each atom’s features, which are geometric tensors (irreducible
representations) of varying orders ( l). Originally, NequIP’s architecture concluded with a simple head that
processed scalar (l= 0) features to predict atomic energies.
QHNetQHNet [ 44] is an efficient SE(3)-equivariant model designed to predict quantum tensors. Its
architecture is distinguished by node-wise interaction layers that use an attention-like mechanism and a Norm
Gate that dynamically rescales higher-order tensor features. QHNet was originally designed with a large,
multi-stage prediction head that used a Tensor Expansion module to construct the final Hamiltonian matrix
from pair-wise atomic features.
Species-dependent Equivariant Prediction HeadWe replace the original prediction heads of both models with
a single, species-dependent equivariant linear layer. This simple layer directly maps the final node features
from the backbone to a new output feature, hi
out, which are the density coefficients containing irreducible
representations from order l= 0to l= 4. The layer’s weights are conditioned on the atomic species, allowing
the model to learn a distinct final mapping for each chemical element. For NequIP, this modification has a
negligible impact on the total parameter count. For QHNet, however, replacing its complex original head
results in a significant efficiency gain, with the final model retaining only about one-quarter of the original
parameters (see Table 1).
5.3 Training Procedure
All density coefficients models were trained by minimizing a composite loss function, L, calculated per atom.
This loss is the sum of the mean absolute error (MAE) and the root mean square error (RMSE) of the
coefficients, averaged over all atoms in the batch:
L=

\section{AAX}

a=11
NaNaX
i=1|ˆca,i−ca,i|!
+vuut1
AAX
a=11
NaNaX
i=1(ˆca,i−ca,i)2 (7)
where Ais the total number of atoms, Nais the number of coefficients for atom a, and ˆca,iandca,iare the
predicted and ground-truth coefficients. Other training details are described in Appendix D.

\section{Experiments}

In this section, we present benchmark results for the SCFbench dataset using the modified NequIP and
QHNet models, focusing on their performance in accelerating SCF calculations. The results are summarized
in Table 1 and illustrated in Figure 1.
6.1 SCF Acceleration with Predicted Density (ID and OOD)
The Hamiltonian prediction model exhibits significant limitations. While it achieves a low RIC on the in-
distribution test set (63.20\%), which is comparable to performance reported in other works [ 45], its performance
collapses on the OOD test set. The relative iteration count increases to 179.47\%. More alarmingly, the model
suffers from a non-convergence problem, failing to converge for over 2.5\% of the OOD molecules. Unlike
chemically-grounded methods like SAD, the ML model can produce unphysical initial guesses, especially for
larger molecules, leading to a failure of the SCF procedure.
7

Table 1Results on the SCFbench benchmark dataset. The best results for each dataset are highlighted in bold and
the second bests are underlined. Best settings for each prediction target are marked in gray. Results for an extended
set of metrics are available in Appendix E.
Prediction Target Model \# Param.ID Test OOD Test
Convergence↑RIC↓Convergence↑RIC↓
HamiltonianGround Truth - 100\% 29.22\% 100\% 26.96\%
QHNet 20.5M 100\% 63.20\% 97.43\% 179.47\%
Density MatrixGround Truth - 100\% 27.57\% 100\% 26.62\%
QHNet 20.5M 100\% 70.45\% 99.71\% 91.69\%
Density Coefficients
def2-universal-jfitGround Truth - 100\% 62.80\% 100\% 60.45\%
QHNet 5.9M 100\% 66.90\% 100\% 73.26\%
NequIP-S 2.7M 100\% 74.90\% 100\% 89.46\%
NequIP-M 36.9M 100\% 64.82\% 100\% 69.10\%
NequIP-L 50.0M 100\% 63.78\% 100\% 66.68\%
Density Coefficients
ETB,β= 2.0Ground Truth - 100\% 58.96\% 100\% 55.05\%
QHNet 5.9M 100\% 68.62\% 100\% 79.36\%
NequIP-S 2.7M 100\% 82.20\% 100\% 93.28\%
NequIP-M 36.9M 100\% 67.31\% 100\% 78.39\%
NequIP-L 50.0M 100\% 62.48\% 100\% 70.42\%
Density Coefficients
ETB,β= 1.5Ground Truth - 100\% 43.26\% 100\% 39.66\%
QHNet 5.9M 100\% 78.05\% 100\% 82.76\%
NequIP-S 2.7M 100\% 89.80\% 99.24\% 127.16\%
NequIP-M 36.9M 100\% 76.82\% 99.62\% 108.17\%
NequIP-L 50.0M 100\% 68.85\% 99.90\% 81.33\%
Predicting the density matrix offers an improvement over the Hamiltonian but still falls short in transferability.
It achieves a solid RIC of 70.45\% relative iteration count on the ID set and maintains a high convergence
rate on the OOD set. However, its performance degrades on larger molecules, with the RIC increasing to
91.69\% for the OOD set. As illustrated in Figure 1, its performance clearly worsens as system size increases,
highlighting that the density matrix remains a challenging target for size transferability.
In stark contrast, our density-based models demonstrate excellent scalability. On the ID test set, the best
models (NequIP-L) achieve RICs of 62-64\%, nearing the theoretical limit imposed by the ground truth density.
Critically, this strong performance is maintained on the OOD test set. For the def2-universal-jfit basis, the
NequIP-L model’s acceleration is remarkably consistent, with an RIC of 63.78\% on the ID set and 66.68\%
on the OOD set, and it achieves a 100\% convergence rate across all tests. This remarkable consistency
proves that electron density is a highly transferable property, enabling models trained on small molecules
to effectively accelerate calculations for much larger ones. Further emphasizing this point, when we task
the QHNet electron density model with predicting electron density instead of the Hamiltonian matrix, its
RIC improves dramatically to 73.26\% on the OOD set, highlighting that the choice of a transferable physical
quantity is more critical than the specific model architecture.
Analyzing the ground truth results reveals the theoretical limits of this approach. The ground truth
Hamiltonian and Density Matrix provide the best possible initial guess, requiring only one SCF cycle in theory;
the remaining 27-29\% RICs are attributed to numerical precision differences. For density coefficients, the
potential acceleration depends on the expressiveness of the auxiliary basis, with larger bases like ETB ( β= 1.5)
offering greater potential for acceleration (a theoretical limit of ∼40\%) than the compact def2-universal-jfit
basis ( ∼60\%). Our ML models come very close to reaching this RIC limit for the def2-universal-jfit basis,
demonstrating the learnability of the task. The performance gap for the larger ETB basis sets, however,
highlights a promising avenue for future work. Improved model architectures could potentially capture more
8

information from these expressive bases, pushing the acceleration even closer to the theoretical limit.
6.2 Functional and Basis Set Transferability
Table 2Transferability of NequIP-L model across different functionals and basis sets. The model is trained on
PBE/def2-SVP and evaluated on various settings. RICs are reported for both ID and OOD sets.
Functional (Family: Name) Basis Set RIC (ID Test)↓RIC (OOD Test)↓
In-distribution setting
GGA: PBE def2-SVP 63.78\% 66.68\%
Transferring to different XC functionals
GGA: BLYP
def2-SVP71.38\% 71.22\%
meta-GGA: SCAN 88.15\% 86.45\%
Hybrid GGA: B3LYP5 84.63\% 83.72\%
Hybrid GGA: PBE0 85.99\% 85.51\%
Transferring to different atomic orbital basis sets
GGA: PBEdef2-TZVP 76.68\% 75.24\%
def2-TZVPPD 77.07\% 75.81\%
def2-QZVP 77.81\% 75.98\%
Transferring to different XC functionals AND basis sets
Hybrid GGA: B3LYP5 def2-TZVP 87.70\% 85.47\%
A key advantage of targeting electron density is its theoretical independence from the specific XC functional
and orbital basis set used in a calculation [ 19]. To test this in practice, we evaluate the transferability
of a single model—the NequIP-L model trained on PBE/def2-SVP with the def2-universal-jfit auxiliary
basis—across a range of different functionals and larger orbital basis sets. The results, presented in Table 2,
demonstrate the practical robustness of the density-based approach. While performance moderately degrades
compared to the original PBE/def2-SVP setting, the model still provides meaningful acceleration, particularly
for the OOD set, showcasing its utility in diverse computational chemistry workflows.
Notably, our model achieves an RIC of 85.47\% (OOD) on B3LYP5/def2-TZVP, where B3LYP5 refers to the
B3LYP hybrid functional [ 21] with the VWN5 correlation component [ 40]—matching the setup used in Li
et al.[25]. Despite training on a dataset consisting of much smaller molecules and with different functional
and basis set choices, our density-based approach delivers comparable acceleration performance for molecules
of similar size. This highlights the strong transferability and data efficiency of electron density prediction,
even when evaluated under conditions aligned with state-of-the-art Hamiltonian-based models.

\section{Conclusion}

By targeting the electron density, this work provides a practical and reliable solution to the long-standing
challenge of creating a scalable initial guess for SCF calculations. We have shown that a single model, trained
on a modest dataset of small molecules, can serve as a “drop-in” accelerator for a wide range of systems,
including those significantly larger than the training data, and across various functionals and basis sets. The
robustness of our method marks a steady step towards a universally applicable tool for the computational
chemistry community.
To facilitate further progress, we have released the SCFbench dataset, a comprehensive benchmark designed
to test these crucial aspects of transferability and scalability. Future work can build on this foundation in
several key directions. While our models approach the theoretical performance limit for compact auxiliary
basis sets, a gap remains for more expressive bases; developing more powerful neural network architectures
could close this gap and unlock even greater acceleration. Furthermore, extending the SCFbench dataset
to include a wider range of the periodic table and periodic systems will be vital for pushing this promising
method towards true universality.

\section{References}

[1]Richard D. Bardo and Klaus Ruedenberg. Even-tempered atomic orbitals. VI. Optimal orbital exponents and
optimal contractions of Gaussian primitives for hydrogen, carbon, and oxygen in molecules. TheJournalof
Chemical Physics, 60(3):918–931, February 1974. ISSN 0021-9606. doi: 10.1063/1.1681168.
[2]Simon Batzner, Albert Musaelian, Lixin Sun, Mario Geiger, Jonathan P. Mailoa, Mordechai Kornbluth, Nicola
Molinari, Tess E. Smidt, and Boris Kozinsky. E(3)-equivariant graph neural networks for data-efficient and
accurate interatomic potentials. NatureCommunications , 13(1):2453, May 2022. ISSN 2041-1723. doi: 10.1038/
s41467-022-29939-5.
[3]Felix Brockherde, Leslie Vogt, Li Li, Mark E. Tuckerman, Kieron Burke, and Klaus-Robert Müller. Bypassing the
kohn-sham equations with machine learning. NatureCommunications , 8(1):872, October 2017. ISSN 2041-1723.
URLhttps://doi.org/10.1038/s41467-017-00839-3.
[4]Brett I. Dunlap. Robust and variational fitting. Physical Chemistry Chemical Physics, 2(10):2113–2116, January
2000. ISSN 1463-9084. doi: 10.1039/B000027M.
[5]J. A. Ellis, L. Fiedler, G. A. Popoola, N. A. Modine, J. A. Stephens, A. P. Thompson, A. Cangi, and S. Rajaman-
ickam. Accelerating finite-temperature kohn-sham density functional theory with deep neural networks. Phys.
Rev.B, 104(3):035120, July 2021. URLhttps://link.aps.org/doi/10.1103/PhysRevB.104.035120.
[6]Jonas Elsborg, Luca Thiede, Alán Aspuru-Guzik, Tejs Vegge, and Arghya Bhowmik. ELECTRA: A Cartesian
Network for 3D Charge Density Prediction with Floating Orbitals, May 2025.
[7]Pol Febrer, Peter Bjørn Jørgensen, Miguel Pruneda, Alberto García, Pablo Ordejón, and Arghya Bhowmik.
Graph2mat: universal graph to matrix conversion for electron density prediction. Machine Learning: Science
andTechnology , 6(2):025013, apr 2025. doi: 10.1088/2632-2153/adc871. URL https://dx.doi.org/10.1088/
2632-2153/adc871.
[8]Bruno Focassio, Michelangelo Domina, Urvesh Patil, Adalberto Fazzio, and Stefano Sanvito. Linear
jacobi-legendre expansion of the charge density for machine learning-accelerated electronic structure calcu-
lations. npjComputational Materials , 9(1):87, May 2023. ISSN 2057-3960. URL https://doi.org/10.1038/
s41524-023-01053-0.
[9]Xiang Fu, Andrew Rosen, Kyle Bystrom, Rui Wang, Albert Musaelian, Boris Kozinsky, Tess Smidt, and Tommi
Jaakkola. A recipe for charge density prediction, 2024. URLhttps://arxiv.org/abs/2405.19276.
[10]Stefan Ganscha, Oliver T. Unke, Daniel Ahlin, Hartmut Maennel, Sergii Kashubin, and Klaus-Robert Müller.
The QCML dataset, Quantum chemistry reference data from 33.5M DFT and 14.7B semi-empirical calculations.
Scientific Data, 12(1):406, March 2025. ISSN 2052-4463. doi: 10.1038/s41597-025-04720-7.
[11] Mario Geiger and Tess Smidt. E3nn: Euclidean Neural Networks, July 2022.
[12]Andrea Grisafi, Alberto Fabrizio, Benjamin Meyer, David M. Wilkins, Clemence Corminboeuf, and Michele
Ceriotti. Transferable machine-learning model of the electron density. ACSCent.Sci., 5(1):57–64, January 2019.
ISSN 2374-7943. doi: 10.1021/acscentsci.8b00551. URLhttps://doi.org/10.1021/acscentsci.8b00551.
[13]S. Hazra, U. Patil, and S. Sanvito. Predicting the one-particle density matrix with machine learning. J.
Chem.TheoryComput. , 20(11):4569–4578, June 2024. ISSN 1549-9618. doi: 10.1021/acs.jctc.4c00042. URL
https://doi.org/10.1021/acs.jctc.4c00042.
[14]P. Hohenberg and W. Kohn. Inhomogeneous Electron Gas. Physical Review, 136(3B):B864–B871, November
1964. ISSN 0031-899X. doi: 10.1103/PhysRev.136.B864.
[15]Peter Bjørn Jørgensen and Arghya Bhowmik. Equivariant graph neural networks for fast electron density estimation
of molecules, liquids, and solids. npjComputational Materials , 8(1):183, August 2022. ISSN 2057-3960. URL
https://doi.org/10.1038/s41524-022-00863-y.
[16]Kuzma Khrabrov, Ilya Shenbin, Alexander Ryabov, Artem Tsypin, Alexander Telepov, Anton Alekseev, Alexander
Grishin, Pavel Strashnov, Petr Zhilyaev, Sergey Nikolenko, and Artur Kadurin. nablaDFT: Large-Scale Confor-
mational Energy and Hamiltonian Prediction benchmark and dataset. Physical Chemistry Chemical Physics, 24
(42):25853–25863, 2022. doi: 10.1039/D2CP03966D.
[17]Kuzma Khrabrov, Anton Ber, Artem Tsypin, Konstantin Ushenin, Egor Rumiantsev, Alexander Telepov, Dmitry
Protasov, Ilya Shenbin, Anton Alekseev, Mikhail Shirokikh, Sergey Nikolenko, Elena Tutubalina, and Artur
10

Kadurin. ∇2DFT: A Universal Quantum Chemistry Dataset of Drug-Like Molecules and a Benchmark for Neural
Network Potentials, December 2024.
[18]Seongsu Kim, Nayoung Kim, Dongwoo Kim, and Sungsoo Ahn. High-order Equivariant Flow Matching for Density
Functional Theory Hamiltonian Prediction, May 2025.
[19]W. Kohn and L. J. Sham. Self-Consistent Equations Including Exchange and Correlation Effects. Physical Review,
140(4A):A1133–A1138, November 1965. ISSN 0031-899X. doi: 10.1103/PhysRev.140.A1133.
[20]Risi Kondor and Shubhendu Trivedi. On the Generalization of Equivariance and Convolution in Neural Networks
to the Action of Compact Groups. In Proceedings ofthe35thInternational Conference onMachine Learning ,
pages 2747–2755. PMLR, July 2018.
[21]Chengteh Lee. Development of the Colle-Salvetti correlation-energy formula into a functional of the electron
density. Physical ReviewB, 37(2):785–789, 1988. doi: 10.1103/PhysRevB.37.785.
[22]Ryong-Gyu Lee and Yong-Hoon Kim. Convolutional network learning of self-consistent electron density via
grid-projected atomic fingerprints. npjComputational Materials , 10(1):248, October 2024. ISSN 2057-3960. URL
https://doi.org/10.1038/s41524-024-01433-0.
[23]Susi Lehtola. Assessment of initial guesses for self-consistent field calculations. superposition of atomic potentials:
Simple yet efficient. J.Chem.TheoryComput. , 15(3):1593–1604, March 2019. ISSN 1549-9618. doi: 10.1021/acs.
jctc.8b01089. URLhttps://doi.org/10.1021/acs.jctc.8b01089.
[24]He Li, Zun Wang, Nianlong Zou, Meng Ye, Runzhang Xu, Xiaoxun Gong, Wenhui Duan, and Yong Xu. Deep-
learning density functional theory Hamiltonian for efficient ab initio electronic-structure calculation. Nature
Computational Science, 2(6):367–377, June 2022. ISSN 2662-8457. doi: 10.1038/s43588-022-00265-6.
[25]Yunyang Li, Zaishuo Xia, Lin Huang, Xinran Wei, Han Yang, Sam Harshe, Zun Wang, Chang Liu, Jia Zhang,
Bin Shao, and Mark B. Gerstein. Enhancing the Scalability and Applicability of Kohn-Sham Hamiltonians for
Molecular Systems, February 2025.
[26]Erpai Luo, Xinran Wei, Lin Huang, Yunyang Li, Han Yang, Zaishuo Xia, Zun Wang, Chang Liu, Bin Shao, and
Jia Zhang. Efficient and Scalable Density Functional Theory Hamiltonian Prediction through Adaptive Sparsity,
May 2025.
[27]Paula Mori-Sánchez, Aron J. Cohen, and Weitao Yang. Localization and Delocalization Errors in Density
Functional Theory and Implications for Band-Gap Prediction. Physical ReviewLetters, 100(14):146401, April
2008. ISSN 0031-9007, 1079-7114. doi: 10.1103/PhysRevLett.100.146401.
[28]Robert G. Parr and Weitao Yang. Density-Functional Theory ofAtomsandMolecules . Oxford University Press,
May 1994. ISBN 978-0-19-535773-8.
[29]John P. Perdew, Kieron Burke, and Matthias Ernzerhof. Generalized Gradient Approximation Made Simple.
Physical ReviewLetters, 77(18):3865–3868, October 1996. doi: 10.1103/PhysRevLett.77.3865.
[30]Angela D. Rabuck and Gustavo E. Scuseria. Improving self-consistent field convergence by varying occupation
numbers. TheJournalofChemical Physics, 110(2):695–700, January 1999. ISSN 0021-9606, 1089-7690. doi:
10.1063/1.478177.
[31]Joshua A Rackers, Lucas Tecot, Mario Geiger, and Tess E Smidt. A recipe for cracking the quantum scaling limit
with machine learned electron densities. Machine Learning: ScienceandTechnology, 4(1):015027, feb 2023. doi:
10.1088/2632-2153/acb314. URLhttps://dx.doi.org/10.1088/2632-2153/acb314.
[32]Raghunathan Ramakrishnan, Pavlo O. Dral, Matthias Rupp, and O. Anatole von Lilienfeld. Quantum chemistry
structures and properties of 134 kilo molecules. Scientific Data, 1(1):140022, August 2014. ISSN 2052-4463. doi:
10.1038/sdata.2014.22.
[33]K. T. Schütt, M. Gastegger, A. Tkatchenko, K.-R. Müller, and R. J. Maurer. Unifying machine learning and
quantum chemistry with a deep neural network for molecular wavefunctions. NatureCommunications , 10(1):5024,
November 2019. ISSN 2041-1723. doi: 10.1038/s41467-019-12875-2.
[34]Xuecheng Shao, Lukas Paetow, Mark E. Tuckerman, and Michele Pavanello. Machine learning electronic structure
methods based on the one-electron reduced density matrix. NatureCommunications , 14(1):6281, October 2023.
ISSN 2041-1723. URLhttps://doi.org/10.1038/s41467-023-41953-9.
11

[35]Qiming Sun, Xing Zhang, Samragni Banerjee, Peng Bao, Marc Barbry, Nick S. Blunt, Nikolay A. Bogdanov,
George H. Booth, Jia Chen, Zhi-Hao Cui, Janus J. Eriksen, Yang Gao, Sheng Guo, Jan Hermann, Matthew R.
Hermes, Kevin Koh, Peter Koval, Susi Lehtola, Zhendong Li, Junzi Liu, Narbe Mardirossian, James D. McClain,
Mario Motta, Bastien Mussard, Hung Q. Pham, Artem Pulkin, Wirawan Purwanto, Paul J. Robinson, Enrico
Ronca, Elvira R. Sayfutyarova, Maximilian Scheurer, Henry F. Schurkus, James E. T. Smith, Chong Sun, Shi-Ning
Sun, Shiv Upadhyay, Lucas K. Wagner, Xiao Wang, Alec White, James Daniel Whitfield, Mark J. Williamson,
Sebastian Wouters, Jun Yang, Jason M. Yu, Tianyu Zhu, Timothy C. Berkelbach, Sandeep Sharma, Alexander Yu.
Sokolov, and Garnet Kin-Lic Chan. Recent developments in the PySCF program package. TheJournalof
Chemical Physics, 153(2):024109, July 2020. ISSN 0021-9606. doi: 10.1063/5.0006074.
[36]Attila Szabo and Neil S. Ostlund. Modern Quantum Chemistry: Introduction to
Advanced Electronic Structure Theory. Courier Corporation, July 1996. ISBN 978-0-486-69186-2.
[37]Zechen Tang, He Li, Peize Lin, Xiaoxun Gong, Gan Jin, Lixin He, Hong Jiang, Xinguo Ren, Wenhui Duan, and
Yong Xu. A deep equivariant neural network approach for efficient hybrid density functional calculations. Nature
Communications, 15(1):8815, October 2024. ISSN 2041-1723. doi: 10.1038/s41467-024-53028-4.
[38]Oliver Unke, Mihail Bogojeski, Michael Gastegger, Mario Geiger, Tess Smidt, and Klaus-Robert
Müller. SE(3)-equivariant prediction of molecular wavefunctions and electronic densities. In Advances in
Neural Information Processing Systems, volume 34, pages 14434–14447. Curran Associates, Inc., 2021.
[39] J. H. Van Lenthe, R. Zwaans, H. J. J. Van Dam, and M. F. Guest. Starting scf calculations by superposition of
atomic densities. JournalofComputational Chemistry , 27(8):926–932, 2006. doi: https://doi.org/10.1002/jcc.
20393. URLhttps://onlinelibrary.wiley.com/doi/abs/10.1002/jcc.20393.
[40] S. H. Vosko, L. Wilk, and M. Nusair. Accurate spin-dependent electron liquid correlation energies for local spin
density calculations: A critical analysis. Canadian JournalofPhysics, 58(8):1200–1211, August 1980. ISSN
0008-4204, 1208-6045. doi: 10.1139/p80-159.
[41]Johannes Voss. Machine learning for accuracy in density functional approximations. JournalofComputational
Chemistry , 45(21):1829–1845, 2024. doi: https://doi.org/10.1002/jcc.27366. URL https://onlinelibrary.wiley.
com/doi/abs/10.1002/jcc.27366.
[42]Florian Weigend. Accurate Coulomb-fitting basis sets for H to Rn. Physical Chemistry Chemical Physics, 8(9):
1057–1065, February 2006. ISSN 1463-9084. doi: 10.1039/B515623H.
[43]Florian Weigend and Reinhart Ahlrichs. Balanced basis sets of split valence, triple zeta valence and quadruple
zeta valence quality for H to Rn: Design and assessment of accuracy. Physical Chemistry Chemical Physics, 7
(18):3297–3305, August 2005. ISSN 1463-9084. doi: 10.1039/B508541A.
[44]Haiyang Yu, Zhao Xu, Xiaofeng Qian, Xiaoning Qian, and Shuiwang Ji. Efficient and Equivariant Graph Networks
for Predicting Quantum Hamiltonian, November 2023.
[45]Haiyang Yu, Meng Liu, Youzhi Luo, Alex Strasser, Xiaofeng Qian, Xiaoning Qian, and Shuiwang Ji. QH9: A
Quantum Hamiltonian Prediction Benchmark for QM9 Molecules, March 2024.
[46]Barbara Zdrazil, Eloy Felix, Fiona Hunter, Emma J Manners, James Blackshaw, Sybilla Corbett, Marleen de Veij,
Harris Ioannidis, David Mendez Lopez, Juan F Mosquera, Maria Paula Magarinos, Nicolas Bosc, Ricardo Arcila,
Tevfik Kizilören, Anna Gaulton, A Patrícia Bento, Melissa F Adasme, Peter Monecke, Gregory A Landrum, and
Andrew R Leach. The ChEMBL Database in 2023: A drug discovery platform spanning multiple bioactivity
data types and time periods. NucleicAcidsResearch , 52(D1):D1180–D1192, January 2024. ISSN 0305-1048. doi:
10.1093/nar/gkad1004.
[47]He Zhang, Chang Liu, Zun Wang, Xinran Wei, Siyuan Liu, Nanning Zheng, Bin Shao, and Tie-Yan Liu.
Self-Consistency Training for Density-Functional-Theory Hamiltonian Prediction, June 2024.
[48]Tianze Zheng, Ailun Wang, Xu Han, Yu Xia, Xingyuan Xu, Jiawei Zhan, Yu Liu, Yang Chen, Zhi Wang, Xiaojie
Wu, Sheng Gong, and Wen Yan. Data-driven parametrization of molecular mechanics force fields for expansive
chemical space coverage. Chemical Science, 16(6):2730–2740, 2025. doi: 10.1039/D4SC06640E.

\section{Appendix
A Error Sources in Constructing Hamiltonian from Density Coefficients}

Table 3 compares the number of SCF cycles required for convergence for the D-Glucose molecule (C 6H12O6)
using different sets of auxiliary basis sets and their corresponding number of basis functions. As shown,
increasing the basis set size can reduce the number of SCF cycles to as low as 38.5\% of the baseline.
Table 3Effect of auxiliary basis set size on SCF convergence for D-Glucose (C 6H12O6). SCF iteration ratios are
reported as the number of iterations required for convergence, normalized to the default minaoinitial guess. With
the def2-SVP basis set, D-Glucose has 228 basis functions. All results are obtained using the ground truth density
coefficients as the initial guess.
Auxiliary Basis Set Number of Basis Functions SCF Cycles SCF Iter. Ratio (\%)
def2-universal-jfit 720 7 53.8
ETB (β= 2.0)1740 7 53.8
ETB (β= 1.5)2898 5 38.5
B Computing the Density Coefficients
In our work, the machine learning target is the set of expansion coefficients\{c k\}that represent the electron
density ρ(r)in a given auxiliary basis set \{χk(r)\}. There are at least two principled ways to determine these
ground-truth coefficients from a converged DFT calculation.
The first approach is to minimize the squared error of the density itself, which corresponds to an L2 projection
of the density onto the auxiliary basis. The objective is to find the coefficients \{ck\}that solve the following
minimization problem:
min
\{ck\}Zρ(r)−X
kckχk(r)2
dr.(8)
This leads to a system of linear equations:
X
lSaux
klcl=Z
ρ(r)χ k(r)dr,(9)
whereSaux
kl=R
χk(r)χ l(r)dris the overlap matrix of the auxiliary basis functions.
The second approach, which is the standard method in density fitting, is to minimize the error in the Coulomb
repulsion energy. The objective is to minimize the self-repulsion of the residual density:
min
\{ck\}ZZ(ρ(r)−P
kckχk(r)) (ρ(r′)−P
lclχl(r′))
|r−r′|drdr′.(10)
This leads to a different system of linear equations:
X
kZZχk(r)χ k(r′)
|r−r′|drdr′
vk=ZZρ(r)χ l(r′)
|r−r′|drdr′.(11)
Here, the matrix on the left-hand side is the two-center two-electron Coulomb repulsion integral matrix for
the auxiliary basis (int2c2ein PySCF).
We tested both approaches and found that they yielded comparable performance for accelerating SCF
convergence. For all results presented in this paper, we used the first method to generate the ground-truth
density coefficients for our training data.

\section{C Methods of Constructing Hamiltonian Matrix from Predicted Density
We outline the construction of the initial Fock matrix for various types of functionals based on the electron}

density. Specifically, this involves the evaluation of the Coulomb matrix ( J) and exchange-correlation (XC)
matrix (V XC).
The Coulomb matrix is evaluated using three-center electron repulsion integrals:
Jµν=X
i(µν|χ i)ci,(12)
where
(µν|χ i) =ZZµ(r1)ν(r 1)χi(r2)
r12dr1dr2.(13)
are the three-center two-electron integrals between the atomic orbital pair µ(r)ν(r)and the auxiliary function
χ(r)
For LDA and GGA functionals, the electron density and its gradients over the auxiliary basis functions can be
readily computed. The XC matrix is then obtained by numerical integration over a set of Becke grids rgand
its weightsω g:X
gVxc[ρ,∇ρ]ω gµ(rg)ν(r g).(14)
For meta-GGA functionals, the XC potential also depends on the kinetic energy density τ. The exact τis
constructed from molecular orbitals, which are not available when our only input is the total electron density.
We therefore approximate τusing the von Weizsäcker kinetic energy density, which provides an estimate
based solely on the density and its gradient:
τ(r) =1
2X
i∇ψi(r)· ∇ψ i(r)≈∇ρ· ∇ρ
8ρ.(15)
This allows for the evaluation of the meta-GGA XC matrix term:
1
2X
gVxc[ρ,∇ρ, τ]ω g∇µ(r g)· ∇ν(r g).(16)
For hybrid and range-separated functionals, the Hartree-Fock (HF) exchange matrix is needed. The HF
exchange term is a function of the density matrix D. For a similar reason as with meta-GGA functionals—the
difficulty of reconstructing Dfrom ρ—we must use an approximate density matrix. We employ the SAD
density matrix (i.e.minao) as a chemically reasonable guess to construct the HF exchange matrix:
DSAD=⊕ ADA,(17)
D Hyperparameters and Training
Hyperparameters for Model Architectures.
For QHNet models, we adopt the same hyperparameters as those used in Yu et al. [45]. The hyperparameters
for the backbone are kept unchanged for different prediction targets to ensure a fair comparison.
For Nequip models, three variants of different sizes are trained and evaluated, namely NequIP-S, NequIP-M
and NequIP-L. The sizes of the models are kept the same across different auxiliary basis set choices.
Hyperparameters for all these four architectures are summarized in Table 4.
Hyperparameters for different prediction targets.Hyperparameters for training models for different prediction
targets are summarized in Table 5. All models have converged after the training finished.
14

Table 4Hyperparameters for model architectures.
Model Hyperparameter Value
QHNet radius cutoff 15.0
Lmax 4
hidden size 128
bottleneck hidden size 32
number of layers 5
radius embed dim 16
NequIP-S radius cutoff 5.0
lmax 4
number of layers 4
hidden size 32
radial MLP width 64
NequIP-M radius cutoff 5.0
lmax 4
number of layers 7
hidden size 64
radial MLP width 128
NequIP-L radius cutoff 5.0
lmax 4
number of layers 9
hidden size 64
radial MLP width 128
Table 5Hyperparameters for training.
Hyperparameter Hamiltonian Density Matrix Density Coefficients
Max Epochs 5000 5000 5000*
Batch Size 1024 1024 1024
Optimizer Adam Adam Adam
Learning Rate Scheduler Polynomial Polynomial Polynomial
Learning Rate 5e-3 5e-3 2e-2
Minimum Learning Rate 1e-7 1e-7 1e-7
*: The NequIP-S and NequIP-M models are trained for 2000 epochs.
E Extended Results on SCFbench
To provide a more comprehensive view of model performance, we report an extended set of evaluation metrics
on SCFbench in Table 6. The definitions of the metrics are detailed below.
MAE(prediction).The mean absolute error of the model predictions. Note that values of this metric cannot be
compared across different prediction targets.
MAE(C).The mean absolute error of the molecular orbital coefficients obtained from the initial guesses
predicted by the model.
Csimilarity.The cosine similarity between the predicted and the ground-truth molecular orbital coefficients.
15

Table 6Extended results on the SCFbench dataset. The unit for energy is Hartree.
Prediction Target ModelID Test OOD Test
MAE(prediction)↓MAE(C)↓CSimilarity↑MAE(prediction)↓MAE(C)↓CSimilarity↑
HamiltonianQHNet 4.0e-5 0.1527 0.8459 1.7e-3 0.1586 0.1572
Density MatrixQHNet 5.3e-4 0.1314 0.9656 1.4e-3 0.1464 0.5101
Density Coefficients
def2-universal-jfitQHNet 1.7e-4 0.0786 0.9814 5.0e-4 0.1085 0.8763
NequIP-S 2.7e-4 0.0995 0.9512 5.7e-4 0.1312 0.6863
NequIP-M 1.1e-4 0.0746 0.9846 4.0e-4 0.0964 0.9136
NequIP-L 8.9e-5 0.0788 0.9865 3.8e-4 0.0928 0.9334
Density Coefficients
ETB,β= 2.0QHNet 1.8e-4 0.0816 0.9815 7.0e-4 0.1166 0.7981
NequIP-S 3.5e-4 0.1151 0.8989 8.7e-4 0.1486 0.4501
NequIP-M 1.6e-4 0.0832 0.9819 7.2e-4 0.1153 0.8113
NequIP-L 1.0e-4 0.0695 0.9907 5.4e-4 0.0985 0.8990
Density Coefficients
ETB,β= 1.5QHNet 1.2e-3 0.1096 0.9395 3.8e-3 0.1367 0.6119
NequIP-S 1.7e-3 0.1329 0.8298 4.2e-3 0.1507 0.4167
NequIP-M 1.2e-3 0.1051 0.9429 4.0e-3 0.1375 0.5738
NequIP-L 8.7e-4 0.0887 0.9784 3.7e-3 0.1189 0.7767
F Results for Wavefunction Alignment Loss (WALoss)
The Wavefunction Alignment Loss (WALoss) is proposed by Li et al. [25]for solving the Scaling-induced
MAE-Applicability Divergence (SAD) problem and enhancing the scalability and applicability of Hamiltonian
prediction models. As the WALoss is originally used to train the model on the PubChemQH dataset consisting
of relatively large molecules, it is thus interesting to find out whether WALoss is able to solve the transferability
problem of Hamiltonian prediction. Therefore, although there is no publicly available code for WALoss, we
reimplement it ourselves and test it with QHNet on our SCFbench dataset.
There are multiple hyperparameters in WALoss, including the λ1,λ2andλ3for weighting the elementwise
error losses and WALoss, and the ρandξfor weighting the WALoss terms for occupied and unoccupied
orbitals, respectively. The optimal values of λs are thoroughly discussed in the original paper, but the values
ofρandξare unspecified except the description of ρ≫ξ. Therefore, we adopt the optimal values for the λs
from the original paper ( λ1= 1.0, λ2= 1.0, λ3= 2.5), fix ρto be1 .0and experiment with various values for ξ.
As listed in Table 7, some settings outperform the original QHNet on the ID test set, but when tested on the
OOD test set, none of them show advantage in system size transferability in terms of RIC. We suspect that
WALoss is great for solving the learnability problem of the Hamiltonian prediction task on large molecules
(such as those in the PubChemQH dataset), but still falls short in overcoming the task’s inherent transferability
issues.
Table 7WALoss results on the SCFbench benchmark dataset.
ModelξID Test OOD Test
Convergence↑RIC↓Convergence↑RIC↓
QHNet w/o WALoss- 100\% 63.20\% 97.43\% 179.47\%
QHNet w/ WALoss1.0 100\% 69.56\% 99.57\% 173.07\%
0.5 100\% 67.97\% 97.61\% 183.31\%
0.3 100\% 67.08\% 98.64\% 179.77\%
0.1 100\% 67.11\% 98.40\% 170.57\%
0.01 100\% 62.65\% 98.40\% 173.08\%
0.001 100\% 62.66\% 97.25\% 177.35\%
16

\end{document}
