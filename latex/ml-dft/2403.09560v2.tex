\documentclass[11pt,a4paper]{article}
\usepackage[utf8]{inputenc}
\usepackage[T1]{fontenc}
\usepackage{amsmath}
\usepackage{amsfonts}
\usepackage{amssymb}
\usepackage{amsthm}
\usepackage{graphicx}
\usepackage{hyperref}
\usepackage{cite}
\usepackage{natbib}
\usepackage{geometry}
\usepackage{booktabs}
\usepackage{multirow}
\usepackage{array}
\usepackage{caption}
\usepackage{subcaption}
\usepackage{float}
\usepackage{url}
\geometry{margin=1in}

\title{Self-Consistency Training for Density-Functional-Theory Hamiltonian Prediction}
\author{He Zhang, Chang Liu, Zun Wang, Xinran Wei, Siyuan Liu, Nanning Zheng, Bin Shao, Tie-Yan Liu}
\date{\today}

\begin{document}
\maketitle

\begin{abstract}
Predicting the mean-field Hamiltonian matrix in
density functional theory is a fundamental formu-
lation to leverage machine learning for solving
molecular science problems. Yet, its applicability
is limited by insufficient labeled data for train-
ing. In this work, we highlight that Hamiltonian
prediction possesses a self-consistency principle,
based on which we propose self-consistency train-
ing, an exact training method that does not require
labeled data. It distinguishes the task from predict-
ing other molecular properties by the following
benefits: (1)it enables the model to be trained on
a large amount of unlabeled data, hence addresses
the data scarcity challenge and enhances general-
ization; (2)it is more efficient than running DFT
to generate labels for supervised training, since it
amortizes DFT calculation over a set of queries.
We empirically demonstrate the better generaliza-
tion in data-scarce and out-of-distribution scenar-
ios, and the better efficiency over DFT labeling.
These benefits push forward the applicability of
Hamiltonian prediction to an ever-larger scale.
1. Introduction
Calculating properties of molecules is the foundation for
a wide range of industry needs including drug design, pro-
tein engineering, and material discovery. The key to these
properties is the electronic structure in the molecule, for
which various computational methods are proposed. Den-
sity functional theory (DFT) (Hohenberg \& Kohn, 1964;
Kohn \& Sham, 1965; Perdew et al., 1996; Teale et al., 2022)
1National Key Laboratory of Human-Machine Hybrid Aug-
mented Intelligence, National Engineering Research Center for
Visual Information and Applications, and Institute of Artificial
Intelligence and Robotics, Xi’an Jiaotong University2Microsoft
Research AI for Science3These authors did this work during an
internship at Microsoft Research AI for Science. Correspondence
to: Chang Liu <changliu@microsoft.com >, Nanning Zheng
<nnzheng@mail.xjtu.edu.cn >.
Proceedings of the 41stInternational Conference on Machine
Learning , Vienna, Austria. PMLR 235, 2024. Copyright 2024 by
the author(s).
෡𝐇𝜃ℳ 𝐸
𝜖HOMO
𝜖LUMO
𝜖Δ
⋯DFT initialization ℳ≔\{𝒵,ℛ\}Self-consistencyFigure 1. Hamiltonian prediction is the task to use a machine learn-
ing model to predict the mean-field Hamiltonian matrix ˆHθ(M)
in density functional theory from a given molecular structure
M:=\{Z,R\}specified by the atomic types Zand coordinates
Rof atoms. It can derive various molecular properties, e.g., the
total energy E, the HOMO and LUMO energies ϵHOMO , ϵLUMO
and their gap ϵ∆for the given molecule, and can also serve as
an accurate DFT initialization. We highlight in this work that the
task has a self-consistency principle (the blue loop arrow), which
allows training the model without labeled data.
is perhaps the most prevailing choice due to its balanced
accuracy and efficiency, but still hard to meet the demand in
industry. Encouraged by the impressive advancement in ma-
chine learning, researchers have trained machine learning
models on datasets with property labels to directly predict
properties of queried molecules (Ramakrishnan et al., 2014;
Chmiela et al., 2019; Chanussot et al., 2021). For each prop-
erty, a separate model (at least a separate output module)
needs to be trained. A more fundamental formulation is to
predict the Hamiltonian matrix (Sch ¨utt et al., 2019), or more
precisely, the effective one-electron mean-field Hamiltonian
matrix, i.e., the Fock matrix in a DFT calculation after con-
vergence. The Hamiltonian matrix can directly provide all
the properties that a DFT calculation can (Fig. 1), waiving
the need to specify the target property or train multiple mod-
els. Moreover, Hamiltonian prediction can also accelerate
running DFT by providing an accurate initialization.
Noticeable progress has been made for Hamiltonian predic-
tion. Hegde \& Bowen (2017) pioneered the direction using
kernel ridge regression to predict semi-empirical Hamil-
tonian for one-dimensional systems. Sch ¨utt et al. (2019)
then proposed a neural network model called SchNorb to
predict Hamiltonian for small molecules, which is further
enhanced for prediction efficiency by Gastegger et al. (2020).
Shmilovich et al. (2022) proposed to employ atomic orbital
features for Hamiltonian prediction. Noting that the Hamil-
1arXiv:2403.09560v2 [cs.LG] 5 Jun 2024
\end{abstract}

Self-Consistency Training for Density-Functional-Theory Hamiltonian Prediction
tonian matrix is composed of tensors in various orders which
are equivariant to coordinate rotation in respective ways,
subsequent works proposed neural network model architec-
tures that guarantee the equivariance. Some works (Unke
et al., 2021; Yu et al., 2023b; Gong et al., 2023; Yin et al.,
2024) include high-order tensorial features into model input,
which are processed in an equivariant way typically with
tensor products. Li et al. (2022) used local frames to anchor
coordinate systems with the molecule so that the predic-
tion target is invariant. Zhang et al. (2022); Nigam et al.
(2022) implemented the prediction by constructing equivari-
ant kernels. There are works that exploited data other than
the Hamiltonian directly, e.g., using orbital energies (Wang
et al., 2021b; Gu et al., 2022; Zhong et al., 2023) to su-
pervise the prediction of Hamiltonian. While these prior
efforts have introduced powerful architectures showing en-
couraging outcomes, they all rely on datasets providing
Hamiltonian or orbital energy labels. Since such datasets
are scarce, the applicability of Hamiltonian prediction is
restricted to molecules with no more than 31 atoms (Yu
et al., 2023a).1
In this work, we highlight a uniqueness of Hamiltonian pre-
diction: it has a self-consistency principle (indicated by the
blue loop arrow in Fig. 1), by leveraging which we design
a training method that guides the model without labeled
data. The self-consistency originates from the basic equa-
tion of DFT (Eq. (1)) that the Hamiltonian needs to satisfy.
Conventional DFT solves the equation using a fixed-point
iteration process called self-consistent field (SCF) iteration.
In contrast, the proposed self-consistency training solves
the equation by directly minimizing the residue of the equa-
tion incurred by the model-predicted Hamiltonian (Fig. 2).
As the equation fully determines the prediction target, no
Hamiltonian label is required, and the loss function is mini-
mized only if the equation is satisfied and the prediction is
exact. Self-consistency training compensates data scarcity
with physical laws, and differentiates Hamiltonian predic-
tion from other machine learning formulations ( e.g., energy
prediction), in that it enables continued self-improvement
without additional labeled data.
We exploit the merit of self-consistency training in two
specific points. (1)Self-consistency training leverages un-
labeled data, which allows substantial improvement of the
generalizability of the Hamiltonian prediction model. We
demonstrate that the predicted Hamiltonian as well as de-
rived molecular properties are indeed improved by a sig-
nificant margin, when labeled data is limited (data-scarce
scenario) and when the model is evaluated on molecules
1There are a few works (Li et al., 2022; Gong et al., 2023)
that have demonstrated applicability to large-scale material sys-
tems. We note that this is achieved by leveraging the periodicity
and locality in material systems, which do not hold perfectly in
molecular systems.larger than those used in training (out-of-distribution sce-
nario).
(2)Self-consistency training on unlabeled data is more ef-
ficient than generating labels using DFT on those data for
supervised learning, as we find that self-consistency train-
ing can be seen as an amortization of DFT calculation over
a set of molecules. DFT requires multiple SCF iterations
on each molecule before providing supervision, while self-
consistency training only requires effectively one SCF it-
eration to return a training signal, hence can provide in-
formation on more molecules given the same amount of
computation. The better efficiency for Hamiltonian predic-
tion training is empirically verified in both data-scarce and
out-of-distribution scenarios. More attractively, regarding
physical quantities derived from the predicted Hamiltonian,
self-consistency training even outperforms supervised train-
ing using full additional labels given sufficient computa-
tional budget, indicating that it is more relevant to molecular
properties and real applications. We also verified the direct
acceleration by self-consistency training to solve a bunch of
molecules upon the conventional DFT calculation.
Finally, we demonstrate that with the above two unique
benefits of self-consistency training, the applicability of
Hamiltonian prediction can overcome the data limit, and is
extended to molecules much larger (56 atoms) than previ-
ously reported, showing increased practical relevance. It
also derives orders better molecular property results on these
large molecules than end-to-end property prediction models,
which are always limited by the availability of labeled data.
2. Self-Consistency Training
2.1. Preliminaries
We first provide a schematic description of the calculation
mechanism of DFT and conventional supervised learning for
Hamiltonian prediction before delving into self-consistency
training. Appendix A provides more details.
For a given molecular structure M:=\{Z,R\}, where
Z:=\{Z(a)\}A
a=1andR:=\{R(a)\}A
a=1specify the atomic
numbers (types) and coordinates of the Anuclei in the
molecule, DFT solves the ground state of the Nelec-
trons in the molecule by minimizing electronic energy un-
der a reduced representation of electronic state, which is
None-electron wavefunctions \{ϕi(r)\}N
i=1, called orbitals.
Here, r∈R3represents the Cartesian coordinates of an
electron. For practical calculation, a basis set of func-
tions on R3is used to expand the orbitals. To roughly
align with the electronic structure, the basis functions de-
pend on the molecular structure, hence are denoted as
\{ηM,α(r)\}B
α=1. Expansion coefficients of the orbitals are
collected into a matrix C∈RB×Nin the following way:
ϕi(r) =PB
α=1CαiηM,α(r).
2

Self-Consistency Training for Density-Functional-Theory Hamiltonian Prediction
𝐇ℳ𝐂 𝐂=𝐒ℳ 𝐂 𝛜Kohn -Sham equation
𝐇ℳ𝐂𝐂ℳ𝐇 :
solution to 𝐇 𝐂=𝐒ℳ 𝐂 𝛜
𝐂 𝐇𝐇(𝟏)𝐂(𝟎)𝐂(𝟏)𝐇(𝟐)𝐇ℳ⋆⋯ ⇔ ෡𝐇𝜃(ℳ)𝐂ℳ(𝐇(𝟏))
𝐇ℳ(𝐂(0)) 𝐇ℳ(𝐂(1))𝐂ℳ(𝐇(𝟐))
ℒlabel loss Supervised training
DFT labeling (SCF iteration)
𝐂ℳ(෡𝐇𝜃(ℳ))
𝐇ℳ(𝐂ℳ,𝜃)Self-Consistency training
෡𝐇𝜃(ℳ)
⇔𝐂ℳ,𝜃
𝐇ℳ(𝐂ℳ,𝜃)ℒself−con lossEq. (1)
Eq. (3)Eq. (2)
Figure 2. Illustration of the proposed self-consistency training with comparison to the conventional DFT calculation and supervised
training. ( Left) The central task of a DFT calculation is to solve the Kohn-Sham equation (Eq. (1)) for the given molecular structure
M. (Middle ) The equation is equivalent to the condition that the eigenvectors CofHrecover Hvia a known function HM(C).
(Top-Right ) To solve the equation, conventional DFT uses a fixed-point iteration (SCF iteration), which, upon convergence, gives the
labelH⋆
Mfor supervised training (Eq. (2)) of a Hamiltonian prediction model ˆHθ(M). (Bottom-Right ) In contrast, self-consistency
training (Eq. (3)) directly minimizes the mismatch between the predicted Hamiltonian ˆHθ(M)and the matrix HM(CM,θ)reconstructed
from its eigenvectors.
DFT typically solves the electronic energy minimization
problem w.r.t Cby directly solving the optimality equation:
HM(C)C=SMCϵ, (1)
which is called the Kohn-Sham equation. Here, HM(C)
is a matrix-valued function with an explicit expression
(given an exchange-correlation functional) (Appendix A.5).
This matrix is called the Hamiltonian matrix (also noted
as the Fock matrix) due to the resemblance of the equa-
tion to the Schr ¨odinger equation. The matrix SM,αβ:=R
ηM,α(r)ηM,β(r) dris the overlap matrix of the basis,
which can be computed analytically for common basis
choices. Eq. (1)can be seen as a generalized eigenvalue
problem defined by the matrices HM(C)andSM, where
the coefficients of orbitals Cin the equation can be under-
stood as eigenvectors, and the diagonal matrix ϵcomprises
eigenvalues which are referred to as orbital energies.
However, the difficulty to solve Eq. (1)is that, the matrix
that defines the problem and the eigenvector solution are
intertwined: the eigenvectors Cneed to recover the Hamil-
tonian matrix that defined the eigenvalue problem through
the explicit function HM(C)(Fig. 2, middle). Conven-
tional DFT calculation solves it using a fixed-point iteration
process called self-consistent field (SCF) iteration. In each
step, orbital coefficients C(k−1)are used to construct the
Hamiltonian matrix H(k):=HM(C(k−1)), which defines
a generalized eigenvalue problem H(k)C=SMCϵ, whose
eigenvectors, denoted as CM(H(k)), are taken as the up-
dated orbital coefficients C(k)(Fig. 2, top right). The con-
verged Hamiltonian H⋆
Mand its eigenvectors hence solve
Eq. (1), which then derive various molecular structures.
Hamiltonian prediction aims to bypass the SCF iteration bydirectly predicting H⋆
Mfrom molecular structure Musing
a machine-learning model ˆHθ(M),2where θdenotes the
model parameters to be learned. The conventional way to
learn such a model is by supervised learning, which requires
running DFT on a set of molecular structures Dto construct
a labeled dataset D, on which the supervised training loss
function is applied:
Llabel(θ;D) :=1
|D|X
(M,H⋆
M)∈D


ˆHθ(M)−H⋆
M


2
F,(2)
where |D|denotes the number of samples in the set D. The
squared Frobenius norm amounts to the mean squared error
(MSE) over the matrix entries. Some works (Unke et al.,
2021; Yu et al., 2023b) also include a mean absolute error
(MAE) loss for more efficient learning.
2.2. Self-Consistency Training
We now describe the proposed self-consistency training
for Hamiltonian prediction. It can be seen as another
way to solve the Kohn-Sham equation (1), which the pre-
diction ˆHθ(M)needs to satisfy. Recall that the equa-
tion is equivalent to the condition that C:=CM(H),
i.e., the eigenvectors of the generalized eigenvalue prob-
lem defined by the Hamiltonian matrix H, construct the
same Hamiltonian matrix, i.e.,HM(C) =H(Fig. 2,
middle). The self-consistency training loss is hence de-
signed to enforce this condition: the difference between
the predicted Hamiltonian ˆHθ(M)and the reconstructed
Hamiltonian from itself should be minimized, where the
2The “hat” or “circumflex” accent in the notation here is meant
to represent “a neural-network estimator”.
3

Self-Consistency Training for Density-Functional-Theory Hamiltonian Prediction
reconstruction is done by first solving for the eigenvec-
torsCM,θ:=CMˆHθ(M)
of the generalized eigenvalue
problem defined by ˆHθ(M)then constructing the Hamil-
tonian using HM(CM,θ)(Fig. 2, bottom right). Explicitly,
the self-consistency loss is:
Lself-con (θ;D) :=
1
|D|X
M∈D


ˆHθ(M)−HM
CMˆHθ(M)


2
F.(3)
Following the practice of previous work (Unke et al., 2021;
Yu et al., 2023b), we also include its MAE counterpart in
place of the squared Frobenius norm into the loss. The im-
plementation process is summarized in Alg. 1. Note that the
loss only requires a set of molecular structures Dunneces-
sarily with Hamiltonian labels. It thus enables leveraging
numerous molecular structures for learning Hamiltonian pre-
diction, which could substantially enhance generalizability
of the prediction model to a wide range of molecules, allow-
ing applicability beyond the limitation of labeled datasets.
We make the following four remarks regarding the un-
derstanding of the self-consistency loss. (1)The self-
consistency loss is distinct from regularization or self-
supervised training, in the sense that the loss by itself can
already drive the model towards the exact target, since the
loss enforces the equation that determines the target. (2)We
emphasize that the loss should notbe interpreted as updating
the prediction ˆHθ(M)towards the reconstructed Hamilto-
nianHM
CMˆHθ(M)
as a fixed target (which is the
case when the stop grad operation is applied to the re-
constructed Hamiltonian), and the back-propagation ( i.e.,
computation of the gradient of the loss w.r.t θ) through the
Hamiltonian reconstruction process is indispensable. This is
because the reconstructed Hamiltonian unnecessarily comes
closer to the target solution (Pulay, 1982; Cances \& Le Bris,
2000), so taking the reconstructed Hamiltonian as a con-
stant when optimizing θmay even make the model worse.
Instead, the loss aims to minimize the change from the re-
construction process. To minimize this change, both the
predicted matrix and the reconstructed matrix are driven
towards the solution. (3)One may also consider enforcing
self-consistency by minimizing the difference in the derived
energy after reconstruction, which meets the common stop-
ping criterion in a DFT calculation and could hold more
physical relevance. But this would require eigen-solving
the reconstructed matrix and evaluating the energy from
the eigenvectors, which is as costly as another Hamiltonian
reconstruction, making the loss unacceptably costly to op-
timize. (4)The self-consistency loss bears some similarity
to the SCF loss in DM21 (Kirkpatrick et al., 2021). Both
connect a DFT solution and an exchange-correlation (XC)
functional defining the DFT calculation (part of HM(C)in
our formulation). The SCF loss is used to regularize an XC
functional model with a label (solution), while we use theAlgorithm 1 Implementation of self-consistency loss (on
one molecular structure)
Require: Molecular structure M=\{Z,R\}compris-
ing types Z:=\{Z(a)\}A
a=1and coordinates R:=
\{R(a)\}A
a=1of its atoms, pre-computed integral matri-
ces (e.g., overlap matrix SM), Hamiltonian prediction
model ˆHθ(·)to be learned.
1:Generate requisite integrals and quadrature grid for con-
structing Hamiltonian (Appendix B.2);
2:Predict Hamiltonian ˆHθ(M)using the model;
3:Solve for the eigenvectors CM,θof the generalized
eigenvalue problem ˆHθ(M)C=SMCϵ.
4:Reconstruct Hamiltonian HM(CM,θ)following an ex-
plicit expression (Appendix A.5);
5:Compute the loss Lself-con (θ;\{M\} )as the addition of
the mean squared error (shown in Eq. (3)) and mean
absolute error between ˆHθ(M)andHM(CM,θ).
output Lself-con (θ;\{M\} ).
self-consistency loss to train a solution-prediction model
given a well-established XC functional. It is future work
to investigate the utility of the SCF loss for unsupervised
Hamiltonian prediction.
2.3. Implementation Considerations
For stable and efficient optimization of the self-consistency
loss, we mention a few technical treatments.
Back-Propagation through Eigensolver. As mentioned,
back-propagation through the reconstruction process
HM
CMˆHθ(M)
is indispensable. This requires dif-
ferentiation through the eigensolver CM(H). We leverage
the eigensolver implemented in an automatic differentia-
tion package PyTorch (Paszke et al., 2019) which automati-
cally provides the differentiation calculation. Nevertheless,
the calculation often appears numerically unstable (Ionescu
et al., 2015; Wang et al., 2019), as it relies on a matrix G
(see Appendix B.2 for detailed derivation),
Gij=(
1/(ϵi−ϵj), i̸=j,
0, i =j,
where ϵiisi-th eigenvalue. When there are two close eigen-
values, the values in Gcan be exceedingly large, causing
unstable training. To mitigate this instability, we introduce
two treatments. The first is simply truncating the values
inGif they are larger than a chosen threshold. The sec-
ond treatment is to skip the model parameter update when
the scale of the gradient w.r.t parameters exceeds a certain
threshold. Appendix B.2 presents more details.
Efficient Hamiltonian Reconstruction. Evaluating the
function HM(C)is also a costly procedure, mainly due
4

Self-Consistency Training for Density-Functional-Theory Hamiltonian Prediction
to two computational components. The first is the evalua-
tion of basis functions on a quadrature grid for evaluating
the exchange-correlation component of the Hamiltonian
matrix (Appendix A.5). To accelerate this part, we imple-
mented a GPU-based evaluation process of basis functions
on grid points. The other costly procedure is the evaluation
of the Hartree component of the Hamiltonian matrix, which
requires O(N4)cost in its vanilla form. For efficient eval-
uation of this term, we adopt the density fitting approach
(Appendix A.5), a widely used technique in DFT to reduce
the complexity to O(N3)with acceptable loss of accuracy.
2.4. Amortization of DFT Calculation
As mentioned in Sec. 2.2, self-consistency training can be
applied to unlimited unlabeled molecular structures, hence
can substantially improve the generalizability of a Hamil-
tonian prediction model. Here, we point out that self-
consistency training is also more efficient to improve gener-
alizability than generating additional labels using DFT on
those data and then supervising the model. This is based on
the interpretation that self-consistency training is an amor-
tization of DFT: for a given molecular structure M, DFT
requires multiple SCF iterations for convergence before it
can provide a supervision on M(Fig. 2, top right), while
self-consistency training only requires oneSCF iteration
to evaluate the loss and guide the training on M(Fig. 2,
bottom right). This indicates that given the same amount of
computational resources measured in the number of SCF it-
erations, self-consistency training can distribute the resource
on more molecular structures, hence providing information
on a larger range of the input space. This is more valu-
able than Hamiltonian labels on fewer molecular structures
for the model to generalize to a broad range of molecular
structures.
Self-consistency training can also be viewed as a way to
carry out DFT calculation. Under this view, the amortiza-
tion effect makes self-consistency training a more efficient
method than the conventional DFT to solve a large amount
of molecular structures. Apart from the amortization effect,
the efficiency is also benefited from the generalization of
a Hamiltonian prediction model to similar molecular struc-
tures, on which the model can already provide close results.
The demand to solve a set of molecular structures is not
uncommon; e.g., high-throughput drug screening requires
investigating a large amount of ligand-receptor compounds
using DFT (Jordaan et al., 2020). Therefore, the applicabil-
ity scope of Hamiltonian prediction with self-consistency
training is enlarged.

\section{Experimental Results}

We now empirically validate the benefits of self-consistency
training. We adopt QHNet (Yu et al., 2023b) as the Hamilto-nian prediction model, which is an SE(3) -equivariant graph
neural network that balances efficiency and accuracy. Addi-
tional results based on alternative architectures ( e.g., PhiS-
Net (Unke et al., 2021)) are provided in Appendix D.2,
which indicate the same conclusions as presented below.
We employ the following metrics to measure prediction ac-
curacy. A direct metric is the mean absolute error (MAE)
over matrix entries between the predicted and DFT-solved
Hamiltonian matrices, as introduced by Sch ¨utt et al. (2019).
Directly derived quantities from Hamiltonian, including
orbital energies ϵand coefficients Csolved from the gener-
alized eigenvalue problem, are also used to assess accuracy,
measured by MAE for ϵand cosine similarity for C. We
also report the MAE for three molecular properties rele-
vant to molecular research, including the highest occupied
molecular orbital energy ϵHOMO , the lowest unoccupied
molecular orbital energy ϵLUMO , and their gap ϵ∆. We
also assess the utility for accelerating DFT by the ratio of
the number of SCF steps to convergence using the predic-
tion as initialization over the number using the standard
initialization, denoted as “SCF Accel.” The conventional
DIIS (Pulay, 1980) strategy is adopted for running SCF it-
eration, while we also present “SCF Accel.” results using
the second-order SCF (SOSCF) (Sun et al., 2017) iteration
strategy in Appendix D.3, considering that DIIS may lead
to non-monotone iterations thereby diminishes the benefit
of a more accurate initialization.
3.1.Self-Consistency Training Improves Generalization
As discussed in Sec. 2.2, self-consistency training can lever-
age unlabeled data to improve generalizability. We validate
this benefit on two challenging generalization scenarios.
Data-Scarce Scenario. For some scientific tasks with lim-
ited labels available, it is difficult for the machine learn-
ing model to achieve meaningful performance even for in-
distribution (ID) generalization. To demonstrate the advan-
tage of self-consistency training in this scenario, we first
conduct generalization experiments over the conformational
space. Conformations of ethanol, malondialdehyde and
uracil from the MD17 dataset (Chmiela et al., 2019; Sch ¨utt
et al., 2019) are considered. The training/validation/test
split setting follows Sch ¨utt et al. (2019). To simulate a
data-scarce setting, for each molecule, only 100 labeled con-
formations (denoted as D(1)) are provided for supervised
training using the supervised loss Llabel(θ;D(1))(Eq. (2)).
With the self-consistency loss (Eq. (3)), a large amount of
additional unlabeled structures in the training set (about
24,900 structures for each molecule; denoted as D(2)) can
be leveraged, in which case the resulting loss function is:
Llabel(θ;D(1)) +λself-conLself-con (θ;D(2)). (4)
See more training details in Appendix C.4.
5

Self-Consistency Training for Density-Functional-Theory Hamiltonian Prediction
Table 1. Generalization improvement by self-consistency training on unlabeled data in the data-scarce scenario (MD17 Hamiltonian).
Evaluated on the test split of conformations of each molecule.
Molecule Setting H[µEh]↓ϵ[µEh]↓C[\%]↑ϵHOMO [µEh]↓ϵLUMO [µEh]↓ϵ∆[µEh]↓SCF Accel. [\%]↓
Ethanollabel 160.36 712.54 99.44 911.64 6800.84 6643.11 68.3
label + self-con 75.65 285.49 99.94 336.97 1203.60 1224.86 61.5
Malondi- label 101.19 456.75 99.09 471.92 1093.22 1115.94 69.1
aldehyde label + self-con 86.60 280.39 99.67 274.45 279.14 324.37 62.1
Uracillabel 88.26 1079.51 95.83 1217.17 12496.1 11850.56 65.8
label + self-con 63.82 315.40 99.58 359.98 369.67 388.30 54.5
Table 2. Generalization improvement by self-consistency training on unlabeled data in the OOD scenario (QH9). The model is trained on
the QH9-small training split, and evaluated on the QH9-large test split directly ( zero-shot ) or after fine-tuned by self-con sistency
loss on QH9-large training split (without labels).
Setting H[µEh]↓ϵ[µEh]↓C[\%]↑ϵHOMO [µEh]↓ϵLUMO [µEh]↓ϵ∆[µEh]↓SCF Accel. [\%]↓
zero-shot 69.67 403.52 95.72 778.86 12230.49 12203.12 66.3
self-con (all-param) 65.74 375.31 97.31 565.50 1130.55 1316.96 64.5
self-con (adapter) 64.48 268.83 97.12 449.80 1220.54 1394.29 65.0
Prediction results on test structures are summarized in Ta-
ble 1. Compared to the results of supervised (label-based)
training, applying self-consistency loss on unlabeled struc-
tures leads to a substantial improvement across all evalua-
tion metrics and molecules. Notably, it achieves a significant
reduction in the Hamiltonian MAE, with decreases from
14.4\% to 52.8\%. The MAE of ϵHOMO ,ϵLUMO andϵ∆are
even reduced by several folds. Applying self-consistency
training also substantially improves the acceleration for con-
ventional DFT. In addition, a point-by-point, instance-level
comparison in Appendix D.5 shows that self-consistency
training leads to faster SCF convergence over various molec-
ular systems consistently, while supervised training does
not. These findings underscore the attractive capability of
self-consistency training in breaking the limitation of data
scarcity.
Out-of-Distribution (OOD) Scenario. Yu et al. (2023a)
introduced the QH9 dataset to benchmark Hamiltonian pre-
diction over the chemical space. Their findings highlight a
challenging out-of-distribution (OOD) generalization sce-
nario: models trained on smaller molecules often struggle
to generalize to larger molecules, restricting the applica-
bility. To demonstrate the effect of better generalization
using self-consistency training, we construct a similar OOD
scenario. We split the molecular structures in QH9 into two
subsets: QH9-small comprising molecules with no more
than 20 atoms, and QH9-large with larger molecules. The
two subsets are then correspondingly divided at random
into distinct training/validation and training/validation/test
splits (see more dataset details in Appendix C.1). The
model is trained and validated on QH9-small using the
supervised loss (Eq. (2)), and is tested on the QH9-large
test split (dubbed zero-shot ). With the self-consistency
loss (Eq. (3)), the model is allowed to be fine-tuned (with-
out the pretraining supervised loss) on relevant but unla-beled molecular structures, for which we take the QH9-large
training split. We consider two fine-tuning settings: fine-
tuning all parameters of the model, dubbed self-con
(all-param) , or introducing an adapter module atop
the model which is the only optimized component, dubbed
self-con (adapter) . We ensure all models are suffi-
ciently trained. Appendix C.4 shows more training details.
From the results shown in Table 2, we observe a significant
improvement by fine-tuning using self-consistency on un-
labeled QH9-large molecules in both fine-tuning settings.
Remarkably, self-consistency reduces the MAE of ϵLUMO
andϵ∆by an order of magnitude. This result demonstrates
that self-consistency training enables the flexibility to adapt
a model to an OOD workload without labeled data.
3.2. Self-Consistency Training is More Efficient than
DFT Labeling
As discussed in Sec. 2.4, self-consistency training can train
a model more efficiently than DFT labeling followed by
supervised learning, due to its amortization effect of DFT
calculation. We demonstrate the empirical efficiency by
comparing self-consistency training/fine-tuning in the above
two scenarios to the alternative approach of generating la-
bels by running DFT on the additional unlabeled molecular
structures then applying supervised training using these ex-
tended labels (dubbed extended-label ). We also con-
sider a variant that conducts DFT labeling along with model
training (dubbed extended-label-online ): DFT is
only run on unlabeled molecular structures in the current
training batch drawn at random, and the generated labels
are stored for possible use in future batches. This could be
more efficient than extended-label .
The efficiency is monitored by the accuracy-cost curve along
training. The accuracy is measured by the validation Hamil-
6

Self-Consistency Training for Density-Functional-Theory Hamiltonian Prediction
0.060.080.10
(DFT labeling)
0 20000 40000 60000 80000 100000
Computation time (s)0.00000.00250.0050Hamiltonian MAE (E h)
0.060.08
(DFT labeling)
0 25000 50000 75000 100000 125000
Computation time (s)0.0000.0020.004
0.040.06
(DFT labeling)
0 100000 200000 300000
Computation time (s)0.0000.0020.004
(a) Ethanol (b) Malondialdehyde (c) Uracillabel + self-con extended-label extended-label-online
Figure 3. Efficiency comparison in the data-scarce scenario (MD17 Hamiltonian) among self-con sistency training on unlabeled data,
supervised training following DFT labeling on unlabeled data ( extended-label ), and supervised training along with DFT labeling
(extended-label-online ). Dotted horizontal lines extend from the last measured point of the respective curves.
0 100000 200000 300000
Computation time (s)0.000060.000080.000100.000120.00014Hamiltonian MAE (E h)
(DFT Labeling)
0 100000 200000 300000
Computation time (s)0.000050.000100.000150.000200.00025
(DFT labeling)
(a) all-param (b) adapterself-con extended-label extended-label-online
Figure 4. Efficiency comparison in the OOD scenario (QH9) among fine-tuning using self-con sistency training on unlabeled data,
supervised training following DFT labeling on unlabeled data ( extended-label ), and supervised training along with DFT labeling
(extended-label-online ). Dotted horizontal lines extend from the last measured point of the respective curves.
tonian MAE. The cost can be measured by the number of
SCF steps along self-consistency training or DFT labeling.
This matches the analysis in Sec. 2.4 and is system- and
implementation-independent. Results are shown in Figs. D.1
and D.2 in Appendix D.4, which validates the better effi-
ciency in all cases.
For better practical relevance and considering the compli-
cation of the interplay between running SCF and model
parameter optimization, we present results measured by the
cost of real computation time here. All methods are imple-
mented on a workstation equipped with an NVIDIA A100
GPU with 80 GiB memory and a 24-core AMD EPYC CPU.
Data-Scarce Scenario. Accuracy-cost curves of the three
training strategies are presented in Fig. 3. We see that
self-consistency training converges rapidly, achieving a
low prediction error with a cheap cost. In contrast, the
extended-label strategy keeps a plateau at first, rep-
resenting the process to generate labels using DFT during
which the model is not optimized. It is only after the DFT
labeling process that the prediction error starts to drop. The
extended-label-online strategy indeed improves
upon extended-label by amortizing the labeling cost
over the course of training, but it is still not as efficient as
self-consistency training, whose amortization capability al-
lows a more frequent model optimization per SCF step. We
note that due to our hardware limitation, DFT labeling andmodel optimization are performed sequentially. The two
processes can be parallelized which may further improve
the efficiency of extended-label-online at the cost
of using more machines.
Out-of-Distribution (OOD) Scenario. All the three train-
ing settings are run for the fine-tuning stage of the model.
Curves on QH9-large validation split are presented in
Fig. 4. We see again that self-consistency training achieves
a high accuracy at a relatively low cost across both fine-
tuning settings. In contrast, extended-label and
extended-label-online require a higher computa-
tional cost to reach a comparable level of accuracy. These
results indicate the better efficiency of the self-consistency
training through the amortization effect.
Performance of Final Results. At the end of the accuracy-
cost curves in Fig. 4, computational resource is sufficient
to generate full extended labels for supervised training in
the OOD scenario, which has the most abundant and direct
supervision information, hence serves as an upper bound
of Hamiltonian prediction performance. But as shown in
Table 3, this only applies to the Hamiltonian MAE, cor-
responding to the directly supervised quantity, while self-
consistency training still excels at derived physical quan-
tities, especially on ϵHOMO ,ϵLUMO andϵ∆, which are di-
rectly concerned molecular properties thus more relevant to
practical applications. Appendix D.1 shows a similar obser-
7

Self-Consistency Training for Density-Functional-Theory Hamiltonian Prediction
Table 3. Performance comparison between self-consistency training, and supervised training using full extended labels , in the OOD
scenario, corresponding to the ending points of Fig. 4 ( extended-label-online is close to extended-label ).
FT mode Setting H[µEh]↓ϵ[µEh]↓C[\%]↑ϵHOMO [µEh]↓ϵLUMO [µEh]↓ϵ∆[µEh]↓SCF Accel. [\%]↓
all-paramextended-label 62.13 365.66 96.89 577.46 5962.16 6137.66 65.0
self-con 65.74 375.31 97.31 565.50 1130.55 1316.96 64.5
adapterextended-label 59.67 330.05 96.63 541.92 6372.12 6445.33 65.2
self-con 64.48 268.83 97.12 449.80 1220.54 1394.29 65.0
Table 4. Efficiency comparison between self-consistency training
andconventional DFT for solving MD17 molecular structures.
Computation times under the same stopping criteria are shown
for solving the unlabeled molecular structures in the data-scarce
scenario.
Molecule criterion [µEh]tself-con [s] tDFT[s]
Ethanol 31.0 4.50×1046.40×104
Malondialdehyde 88.9 4.81×1041.05×105
Uracil 177.2 1.23×1052.15×105
vation in the data-scarce scenario. This indicates that even
with unlimited computational resource, self-consistency
training could still be preferred than generating labels to
better support real applications.
Direct Acceleration over DFT Calculation. As men-
tioned at the end of Sec. 2.4, self-consistency training is
also a way to directly accelerate DFT calculation on a large
amount of molecular structures by leveraging its amortiza-
tion effect. To demonstrate the advantage empirically, we
compare the computation time of self-consistency training
(using Eq. (4))tself-con and of DFT calculation tDFTfor solv-
ing the unlabeled molecular structures in the data-scarce
scenario (see Sec. 3.1). The same stopping criterion is ap-
plied to both methods in each case, which is taken as the
error of electronic energy (Eq. (A.24) ) derived from the
Hamiltonian following the convention in DFT calculation.
Results are shown in Table 4. We see that self-consistency
training indeed requires lower computational cost than DFT
to reach the same level of accuracy, demonstrating the prac-
tical benefit of amortization. Appendix D.4 shows more
implementation details.
3.3. Self-Consistency Training Extends the Scale of
Hamiltonian Prediction
After verifying the advantages of self-consistency training,
we now wield this powerful tool to extend Hamiltonian
prediction to molecules larger than previously reported in
the field, hence enhancing the relevance to real applications.
Extension to a Larger Scale. For molecules larger than
those covered by Hamiltonian prediction previously, we
consider two molecules in the MD22 dataset (Chmielaet al., 2023): Ac-Ala3-NHMe (ALA3, 42 atoms) and DHA
(56 atoms). Both molecules exceed the size of the largest
molecule in the QH9 dataset (31 atoms) with a significant
gap. Due to the extensive DFT cost, we generate labels for
each molecule only on 500 randomly selected structures
from MD22, which are used only for evaluation. The model
is pre-trained on nearly all QH9 molecules (the QH9-full set-
ting in Table C.2), then got all-param fine-tuned using
the self-consistency loss (Eq. (3)) on the selected structures
but without using the labels. For ease of training, we employ
the MINAO initialization (Sun et al., 2018) as a base Hamil-
tonian and let the model predict the residual correction.
The results are presented in Table 5. Compared to the previ-
ously best setting zero-shot which can only be directly
used right after pre-trained on QH9, fine-tuning with self-
consistency substantially improves the performance on the
two large molecules, with the MAE of ϵLUMO andϵ∆re-
duced by an order, and at least 3x less for other properties.
We note the inadequate performance of zero-shot is not
due to insufficient training on QH9, since the validation
Hamiltonian MAE of 29.06 µEhis sufficiently low. The
performance gap is due to the substantial scale gap between
QH9 and MD22. This gap indicates generalization to larger-
scale molecules is highly challenging. Remarkably, self-
consistency training breaks the data limitation and achieves
a significant performance improvement. Moreover, when
considering the acceleration benefit for SCF, zero-shot
prediction only brings a limited acceleration and even re-
sults in deceleration (on DHA). In contrast, self-consistency
training consistently achieves more significant SCF accel-
eration. Appendix D.3 presents SCF acceleration results
under the SOSCF iteration strategy, where the improvement
by self-consistency training is even more significant.
Comparison to End-to-End Property Predictors. The
conventional paradigm for molecular property prediction is
to predict each property with a devoted model in an end-to-
end (e2e) manner (Sch ¨utt et al., 2018; Gasteiger et al., 2020;
Th¨olke \& De Fabritiis, 2021; Liao \& Smidt, 2022), while
Hamiltonian prediction offers the advantage to provide all
properties that DFT can provide using a single model. More-
over, self-consistency training differentiates Hamiltonian
prediction from other property prediction tasks in that it
enables continued improvement of Hamiltonian prediction
8

Self-Consistency Training for Density-Functional-Theory Hamiltonian Prediction
Table 5. Generalization to larger-scale molecules (from MD22) than previously reported in Hamiltonian prediction, with comparison to
generalization results of e2e property predictors. Models are pretrained on QH9-full with labels and directly evaluated on the molecules
(zero-shot ,e2e), or, for the Hamiltonian model, after fine-tuned by self-con sistency without labels. Self-consistency training
enables meaningful prediction on the larger molecules by bridging generalization gap, and significantly outperforms e2e predictors in
molecular properties.
Molecule Setting H[µEh]↓ϵ[µEh]↓C[\%]↑ϵHOMO [µEh]↓ϵLUMO [µEh]↓ϵ∆[µEh]↓SCF Accel. [\%]↓
ALA3zero-shot 237.71 6.54 ×10352.24 6.90 ×1039.51×1049.79×10484.6
self-con 52.49 1.22 ×10394.46 2.07 ×1033.76×1032.69×10364.7
e2e (ET) N/A N/A N/A 1.74 ×1057.72×1032.38×105N/A
e2e (Equiformer) N/A N/A N/A 2.38 ×1051.16×1042.27×105N/A
DHAzero-shot 397.87 1.84 ×10420.15 1.11 ×1041.90×1051.85×105170.8
self-con 56.12 1.81 ×10383.51 1.99 ×1034.01×1032.34×10367.0
e2e (ET) N/A N/A N/A 2.92 ×1052.58×1043.39×105N/A
e2e (Equiformer) N/A N/A N/A 3.76 ×1052.31×1044.17×105N/A
without labeled data. This continued improvement can even
spread to various molecular properties, which cannot be
achieved by e2e predictors without additional labeled data.
We showcase this unique merit in this scenario of gener-
alizing to larger-scale molecules, where the continued im-
provement could bridge the generalization gap. We compare
the properties ϵHOMO ,ϵLUMO andϵ∆derived from the pre-
dicted Hamiltonian with the direct prediction results by
the respective e2e predictors. Same as the Hamiltonian
predictor, the e2e property predictors are trained with la-
bels on nearly all QH9 molecules, but they do not have a
self-consistency training strategy hence can only be directly
applied to predict the respective properties on the much
larger molecules. We consider two representative imple-
mentations of the e2e predictors, using the ET (Th ¨olke
\& De Fabritiis, 2021) and the Equiformer (Liao \& Smidt,
2022) architectures.
Results are shown in Table 5. They indicate that the e2e
predictors significantly suffer from the generalization gap
when comparing the evaluated error to the validation error
(Table D.3) which are around 1 ×103µEh. The Hamilto-
nian predictor can also predict such properties as derived
from the predicted Hamiltonian. Even applied right af-
ter pre-training ( zero-shot ), the Hamiltonian predictor
can already provide better results than e2e predictors on
ϵHOMO andϵ∆. Using the unique lever of self-consistency
training, the Hamiltonian predictor provides much more ac-
curate results on all three properties, with one to two orders
less MAE than e2e predictors. These results indicate that
self-consistency training offers a promising avenue towards
improving OOD prediction for molecular properties in a
label-free manner.

\section{Conclusion}

We have presented self-consistency training for Hamiltonian
prediction, a novel training method that does not requirelabeled data. This is a unique advantage of Hamiltonian
prediction. As the self-consistency loss is designed to en-
force the basic equation of DFT, it provides complete and
exact information of the prediction target. Self-consistency
training opens the access to gain supervision from vastly
available unlabeled data, which substantially solves the data-
scarce problem and allows generalization to challenging
domains. We have also pointed out and empirically verified
that self-consistency training is more efficient than running
DFT to generate data for supervised learning, benefited
from its amortization effect. Using self-consistency training,
we have pushed Hamiltonian prediction to solve molecules
larger than ever reported.
More broadly, since Hamiltonian matrix can derive rich
molecular properties ( e.g., energy, HOMO-LUMO gap), the
self-consistency training can also improve the prediction of
these properties without labeled data, and can even supervise
end-to-end prediction models. Since labeled data in the
science domain in general is much less than in conventional
AI domains, and generalization is more challenging due to
the flexibility in the input, this way to leverage fundamental
physical laws to train the model would be especially helpful.
Acknowledgements
We thank Lin Huang, Han Yang, and Yue Wang for insight-
ful discussions on the idea and techniques; Erpai Luo for dis-
cussions on model design and help with dataset preparation;
Jan Hermann, Michael Gastegger and Sebastian Ehlert for
suggestions on evaluation and writing; Tao Qin, Jia Zhang
and Huanhuan Xia for constructive feedback. Additionally,
we acknowledge Lin Huang, Han Yang, and Jia Zhang for
providing their CuDA implementation of Hamiltonian con-
struction. We also thank anonymous reviewers and area
chair for their feedback. He Zhang and Nanning Zheng
were supported by the National Natural Science Foundation
of China under Grant 62088102.
9

Self-Consistency Training for Density-Functional-Theory Hamiltonian Prediction
Impact Statement
This paper presents work whose goal is to advance the field
of Machine Learning. There are many potential societal
consequences of our work, none which we feel must be
specifically highlighted here.
References
Becke, A. D. Density-functional thermochemistry. I. the
effect of the exchange-only gradient correction. The
Journal of chemical physics , 96(3):2155–2160, 1992.
Becke, A. D. Density-functional thermochemistry. III. The
role of exact exchange. The Journal of Chemical Physics ,
98, 1993. ISSN 00219606. doi: 10.1063/1.464913.
Cances, E. and Le Bris, C. On the convergence of SCF
algorithms for the Hartree-Fock equations. ESAIM: Math-
ematical Modelling and Numerical Analysis , 34(4):749–
774, 2000.
Chanussot, L., Das, A., Goyal, S., Lavril, T., Shuaibi, M.,
Riviere, M., Tran, K., Heras-Domingo, J., Ho, C., Hu, W.,
et al. Open catalyst 2020 (OC20) dataset and community
challenges. Acs Catalysis , 11(10):6059–6072, 2021.
Chen, Y ., Zhang, L., Wang, H., and E, W. DeePKS: A
comprehensive data-driven approach toward chemically
accurate density functional theory. Journal of Chemical
Theory and Computation , 17(1):170–181, 2021.
Chmiela, S., Sauceda, H. E., Poltavsky, I., M ¨uller, K.-R.,
and Tkatchenko, A. sGDML: Constructing accurate and
data efficient molecular force fields using machine learn-
ing. Computer Physics Communications , 240:38–45,
2019.
Chmiela, S., Vassilev-Galindo, V ., Unke, O. T., Kabylda,
A., Sauceda, H. E., Tkatchenko, A., and M ¨uller, K.-
R. Accurate global machine learning force fields for
molecules with hundreds of atoms. Science Advances , 9
(2):eadf0873, 2023.
del Mazo-Sevillano, P. and Hermann, J. Variational principle
to regularize machine-learned density functionals: the
non-interacting kinetic-energy functional. arXiv preprint
arXiv:2306.17587 , 2023.
Dick, S. and Fernandez-Serra, M. Machine learning accu-
rate exchange and correlation functionals of the electronic
density. Nature communications , 11(1):3509, 2020.
Ditchfield, R., Hehre, W. J., and Pople, J. A. Self-consistent
molecular-orbital methods. ix. an extended gaussian-type
basis for molecular-orbital studies of organic molecules.
The Journal of Chemical Physics , 54(2):724–728, 1971.Dunlap, B. I. Robust and variational fitting. Phys.
Chem. Chem. Phys. , 2:2113–2116, 2000. doi: 10.
1039/B000027M. URL http://dx.doi.org/10.
1039/B000027M .
Dunning Jr, T. H. Gaussian basis sets for use in correlated
molecular calculations. i. the atoms boron through neon
and hydrogen. The Journal of chemical physics , 90(2):
1007–1023, 1989.
Fey, M. and Lenssen, J. E. Fast graph representa-
tion learning with PyTorch Geometric. arXiv preprint
arXiv:1903.02428 , 2019.
Gastegger, M., McSloy, A., Luya, M., Sch ¨utt, K. T., and
Maurer, R. J. A deep neural network for molecular wave
functions in quasi-atomic minimal basis representation.
The Journal of Chemical Physics , 153(4), 2020.
Gasteiger, J., Groß, J., and G ¨unnemann, S. Directional
message passing for molecular graphs. arXiv preprint
arXiv:2003.03123 , 2020.
Gong, X., Li, H., Zou, N., Xu, R., Duan, W., and Xu, Y .
General framework for E(3)-equivariant neural network
representation of density functional theory Hamiltonian.
Nature Communications , 14(1):2848, 2023.
Gu, Q., Zhang, L., and Feng, J. Neural network represen-
tation of electronic structure from ab initio molecular
dynamics. Science Bulletin , 67(1):29–37, 2022.
Hegde, G. and Bowen, R. C. Machine-learned approxima-
tions to density functional theory hamiltonians. Scientific
reports , 7(1):42669, 2017.
Hellweg, A. and Rappoport, D. Development of new auxil-
iary basis functions of the karlsruhe segmented contracted
basis sets including diffuse basis functions (def2-svpd,
def2-tzvppd, and def2-qvppd) for ri-mp2 and ri-cc cal-
culations. Physical Chemistry Chemical Physics , 17(2):
1010–1017, 2015.
Hohenberg, P. and Kohn, W. Inhomogeneous electron gas.
Physical review , 136(3B):B864, 1964.
Imoto, F., Imada, M., and Oshiyama, A. Order-N orbital-
free density-functional calculations with machine learn-
ing of functional derivatives for semiconductors and met-
als.Physical Review Research , 3(3):033198, 2021.
Ionescu, C., Vantzos, O., and Sminchisescu, C. Matrix
backpropagation for deep networks with structured layers.
InProceedings of the IEEE international conference on
computer vision , pp. 2965–2973, 2015.
Jain, A., Ong, S. P., Hautier, G., Chen, W., Richards, W. D.,
Dacek, S., Cholia, S., Gunter, D., Skinner, D., Ceder,
10

Self-Consistency Training for Density-Functional-Theory Hamiltonian Prediction
G., and Persson, K. A. Commentary: The Materials
Project: A materials genome approach to accelerating
materials innovation. APL Materials , 1(1):011002, 07
2013. ISSN 2166-532X. doi: 10.1063/1.4812323. URL
https://doi.org/10.1063/1.4812323 .
Jensen, F. Polarization consistent basis sets: Principles. The
Journal of Chemical Physics , 115(20):9113–9125, 2001.
Jordaan, M. A., Ebenezer, O., Damoyi, N., and Shapi, M.
Virtual screening, molecular docking studies and dft cal-
culations of fda approved compounds similar to the non-
nucleoside reverse transcriptase inhibitor (nnrti) efavirenz.
Heliyon , 6(8), 2020.
Kirkpatrick, J., McMorrow, B., Turban, D. H., Gaunt, A. L.,
Spencer, J. S., Matthews, A. G., Obika, A., Thiry, L.,
Fortunato, M., Pfau, D., et al. Pushing the frontiers
of density functionals by solving the fractional electron
problem. Science , 374(6573):1385–1389, 2021.
Kohn, W. and Sham, L. J. Self-consistent equations includ-
ing exchange and correlation effects. Physical review ,
140(4A):A1133, 1965.
Kudin, K. N., Scuseria, G. E., and Canc `es, E. A black-
box self-consistent field convergence algorithm: One
step closer. The Journal of Chemical Physics , 116
(19):8255–8261, 04 2002. ISSN 0021-9606. doi:
10.1063/1.1470195. URL https://doi.org/10.
1063/1.1470195 .
Levy, M. Universal variational functionals of electron densi-
ties, first-order density matrices, and natural spin-orbitals
and solution of the v-representability problem. Proceed-
ings of the National Academy of Sciences , 76(12):6062–
6065, 1979.
Li, H., Wang, Z., Zou, N., Ye, M., Xu, R., Gong, X., Duan,
W., and Xu, Y . Deep-learning density functional theory
Hamiltonian for efficient ab initio electronic-structure
calculation. Nature Computational Science , 2(6):367–
377, 2022.
Liao, Y .-L. and Smidt, T. Equiformer: Equivariant graph
attention transformer for 3d atomistic graphs. arXiv
preprint arXiv:2206.11990 , 2022.
Lieb, E. H. Density functionals for Coulomb systems. Inter-
national Journal of Quantum Chemistry , 24(3):243–277,
1983.
Nigam, J., Willatt, M. J., and Ceriotti, M. Equivariant
representations for molecular Hamiltonians and N-center
atomic-scale properties. The Journal of Chemical Physics ,
156(1), 2022.Paszke, A., Gross, S., Massa, F., Lerer, A., Bradbury, J.,
Chanan, G., Killeen, T., Lin, Z., Gimelshein, N., Antiga,
L., et al. Pytorch: An imperative style, high-performance
deep learning library. Advances in neural information
processing systems , 32, 2019.
Perdew, J. P., Burke, K., and Ernzerhof, M. Generalized
gradient approximation made simple. Physical review
letters , 77(18):3865, 1996.
Perdew, J. P., Ruzsinszky, A., Csonka, G. I., Vydrov, O. A.,
Scuseria, G. E., Constantin, L. A., Zhou, X., and Burke,
K. Restoring the density-gradient expansion for exchange
in solids and surfaces. Physical review letters , 100(13):
136406, 2008.
Pulay, P. Convergence acceleration of iterative sequences.
the case of scf iteration. Chemical Physics Letters , 73(2):
393–398, 1980.
Pulay, P. Improved SCF convergence acceleration. Journal
of Computational Chemistry , 3(4):556–560, 1982.
doi: https://doi.org/10.1002/jcc.540030413. URL
https://onlinelibrary.wiley.com/doi/
abs/10.1002/jcc.540030413 .
Ramakrishnan, R., Dral, P. O., Rupp, M., and V on Lilienfeld,
O. A. Quantum chemistry structures and properties of
134 kilo molecules. Scientific data , 1(1):1–7, 2014.
Remme, R., Kaczun, T., Scheurer, M., Dreuw, A., and Ham-
precht, F. A. KineticNet: Deep learning a transferable
kinetic energy functional for orbital-free density func-
tional theory. arXiv preprint arXiv:2305.13316 , 2023.
Sch¨utt, K. T., Sauceda, H. E., Kindermans, P.-J.,
Tkatchenko, A., and M ¨uller, K.-R. Schnet–a deep learn-
ing architecture for molecules and materials. The Journal
of Chemical Physics , 148(24), 2018.
Sch¨utt, K. T., Gastegger, M., Tkatchenko, A., M ¨uller, K.-
R., and Maurer, R. J. Unifying machine learning and
quantum chemistry with a deep neural network for molec-
ular wavefunctions. Nature communications , 10(1):5024,
2019.
Seminario, J. M. Recent developments and applications of
modern density functional theory. 1996.
Shmilovich, K., Willmott, D., Batalov, I., Kornbluth, M.,
Mailoa, J., and Kolter, J. Z. Orbital Mixer: Using atomic
orbital features for basis-dependent prediction of molec-
ular wavefunctions. Journal of Chemical Theory and
Computation , 18(10):6021–6030, 2022.
Slater, J. C. A simplification of the Hartree-Fock method.
Physical review , 81(3):385, 1951.
11

Self-Consistency Training for Density-Functional-Theory Hamiltonian Prediction
Snyder, J. C., Rupp, M., Hansen, K., M ¨uller, K.-R., and
Burke, K. Finding density functionals with machine
learning. Physical review letters , 108(25):253002, 2012.
Song, Y ., Sebe, N., and Wang, W. Why approximate matrix
square root outperforms accurate SVD in global covari-
ance pooling? In Proceedings of the IEEE/CVF Interna-
tional Conference on Computer Vision , pp. 1115–1123,
2021.
Stephens, P. J., Devlin, F. J., Chabalowski, C. F., and Frisch,
M. J. Ab initio calculation of vibrational absorption and
circular dichroism spectra using density functional force
fields. The Journal of physical chemistry , 98(45):11623–
11627, 1994.
Sun, Q. Co-iterative augmented hessian method for orbital
optimization. arXiv preprint arXiv:1610.08423 , 2016.
Sun, Q., Yang, J., and Chan, G. K.-L. A general second
order complete active space self-consistent-field solver
for large-scale systems. Chemical Physics Letters , 683:
291–299, 2017.
Sun, Q., Berkelbach, T. C., Blunt, N. S., Booth, G. H., Guo,
S., Li, Z., Liu, J., McClain, J. D., Sayfutyarova, E. R.,
Sharma, S., et al. PySCF: the python-based simulations
of chemistry framework. Wiley Interdisciplinary Reviews:
Computational Molecular Science , 8(1):e1340, 2018.
Teale, A. M., Helgaker, T., Savin, A., Adamo, C., Aradi,
B., Arbuznikov, A. V ., Ayers, P. W., Baerends, E. J.,
Barone, V ., Calaminici, P., et al. DFT exchange: sharing
perspectives on the workhorse of quantum chemistry and
materials science. Physical chemistry chemical physics ,
24(47):28700–28781, 2022.
Th¨olke, P. and De Fabritiis, G. Equivariant transformers for
neural network based molecular potentials. In Interna-
tional Conference on Learning Representations , 2021.
Unke, O., Bogojeski, M., Gastegger, M., Geiger, M., Smidt,
T., and M ¨uller, K.-R. SE(3)-equivariant prediction of
molecular wavefunctions and electronic densities. Ad-
vances in Neural Information Processing Systems , 34:
14434–14447, 2021.
Wang, W., Dang, Z., Hu, Y ., Fua, P., and Salzmann, M.
Backpropagation-friendly eigendecomposition. Advances
in Neural Information Processing Systems , 32, 2019.
Wang, W., Dang, Z., Hu, Y ., Fua, P., and Salzmann, M.
Robust differentiable svd. IEEE Transactions on Pattern
Analysis and Machine Intelligence , 44(9):5472–5487,
2021a.Wang, Y . A., Govind, N., and Carter, E. A. Orbital-
free kinetic-energy density functionals with a density-
dependent kernel. Physical Review B , 60(24):16350,
1999.
Wang, Z., Ye, S., Wang, H., He, J., Huang, Q., and Chang,
S. Machine learning method for tight-binding hamilto-
nian parameterization from ab-initio band structure. npj
Computational Materials , 7(1):11, 2021b.
Witt, W. C., Beatriz, G., Dieterich, J. M., and Carter, E. A.
Orbital-free density functional theory for materials re-
search. Journal of Materials Research , 33(7):777–795,
2018.
Yin, S., Zhu, X., Gao, T., Zhang, H., Wu, F., and He,
L. Harmonizing covariance and expressiveness for deep
Hamiltonian regression in crystalline material research:
a hybrid cascaded regression framework. arXiv preprint
arXiv:2401.00744 , 2024.
Yu, H., Liu, M., Luo, Y ., Strasser, A., Qian, X., Qian, X., and
Ji, S. QH9: A quantum hamiltonian prediction benchmark
for QM9 molecules. arXiv preprint arXiv:2306.09549 ,
2023a.
Yu, H., Xu, Z., Qian, X., Qian, X., and Ji, S. Efficient
and equivariant graph networks for predicting quantum
Hamiltonian. arXiv preprint arXiv:2306.04922 , 2023b.
Zhang, H., Liu, S., You, J., Liu, C., Zheng, S., Lu, Z., Wang,
T., Zheng, N., and Shao, B. Overcoming the barrier of
orbital-free density functional theory for molecular sys-
tems using deep learning. Nature Computational Science ,
pp. 1–14, 2024.
Zhang, L., Onat, B., Dusson, G., McSloy, A., Anand, G.,
Maurer, R. J., Ortner, C., and Kermode, J. R. Equivari-
ant analytical mapping of first principles Hamiltonians to
accurate and transferable materials models. Npj Compu-
tational Materials , 8(1):158, 2022.
Zhong, Y ., Yu, H., Su, M., Gong, X., and Xiang, H. Transfer-
able equivariant graph neural networks for the hamiltoni-
ans of molecules and solids. npj Computational Materials ,
9(1):182, 2023.
12

Self-Consistency Training for Density-Functional-Theory Hamiltonian Prediction
A. Brief Introduction to Density Functional Theory
A.1. Background of Electronic Structure Methods
All properties of a molecule is determined by the result of interaction among the electrons and nuclei in the molecule. As
nuclei are much heavier than electrons, they are typically treated as classical particles, while the electrons are governed
by the Schr ¨odinger equation. Therefore, the state of the Anuclei is specified by the molecular structure M:=\{R,Z\},
where Z:=\{Z(a)\}A
a=1andR:=\{R(a)\}A
a=1specifies the atomic numbers (species) and coordinates of the nuclei, while
the state of the Nelectrons is specified by the wavefunction ψ(r(1),···,r(N)). The squared modulus of the wavefunctionψ(r(1),···,r(N))2represents the joint distribution of the Nelectrons. Since electrons are indistinguishable, the density
functionψ(r(1),···,r(N))2is permutation symmetric, hence the wavefunction ψ(r(1),···,r(N))is permutation sym-
metric or antisymmetric, i.e., it keeps or changes sign ( i.e., phase change of 0 or π) when the coordinates of two particles
are exchanged. For electrons, their statistical behavior indicates that their wavefunction is antisymmetric (electrons are
fermions).
Commonly, only the stationary states of electrons in a given molecular structure Mare concerned, since the evolution of
electrons is much faster than the motion of nuclei so their state instantly becomes stationary for any given molecular structure.
The stationary states are determined by the stationary Schr ¨odinger equation: ˆHMψ=Eψ,i.e., they are eigenstates of
the Hamiltonian operator ˆHM. The Hamiltonian operator ˆHM:=ˆT+ˆVee+ˆVext,Mis composed of the kinetic energy
operator ˆTψ:=−1
2PN
i=1∇2
r(i)ψ(atomic units are used throughout), the internal potential energy operator among electrons
ˆVeeψ:=P
1⩽i<j⩽N1
∥r(i)−r(j)∥ψ, and the external potential energy operator ˆVext,Mψ:=PN
i=1Vext,M(r(i))ψwhere
Vext,M(r) :=−PA
a=1Z(a)
∥r−R(a)∥is the external potential generated by the nuclei. Note that the latter two operators are
multiplicative, i.e., their action on a wavefunction is the multiplication with the corresponding potential function. The
Hamiltonian operator is Hermitian, hence its eigenvalues are always real. Moreover, since this Hamiltonian operator is also
real (in physics term, it is time-reversible) (particularly, it does not involve magnetic fields or spin-orbital coupling), every
eigenstate of it has a real-valued eigenfunction. Hence from now on, it suffices to only consider real-valued wavefunctions.
Solving an eigenvalue problem is challenging especially when Nis large. On the other hand, most of the concerned
properties of a molecule only involve the ground state, i.e., the eigenstate with the lowest eigenvalue (energy). Hence an
alternative form to solve the electronic ground state can be composed as an optimization problem, known as the variational
formulation:
E⋆
M= min
ψ:antisym ,⟨ψ|ψ⟩=1⟨ψ|ˆHM|ψ⟩, (A.1)
where ⟨ψ|ϕ⟩denotes the integral of ψ∗ϕw.r.t all their arguments, and ⟨ψ|ˆHM|ϕ⟩:=⟨ψ|ˆHMϕ⟩(which is also ⟨ˆHMψ|ϕ⟩
since ˆHMis Hermitian). Various ways to parameterize the wavefunction ψand estimate and optimize the energy are
proposed. Regardless, this formulation optimizes a function on R3N, whose complexity may increase exponentially w.r.t
system size N, limiting the scale of practically applicable systems.
Before going on, we note that although wavefunctions are in general complex-valued, it is sufficient to only consider
real-valued wavefunctions for solving static ( i.e., no time evolution) electronic state without magnetic fields and ignoring
spin-orbital coupling, since the Hamiltonian operator ˆHMin such cases are not only Hermitian but also time-reversible
(meaning that ˆHMis “real”, ˆH∗
M=ˆHM), each of whose eigenstate has a real-valued eigenfunction. For this reason, we
only consider real-valued functions (including orbitals and basis functions), and do not distinguish matrix transpose and
Hermitian conjugate.
A.2. Basic Idea of DFT
Density functional theory (DFT) is motivated to address the exponentially complex optimization space. It aims to optimize
the (one-electron reduced) density ρ(r), a function on a fixed-dimensional space R3. It is a reduced quantity to describe
the electronic state. The electron density corresponding to the electronic state specified by wavefunction ψis the marginal
distribution (up to a factor of the number of electrons) of the joint distribution:
ρ[ψ](r) :=NZ
|ψ(r,r(2),···,r(N))|2dr(2)···dr(N), (A.2)
13

Self-Consistency Training for Density-Functional-Theory Hamiltonian Prediction
which is independent of the variable for which the marginalization is conducted due to the indistinguishability. It is how one
straightforwardly perceives electron density, which is a valid concept also under the classical view.
Now the question is, whether optimizing the density is sufficient to determine the electronic ground state, considering that
the density is only a reduced quantity. This is first answered affirmatively in the seminal work by Hohenberg \& Kohn
(1964), but it would be more explicit to deduce the answer following Levy’s constrained search formulation (Levy, 1979) of
Eq. (A.1):
E⋆
M= min
ψ:antisym ,⟨ψ|ψ⟩=N⟨ψ|ˆHM|ψ⟩= min
ρ:⩾0,⟨1|ρ⟩=N
min
ψ:antisym ,ρ[ψ]=ρ⟨ψ|ˆHM|ψ⟩
, (A.3)
where 1denotes the constant 1-valued function. Note that when viewing minψ:antisym ,ρ[ψ]=ρ⟨ψ|ˆHM|ψ⟩as a functional of
density ρ, the optimization problem in Eq. (A.3) indicates that the ground-state energy and density can indeed be solved by
optimizing the density. Among the three components of ˆHM, the ˆVext,Mterm already makes a density functional, since
⟨ψ|ˆVext,M|ψ⟩=PN
i=1R
Vext,M(r(i))ψ(r(1),···,r(N))2dr(1)···dr(N)=1
NPN
i=1R
Vext,M(r(i))ρ[ψ](r(i)) dr(i)=
⟨Vext,M|ρ[ψ]⟩is independent of ψonceρ[ψ]is fixed. So Eq. (A.3) can be formulated as:
E⋆
M= min
ρ:⩾0,⟨1|ρ⟩=N
min
ψ:antisym ,ρ[ψ]=ρ⟨ψ|ˆT+ˆVee|ψ⟩
| \{z \}
=:F[ρ]+⟨Vext,M|ρ⟩,
where F[ρ]is called the universal functional comprising the kinetic and internal potential energy minimally attainable for
the given density ρ. Its name follows the fact that it does not depend on molecular structure Mand applies to any system.
As the universal functional is still quite implicit to carry out practical calculation, approximations are considered to cover
the major part of the kinetic energy and of the internal potential energy. For the latter, the classical internal potential energy
can be used, which ignores electron correlation and adopts an explicit expression in terms of ρ(r):
EH[ρ] :=1
2Zρ(r)ρ(r′)
∥r−r′∥drdr′. (A.4)
It is also called the Hartree energy, hence the notation. For the kinetic part, the kinetic energy density functional (KEDF) is
introduced following a similar formulation as the definition of the universal functional:
TS[ρ] := min
ψ:antisym ,ρ[ψ]=ρ⟨ψ|ˆT|ψ⟩. (A.5)
The rest part of the kinetic and internal potential energy is called the exchange-correlation (XC) energy:
EXC[ρ] :=F[ρ]−TS[ρ]−EH[ρ],
and the variational problem to solve the electronic ground state of molecule in structure Mbecomes:
E⋆
M= min
ρ:⩾0,⟨1|ρ⟩=NTS[ρ] +EH[ρ] +EXC[ρ] +⟨Vext,M|ρ⟩. (A.6)
Although the exact expression of EXCin terms of ρis still unknown, it makes only a minor part of the total electronic
energy and is more flexible to approximate. Over the past decades, researchers have developed many successful approxima-
tions (Becke, 1993; Stephens et al., 1994; Perdew et al., 1996; 2008). Deep learning has also been leveraged for developing
an approximation (Dick \& Fernandez-Serra, 2020; Chen et al., 2021; Kirkpatrick et al., 2021). As for the KEDF, there are
methods that also directly approximate the density functional (Slater, 1951; Wang et al., 1999; Witt et al., 2018), which are
now called orbital-free density functional theory. Nevertheless, approximating KEDF is harder and requires higher accuracy,
since it accounts for a major part of energy. It is also an active research direction to leverage machine learning models to
approximate the functional more accurately (Snyder et al., 2012; Imoto et al., 2021; Remme et al., 2023; del Mazo-Sevillano
\& Hermann, 2023; Zhang et al., 2024).
A.3. Kohn-Sham DFT
Considering the difficulty of directly approximating the KEDF, Kohn \& Sham (1965) exploited properties of the KEDF and
developed a method that evaluates the kinetic energy directly. Note that in the optimization of the definition of KEDF in
14

Self-Consistency Training for Density-Functional-Theory Hamiltonian Prediction
Eq.(A.5) , there is no interaction among electrons ( ˆToperates one-body-wise). It is known that the ground-state wavefunction
solution of non-interacting systems is in the form of a determinant (at least in absense of degeneracy (Lieb, 1983, Thm. 4.6)),
which, instead of a general function on R3N, is composed of Nfunctions Φ :=\{ϕi(r)\}N
i=1onR3called orbitals:
ψ[Φ](r(1),···,r(N)) :=1√
N!det[ϕi(r(j))]ij. (A.7)
The optimization problem in the definition of KEDF in Eq. (A.5) can be equivalently formulated as:3
TS[ρ] = min
\{ϕi\}N
i=1:ρ[ψ[Φ]]=ρ⟨ψ[Φ]|ˆT|ψ[Φ]⟩= min
\{ϕi\}N
i=1:orthonormal ,
ρ[ψ[Φ]]=ρNX
i=1⟨ϕi|ˆT|ϕi⟩, (A.8)
where the second expression to optimize orthonormal orbitals, ⟨ϕi|ϕj⟩=δij, is valid since any set of functions can be
orthonormalized by e.g., the Gram-Schmidt process without changing the corresponding density and kinetic energy. This is
desired to simplify calculation, for which the kinetic energy calculation is simplified in Eq. (A.8) , and the density (Eq. (A.2) )
substituted by Eq. (A.7) can also be simplified as:
ρ[ψ[Φ]](r) =NX
i=1|ϕi(r)|2.
Using this simplified formulation Eq. (A.8) , the original optimization problem Eq. (A.6) for solving the electronic structure
given molecular structure Mbecomes:
E⋆
M= min
ρ:⩾0,⟨1|ρ⟩=N
min
\{ϕi\}N
i=1:orthonormal ,
ρ[ψ[Φ]]=ρNX
i=1⟨ϕi|ˆT|ϕi⟩
+EH[ρ] +EXC[ρ] +⟨Vext,M|ρ⟩
= min
\{ϕi\}N
i=1:
orthonormal
EM[Φ] :=NX
i=1⟨ϕi|ˆT|ϕi⟩+EH[ρ[ψ[Φ]]] +EXC[ρ[ψ[Φ]]] +⟨Vext,M|ρ[ψ[Φ]]⟩
. (A.9)
In this way, the query for directly evaluating TS[ρ]is avoided by an exact estimation using the orbitals. This formulation
optimizes Nfunctions on R3instead of one function on R3, hence the complexity is increased by at least an order of N.
This formulation is called the Kohn-Sham DFT, and has become the default DFT formulation due to its success to solve
molecular system problems computationally (Seminario, 1996; Jain et al., 2013).
To solve Eq. (A.9) , standard DFT solves the equation of optimality, which is derived by taking the variation of EM[Φ]w.r.t
each orbital under the orthonormality constraint. The variation of EM[Φ]is:
δEM[Φ]
δϕi(r) =δPN
j=1(−1
2)⟨ϕj|∇2|ϕj⟩
δϕi(r) +Zδ
EH[ρ] +EXC[ρ] +⟨Vext,M|ρ⟩
δρ(r′)
ρ=ρ[ψ[Φ]]δρ[ψ[Φ]](r′)
δϕi(r)dr′
= 2ˆTϕi(r) + 2 Zρ(r′)
∥r′−r∥dr′
|\{z \}
=:VH[ρ](r)+δEXC[ρ]
δρ(r)
|\{z\}
=:VXC[ρ](r)!
ρ=ρ[ψ[Φ]]ϕi(r) + 2Vext,M(r)ϕi(r). (A.10)
By introducing the (one-electron effective) Hamiltonian operator, or more commonly called the Fock operator in DFT,
ˆHM,[ρ]:=ˆT+ˆVH[ρ]+ˆVXC[ρ]+ˆVext,M, (A.11)
where the latter three operators act on a function by multiplying the function with the respective potential energy function,
the variation can be written as:
δEM[Φ]
δϕi(r) = 2 ˆHM,[ρ[ψ[Φ]]]ϕi. (A.12)
3Assume the queried density ρcomes from the set of densities of the ground state of all non-interacting systems. Although this set
still has ρthat violates the equivalence to Eq. (A.5) (Lieb, 1983, Thm. 4.8), the determinantal definition Eq. (A.8) still recovers the
ground-state energy if optimized on this set (Lieb, 1983, Thm. 4.9).
15

Self-Consistency Training for Density-Functional-Theory Hamiltonian Prediction
For the orthonormality constraint, first consider the normalization constraint and introduce Lagrange multipliers \{εi\}N
i=1for
them. The corresponding variation is:
δPN
j=1εj(⟨ϕj|ϕj⟩ −1)
δϕi(r) = 2 εiϕi(r),
which leads to the optimality equation:
ˆHM,[ρ[ψ[Φ]]]ϕi(r) =εiϕi(r),∀i= 1,···, N. (A.13)
This is known as the Kohn-Sham equation (in function form). From this equation, the optimal solution of orbitals are
eigenstates of the operator ˆHM,[ρ[ψ[Φ]]], which can be verified to be Hermitian. Hence, in the general case where there is no
degeneracy, different orbitals in the solution are naturally orthogonal, so there is no need to further enforce this constraint
explicitly.
A.4. Practical Calculation under a Basis
Vectorizing a function as the expansion coefficient vector on a basis function set is an effective and controllable way to
represent a function numerically. For molecules, as the electrons distribute around atoms in the molecule, commonly adopted
basis functions are atom-centered functions. To allow analytical calculation of integrals, the functions typically take a
Gaussian form for the radial variable ( i.e., the distance from the center nucleus of this basis function) multiplied with a
spherical harmonic function for the angular variables (or equivalently a monomial of the three coordinates) (Ditchfield et al.,
1971; Hellweg \& Rappoport, 2015; Dunning Jr, 1989; Jensen, 2001). Different chemical elements usually have different
sets of basis functions. To expand the orbitals in a molecule, the basis set is the union of basis functions centered at each of
the atoms in the molecule. We collectively label them with one index α, and denote them as \{ηM,α(r)\}B
α=1. The number of
basis functions Bfor a molecular system typically increases linearly with the number of electrons Nin the system.
The orbitals can then be represented as expansion coefficients C:
ϕi(r) =BX
α=1CαiηM,α(r). (A.14)
Next we show the derivation for the optimality equation for C. Given that orthonormality constraint is satisfied, the density
corresponding to the orbital state specified by Cis:
ρM,C(r) =X
α,βNX
i=1CαiCβiηM,α(r)ηM,β(r) =X
α,β(CC⊤)αβηM,α(r)ηM,β(r). (A.15)
The Kohn-Sham equation presented in Eq. (A.13) is turned intoP
αCαiˆHM,[ρM,C]ηM,α(r) =P
αεiCαiηM,α(r).
Integrating both sides with basis function ηM,β(r)gives:
HM(C)C=S Cε, (A.16)
where:

HM(C)
αβ:=⟨ηM,α|ˆHM,[ρM,C]|ηM,β⟩, (A.17)
is the Hamiltonian matrix ( ˆHM,[ρM,C]defined in Eq. (A.11)),
(SM)αβ:=⟨ηM,α|ηM,β⟩,
is the overlap matrix of the atomic basis, and
ε:= Diag( ε1,···, εN),
is a diagonal matrix comprising the eigenvalues. This is the matrix form of the Kohn-Sham equation Eq. (A.13) , as presented
in Eq. (1) in the main paper.
16

Self-Consistency Training for Density-Functional-Theory Hamiltonian Prediction
To solve Eq. (A.16) , conventional DFT calculation uses a fixed-point iteration process known as the self-consistent field
(SCF) iteration. At each iteration step k, the last orbital solution C(k−1)is used to construct the Hamiltonian matrix
H(k):=HM(C(k−1)), and the updated orbital solution C(k)for this step is derived by solving H(k)C=SCε. There are
variants that accelerate the iteration, e.g., the direct inversion in the iterative subspace (DIIS) method (Pulay, 1982; Kudin
et al., 2002), which constructs H(k)not only using HM(C(k−1))but also using Hamiltonian matrices from previous steps.
In contrast, our self-consistency approach (Eq. (3)) solves Eq. (A.16) directly, by minimizing the violation of the equal-
ity in terms of the Hamiltonian matrix,


ˆHθ(M)−HM
CMˆHθ(M)


2
F, where CMˆHθ(M)
denotes the orbital
coefficients solved from ˆHθ(M)C=SMCε.
A.5. Details to Construct the Hamiltonian Matrix
The definition of the Hamiltonian matrix is given by Eq. (A.17) as the product of the Hamiltonian operator on basis functions.
The operator is in turn defined by Eq. (A.11) and Eq. (A.10) , following which the Hamiltonian matrix can be computed
from the equations below:
HM(C) =TM+VH,M(C) +VXC,M(C) +Vext,M, (A.18)
where:
(TM)αβ:=⟨ηM,α|ˆT|ηM,β⟩=−1
2Z
ηM,α(r)∇2ηM,β(r) dr,
(VH,M(C))αβ:=⟨ηM,α|VH[ρM,C]|ηM,β⟩Eqs. (A.10, A.15)=X
γδ(˜DM)αβ,γδ (CC⊤)γδ, (A.19)
where (˜DM)αβ,γδ :=ZZηM,α(r)ηM,β(r)ηM,γ(r′)ηM,δ(r′)
∥r−r′∥dr′dr,
(VXC,M(C))αβ:=⟨ηM,α|VXC[ρM,C]|ηM,β⟩=Z
VXC[ρM,C](r)ηM,α(r)ηM,β(r) dr,
(Vext,M)αβ:=⟨ηM,α|Vext,M|ηM,β⟩=−AX
a=1Z(a)ZηM,α(r)ηM,β(r)

r−R(a)

dr. (A.20)
Under the mentioned type of basis functions \{ηM,α(r)\}B
α=1, integrals SM,TM,˜DM, andVext,Mcan be evaluated
analytically. To evaluate VXC,M(C), the integral can be evaluated on a quadrature grid, for which common XC functional
approximations provide a way to evaluate VXC[ρM,C]on each grid point.
Density Fitting Note that directly calculating VH,M(C)following Eq. (A.19) requires O(B4) =O(N4)complexity,
which soon dominates the cost and restricts the applicability to large systems. There is a widely adopted approach in
DFT to reduce the complexity for this term, called density fitting (Dunlap, 2000). It is motivated by noting VH[ρM,C](r)
as defined in Eq. (A.10) involves an integral with density function ρM,C(r), which, by Eq. (A.15) , involves a double
summation that incurs O(N2)cost. Eq. (A.15) can be seen as expanding the density function onto the paired basis
set\{ηM,α(r)ηM,β(r)\}α,β=1,···,Bof size B2. It is hence possible to reduce the complexity by projecting the density
function onto an auxiliary basis set \{ωM,µ(r)\}M
µ=1of size M=O(N). The projected density can be represented by the
corresponding coefficients pin the way that:
ρM,p(r) :=MX
µ=1pµωM,µ(r). (A.21)
The projection is done by finding the coefficients pthat minimizes the difference from ρM,p(r)toρM,C(r). Note that the
purpose of density fitting here is to reduce cost complexity for calculating VH,M, the operator matrix corresponding to the
Hartree energy defined in Eq. (A.4). Therefore, the difference is preferred to be measured in Hartree energy:
EH[ρM,p−ρM,C] =ZZ
ρM,p(r)−ρM,C(r)
ρM,p(r′)−ρM,C(r′)
∥r−r′∥drdr′
=p⊤˜WMp−2p⊤˜LMvec(CC⊤) + vec( CC⊤)⊤˜DMvec(CC⊤),
17

Self-Consistency Training for Density-Functional-Theory Hamiltonian Prediction
where vec(CC⊤)∈RB2denotes the vector of the flattened density matrix CC⊤∈RB×B, and the pre-computed constant
integral matrices are defined by: (˜WM)µν:=RRωM,µ(r)ωM,ν(r′)
∥r−r′∥drdr′,(˜LM)µ,αβ:=RRωM,µ(r)ηM,α(r′)ηM,β(r′)
∥r−r′∥drdr′,
and(˜DM)αβ,γδ :=RRηM,α(r)ηM,β(r)ηM,γ(r′)ηM,δ(r′)
∥r−r′∥drdr′, which can be computed analytically using common basis sets.
As a quadratic form, the solution is:
pM(C) := ˜W−1
M˜LMvec(CC⊤). (A.22)
Note that since the auxiliary basis is usually not complete to expand the paired basis, the projected density ρM,pM(C)is an
approximation to the original density ρM,C.
Using density fitting, the Hartree operator matrix VH,M(C)defined by Eq. (A.19) can be approximately estimated by
substituting the projected density ρM,pM(C)(r), given by Eq. (A.21) and Eq. (A.22) , into the Hartree potential VH[ρM,pM(C)]
defined by Eq. (A.10):
(VH,M(C))αβ≈
pM(C)⊤˜LM
αβ. (A.23)
Since the calculation of pM(C)following Eq. (A.22) has complexity O(M3) +O(MB2) +O(NB2) =O(N3), and the
complexity of Eq. (A.23) itself has complexity O(MB2) =O(N3), the overall complexity for estimating VH,M(C)using
density fitting is O(N3), which reduces the original quartic O(N4)complexity.
Alternative Derivation We would like to mention an alternative derivation of the Hamiltonian matrix as an amendment.
This derivation is to first parameterize the optimization problem using a function basis then deriving the optimality condition
in matrix form. Noting that under a basis set \{ηM,α(r)\}B
α=1, the orbital functions can be parameterized using the orbital
coefficient matrix CasΦM,Cin the form of Eq. (A.14) , the corresponding optimization problem Eq. (A.9) can be converted
into a usual optimization problem on vectors/matrix (instead of on functions): E⋆
M=
min
C∈RB×N:
C⊤SMC=I(
EM(C) :=EM[ΦM,C] = vec(TM)⊤vec(Γ(C)) +1
2vec(Γ(C))⊤˜DMvec(Γ(C))
+EXCX
α,βΓ(C)αβηM,αηM,β
+ vec( Vext,M)⊤vec(Γ(C))!
=:EM
Γ(C))
,
(A.24)
where the constraint comes from the orthonormality of orbitals δij=⟨ϕM,C,i|ϕM,C,j⟩=P
α,βCαiCβj⟨ηM,α|ηM,β⟩=P
α,βCαiCβj(SM)αβ= (C⊤SMC)ij, and the density matrix is defined by Γ(C) :=CC⊤. The expression for the
XC energy part comes from the density function expression under a basis, i.e., Eq. (A.15) . Noting that the energy
expression depends on Conly through the density matrix Γ(C), we finally denote the optimization objective as EM
Γ(C)
.
Introducing Lagrange multipliers grouped into a symmetric matrix ϵ(since the constraint is symmetric) for the constraint
and taking the gradient w.r.t C, we have the optimality condition: ∇CEM
Γ(C)
=∇Ctr
ϵ⊤(C⊤SMC−I)
. Using
the chain rule and that all matrices except Care symmetric, we have:
∇CEM
Γ(C)
= 2∇ΓEM
Γ(C)
C, (A.25)
and that ∇Ctr
ϵ⊤(C⊤SMC−I)
= 2SMCϵ. The optimality equation then becomes:
∇ΓEM
Γ(C)
C=SMCϵ. (A.26)
When optimality is achieved, Cis the eigenvectors of the Hermitian (symmetric) matrix ∇ΓEM
Γ(C)
, so in the common
situation that there is no degenerated state, the eigenvectors are already orthogonal, i.e., the non-diagonal part of the
constraint C⊤SMC=Iis satisfied. Therefore, the multipliers only need to handle the normalization constraints hence
only the diagonal part of ϵis effective. This reduces ϵin Eq. (A.26) to a diagonal matrix. In this way, Eq. (A.26) becomes
identical to Eq. (A.16), which indicates:
HM(C) =∇ΓEM
Γ(C)
. (A.27)
The equivalence to the first definition in Eq. (A.17) together with Eq. (A.11) and Eq. (A.10) can be seen from the relation
between variation and gradient: for a general functional F[·]and a general parameterized function fθ(x), the relation is:
∂F[fθ]
∂θ=ZδF[fθ]
δf(x)∂fθ
∂θ(x) dx.
18

Self-Consistency Training for Density-Functional-Theory Hamiltonian Prediction
Using this equation and noting that (∇CE)αimeans∂E
∂CαiandEM(C) :=EM[ΦM,C]from Eq. (A.24), we have:

∇CEM(C)
αi=ZNX
j=1δEM[ΦM,C]
δϕM,C,j(r)
∇CϕM,C,j(r)
αidr=ZδEM[ΦM,C]
δϕM,C,i(r)
∇CϕM,C,i(r)
αidr
Eqs. (A.12, A.14)= 2Z
ˆHM,[ρM,C]ϕM,C,i(r)ηM,α(r) dr= 2ZX
βCβiˆHM,[ρM,C]ηM,β(r)ηM,α(r) dr
= 2X
βCβi⟨ηM,α|ˆHM,[ρM,C]|ηM,β⟩Eqs. (A.17)= 2
HM(C)C
αi,
which means ∇CEM(C) = 2HM(C)C. On the other hand, noting EM(C) =EM
Γ(C)
from Eq. (A.24) and noting
Eq.(A.25) , we also have ∇CEM(C) = 2∇ΓEM
Γ(C)
C. This also gives HM(C) =∇ΓEM
Γ(C)
,i.e., Eq. (A.27) .
From Eq. (A.27) , the detailed construction of the Hamiltonian matrix Eq. (A.18) to (A.20) can be recovered using the
detailed expressions in Eq. (A.24).
B. Additional Technical Details
In this section, we present additional details regarding the implementation of the Hamiltonian prediction model and the
self-consistency loss.
B.1. Model Implementation Details
QHNet. We build our model upon the official QHNet codebase4, which is an SE(3) -equiavariant graph neural network
for Hamiltonian prediction (Yu et al., 2023b). With careful architecture design, QHNet achieves a good balance between
inference efficiency and accuracy. Its architecture is composed of four key modules: node-wise interaction, diagonal pair,
non-diagonal pair and expansion. Given the atom types Zand positions Ras inputs, the QHNet model employs five layers
of node-wise interaction to extract SE(3) -equivariant atomic features. Subsequently, the features of diagonal/non-diagonal
atom pairs are fed into diagonal/non-diagonal pair modules respectively to build pairwise representations faa(diagonal)
andfab(non-diagonal), where aandbdenote the atom index. The expansion module then transforms these pairwise
representations into blocks of the Hamiltonian matrix. Further information can be found in the original paper (Yu et al.,
2023b). For our experimental studies, all models are configured with the default parameters specified for QHNet. The neural
network codebase is developed using PyTorch(Paszke et al., 2019) and PyTorch-Geometric (Fey \& Lenssen, 2019).
Adapter Module. As outlined in Sec. 3.1, to facilitate the generalization of the Hamiltonian model in the OOD scenario,
we apply self-consistency loss for fine-tuning the QHNet model with two fine-tuning approaches: all-param and
adapter . Specifically, we construct the adapter using three modules: diagonal pair module, non-diagonal pair module
and expansion module. and then insert it atop the original QHNet model. A schematic illustration of the adapter module
is provided in Fig. B.1. Given the input molecule, the pretrained QHNet model is initially used to produce the initial
Hamiltonian matrix blocks ˆH5, along with the final atomic representations hand the final pairwise representations f.
Subsequently, the atomic representations are fed into corresponding diagonal or non-diagonal pair modules respectively
to build pairwise representations f′. Afterward, the pairwise representations of the QHNet model and the adapter module
are combined ( e.g.,f′
aa+t1·faa, with t1as a learnable combination coefficient). The combined pairwise representations
are first fed into a linear layer and then employed by the expansion module to produce the refinement Hamiltonian ( ˆH′).
Finally, we take the combination of the initial Hamiltonian and the refinement Hamiltonian ( e.g.,ˆH′
aa+o1·ˆHaa, with
o1as a learnable combination coefficient) as the final output ˆH′′. It is important to note that the combination of pairwise
representations and Hamiltonian blocks, whether diagonal or non-diagonal, is conducted independently, and the combination
coefficients are distinct for each pair ( i.e.,t1̸=t2ando1̸=o2).
B.2. Self-Consistency Loss
Back-Propagation through Eigensolver. As described in Sec. 2.3, the evaluation of self-consistency loss Lself-con (Eq. (3))
requires the eigenvectors CM,θof the generalized eigenvalue problem (Line 3 in Alg. 1), necessitating the back-propagation
4https://github.com/divelab/AIRS/tree/main/OpenDFT/QHNet, the code is available under the terms of the GPL-3.0 license
5The model-predicted Hamiltonian should be formally denoted as ˆHθ(M), and we omit θandMfor brevity
19

Self-Consistency Training for Density-Functional-Theory Hamiltonian Prediction
QHNet 𝒵,ℛDiagonal pair
Non -diagonal 
pairLinea rLinea r Expansion
Expansion𝐡
𝑡2⋅𝐟𝑎𝑏𝑡1⋅𝐟𝑎𝑎
𝐟𝑎𝑎′
𝐟𝑎𝑏′
𝑜2⋅෡𝐇𝑎𝑏𝑜1⋅෡𝐇𝑎𝑎
෡𝐇𝑎𝑎′
෡𝐇𝑎𝑏′෡𝐇𝑎𝑏′′෡𝐇𝑎𝑎′′Adapter
Figure B.1. The whole architecture of the adapter module. Given the atom types Zand positions Ras inputs, the pretrained QHNet
model is used to produce atomic representations h, pairwise representations fand the initial Hamiltonian prediction ˆH. Subsequently, the
adapter module is utilized to produce refinement Hamiltonian ˆH′based on handf. Finally, the refinement Hamiltonian is combined with
the initial Hamiltonian prediction as the final output ˆH′′.t1,t2,o1ando2denote learnable combination coefficients. aandbdenote the
indexes of atoms.
through an eigensolver. Thus we need to compute the gradient of the loss function Lself-con w.r.t the matrix ˆHθ(M). In our
practical implementation, we solve the generalized eigenvalue problem for each molecule with two steps: (1)Solve the
eigenvalue problem for matrix SM,Λ,U=EigSol (SM)and then define A=UΛ−1/2. This leads to the transformation
of the Hamiltonian matrix to ˜Hθ=A⊤ˆHθ(M)A;(2)Solve the eigenvalue problem for the transformed Hamiltonian
matrix ˜Hθ,ϵ,˜C=EigSol (˜Hθ), from which the eigenvectors of the original problem are recovered as CM,θ=A˜C.
Following these steps, the self-consistency loss Lself-con (HM(CM,θ))is calculated using the eigenvectors CM,θthat have
been derived.
The partial derivatives of self-consistency loss Lself-con w.r.t the transformed matrix ˜Hθcan be expressed as: ∇˜HθLself-con =
˜C
G◦(˜C⊤∇˜CLself-con )˜C⊤, where
Gij=(
1/(ϵi−ϵj), i̸=j,
0, i =j,
with ϵirepresenting the i-th eigenvalues. The gradient ∇˜CLself-con is calculated using the chain rule ∇˜CLself-con =
∇˜CC∇CM,θLself-con = vec−1
(IN⊗A⊤) vec(∇CM,θLself-con )
, where vecandvec−1denote the vectorization oper-
ator and its inverse operator, ⊗denotes the Kronecker product operator, and INdenotes the N-dimensional identity
matrix. Then we can derive the partial derivative w.r.t the original Hamiltonian matrix ˆHθ(M)as:∇ˆHθ(M)Lself-con =
∇˜HθˆHθ(M)∇˜HθLself-con = vec−1
A⊗Avec(∇˜HθLself-con )
. Consequently, the partial derivatives w.r.t ˆHθ(M)rely on
the matrix G, which can lead to a large gradient when two eigenvalues are close.
To mitigate this instability and promote stable training, we introduce two treatments. The first is to limit the magnitude of
gradient by applying truncation on matrix Gin the backward function of PyTorch:
˜Gij=(
T·sgn(ϵi−ϵj),if1/|ϵi−ϵj|> T,
Gij, if1/|ϵi−ϵj| ≤T,
where Tis a threshold determined by taking the 60-th percentile of absolute values of all Gentries, and sgn(·)denotes the
sign function. The technique is chosen for its simplicity and effectiveness, and there exist other methods for addressing this
issue (Song et al., 2021; Wang et al., 2021a). The second treatment is to skip model parameter update when the scale of the
gradient w.r.t parameters exceeds a certain threshold gs, which is determined through cross-validation.
Efficient Hamiltonian Reconstruction To reconstruct the Hamiltonian HM(CM,θ), we first generate requisite integrals
and quadrature grid using PySCF and then compute the Hamiltonian using PyTorch according to standard SCF procedure. Yet,
this involves two costly steps. (1)Evaluating atomic basis functions on generated quadrature grid points is computationally
20

Self-Consistency Training for Density-Functional-Theory Hamiltonian Prediction
Table B.1. Comparison of training time per iteration for the QHNet model with and without the incorporation of the self-consistency loss.
The batch size is maintained at 5 across all configurations, and the average training time is calculated over 50 iterations. Unit: s.
Model Ethanol Malondialdehyde Uracil
QHNet without self-con 0.283 0.289 0.355
QHNet with self-con 0.375 0.401 0.613
expensive. To accelerate this computation, we re-implement the evaluation of basis functions on GPU. Moreover, the grid
level determines the number of grid points and in turn influences the construction accuracy of the exchange-correlation
potential VXC,M(C). Empirically, we find that a grid level of 2 strikes an optimal balance between construction accuracy
and computational efficiency. (2)The computation of the Hartree component entails a O(N4)complexity (Line 6 in Alg. 1,
Nis the number of electrons). As the molecular size increases, this computation becomes highly costly. To enable an
efficient evaluation of the Hartree matrix, we apply the density fitting technique widely used in DFT programs to reduce the
computational complexity from O(N4)toO(N3). Leveraging the two techniques leads to a significant acceleration for the
Hamiltonian reconstruction, enabling faster self-consistency training. Moreover, integral matrices that are solely dependent
on the molecular conformation ( i.e.,TM,Vext,M) are pre-computed and stored in the database. These pre-computed
matrices are then loaded as needed during the training process.
Computational Complexity. As the self-consistency loss is constructed following the standard SCF procedure, it possesses
the same computational complexity as one SCF iteration under the Kohn-Sham DFT formulation. After the application
of the density fitting technique, the computational complexity becomes O(N3)(Ndenotes the number of electrons in a
molecule). Note that self-consistency training only brings extra computational cost during training, while keeps the same
cost for Hamiltonian prediction. The empirical time cost of the Hamiltonian prediction model, both with and without the
incorporation of self-consistency loss, are detailed in Table B.1.
C. Experimental Study Settings
In this section, we provide further data preparation and training details for the empirical study presented in Sec. 3.
C.1. Dataset Preparation
To demonstrate the benefits of self-consistent training, we first conduct experiments on two generalization scenarios,
corresponding to two molecular datasets, MD17 and QH9, respectively. Afterward, we evaluate the applicability of the
Hamiltonian prediction model on large-scale molecules, for which we adopt the MD22 dataset.
Table C.1. Statistics of the MD17 dataset (Sch ¨utt et al., 2019).
Molecule Train (labeled) Train (unlabeled) Validation Test Molecular size
Ethanol 100 24,900 500 4,500 9
Malondialdehyde 100 24,900 500 1,478 19
Uracil 100 24,900 500 4,500 26
MD17. To evaluate the benefit of self-consistency training in improving generalization for the data-scarce scenario, we
adopt the MD17 dataset (Sch ¨utt et al., 2019), and focus on three conformational spaces of ethanol ( C2H5OH), malondialde-
hyde ( CH2(CHO) 2) and uracil ( C4H4N2O2). The Hamiltonian matrices in this dataset are calculated with the PBE (Perdew
et al., 1996) exchange-correlation functional and the Def2SVP Gaussian-type orbital (GTO) basis set. We follow the split
setting used by Sch ¨utt et al. (2019) to divide the structures of each molecule into training/validation/test sets. Moreover, we
randomly select 100 labels from the training set for supervised training, while the remaining training structures are utilized
as unlabeled data for self-consistency training. The detailed statistics of three conformational spaces are summarized in
Table C.1.
21

Self-Consistency Training for Density-Functional-Theory Hamiltonian Prediction
Table C.2. Statistics of the QH9 dataset (Yu et al., 2023a).
Data setting Training Validation Test
QH9-small 94,001 10,000 N/A
QH9-large 18,000 2,000 6,830
QH9-full 124,289 6,542 N/A
QH9. To evaluate the benefit of self-consistency training in improving generaliation for the out-of-distribution (OOD)
scenario, we adopt the QH9 dataset6(Yu et al., 2023a). This dataset is proposed to benchmark Hamiltonian prediction
methods in chemical space, consisting of two subsets: QH-stable and QH-dynamic. Here we adopt the QH-stable subset
(dubbed as QH9 hereafter), which consists of 130,831 stable small organic molecules with no more than 9 heavy atoms, as
well as their corresponding Hamiltonian matrices. The Hamiltonian matrices are calculated with the B3LYP (Becke, 1992)
exchange-correlation functional and the Def2SVP GTO basis set. To simulate an OOD benchmark, we divide the QH9
dataset into two subsets by molecular size (QH9-small and QH9-large) and further partition them into training/validation/test
sets. Additionally, we establish a separate split setting for the generalization study on large-scale molecules(referred to as
QH9-full). Comprehensive statistics related to these division settings are detailed in Table C.2.
MD22. To evaluate the applicability of the Hamiltonian prediction model on large-scale molecules, we adopt the MD22
dataset (Chmiela et al., 2023) and focus on the Ac-Ala3-NHMe (ALA3) and DHA molecules. Since the MD22 does not
provide Hamiltonian labels (and energy and force labels are provided under a different exchange-correlation functional
PBE), we randomly sample 500 structures for each molecule as our benchmark and use PySCF (Sun et al., 2018) to generate
Hamiltonian matrices for these structures with the B3LYP exchange-correlation functional and the Def2SVP GTO basis set.
C.2. DFT Implementation Details
For this study, all DFT calculations, including those for evaluating the SCF acceleration ratio (Sec. 3.1-3.3) and for
benchmarking the DFT computation cost (Sec. 3.2) are performed using the PySCF software (Sun et al., 2018) with its
default parameter settings.
C.3. Hardware Configurations
All neural network models are trained and evaluated on a workstation equipped with a Nvidia A100 GPU with 80 GiB
memory and a 24-core AMD EPYC CPU, which is also used for DFT calculations. Note that all the computation times
reported in the empirical study are benchmarked on this specific hardware configuration to ensure consistency in comparison.
However, it is also recognized that both neural network models and DFT computations have the potential to be parallelized
and accelerated using multiple GPUs or CPU cores. Given this capability for parallel processing, establishing a perfectly
equitable hardware benchmarking environment for both approaches is challenging.
C.4. Training Details
Data-Scarce Scenario. We first describe the training details utilized in the empirical study of Sec. 3.1. For the self-
consistency training setting, we set the total training iterations to 200k for three conformational spaces following Yu et al.
(2023b). The weighting factor λself-con is set to 10 across all molecules. Considering that the efficacy of the supervised
learning setting might be limited by scarce labeled data, we allocate a higher number of training iterations ( i.e., 500k) to
this strategy to ensure the model reaches its optimal performance within the constraints of the available labels. For all
experimental conditions and datasets, we maintain a consistent batch size of 5. We utilize a polynomial decay learning rate
scheduler to modulate the learning rate (LR) during training, where the polynomial power is set to 5 for self-consistency
training and 14 for supervised learning based on empirical trials. Notably, the scheduler increases the learning rate gradually
during the first 10k warm-up iterations. The learning rate starts at 0 and peaks at a maximum of 1 ×10−3across all training
scenarios. When addressing the supervised training settings with extended labeled data as mentioned in Sec. 3.2, we adopt
the same training hyperparameters as those used in the supervised learning setting of Sec. 3.1, with the singular adjustment
of setting the polynomial power to 5.
6The dataset is licensed under a Creative Commons Attribution-NonCommercial-ShareAlike 4.0 International License.
22

Self-Consistency Training for Density-Functional-Theory Hamiltonian Prediction
Out-of-Distribution Scenario. As outlined in Sec. 3.1, to benchmark the OOD generalization performance, we initially
train the QHNet model on the QH9-small subset, and then fine-tune the model on unlabeled large molecules. We continue
to use the polynomial learning rate schedule, which includes a warm-up phase. Additional training hyperparameters are
detailed in Table C.3.
Table C.3. Training hyper-parameters in the OOD scenario.
Training Phase Batch size Maximum LR Polynomial Power Iterations Warm-up Iterations
Pretraining 32 5 ×10−41 300k 1k
Fine-tuning 5 2 ×10−58 200k 3k
Large-Scale Molecular Systems. As discussed in Sec. 3.3, we adopt a two-stage training strategy to generalize the
Hamiltonian model to large-scale MD22 structures. We still employ the polynomial learning rate schedule with the warm-up
stage, and detail other training hyper-parameters in Table C.4.
D. Additional Experimental Results
D.1. Generalization Results
As noted in Sec. 3.1, while supervised training can outperform self-consistency training with adequate computational
resources, the advantage in terms of Hamiltonian MAE does not consistently extend to molecular properties. This
discrepancy has been observed in the OOD generalization scenario in Table 3 of Sec. 3.1. Correspondingly, the results
presented in Table C.5 show that there exists a comparable trend in the data-scarce scenario.
D.2. Results of Alternative Model Architectures
For further validate the advantage of self-consistency training with alternative architectures, we investigate its benefit using
the PhiSNet (Unke et al., 2021) architecture, which is another performant model for Hamiltonian prediction on molecules.
As shown in Table D.1, the results exhibit the same conclusion as shown in Table 1: compared to the results of supervised
training ( label ), applying self-consistency loss ( label+self-con ) on unlabeled structures leads to a remarkable
improvement across all evaluation metrics. Notably, the MAEs for HOMO ϵHOMO , LUMO ϵLUMO and HOMO-LUMO gap
ϵ∆are reduced by several folds. These results demonstrate the generality of self-consistency training for improving the
performance of general architectures.
D.3. The Impact of SCF Iteration Strategies
As mentioned in Sec. 3, to illustrate the accuracy of Hamiltonian prediction, we assess its capability for accelerating DFT
when using the prediction as initialization. Following previous studies (Yu et al., 2023b;a), all DFT calculations are carried
out using the PySCF (Sun et al., 2018) software, with the PBE XC functional and the Def2SVP basis set being adopted.
The direct inversion in iterative subspace (DIIS) (Pulay, 1980) iteration strategy is employed, following the defaults. The
results in Tables 1-3 and 5 show that the Hamiltonian prediction model leads to a substantial SCF acceleration across various
molecular systems, and applying self-consistency training can further improve the acceleration gains. Nevertheless, we
find that the iteration strategy can considerably influence SCF convergence, which may diminish the benefit of a more
accurate initialization. For example, it is known that DIIS may show a non-monotone iteration behavior, meaning that the
Hamiltonian in the next iteration may not be closer to the final solution than the Hamiltonian in the current iteration (Sun,
2016; Sun et al., 2017). Hence, even when the initial Hamiltonian (predicted by the model) is closer to the final solution, the
Hamiltonian in the next iteration may still be farther away from the solution (“DIIS algorithm does not honor the initial guess
well. The optimization procedure may lead the wavefunction anywhere in the variational space” (Sun, 2016)). To verify this
point, we attempt to use the second-order SCF (SOSCF) iteration strategy (Sun et al., 2017) in place of DIIS for running the
SCF iteration and summarize the results in Table D.2. SOSCF directly engages in orbital optimization, hence guaranteeing
monotonicity. In the evaluation setting of OOD generalization on QH9-large test molecules (in parallel with Table 3),
we observe a 57.8\% and 56.2\% SCF acceleration for the extended-label andself-con settings respectively. The
speedup is indeed improved to the DIIS speedup of 65.0\% and 64.5\% for the respective settings, justifying the speculation.
Moreover, in the evaluation setting of large-scale generalization on ALA3 and DHA structures from MD22 (in parallel with
23

Self-Consistency Training for Density-Functional-Theory Hamiltonian Prediction
Table C.4. Training hyper-parameters in the large-scale molecular systems.
Training Phase Batch size Maximum LR Polynomial Power Iterations Warm-up Iterations
Pretraining 32 5 ×10−41 400k 5k
Fine-tuning (ALA3) 2 2 ×10−58 100k 10k
Fine-tuning (DHA) 1 2 ×10−58 200k 10k
Table C.5. Performance comparison between self-con sistency training, and supervised training using full extended labels , in the
data-scarce scenario (in parallel with Table 3), corresponding to the ending points of Fig. 3 ( extended-label-online is close to
extended-label ).
Molecule Setting H[µEh]↓ϵ[µEh]↓C[\%]↑ϵHOMO [µEh]↓ϵLUMO [µEh]↓ϵ∆[µEh]↓SCF Accel. [\%]↓
Ethanolextended-label 58.28 986.84 99.94 230.20 2902.14 2723.64 63.5
label + self-con 95.90 340.56 99.94 403.60 1426.20 1370.35 61.5
Malondi- extended-label 71.45 1014.12 99.63 199.48 414.58 415.91 66.6
aldehyde label + self-con 86.60 280.39 99.67 274.45 279.14 324.37 62.1
Uracilextended-label 52.53 288.29 99.38 306.05 294.54 398.08 58.1
label + self-con 63.82 315.40 99.58 359.98 369.67 388.30 54.5
Table 5), self-consistency training achieves a 47.5\% and 37.0\% SCF acceleration respectively, substantially better than
DIIS speedup of 64.7\% and 67.0\%. These results support that employing the SOSCF convergence method can better honor
the quality of the initial guess. Additionally, for DHA structures where the low-quality zero-shot prediction results in
a deceleration (170.8\%), SOSCF can lead to worse convergence performance (231.8\%). Notably, the observed speedup
appears to be more pronounced on the larger molecular systems ( e.g., ALA3 and DHA), which indicates that molecular
systems listed in Table 3 may be already easy to converge with DFT and thereby difficult to accelerate further.
D.4. Amortization Effect of Self-Consistency Training
As mentioned in Sec. 3.2, we directly access the amortization effect by comparing the computational cost of self-consistency
training with that of DFT for solving a bunch of structures. It should be noted that measuring the computational cost of DFT
on all unlabeled training structures is impractical, thus we run DFT on 50 randomly picked structures for each molecule. The
mean computation time derived from these 50 structures serves as a benchmark to approximate the overall computational
time required for the complete set of structures.
To further demonstrate the amortization efficiency of self-consistency training, we also measure the computational cost by
“the number of consumed SCF iterations” and present the accuracy-cost curves in Figs. D.1 and D.2. The results indicate
the same conclusion as shown in Figs. 3 and 4: self-consistency training can achieve a satisfying prediction accuracy even
with less SCF steps than the number of SCF steps in the DFT calculation for labeling the molecular structures. Even in the
‘extended-label-online’ setting where the data is generated along with the training of the model, self-consistency training can
still achieve a better accuracy given the same budget of SCF iterations.
D.5. More Results of SCF Acceleration
To comprehensively investigate the significance of our method in accelerating SCF convergence, we present a detailed
point-by-point comparison of SCF acceleration for different Hamiltonian prediction models. The results on three datasets are
summarized in Figs. D.3-D.5. Remarkably, the models with self-consistency training always lead to faster convergence than
conventional MINAO guess across various settings. In contrast, the label-based training method for uracil in Table 1 (see
Fig. D.3(c)) and the zero-shot setting for two MD22 molecules in Table 5 (see Fig. D.5) result in slower convergence
than MINAO on some structures, while applying self-consistency training can eliminate this issue.
D.6. Large-Scale Generalization Results
As discussed in Sec. 3.3, we assess the generalization of self-consistency training to large-scale MD22 structures by
comparing it with two state-of-the-art end-to-end ( e2e) property prediction models. For this purpose, we choose two
24

Self-Consistency Training for Density-Functional-Theory Hamiltonian Prediction
Table D.1. Generalization improvement by self-consistency training on unlabeled data on various model architectures in the data-scarce
scenario ( MD17-Ethanol Hamiltonian). Evaluated on the test split of conformations of the molecule. The setting is in parallel with
Table 1.
Architecture Setting H[µEh]↓ϵ[µEh]↓C[\%]↑ϵHOMO [µEh]↓ϵLUMO [µEh]↓ϵ∆[µEh]↓SCF Accel. [\%]↓
QHNetlabel 160.36 712.54 99.44 911.64 6800.84 6643.11 68.3
label + self-con 75.65 285.49 99.94 336.97 1203.60 1224.86 61.5
PhiSNetlabel 116.72 2702.13 98.63 1887.50 7954.97 6834.62 65.69
label + self-con 93.77 475.29 99.91 602.96 1645.25 1689.17 62.87
Table D.2. SCF acceleration performance under two SCF iteration strategies, DIIS and SOSCF. (Results in the main paper are under
DIIS.) Evaluated on the QH9-large test split in the OOD scenario (supervised training uses full extended labels; same setting as Table 3)
and on the MD22 molecules in the larger-scale generalization scenario (same setting as Table 5).
Evaluation setting Model setting Iteration Strategy SCF Accel. [\%]↓
QH9-largeextended-labelDIIS 65.0
SOSCF 57.8
self-conDIIS 64.5
SOSCF 56.2
MD22 (ALA3)zero-shotDIIS 84.6
SOSCF 71.8
self-conDIIS 64.7
SOSCF 47.5
MD22 (DHA)zero-shotDIIS 170.8
SOSCF 231.8
self-conDIIS 67.0
SOSCF 37.0
advanced e2e architectures, ET7(Th¨olke \& De Fabritiis, 2021) and Equiformer8(Liao \& Smidt, 2022), as our baselines.
They are representatives for equivariant architectures that utilize vector features and high-order tensor features, respectively.
Despite the availability of pretrained models for these methods, they are not directly applicable for our analysis because
they were originally trained on the QM9 dataset, which differs from the QH9 dataset used in our study. The two datasets
use distinct orbital basis sets for DFT calculations, resulting in slightly different label distributions. To ensure a fair
comparison, we retrain the ET and Equiformer models on the QH9-full training split, with the default hyper-parameter
settings specified in their codebases. The performance of the two e2e predictors on the QH9-full validation set, as shown
in Table D.3, aligns with the outcomes reported in their respective original publications, validating the effectiveness of our
model replication. The substantial performance disparity between the QH9-full validation set and two MD22 molecules
implies a significant generalization challenge for e2e predictors. Fortunately, this challenge can be potentially alleviated by
applying self-consistency training to Hamiltonian prediction.
E. Limitations and Future work
Even though self-consistency training has shown improved efficiency than conventional DFT calculation for training a
Hamiltonian prediction model or for solving a bunch of molecular structures (Sec. 3.2) by amortizing the cost of SCF
iterations over queried molecular structures, the computational complexity remains the same as that of DFT. This complexity
may still limit the applicability (but to a higher level) of Hamiltonian prediction to large molecular systems such as
biomacromolecules. It would be a promising future work to reduce the complexity of evaluating the self-consistency loss
by leveraging techniques from linear-scaling DFT algorithms. The Hamiltonian prediction model we used in this study,
although is already more efficient than a few alternatives, still requires considerable cost to evaluate, due to e.g.the use of
computationally expensive tensor product operations. This calls for designing more efficient neural network architectures for
Hamiltonian prediction. Moreover, current Hamiltonian prediction models only support prediction under a specific basis set,
which has a restricted flexibility to trade-off efficiency and accuracy, and is hard to leverage data under different choices of
7https://github.com/torchmd/torchmd-net, the code is freely available under the terms of the MIT license.
8https://github.com/atomicarchitects/equiformer, the code is freely available under the terms of the MIT license.
25

Self-Consistency Training for Density-Functional-Theory Hamiltonian Prediction
0.060.080.10
(DFT labeling)
0 100000 200000 300000
Effective SCF iterations0.00000.00250.0050Hamiltonian MAE (Eh)
0.060.08 (DFT labeling)
0 100000 200000 300000 400000
Effective SCF iterations0.0000.0020.0040.040.06 (DFT labeling)label + self-con extended-label extended-label-online
0 100000 200000 300000 400000
Effective SCF iterations0.0000.0020.004
Figure D.1. Efficiency comparison in the data-scarce scenario (MD17 Hamiltonian) among self-con sistency training on unlabeled
data, supervised training following DFT labeling on unlabeled data ( extended-label ), and supervised training along with DFT
labeling ( extended-label-online ). The setting is in parallel with Fig. 3, with the only difference that the cost is measured by the
effective number of SCF iterations consumed along the training process.
0 50000 100000 150000 200000
Effective SCF iterations0.000050.000100.000150.000200.00025Hamiltonian MAE (Eh)
 (DFT labeling)extended-label self-con extended-label-online
Figure D.2. Efficiency comparison in the OOD scenario (QH9) among fine-tuning using self-con sistency training on unlabeled
data, supervised training following DFT labeling on unlabeled data ( extended-label ), and supervised training along with DFT
labeling ( extended-label-online ). The setting is in parallel with Fig. 4(b) using the adapter fine-tuning strategy, with the only
difference that the cost is measured by the effective number of SCF iterations consumed along the training process.
basis. A possible solution to this restriction is including the overlap matrix into the model input, which conveys information
about the basis set in a form relevant to the given molecular structure.
26

Self-Consistency Training for Density-Functional-Theory Hamiltonian Prediction
Table D.3. Generalization results of e2e property predictors on larger-scale MD22 molecules. Models are pretrained on the QH9-full
training split and directly evaluated on the large-scale molecules. The e2e predictors significantly suffer from the generalization gap.
QH9(valid) denotes the QH9-full validation split.
Model ϵHOMO [µEh]↓ ϵLUMO [µEh]↓ ϵ∆[µEh]↓
QH9 (valid) ALA3 DHA QH9 (valid) ALA3 DHA QH9 (valid) ALA3 DHA
e2e (ET) 818.77 1.74 ×1052.92×105540.22 7.72 ×1042.58×1041.38×1032.38×1053.39×105
e2e (Equiformer) 646.42 2.38 ×1053.76×105488.40 1.16 ×1042.31×1041.15×1032.27×1064.17×106
5 6 7 8 9 10
\#SCF (MINAO)5678910\#SCF (Model)y=xlabel
5 6 7 8 9 10
\#SCF (MINAO)5678910\#SCF (Model)y=xlabel + self-con
(a) Comparison of SCF acceleration on MD17-ethanol structures
678910111213
\#SCF (MINAO)678910111213\#SCF (Model)y=xlabel
678910111213
\#SCF (MINAO)678910111213\#SCF (Model)y=xlabel + self-con
(b) Comparison of SCF acceleration on MD17-malondialdehyde structures
7891011121314151617
\#SCF (MINAO)7891011121314151617\#SCF (Model)y=xlabel
7891011121314151617
\#SCF (MINAO)7891011121314151617\#SCF (Model)y=xlabel + self-con
(c) Comparison of SCF acceleration on MD17-uracil structures
Figure D.3. Comparison of SCF acceleration on MD17 structures (in parallel with Table 1). Each sub-figure shows a scatter plot of the
number of converged SCF steps from two initial guesses: MINAO ( x-axis), and the predicted Hamiltonian ( y-axis) by a model trained
using labels (left) and additionally using self-consistency training (right). All figures are plotted using 50 data points.
27

Self-Consistency Training for Density-Functional-Theory Hamiltonian Prediction
5678910
\#SCF (MINAO)4567\#SCF (Model)pretrain
5678910
\#SCF (MINAO)4567\#SCF (Model)extended-label (all-param)
5678910
\#SCF (MINAO)4567\#SCF (Model)extended-label (adapter)
5678910
\#SCF (MINAO)4567\#SCF (Model)self-con (all-param)
5678910
\#SCF (MINAO)4567\#SCF (Model)self-con (adapter)
Figure D.4. Comparison of SCF acceleration on QH9-large structures (in parallel with Table 3). Each figure shows a scatter plot of the
number of converged SCF steps from two initial guesses: MINAO ( x-axis), and the predicted Hamiltonian ( y-axis) by a model pretrained
using labels (top left) and additionally finetuned using labels (top middle and right) or self-consistency loss (bottom left and middle) with
two fine-tuning strategies. All figures are plotted using 50 data points.
28

Self-Consistency Training for Density-Functional-Theory Hamiltonian Prediction
789101112131415
\#SCF (MINAO)67891011121314151617\#SCF (Model)y=xpretrain
789101112131415
\#SCF (MINAO)67891011121314151617\#SCF (Model)y=xself-con (all-param)
(a) Comparison of SCF acceleration on MD22-ALA3 structures
5 10 15
\#SCF (MINAO)5101520253035404550\#SCF (Model)
y=xpretrain
5 10 15
\#SCF (MINAO)5101520253035404550\#SCF (Model)
y=xself-con (all-param)
(b) Comparison of SCF acceleration on MD22-DHA structures
Figure D.5. Comparison of SCF acceleration on MD22 structures (in parallel with Table 5). Each sub-figure shows a scatter plot of the
number of converged SCF steps from two initial guesses: MINAO ( x-axis), and the predicted Hamiltonian ( y-axis) by a model pretrained
using labels (left) and additionally finetuned using self-consistency loss with the all-param strategy (right). All figures are plotted
using 50 data points.
29

\end{document}
