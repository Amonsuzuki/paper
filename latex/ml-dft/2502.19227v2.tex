\documentclass[11pt,a4paper]{article}
\usepackage[utf8]{inputenc}
\usepackage[T1]{fontenc}
\usepackage{amsmath}
\usepackage{amsfonts}
\usepackage{amssymb}
\usepackage{amsthm}
\usepackage{graphicx}
\usepackage{hyperref}
\usepackage{cite}
\usepackage{natbib}
\usepackage{geometry}
\usepackage{booktabs}
\usepackage{multirow}
\usepackage{array}
\usepackage{caption}
\usepackage{subcaption}
\usepackage{float}
\usepackage{url}
\geometry{margin=1in}

\title{Published as a conference paper at ICLR 2025}
\author{}
\date{\today}

\begin{document}
\maketitle

\begin{abstract}
Density Functional Theory (DFT) is a pivotal method within quantum chemistry
and materials science, with its core involving the construction and solution of
the Kohn-Sham Hamiltonian. Despite its importance, the application of DFT is
frequently limited by the substantial computational resources required to construct
the Kohn-Sham Hamiltonian. In response to these limitations, current research
has employed deep-learning models to efficiently predict molecular and solid
Hamiltonians, with roto-translational symmetries encoded in their neural networks.
However, the scalability of prior models may be problematic when applied to large
molecules, resulting in non-physical predictions of ground-state properties. In
this study, we generate a substantially larger training set (PubChemQH) than used
previously and use it to create a scalable model for DFT calculations with physical
accuracy. For our model, we introduce a loss function derived from physical princi-
ples, which we call Wavefunction Alignment Loss (WALoss). WALoss involves
performing a basis change on the predicted Hamiltonian to align it with the ob-
served one; thus, the resulting differences can serve as a surrogate for orbital energy
differences, allowing models to make better predictions for molecular orbitals and
total energies than previously possible. WALoss also substantially accelerates
self-consistent-field (SCF) DFT calculations. Here, we show it achieves a reduc-
tion in total energy prediction error by a factor of 1347 and an SCF calculation
speed-up by a factor of 18\%. These substantial improvements set new benchmarks
for achieving accurate and applicable predictions in larger molecular systems.
1 I NTRODUCTION
Density functional theory (DFT) (Kohn \& Sham, 1965; Hohenberg \& Kohn, 1964; Martin, 2020)
has been widely used in physics (Argaman \& Makov, 2000; Jones, 2015; Van Mourik et al., 2014),
chemistry (Levine et al., 2009; Van Mourik et al., 2014), and materials science (March, 1999;
Neugebauer \& Hickel, 2013) to study the electronic properties of molecules and solids. This
methodology is particularly valued for its balanced blend of computational efficiency and accuracy,
rendering it a versatile choice for investigating electronic structure (Kohn et al., 1996; Parr \& Yang,
1995), spectroscopy (Neese, 2009; Orio et al., 2009), lattice dynamics (Dal Corso et al., 1997;
Wang et al., 2021), transport properties (Bhamu et al., 2018), and more. The most critical step in
applying DFT to a molecule is constructing the Kohn-Sham Hamiltonian, which consists of the
kinetic operator, the external potential, the Coulomb potential (also known as the Hartree potential),
and the exchange-correlation potential (Kohn \& Sham, 1965; Hohenberg \& Kohn, 1964; Martin,
2020).
The Hamiltonian matrix contains crucial information about molecular systems and their quantum
states (Kohn \& Sham, 1965; Hohenberg \& Kohn, 1964; Yu et al., 2023b; Zhang et al., 2024). This
matrix facilitates the extraction of various properties, including the highest occupied molecular orbital
(HOMO) and lowest unoccupied molecular orbital (LUMO) energies, the HOMO-LUMO gap, total
∗Equal contribution.
1arXiv:2502.19227v2 [physics.chem-ph] 20 Mar 2025
\end{abstract}

Published as a conference paper at ICLR 2025
energy, and spectral characteristics (Eisberg \& Resnick, 1985; Zhang et al., 2024). These properties
are vital for analyzing conformational energies (St.-Amant et al., 1995), reaction pathways (Farberow
et al., 2014), and vibrational frequencies (Watson \& Hirst, 2002). However, the efficiency of DFT is
constrained by the self-consistent field (SCF) iterations required to solve the Kohn-Sham equations
and achieve consistent charge density. These iterations scale as O(N3T)∼O(N4T)(Tirado-Rives
\& Jorgensen, 2008; Yu et al., 2023a), where Nrepresents the number of electrons and Tdenotes the
number of SCF cycles, making them particularly resource-intensive for large systems.
Consequently, this computational intensity underscores the critical need to develop more efficient
methodologies to determining the Hamiltonian without relying on SCF iterations. This challenge
has catalyzed interest in leveraging deep learning to predict Hamiltonians directly from atomic
configurations while adhering to the inherent symmetries of molecular systems (Unke et al., 2021a;
Kochkov et al., 2021; Yu et al., 2023b; Li et al., 2022; Schütt et al., 2019; Gastegger et al., 2020;
Hermann et al., 2020; Yin et al., 2024; Zhang et al., 2024; Luo et al., 2025). Notably, the Hamiltonian
matrix is subject to unitary transformations under molecular rotations due to its spherical harmonics
component. To tackle these transformations, researchers have developed SE(3)-equivariant neural
networks such as PhisNet (Unke et al., 2021b) and QHNet (Yu et al., 2023b). These networks
employ high-order spherical tensors and the Clebsch-Gordon tensor product to construct the predicted
Hamiltonians.
However, current implementations of SE(3)-equivariant neural networks face considerable scalability
challenges when applied to large molecular structures. Notably, prevailing techniques for predicting
the Hamiltonian matrix predominantly utilize mean absolute error (MAE) or mean square error
(MSE) as the loss function (Unke et al., 2021b; Yu et al., 2023b). We contend that these elementwise
losses alone are insufficient for accurately learning Hamiltonians in large systems. To underscore
this point, we introduce the PubChemQH dataset, which comprises molecular Hamiltonians for
structures with atom counts ranging from 40 to 100. In contrast, the previously curated dataset
(Yu et al., 2023a) is limited to molecules with no more than 31 atoms. Figure 1 demonstrates a
phenomenon enabled by the introduction of the PubChemQH dataset, which we term Scaling-Induced
MAE- Applicability Divergence (SAD). The applicability of the Hamiltonian is assessed by evaluating
the system energy error derived from pyscf (Sun et al., 2020; Sun, 2015; Sun et al., 2018). While
small molecules yield system energies close to the ground truth with relative Hamiltonian MAEs
up to 200\%, larger molecules exhibit profound inaccuracies. Energy errors escalate to as much as
1,000,000 kcal/mol at only 0.01\% relative MAE, rendering the Hamiltonians inapplicable. This
striking disparity emphasizes the inadequacy of MAE for large systems and highlights the pressing
need for new methodologies to enhance the scalability1and applicability of predicted Hamiltonians.
To address these limitations, this work aims to enhance Hamiltonian learning for large systems
by utilizing wavefunctions and their corresponding energies as surrogates to improve Hamiltonian
applicability. The wavefunction is crucial for Hamiltonian prediction as it encapsulates the quantum
state of a system, enabling the validation of Hamiltonians based on their ability to accurately
reflect the system’s energy and dynamics. However, learning the electron wavefunction is non-
trivial; inaccuracies in the machine learning-based Hamiltonian may lead to significant error in both
electronic structure and wavefunctions. This challenge motivated us to introduce the Wavefunction
Alignment Loss function (WALoss), which aligns the eigenspaces of predicted and ground-truth
Hamiltonians without explicit backpropagation through eigensolvers. Additionally, to improve the
scalability of Hamiltonian learning, we present WANet , a modernized architecture for Hamiltonian
prediction that leverages eSCN (Passaro \& Zitnick, 2023) convolution and a sparse mixture of pair
experts. Our pipeline and main contributions are summarized in Figure 2.

\section{B ACKGROUND}

In the framework of Density Functional Theory (DFT), a molecular system is defined by its nuclear
configuration M:=\{Z,R\}, where Zrepresents the atomic numbers of the nuclei and Rtheir
positions within the molecule. DFT focuses on determining the ground state of a system consisting
ofNelectrons by minimizing the total electronic energy with respect to the electron density ρ(r).
Here, ρ(r)is a functional of the set of None-electron orbitals \{ψi(r)\}N
i=1, where r∈R3specifies
the spatial coordinates of an electron.
1We define the scalability as the model’s accuracy under large molecules.
2

Published as a conference paper at ICLR 2025
Figure 1: Visualization of the SAD phenomenon. The y-axis (in symmetrical log scale (Webber, 2012))
represents the system energy error derived from the perturbed Hamiltonian, while the x-axis shows the relative
MAE, defined asMAE( ˆH,H⋆)
Mean(H⋆), whereH⋆represents the ground-truth Hamiltonian and ˆHdenotes the predicted
or perturbed Hamiltonian. The relative MAE is induced by model learning errors or Gaussian perturbation. The
Gaussian perturbation ensures that the perturbed matrix remains Hermitian. The initial guess line is derived
from minao (Almlöf et al., 1982; Van Lenthe et al., 2006). Current state-of-the-art models, such as QHNet,
achieve a relative MAE of up to 10−2on PubChemQH molecules. For small molecules, a 10−2relative MAE is
sufficient for accurate system energy predictions (left panel), but this accuracy does not extend to large systems
(right panel).
For computational efficiency, these orbitals are represented using a basis set \{ϕα(r)\}B
α=1depending
on the molecular geometry, Bdenotes the number of basis and varies for different basis sets.
The expansion coefficients of the orbitals are organized in a matrix C∈RB×N, allowing each
orbital ψi(r)to be expressed as: ψi(r) =PB
α=1Cαiϕα(r).DFT seeks the minimal electronic
energy by solving for the optimal coefficient matrix Cvia the Kohn-Sham equations, represented as
H(C)C=SCϵ,where H(C)∈RB×Bdenotes the Hamiltonian matrix. Notice that His a function
ofC, and can be computed using density fitting with a complexity of O(B3).S∈RB×Bis the
overlap matrix with elements Sαβ:=R
ϕ†
α(r)ϕβ(r)dr, where ϕ†denotes the complex conjugate of
ϕ.ϵrepresents a diagonal matrix containing the orbital energies. This forms a generalized eigenvalue
problem, where Ccomprises the eigenvectors and the diagonal elements of ϵare the eigenvalues.
However, the solution to this problem is complicated by the interdependence between H(C)and
C. To resolve this, traditional DFT employs the self-consistent field (SCF) method, an iterative
process that refines the coefficients C(k)through successive approximations of the Hamiltonian
matrix H(k). Each iteration commences by computing H(k)leveraging C(k−1)and solves a new
eigenvalue problem:
H(k)
C(k−1)
C(k)=SC(k)ϵ(k),
converging to a stable Hamiltonian H⋆and its corresponding eigenvectors C⋆, which define the
electron density and other molecular properties derived from the Kohn-Sham equations.
Problem Formulation The objective of Hamiltonian prediction is to obviate the need for self-
consistent field (SCF) iteration by directly estimating the target Hamiltonian H⋆from a given
molecular structure M. To achieve this, one could parameterize a machine learning model ˆHθ(M).
The learning process is guided by an optimization process defined as:
θ⋆= arg min
θ1
|D|X
(M,H⋆
M)∈Ddist
ˆHθ(M),H⋆
M
, (1)
where |D|denotes the cardinality of dataset D, and dist(·,·)is a predefined distance metric.
SE(3) Equivariant Networks SE(3)-equivariant neural networks incorporate strong prior knowl-
edge through equivariance. These networks utilize equivariant irreducible representation (irreps)
features built from vector spaces of irreducible representations to achieve 3D rotation equivariance.
The vector spaces are (2ℓ+ 1) -dimensional, where degree ℓ∈Nrepresents the angular frequency of
the vectors. Higher ℓvalues are critical for tasks sensitive to angular information, such as predicting
the Hamiltonian matrix. Vectors of degree ℓ, referred to as type- ℓvectors, are rotated using Wigner-D
matrices D(ℓ)∈R(2ℓ+1)×(2ℓ+1)when rotating coordinate systems. Eigenfunctions of rotation in
R3can be projected into type- ℓvectors using the real spherical harmonics Y(ℓ):S2→R2ℓ+1,
whereS2=\{ˆr∈R3:∥ˆr∥= 1\}is the unit sphere. Equivariant GNNs update irreps fea-
tures through message passing of transformed irreps features between nodes, using tensor prod-
3

Published as a conference paper at ICLR 2025
Figure 2: (a) We introduce PubChemQH, a new resource for Hamiltonian learning that facilitates the exploration
of the scalability challenge known as SAD. (b) We present WANet, a modern architecture designed for accurate
Hamiltonian prediction. WANet incorporates SO(2) convolution, a mixture of pair experts, and a many-body
interaction layer. The mixture of pair experts constructs the non-diagonal block, while the many-body interaction
layer constructs the diagonal block. (c) Our loss module, WALoss, performs a basis transformation of the
predicted Hamiltonian using the ground-truth Hamiltonian and the overlap matrix. This enhancement aims to
improve the applicability of the predicted Hamiltonian in real-world scenarios. Our final loss function combines
MAE-MSE loss with WALoss.
ucts: (uℓ1⊗vℓ2)ℓ3m3=Pℓ1
m1=−ℓ1Pℓ2
m2=−ℓ2C(ℓ3,m3)
(ℓ1,m1),(ℓ2,m2)uℓ1m1vℓ2m2,where Cdenotes the Clebsch-
Gordan coefficient, ℓ3satisfies |ℓ1−ℓ2| ≤ℓ3≤ℓ1+ℓ2. Note that mdenotes the m-th element in the
irreducible representation with −ℓ≤m≤ℓandm∈N. In equivariant graph neural networks, the
message function mtsfrom source node sto target node tis calculated using SO(3) convolution. The
ℓ0-th degree of mtscan be expressed as m(ℓo)
ts=P
ℓi,ℓfWℓi,ℓf,ℓo
x(ℓi)
s⊗Y(ℓf)(ˆrts)ℓo
,where
Wℓi,ℓf,ℓoare the weight vectors, xsrepresents the irreducible representation of the source node sand
ˆrts:=rts
∥rts∥2.
3 W AVEFUNCTION -ALIGNMENT LOSS
The applicability of the predicted Hamiltonian ˆHdepends significantly on the accuracy of its
eigenvalues (orbital energies) and eigenvectors (basis coefficients). To assess the alignment between
ˆHand the actual Hamiltonian H⋆with respect to their eigenspaces, we present the following theorem:
Theorem 1. LetHandˆHrepresent Hamiltonian matrices, and Sthe overlap matrix. Define
the perturbation matrix as ∆H:=ˆH−H. Let λi(H,S)andλi(ˆH,S)be the ithgeneralized
eigenvalues of HandˆH, respectively. Assume a spectral gap δseparates the generalized eigenvalues
ofH,ˆH.κ(·)denotes the conditional number of a given matrix, ∥ · ∥ 2represents the spectral
norm. ∥∆H∥1,1=P
i,j|∆Hij|. Then, the difference in eigenvalues and the angle θbetween the
eigenspace of HandˆHsatisfy:
|λi(ˆH,S)−λi(H,S)| ≤κ(S)
∥S∥2· ∥∆H∥1,1,sinθ≤κ(S)
∥S∥2·∥∆H∥1,1
δ.
4

Published as a conference paper at ICLR 2025
Proof. The proof is delegated to Appendix J.1.
Corollary 1 (Perturbation Sensitivity Scaling) .Assume λmin(S)of the overlap matrix Sscales as:
λmin(S) =c+A
1+
B
N0
α,where c >0,A >0, and α >0describe the saturation behavior. Then,
for a perturbation matrix ∆H, under the first-order perturbation approximation, the eigenvalue
perturbation is bounded by:
λi(ˆH,S)−λi(H,S)≤O
σB1/2+B|µ|
,
Proof. The proof is delegated to Appendix J.2.
Remark The theorem highlights that the difference between the predicted and actual Hamiltonian
matrices, when only considering the element-wise norm, can lead to unbounded differences in eigen-
values/eigenvectors due to a significantκ(S)
∥S∥ratio. The corollary further elucidates the sensitivity of
eigenvalue perturbations to the size of the basis B.This phenomenon underscores the catastrophic
scaling associated with increasing B, which is a manifestation of the aforementioned SAD phe-
nomenon. We provide a thorough discussion on αin Appendix J.3. To validate our theoretical
analysis, we empirically evaluated the distribution ofκ(S)
∥S∥2across both the QH9 and PubChemQH
datasets (Figure 6). The results demonstrate that molecules in PubChemQH exhibit substantially
higher ratios compared to QH9, indicating increased perturbation sensitivity in larger molecular
systems. This provides strong empirical evidence for the SAD phenomenon and aligns with our
theoretical predictions. Consequently, these findings suggest that when designing an effective super-
visory signal for learning or optimization tasks involving Hamiltonian matrices, it is crucial to take
into account the interaction of the overlap matrix and the corresponding Hamiltonian to mitigate the
potential instability caused by perturbations.
In light of this, we propose a new loss function: the Wavefunction Alignment Loss. It is designed to
preserve the integrity of the eigenstructure related to molecular orbitals. Let ˆϵandϵ⋆represent the
eigenvalues (orbital energies) of ˆHandH⋆, respectively, and ˆCandC⋆denote the corresponding
eigenvectors (basis coefficients). We define a primary form of the loss function by directly applying
the Frobenius norm to the eigenvalues Lalign=1
nPn
i=1∥ˆϵ(i)−ϵ⋆
(i)∥2
F,where ˆϵ(i)andϵ⋆
(i)are derived
from solving the generalized eigenvalue problems: ˆHˆC=SˆCˆϵandH⋆C⋆=SC⋆ϵ⋆, respectively.
However, this formulation has notable limitations: (1) Generalized eigenvalue problems are suscepti-
ble to numerical instabilities due to ill-conditioned matrices, leading to erroneous gradients during
backpropagation through iterative eigensolvers, complicating optimization. (2) The loss function
assigns uniform weights to all orbital energies, which is not practical as some orbital energies hold
more significance. To address these issues, we begin by applying the following algorithm (Ghojogh
et al., 2019; Golub \& Van Loan, 2013) to perform a simultaneous reduction of the matrix pair (H⋆,S):
Algorithm 1 Simultaneous reduction of a matrix pair (H⋆,S)
Require: Groud-truth Hamiltonian matrix H⋆and overlap matrix S
Ensure: Diagonal matrix ϵ⋆and matrix C⋆such that (C⋆)⊤SC⋆=Iand(C⋆)⊤H⋆C⋆=ϵ⋆
1:Compute the Cholesky decomposition S=GG⊤.
2:Define M⋆=G−1H⋆G−⊤.
3:Apply the symmetric QR algorithm to find the Schur form (Q⋆)⊤M⋆Q⋆=ϵ⋆.
4:Compute C⋆=G−⊤Q⋆.
When the overlap matrix Sis ill-conditioned, the eigenvalues ϵ⋆computed by Algorithm 1 can suffer
from significant roundoff errors. To mitigate this, we modify the algorithm by replacing the Cholesky
decomposition of Swith its eigen (Schur) decomposition V⊤SV=Σ, where Vis the matrix of
eigenvectors and Σis the diagonal matrix of eigenvalues. We then substitute GwithVΣ−1/2. This
modification effectively reorders the entries of M⋆, placing larger values towards the upper left-hand
corner, thereby enhancing the precision in computing smaller eigenvalues (Golub \& Van Loan, 2013;
Wilkinson, 1988).
5

Published as a conference paper at ICLR 2025
Claim 1. Under optimal convergence condition, the ground truth eigenvector C⋆should diagonalize
the predicted transformed matrix ˆH, thereby satisfying the relation (C⋆)⊤ˆHC⋆=ϵ⋆.
Accordingly, we propose a refined loss function:
LWA=1
nnX
i=1∥(C⋆
(i))⊤ˆH(i)C⋆
(i)−ϵ⋆
(i)∥2
F, (2)
where the subscript (·)(i)denotes the i-th sample. This modified loss function explicitly penalizes
deviations from the expected eigenstructure. It is important to note that while (C⋆)⊤ˆHC⋆may not
yield a diagonal matrix, the diagonal nature of ϵ⋆implicitly enforces the predicted eigenstructure
through the non-diagonal elements of the loss function. Here, we provide another derivation of
the WALoss by drawing a parallel to first-order perturbation theory. In perturbation theory, for
a Hamiltonian ˆH=H⋆+ ∆H, where ∆His the perturbation. The first-order energy shift is
δϵ=R
ψ⋆†∆Hψ⋆dr, using the unperturbed eigenvectors ψ⋆ofH⋆, with the assumption that these
eigenvectors remain fixed while adjusting the eigenvalues. Similarly, WALoss employs the ground-
truth eigenvectors C⋆fromH⋆C⋆=SC⋆ϵ⋆to evaluate the predicted Hamiltonian H⋆+ ∆H. The
loss could be rewritten as LWA=1
nPn
i=1∥(C⋆
(i))⊤(H⋆+ ∆H)(i)C⋆
(i)−ϵ⋆
(i)∥2
F, assessing how
wellH⋆+ ∆Hreproduces ϵ⋆in the C⋆basis. Here, (C⋆)⊤(H⋆+ ∆H)C⋆=ϵ⋆+ (C⋆)⊤∆HC⋆,
where (C⋆)⊤∆HC⋆mirrors the perturbation termR
ψ⋆†∆Hψ⋆dr.
It is widely known that energies associated with occupied molecular orbitals (and LUMO) make
up the most energy of the molecule. Thus, to capture the most relevant part of the eigenspectrum,
we calculate the loss function with increased weight on the k+ 1 lowest eigenvalues. Here, k
corresponds to the number of occupied orbitals2, and an additional eigenvalue accounts for the
Lowest Unoccupied Molecular Orbital (LUMO). Let Irepresent the set of indices corresponding to
thek+ 1lowest eigenvalues, and ((·)(i))jindexes the j-th eigenvalue for the i-th sample. The loss
function is defined as follows:
LWA=1
nnX
i=1
ρX
j∈I∥(C⋆
(i))⊤
jˆH(i)(C⋆
(i))j−(ϵ⋆
(i))j∥2
F+ξX
j /∈I∥(C⋆
(i))⊤
jˆH(i)(C⋆
(i))j−(ϵ⋆
(i))j∥2
F
,
(3)
where ndenotes the number of samples, and ρ, ξare hyperparameters where ρ≫ξ. This adaptation
ensures that the loss function places greater emphasis on the eigenvectors and eigenvalues corre-
sponding to the occupied orbitals and LUMO. This improves the model’s focus on the critical part of
the eigenspace, which is crucial for practical applications.

\section{WAN ET}

Here, we present WANet, a modernized architecture for Hamiltonian prediction. First, unlike previous
approaches, we propose a streamlined design for Hamiltonian prediction that consists of two essential
components: the Node Convolution Layer and the Hamiltonian Head. The Node Convolution Layer
operates on a localized radius graph, performing graph convolution to capture intricate atomic
interactions. This block serves a dual purpose: first, it generates an irreducible node representation,
providing a powerful input for the subsequent Hamiltonian Head. Second, it can be initialized with a
pretrained EGNN or reprogrammed for other downstream tasks, constituting a unified framework for
molecular modeling. The Hamiltonian Head constructs both pairwise and many-body irreducible
representations using the Clebsch-Gordon tensor product. These representations are then utilized to
assemble both the non-diagonal and diagonal components of the Hamiltonian matrix. The model
architecture and ablation studies are detailed in Figure 10 and Table 9, respectively.
4.1 N ODE CONVOLUTION LAYER
For the Node Convolution Layer, we replace the traditional SO(3) convolutions with Equivariant
Spherical Channel Network (eSCN) (Passaro \& Zitnick, 2023; Liao et al., 2023). The eSCN
2k=⌊N
2⌋for paired orbitals.
6

Published as a conference paper at ICLR 2025
framework primarily utilizes SO(2) linear operations, optimizing the computation of tensor products
involved in the convolution process. Traditionally, SO(3) convolutions operate on input irreducible
representation (irrep) features uℓimiand spherical harmonic projections Yℓfmf(ˆrts). By applying a
rotation matrix Dtstoˆrts, aligning it with the canonical axis where ℓ= 0andm= 0, the spherical
harmonic projection Yℓfmf(Dtsˆrts)becomes non-zero exclusively for mf= 0. This condition
simplifies the tensor product to C(ℓo,mo)
(ℓi,mi),(ℓf,0), which remains non-zero only when mi=±mo.
Subsequently, eSCN has demonstrated that this reformulation could be reparameterized using SO(2)
operations on the rotated tensors, and simplify the computation from O(L6)toO(L3),Lis the
degree of the representaion. A detailed mathematical framework of this method is elaborated in the
Appendix D.
4.2 H AMILTONIAN HEAD
Sparse Mixture of Long-Short-Range Experts We introduce a variant of the Gated Mixture-of-
Experts (GMoE) (Shazeer et al., 2017; Clark et al., 2022; Riquelme et al., 2021; Zoph et al., 2022;
Jiang et al., 2024) model by incorporating a Mixture-of-Experts (MoE) layer tailored for pairwise
molecular interactions. This enhancement draws inspiration from the Long-Short-Range Message
Passing framework (Li et al., 2023), which differentiates between handling proximal and distal
interactions through specialized layers. Our approach differentiates interaction dynamics based on
distance, with closer pairs experiencing distinct interaction profiles compared to more distant pairs.
This differentiation is achieved through a novel layer that substitutes the conventional pair interaction
layer with a sparse assembly of expert modules, each functioning autonomously as a Pair Construction
Layer. We define the Pair Construction layer with the function Fn
pairfor the n-th expert, and delineate
the output of the MoE layer with Nexperts as: FMoE(xt, xs) =PN
n=1pn(xt, xs)·Fn
pair(xt, xs),
where pn(xt, xs)are the gating probabilities computed by the gating network, and ·denotes scalar
multiplication. The gating probabilities are obtained by applying the Softmax function over the gating
scores of all experts: pn(xt, xs) =exp(Gn(z))PN
m=1exp(Gm(z)),where z= rbf( ∥rts∥2)applies a radial basis
function to the Euclidean distance ∥rts∥2with a distance cutoff, and Gn(z)represents the gating
score for the n-th expert, computed as: Gn(z) =z·Wgn+ϵn.Here, Wgnare learned gating weights
for expert n, and ϵnis injected noise to encourage exploration and promote load balancing among
experts. Specifically, we use Gumbel noise (Jang et al., 2016): ϵn=−log (−log (Un)), U n∼
Uniform(0 ,1).This noise enables a differentiable approximation of the top- Kselection, allowing
for sparse expert utilization while maintaining gradient flow during training. To further promote load
balancing among the experts, we introduce an auxiliary load balancing loss (Shazeer et al., 2017):
Lload \_balancing =NPN
n=1P
(t,s)pn(xt,xs)P
(t,s)12
,which encourages the gating network to allocate
routing probabilities evenly across experts, preventing underutilization of any single expert. The Pair
Construction layer for each expert is defined as:
Fn
pair(xt, xs)lo=P
li,ljWn
li,lj,lo
xlis⊗xlj
tlo
,
where xlisandxlj
tare the li-th and lj-th irreducible representations of source node sand target node
t, respectively; Wn
li,lj,loare the learned weights that couple these representations into the output
representation lo; and⊗denotes the tensor product.
Many-Body Interaction Layer Considering many-body interactions for the diagonal components
of the Hamiltonian in molecular systems captures essential electron correlation effects (Szabo \&
Ostlund, 2012; Jensen, 2017). These interactions provide a more accurate representation of the
collective behavior of atoms, beyond pairwise approximations. This leads to a precise description of
key quantum phenomena like electron delocalization and exchange interactions (Szabo \& Ostlund,
2012). For this purpose, we employ the methodologies of the MACE framework (Batatia et al.,
2022; Kovács et al., 2023; Batatia et al., 2023). Central to MACE is the adept conversion of first-
order features into higher-order features using the so-called density trick (Duval et al., 2023). This
procedure initiates with the formation of generalized Clebsch-Gordon tensor products from the
first-order features:
BL
M,ν=X
lmCLM
lm,ννY
ξ=1wlξfl
m;where lm= (l1m1, . . . , l νmν),
7

Published as a conference paper at ICLR 2025
where fl
mrepresents the input tensor, and BL
M,νthe resultant tensor for the ν-body. The coefficients
CLM
lm,ν are the generalized Clebsch-Gordan coefficients, ensuring the L-equivariance of the output
tensor BL
M,ν. Moreover, CLM
lm,ν is notably sparse and can be pre-computed efficiently.

\section{E XPERIMENTS}

In this section, we evaluate WANet with WALoss on the QH9 and PubChemQH datasets. Our
evaluation metrics include MAE for the Hamiltonian, ϵHOMO ,ϵLUMO ,ϵ∆,ϵocc, cosine similarity for
the eigenvectors C, and relative SCF iterations compared to the initial guess, which are commonly
used in previous works (Unke et al., 2021b; Yu et al., 2023b;a). Additionally, we introduce two
new physics-related metrics—MAE for ϵorband System Energy—to provide a more comprehensive
evaluation. Detailed descriptions of these metrics are provided in Appendix C.1.
5.1 R ESULTS ON THE PUBCHEM QH DATASET
Dataset Generation Process In our study, we investigated the scalability of Hamiltonian learning
by utilizing a CUDA-accelerated SCF implementation (Ju et al., 2024) to perform computational
quantum chemistry calculations, thereby generating the PubChemQH dataset. We began with ge-
ometries from the PubChemQC dataset by (Nakata \& Maeda, 2023), selecting only molecules with
a molecular weight above 400. This filtration process resulted in a dataset comprising molecules
with 40 to 100 atoms, totaling over 50,000 samples. We chose the B3LYP exchange-correlation
functional (Lee et al., 1988; Beeke, 1993; V osko et al., 1980; Stephens et al., 1994) and the Def2TZV
basis set (Weigend \& Ahlrichs, 2005; Weigend, 2006) to approximate electronic wavefunctions.
Generating this comprehensive dataset represents a substantial computational effort, requiring ap-
proximately one month of continuous processing using 128 NVIDIA-V100 GPUs . We provide a
comparison between the PubChemQH dataset and the QH9 dataset in Appendix H.
Table 1: PubChemQH experimental results. The energy units are presented in kcal/mol. N/A indicates that
the metric is not applicable to a specific model. The best-performing models are highlighted in bold. Detailed
training setups are provided in Table 12 and Table 13.
Model Hamiltonian MAE ↓ϵHOMO MAE↓ϵLUMO MAE↓ϵ∆MAE↓ϵoccMAE↓ϵorbMAE↓CSimilarity ↑System Energy MAE ↓relative SCF Iterations ↓
QHNet 0.7765 71.250 83.890 5.790 2087.45 1532.672 2.32\% 65721.028 371\%
WANet 0.6274 60.140 62.35 4.723 734.258 502.43 3.13\% 63579.233 334\%
QHNet w/ WALoss 0.5207 13.945 14.087 4.3982 21.805 10.930 46.66\% 75.625 90\%
PhisNet w/ WALoss 0.5166 11.872 12.075 4.054 20.045 8.917 46.78\% 60.325 90\%
WANet w/ WALoss 0.4744 0.7122 0.730 1.327 18.835 7.330 48.03\% 47.193 82\%
init guess ( minao ) 0.5512 29.430 28.521 4.955 42.740 35.183 0.3293 374.313 100\%
Equiformer V2 Regression N/A 6.955 6.562 3.222 N/A N/A N/A N/A N/A
UniMol+ Regression N/A 12.250 9.472 13.001 N/A N/A N/A N/A N/A
UniMol 2 Regression N/A 9.573 7.414 10.638 N/A N/A N/A N/A N/A
Table 1 presents a comparative analysis of various models’ performance on the PubChemQH dataset.
Despite a higher Hamiltonian MAE, WANet with WALoss significantly outperforms the other
models in practical utility, as evidenced by the System Energy MAE and the required SCF iterations.
Specifically, the System Energy MAE for WANet with WALoss is dramatically reduced from
63579.233 kcal/mol to 47.193 kcal/mol. Additionally, the relative SCF iterations required for WANet
with WALoss is only 82\%, compared to 371\% for QHNet and 334\% for WANet without WALoss.
This substantial reduction in SCF iterations demonstrates the effectiveness of WALoss in accelerating
the convergence process. Furthermore, it is worth noting that the initial guess matrix, although not
achieving as low an MAE as QHNet or WANet without WALoss, shows improved utility with a
System Energy MAE of 374.313 kcal/mol. This finding reinforces the idea that elementwise losses
are insufficient . Additional evidence supporting this idea is provided in Appendix C.2.
5.2 R ESULTS ON THE QH9 D ATASET
The QH9 dataset is a comprehensive quantum chemistry resource designed to support the devel-
opment and evaluation of machine learning models for predicting quantum Hamiltonian matrices.
Built upon the QM9 dataset, QH9 contains Hamiltonian matrices for 130,831 stable molecular
geometries, encompassing molecules with up to nine heavy atoms of elements C, N, O, and F. These
Hamiltonian matrices were generated using pyscf with the B3LYP functional (Lee et al., 1988;
Beeke, 1993; V osko et al., 1980; Stephens et al., 1994) and the def2SVP basis set. Table 2 presents a
comparative analysis of the performance of our model, WANet, against the baseline model, QHNet,
on the QH9 dataset in both stable and dynamic settings. In the QH9-stable experiments, WANet
8

Published as a conference paper at ICLR 2025
Table 2: Experimental results on the QH9 dataset. The energy units are presented in kcal/mol.
Model Hamiltonian MAE ↓ϵoccMAE↓Csimilarity ↑
QH9 stableQHNet 0.0513 0.5366 95.85\%
QHNet w/WAloss 0.0780 0.4901 96.35\%
WANet 0.0502 0.5231 96.86\%
WANet w/WAloss 0.0914 0.4587 96.95\%
QH9 dynamicQHNet 0.0471 0.2744 97.13\%
QHNet w/WAloss 0.0495 0.2658 98.54\%
WAnet 0.0469 0.2614 99.68\%
WAnet w/WAloss 0.0512 0.2500 99.81\%
demonstrates superior performance, achieving higher accuracy compared to QHNet. Specifically,
WANet significantly reduces both the Hamiltonian and occupied energy MAE while improving the
cosine similarity of C. For the QH9-dynamic dataset, WANet consistently outperforms QHNet,
further enhancing prediction accuracy. These results underscore the robustness and effectiveness of
WANet in both stable and dynamic scenarios.
5.3 C OMPARISON WITH A PROPERTY REGRESSION MODEL
Conventional machine learning approaches typically employ property regression, mapping molecular
features directly to the desired property value (Blum \& Reymond, 2009; Montavon et al., 2013).
A common question arises: why use Hamiltonians instead of a property regression model? We
argue that property regression methods often fail to incorporate underlying quantum mechanical
principles, limiting their generalization capability. To illustrate this, we compared the performance of
WANet with WALoss to a model utilizing Equiformer V2 (Liao \& Smidt, 2023), UniMol+ (Lu et al.,
2023), and UniMol2 (Xiaohong et al., 2024) with invariant regression heads, using identical training
and test sets. As shown in Table 1, WANet with WALoss demonstrates significantly lower MAE
values in predicting key quantum chemical properties. Specifically, WANet with WALoss achieves
an 88.88\% improvement in ϵLUMO MAE and a 58.81\% improvement in ϵ∆MAE. Moreover, the
Hamiltonian predicted by WANet with WALoss is not limited to specific properties. It enables the
accurate calculation of various critical properties, such as electronic densities, dipole moments, and
excited-state energies, all from a single model. Additionally, it can be applied to SCF acceleration. In
contrast, the Equiformer V2 regression model is constrained to predicting a narrow set of specific
properties, necessitating the training of a new model for each new property.
5.4 E FFICIENCY EVALUATION OF WAN ET
Figure 3(a): Wall-clock comparison
of WANet-Augmented DFT with tra-
ditional SCF iterations.
Training
Time(hr)Inference
Speed(it/s)Peak GPU
Memory(GB)0255075
39.22
1.0915.8690.43
0.4526.49WANet w/ WALoss
QHNet w/ WALoss
(b): Comparison of training and in-
ference efficiency and resource usage
between QHNet and WANet on the
PubChemQH dataset.WANet exhibits efficiency advantages in two aspects: (1) its ap-
plication to SCF relative to traditional DFT calculations, and (2)
its efficiency and resource usage compared to existing state-of-
the-art neural networks. Specifically, WANet can predict Hamil-
tonians for large molecular systems significantly faster than tra-
ditional DFT methods. Figure 3a presents a wall-clock time com-
parison between WANet-augmented SCF and traditional DFT
calculations. Although the neural network evaluation introduces
a small overhead, WANet substantially reduces the number of
required SCF iterations, resulting in a faster overall computation
time. This notable speed-up makes WANet particularly advanta-
geous for applications requiring rapid predictions for large molec-
ular systems, such as high-throughput virtual screening. Fur-
thermore, WANet outperforms QHNet in training and inference
efficiency on the PubChemQH dataset, offering faster training
times, improved inference speeds, and lower peak GPU memory
usage (Figure 3b).
5.5 M OLECULAR PROPERTIES BEYOND ENERGY PREDICTIONS
Table 3: Additional property predictions. We evaluated WANet in
deriving dipole moment and electronic spatial extent.
Model Dipole Moment MAE (D) Electronic Spatial Extent MAE (a.u.)
Equiformer V2 Regression 3.3221 0.1098
EGNN Regression 4.3412 0.0789
ViSNet Regression 4.5211 0.0982
Initial Guess 4.0170 0.0161
WANet w/ WALoss 3.3928 0.0076To validate the versatility of the
Hamiltonian predicted by WANet
with WALoss, we extended our exper-
iments to include predictions beyond
energy properties, specifically dipole
moment and electronic spatial extent.
As shown in Table 3, WANet with WALoss performs competitively in dipole moment prediction and
excels in predicting electronic spatial extent. This demonstrates the model’s ability to generalize
across multiple molecular properties, highlighting its potential for broader quantum mechanical
9

Published as a conference paper at ICLR 2025
calculations beyond energy-based properties. Further details on the derivation of these properties
from the Hamiltonian are provided in Appendix F.
5.6 S CALABILITY IN ELONGATED CARBON CHAIN
To evaluate the scalability of our model trained on PubChemQH, we conducted inference us-
ing elongated alkanes (C xH2x+2)3, a series of saturated hydrocarbons. We compared three mod-
els: our model with WALoss, our model without WALoss, and an initial guess algorithm.
10 20 30 40 50 60
Number of Carbon Atoms020406080MAE (kcal/mol)
HOMO MAE vs Number of Carbon Atoms
w/ W ALoss
init guess
w/o W ALoss
10 20 30 40 50 60
Number of Carbon Atoms050100150200250MAE (kcal/mol)LUMO MAE vs Number of Carbon Atoms
w/ W ALoss
init guess
w/o W ALossD1 D1 D2 D2
Figure 4: Model performance in predicting HOMO and LUMO energies for
elongated alkanes. The left panel shows the MAE for HOMO predictions,
while the right panel shows the MAE for LUMO predictions. “D1” indicates
that the atom count is within the range of the PubChemQH dataset, whereas
“D2” indicates that the atom count exceeds this range. The models compared
include WANet with WALoss (w/ WALoss), our model without WALoss (w/o
WALoss), and the initial guess (init guess). Notably, our model with WALoss
demonstrates superior performance in LUMO predictions and matches the best
HOMO performance, particularly in the “D2” region.The results, shown in Fig-
ure 4, demonstrate that
our model with WALoss
achieves enhanced perfor-
mance in predicting LUMO
and HOMO energies, partic-
ularly in the “D2” region,
where the atom count ex-
ceeds the range of the Pub-
ChemQH dataset. Notably,
our model with WALoss
also performs well on elon-
gated alkanes with up to
182 atoms—three times the
average atom count of the
PubChemQH training set
(60). These findings high-
light the effectiveness of
WALoss in enhancing the
scalability and applicability
of our model for predicting electronic properties in scalable homogeneous series, demonstrating its
potential for application to larger and more complex molecular systems. Additional analysis of the
scaling performance is provided in Appendix C.3.
5.7 A BLATION STUDY ON WAL OSS
To evaluate the effectiveness of our proposed WALoss, we conducted an ablation study with three
variations: full WALoss, naive WALoss, and WALoss without reweighting. The naive WALoss
applies the Frobenius norm to the eigenvalues leveraging backpropagation through eigensolvers,
defined as Lnaive=1
nPn
i=1∥ˆϵ(i)−ϵ⋆
(i)∥2
F, where ˆϵ(i)andϵ⋆
(i)are derived from solving generalized
eigenvalue problems. The WALoss without reweighting calculates the loss uniformly across all
eigenvalues. As shown in Table 4, the full WALoss achieves the lowest MAEs across all metrics. The
naive WALoss performs poorly, highlighting several challenges associated with optimizing the naive
loss function. Removing reweighting also degrades performance, though not as drastically. Overall,
these results validate the design choices in formulating WALoss to improve Hamiltonian prediction.
Table 4: Ablation study of WALoss on the PubChemQH dataset. The best-performing models are highlighted
in bold.
Model Hamiltonian MAE ↓ϵHOMO MAE↓ϵLUMO MAE↓ ϵ∆ ϵoccMAE↓ϵorbMAE↓ C↑ System Energy MAE ↓SCF Iteration ↓
Naive Loss 0.4912 50.174 55.630 3.634 632.220 486.322 5.36\% 13562.7 306\%
WALoss without Reweighting 0.4973 8.241 7.993 3.988 41.230 21.614 28.28\% 55.492 88\%
WALoss Complete 0.4744 0.7122 0.730 1.327 18.835 7.330 48.03\% 47.193 82\%

\section{C ONCLUSION AND LIMITATIONS}

In this work, we introduced WALoss, a loss function designed to improve the accuracy of predicted
Hamiltonians. Our experiments demonstrate that incorporating WALoss achieves state-of-the-art
performance by reducing prediction errors and accelerating SCF convergence. Additionally, we
introduced a new dataset, PubChemQH , and an efficient model, WANet . However, limitations remain,
such as the high computational cost of generating large training sets. Despite these challenges,
deep learning approaches incorporating WALoss show great promise in advancing computational
chemistry and materials science.
3Elongated alkanes (C xH2x+2) are only present in the test set.
10

Published as a conference paper at ICLR 2025
ACKNOWLEDGMENT
Tha authors would like to thank Gaoyuan Wang from Yale University for the discussion on the scaling
of the eigenvalues. This work is supported by ALW professorship funds to Mark Gerstein.
REFERENCES
Jan Almlöf, Knut Fægri Jr, and Knut Korsell. Principles for a direct scf approach to licao–moab-initio
calculations. Journal of Computational Chemistry , 3(3):385–399, 1982.
Brandon Anderson, Truong Son Hy, and Risi Kondor. Cormorant: Covariant molecular neural
networks. Advances in neural information processing systems , 32, 2019.
Nathan Argaman and Guy Makov. Density functional theory: An introduction. American Journal of
Physics , 68(1):69–79, 2000.
Ilyes Batatia, David P Kovacs, Gregor Simm, Christoph Ortner, and Gábor Csányi. Mace: Higher
order equivariant message passing neural networks for fast and accurate force fields. Advances in
Neural Information Processing Systems , 35:11423–11436, 2022.
Ilyes Batatia, Philipp Benner, Yuan Chiang, Alin M Elena, Dávid P Kovács, Janosh Riebesell,
Xavier R Advincula, Mark Asta, William J Baldwin, Noam Bernstein, et al. A foundation model
for atomistic materials chemistry. arXiv preprint arXiv:2401.00096 , 2023.
Simon Batzner, Albert Musaelian, Lixin Sun, Mario Geiger, Jonathan P Mailoa, Mordechai Kornbluth,
Nicola Molinari, Tess E Smidt, and Boris Kozinsky. E(3)-equivariant graph neural networks for
data-efficient and accurate interatomic potentials. Nature communications , 13(1):2453, 2022.
Axel D Beeke. Density-functional thermochemistry. iii. the role of exact exchange. J. Chem. Phys ,
98(7):5648–6, 1993.
KC Bhamu, Amit Soni, and Jagrati Sahariya. Revealing optoelectronic and transport properties of
potential perovskites cs2pdx6 (x= cl, br): a probe from density functional theory (dft). Solar
Energy , 162:336–343, 2018.
L. C. Blum and J.-L. Reymond. 970 million druglike small molecules for virtual screening in the
chemical universe database GDB-13. J. Am. Chem. Soc. , 131:8732, 2009.
Chi Chen, Weike Ye, Yunxing Zuo, Chen Zheng, and Shyue Ping Ong. Graph networks as a universal
machine learning framework for molecules and crystals. Chemistry of Materials , 31(9):3564–3572,
2019.
Aidan Clark, Diego de Las Casas, Aurelia Guy, Arthur Mensch, Michela Paganini, Jordan Hoffmann,
Bogdan Damoc, Blake Hechtman, Trevor Cai, Sebastian Borgeaud, et al. Unified scaling laws for
routed language models. In International conference on machine learning , pp. 4057–4086. PMLR,
2022.
Taco Cohen and Max Welling. Group equivariant convolutional networks. In Maria Florina Balcan
and Kilian Q. Weinberger (eds.), Proceedings of The 33rd International Conference on Machine
Learning , volume 48 of Proceedings of Machine Learning Research , pp. 2990–2999, New York,
New York, USA, 20–22 Jun 2016. PMLR. URL https://proceedings.mlr.press/
v48/cohenc16.html .
Andrea Dal Corso, Alfredo Pasquarello, and Alfonso Baldereschi. Density-functional perturbation
theory for lattice dynamics with ultrasoft pseudopotentials. Physical Review B , 56(18):R11369,
1997.
Weitao Du, He Zhang, Yuanqi Du, Qi Meng, Wei Chen, Nanning Zheng, Bin Shao, and Tie-Yan Liu.
Se (3) equivariant graph neural networks with complete local frames. In International Conference
on Machine Learning , pp. 5583–5608. PMLR, 2022.
11

Published as a conference paper at ICLR 2025
Alexandre Duval, Simon V Mathis, Chaitanya K Joshi, Victor Schmidt, Santiago Miret, Fragkiskos D
Malliaros, Taco Cohen, Pietro Liò, Yoshua Bengio, and Michael Bronstein. A hitchhiker’s guide
to geometric gnns for 3d atomic systems. arXiv preprint arXiv:2312.07511 , 2023.
Robert Eisberg and Robert Resnick. Quantum physics of atoms, molecules, solids, nuclei, and
particles . 1985.
Carrie A Farberow, James A Dumesic, and Manos Mavrikakis. Density functional theory calculations
and analysis of reaction pathways for reduction of nitric oxide by hydrogen on pt (111). Acs
Catalysis , 4(10):3307–3319, 2014.
Thorben Frank, Oliver Unke, and Klaus-Robert Müller. So3krates: Equivariant attention for interac-
tions on arbitrary length-scales in molecular systems. Advances in Neural Information Processing
Systems , 35:29400–29413, 2022.
Fabian Fuchs, Daniel Worrall, V olker Fischer, and Max Welling. Se (3)-transformers: 3d roto-
translation equivariant attention networks. Advances in neural information processing systems , 33:
1970–1981, 2020.
Michael Gastegger, Adam McSloy, Mathis Luya, K. T. Schütt, and R. J. Maurer. A deep neural
network for molecular wave functions in quasi-atomic minimal basis representation. The Journal
of Chemical Physics , 153(4):044123, 7 2020. ISSN 0021-9606. doi: 10.1063/5.0012911. URL
https://doi.org/10.1063/5.0012911 .
Johannes Gasteiger, Janek Groß, and Stephan Günnemann. Directional message passing for molecular
graphs. arXiv preprint arXiv:2003.03123 , 2020.
Johannes Gasteiger, Florian Becker, and Stephan Günnemann. Gemnet: Universal directional
graph neural networks for molecules. Advances in Neural Information Processing Systems , 34:
6790–6802, 2021.
Benyamin Ghojogh, Fakhri Karray, and Mark Crowley. Eigenvalue and generalized eigenvalue
problems: Tutorial. arXiv preprint arXiv:1903.11240 , 2019.
Gene H Golub and Charles F Van Loan. Matrix computations . JHU press, 2013.
Ganesh Hegde and R. Chris Bowen. Machine-learned approximations to density functional theory
hamiltonians. Scientific Reports , 7(1):42669, 2017. doi: 10.1038/srep42669. URL https:
//doi.org/10.1038/srep42669 .
Jan Hermann, Zeno Schätzle, and Frank Noé. Deep-neural-network solution of the electronic
schrödinger equation. Nature Chemistry , 12(10):891–897, 2020. doi: 10.1038/s41557-020-0544-y.
URLhttps://doi.org/10.1038/s41557-020-0544-y .
Pierre Hohenberg and Walter Kohn. Inhomogeneous electron gas. Physical review , 136(3B):B864,
1964.
Ida-Marie Høyvik. The spectrum of the atomic orbital overlap matrix and the locality of the virtual
electronic density matrix. Molecular Physics , 118:e1765034, 06 2020. doi: 10.1080/00268976.
2020.1765034.
Eric Jang, Shixiang Gu, and Ben Poole. Categorical reparameterization with gumbel-softmax. arXiv
preprint arXiv:1611.01144 , 2016.
Frank Jensen. Introduction to computational chemistry . John wiley \& sons, 2017.
Albert Q Jiang, Alexandre Sablayrolles, Antoine Roux, Arthur Mensch, Blanche Savary, Chris
Bamford, Devendra Singh Chaplot, Diego de las Casas, Emma Bou Hanna, Florian Bressand, et al.
Mixtral of experts. arXiv preprint arXiv:2401.04088 , 2024.
Bowen Jing, Stephan Eismann, Patricia Suriana, Raphael John Lamarre Townshend, and Ron Dror.
Learning from protein structure with geometric vector perceptrons. In International Conference on
Learning Representations , 2020.
12

Published as a conference paper at ICLR 2025
R. O. Jones. Density functional theory: Its origins, rise to prominence, and future. Rev. Mod. Phys. ,
87:897–923, Aug 2015. doi: 10.1103/RevModPhys.87.897. URL https://link.aps.org/
doi/10.1103/RevModPhys.87.897 .
Fusong Ju, Xinran Wei, Lin Huang, Andrew J. Jenkins, Leo Xia, Jia Zhang, Jianwei Zhu, Han Yang,
Bin Shao, Peggy Dai, Ashwin Mayya, Zahra Hooshmand, Alexandra Efimovskaya, Nathan A.
Baker, Matthias Troyer, and Hongbin Liu. Acceleration without disruption: Dft software-as-a-
service. 2024.
John Jumper, Richard Evans, Alexander Pritzel, Tim Green, Michael Figurnov, Olaf Ronneberger,
Kathryn Tunyasuvunakool, Russ Bates, Augustin Žídek, Anna Potapenko, et al. Highly accurate
protein structure prediction with alphafold. Nature , 596(7873):583–589, 2021.
Kuzma Khrabrov, Ilya Shenbin, Alexander Ryabov, Artem Tsypin, Alexander Telepov, Anton
Alekseev, Alexander Grishin, Pavel Strashnov, Petr Zhilyaev, Sergey Nikolenko, and Artur Kadurin.
nabladft: Large-scale conformational energy and hamiltonian prediction benchmark and dataset.
Phys. Chem. Chem. Phys. , 24:25853–25863, 2022. doi: 10.1039/D2CP03966D. URL http:
//dx.doi.org/10.1039/D2CP03966D .
Kuzma Khrabrov, Anton Ber, Artem Tsypin, Konstantin Ushenin, Egor Rumiantsev, Alexander
Telepov, Dmitry Protasov, Ilya Shenbin, Anton Alekseev, Mikhail Shirokikh, Sergey Nikolenko,
Elena Tutubalina, and Artur Kadurin. ∇2dft: A universal quantum chemistry dataset of drug-like
molecules and a benchmark for neural network potentials, 2024. URL https://arxiv.org/
abs/2406.14347 .
Dmitrii Kochkov, Tobias Pfaff, Alvaro Sanchez-Gonzalez, Peter Battaglia, and Bryan K. Clark.
Learning ground states of quantum hamiltonians with graph networks. 2021.
Walter Kohn and Lu Jeu Sham. Self-consistent equations including exchange and correlation effects.
Physical review , 140(4A):A1133, 1965.
Walter Kohn, Axel D Becke, and Robert G Parr. Density functional theory of electronic structure.
The journal of physical chemistry , 100(31):12974–12980, 1996.
Risi Kondor and Shubhendu Trivedi. On the generalization of equivariance and convolution in neural
networks to the action of compact groups. In Jennifer Dy and Andreas Krause (eds.), Proceedings
of the 35th International Conference on Machine Learning , volume 80 of Proceedings of Machine
Learning Research , pp. 2747–2755. PMLR, 10–15 Jul 2018. URL https://proceedings.
mlr.press/v80/kondor18a.html .
Dávid Péter Kovács, Ilyes Batatia, Eszter Sara Arany, and Gabor Csanyi. Evaluation of the mace
force field architecture: From medicinal chemistry to materials science. The Journal of Chemical
Physics , 159(4), 2023.
Chengteh Lee, Weitao Yang, and Robert G Parr. Development of the colle-salvetti correlation-energy
formula into a functional of the electron density. Physical review B , 37(2):785, 1988.
Ira N Levine, Daryle H Busch, and Harrison Shull. Quantum chemistry , volume 6. Pearson Prentice
Hall Upper Saddle River, NJ, 2009.
He Li, Zun Wang, Nianlong Zou, Meng Ye, Runzhang Xu, Xiaoxun Gong, Wenhui Duan, and
Yong Xu. Deep-learning density functional theory hamiltonian for efficient ab initio electronic-
structure calculation. Nature Computational Science , 2(6):367–377, 2022. doi: 10.1038/
s43588-022-00265-6. URL https://doi.org/10.1038/s43588-022-00265-6 .
Yunyang Li, Yusong Wang, Lin Huang, Han Yang, Xinran Wei, Jia Zhang, Tong Wang, Zun Wang,
Bin Shao, and Tie-Yan Liu. Long-short-range message-passing: A physics-informed framework to
capture non-local interaction for scalable molecular dynamics simulation, 2023.
Yunyang Li, Lin Huang, Zhihao Ding, Chu Wang, Xinran Wei, Han Yang, Zun Wang, Chang Liu,
Yu Shi, Peiran Jin, et al. E2former: A linear-time efficient and equivariant transformer for scalable
molecular modeling. arXiv e-prints , pp. arXiv–2501, 2025.
13

Published as a conference paper at ICLR 2025
Yi-Lun Liao and Tess Smidt. Equiformer: Equivariant graph attention transformer for 3d atomistic
graphs. arXiv preprint arXiv:2206.11990 , 2022.
Yi-Lun Liao and Tess Smidt. Equiformer: Equivariant graph attention transformer for 3d atomistic
graphs. In International Conference on Learning Representations , 2023. URL https://
openreview.net/forum?id=KwmPfARgOTD .
Yi-Lun Liao, Brandon Wood, Abhishek Das, and Tess Smidt. Equiformerv2: Improved equivariant
transformer for scaling to higher-degree representations. arxiv preprint arxiv:2306.12059 , 2023.
Yi Liu, Limei Wang, Meng Liu, Yuchao Lin, Xuan Zhang, Bora Oztekin, and Shuiwang Ji. Spherical
message passing for 3d molecular graphs. In International Conference on Learning Representations
(ICLR) , 2022.
Shuqi Lu, Zhifeng Gao, Di He, Linfeng Zhang, and Guolin Ke. Highly accurate quantum chemical
property prediction with uni-mol+. arXiv preprint arXiv:2303.16982 , 2023.
Erpai Luo, Xinran Wei, Lin Huang, Yunyang Li, Han Yang, Zun Wang, Chang Liu, Zaishuo Xia,
Jia Zhang, and Bin Shao. Efficient and scalable density functional theory hamiltonian prediction
through adaptive sparsity. arXiv preprint arXiv:2502.01171 , 2025.
Norman H March. Electron Correlations in the Solid State . World Scientific Publishing Company,
1999.
Richard M Martin. Electronic structure: basic theory and practical methods . Cambridge university
press, 2020.
Grégoire Montavon, Matthias Rupp, Vivekanand Gobre, Alvaro Vazquez-Mayagoitia, Katja Hansen,
Alexandre Tkatchenko, Klaus-Robert Müller, and O Anatole von Lilienfeld. Machine learning
of molecular electronic properties in chemical compound space. New Journal of Physics , 15(9):
095003, 2013. URL http://stacks.iop.org/1367-2630/15/i=9/a=095003 .
Albert Musaelian, Simon Batzner, Anders Johansson, Lixin Sun, Cameron J Owen, Mordechai
Kornbluth, and Boris Kozinsky. Learning local equivariant representations for large-scale atomistic
dynamics. Nature Communications , 14(1):579, 2023.
Maho Nakata and Toshiyuki Maeda. Pubchemqc b3lyp/6-31g*//pm6 dataset: the electronic structures
of 86 million molecules using b3lyp/6-31g* calculations. arXiv preprint arXiv:2305.18454 , 2023.
Frank Neese. Prediction of molecular properties and molecular spectroscopy with density functional
theory: From fundamental theory to exchange-coupling. Coordination Chemistry Reviews , 253
(5-6):526–563, 2009.
Jörg Neugebauer and Tilmann Hickel. Density functional theory in materials science. Wiley
Interdisciplinary Reviews: Computational Molecular Science , 3(5):438–448, 2013.
Maylis Orio, Dimitrios A Pantazis, and Frank Neese. Density functional theory. Photosynthesis
research , 102:443–453, 2009.
Robert G Parr and Weitao Yang. Density-functional theory of the electronic structure of molecules.
Annual review of physical chemistry , 46(1):701–728, 1995.
Saro Passaro and C Lawrence Zitnick. Reducing so (3) convolutions to so (2) for efficient equivariant
gnns. In International Conference on Machine Learning , pp. 27420–27438. PMLR, 2023.
Carlos Riquelme, Joan Puigcerver, Basil Mustafa, Maxim Neumann, Rodolphe Jenatton, André
Susano Pinto, Daniel Keysers, and Neil Houlsby. Scaling vision with sparse mixture of experts.
Advances in Neural Information Processing Systems , 34:8583–8595, 2021.
Soumya Sanyal, Janakiraman Balachandran, Naganand Yadati, Abhishek Kumar, Padmini Ra-
jagopalan, Suchismita Sanyal, and Partha Talukdar. Mt-cgcnn: Integrating crystal graph convo-
lutional neural network with multitask learning for material property prediction. arXiv preprint
arXiv:1811.05660 , 2018.
14

Published as a conference paper at ICLR 2025
Vıctor Garcia Satorras, Emiel Hoogeboom, and Max Welling. E (n) equivariant graph neural networks.
InInternational conference on machine learning , pp. 9323–9332. PMLR, 2021.
Kristof T Schütt, Huziel E Sauceda, P-J Kindermans, Alexandre Tkatchenko, and K-R Müller.
Schnet–a deep learning architecture for molecules and materials. The Journal of Chemical Physics ,
148(24), 2018.
K. T Schütt, Michael Gastegger, Alexandre Tkatchenko, K. R. Müller, and R. J. Maurer. Unifying
machine learning and quantum chemistry with a deep neural network for molecular wavefunctions.
Nature Communications , 10(1):5024, 2019. doi: 10.1038/s41467-019-12875-2. URL https:
//doi.org/10.1038/s41467-019-12875-2 .
Noam Shazeer, Azalia Mirhoseini, Krzysztof Maziarz, Andy Davis, Quoc Le, Geoffrey Hinton, and
Jeff Dean. Outrageously large neural networks: The sparsely-gated mixture-of-experts layer. arXiv
preprint arXiv:1701.06538 , 2017.
Guillem Simeon and Gianni De Fabritiis. Tensornet: Cartesian tensor representations for efficient
learning of molecular potentials. Advances in Neural Information Processing Systems , 36, 2024.
Alain St.-Amant, Wendy D Cornell, Peter A Kollman, and Thomas A Halgren. Calculation of molec-
ular geometries, relative conformational energies, dipole moments, and molecular electrostatic
potential fitted charges of small organic molecules of biochemical interest by density functional
theory. Journal of computational chemistry , 16(12):1483–1506, 1995.
Philip J Stephens, Frank J Devlin, Cary F Chabalowski, and Michael J Frisch. Ab initio calculation
of vibrational absorption and circular dichroism spectra using density functional force fields. The
Journal of physical chemistry , 98(45):11623–11627, 1994.
Qiming Sun. Libcint: An efficient general integral library for g aussian basis functions. Journal of
computational chemistry , 36(22):1664–1671, 2015.
Qiming Sun, Timothy C Berkelbach, Nick S Blunt, George H Booth, Sheng Guo, Zhendong Li, Junzi
Liu, James D McClain, Elvira R Sayfutyarova, Sandeep Sharma, et al. Pyscf: the python-based
simulations of chemistry framework. Wiley Interdisciplinary Reviews: Computational Molecular
Science , 8(1):e1340, 2018.
Qiming Sun, Xing Zhang, Samragni Banerjee, Peng Bao, Marc Barbry, Nick S Blunt, Nikolay A
Bogdanov, George H Booth, Jia Chen, Zhi-Hao Cui, et al. Recent developments in the pyscf
program package. The Journal of chemical physics , 153(2), 2020.
Attila Szabo and Neil S Ostlund. Modern quantum chemistry: introduction to advanced electronic
structure theory . Courier Corporation, 2012.
Nathaniel Thomas, Tess Smidt, Steven Kearnes, Lusann Yang, Li Li, Kai Kohlhoff, and Patrick Riley.
Tensor field networks: Rotation-and translation-equivariant neural networks for 3d point clouds.
arXiv preprint arXiv:1802.08219 , 2018.
Julian Tirado-Rives and William L Jorgensen. Performance of b3lyp density functional methods for a
large set of organic molecules. Journal of chemical theory and computation , 4(2):297–306, 2008.
Artur P Toshev, Gianluca Galletti, Johannes Brandstetter, Stefan Adami, and Nikolaus A Adams.
E (3) equivariant graph neural networks for particle-based fluid mechanics. arXiv preprint
arXiv:2304.00150 , 2023.
Oliver Thorsten Unke, Mihail Bogojeski, Michael Gastegger, Mario Geiger, Tess Smidt, and
Klaus Robert Muller. SE(3)-equivariant prediction of molecular wavefunctions and electronic
densities. In A. Beygelzimer, Y . Dauphin, P. Liang, and J. Wortman Vaughan (eds.), Advances in
Neural Information Processing Systems , 2021a. URL https://openreview.net/forum?
id=auGY2UQfhSu .
Oliver Thorsten Unke, Mihail Bogojeski, Michael Gastegger, Mario Geiger, Tess Smidt, and
Klaus Robert Muller. SE(3)-equivariant prediction of molecular wavefunctions and electronic
densities. In A. Beygelzimer, Y . Dauphin, P. Liang, and J. Wortman Vaughan (eds.), Advances in
Neural Information Processing Systems , 2021b. URL https://openreview.net/forum?
id=auGY2UQfhSu .
15

Published as a conference paper at ICLR 2025
JH Van Lenthe, R Zwaans, Huub JJ Van Dam, and MF Guest. Starting scf calculations by superposi-
tion of atomic densities. Journal of computational chemistry , 27(8):926–932, 2006.
Tanja Van Mourik, Michael Bühl, and Marie-Pierre Gaigeot. Density functional theory across
chemistry, physics and biology, 2014.
Seymour H V osko, Leslie Wilk, and Marwan Nusair. Accurate spin-dependent electron liquid
correlation energies for local spin density calculations: a critical analysis. Canadian Journal of
physics , 58(8):1200–1211, 1980.
Limei Wang, Yi Liu, Yuchao Lin, Haoran Liu, and Shuiwang Ji. Comenet: Towards complete and
efficient message passing for 3d molecular graphs. Advances in Neural Information Processing
Systems , 35:650–664, 2022.
Yi Wang, Mingqing Liao, Brandon J Bocklund, Peng Gao, Shun-Li Shang, Hojong Kim, Allison M
Beese, Long-Qing Chen, and Zi-Kui Liu. Dfttk: Density functional theory toolkit for high-
throughput lattice dynamics calculations. Calphad , 75:102355, 2021.
Tim M Watson and Jonathan D Hirst. Density functional theory vibrational frequencies of amides
and amide dimers. The Journal of Physical Chemistry A , 106(34):7858–7867, 2002.
J Beau W Webber. A bi-symmetric log transformation for wide-range data. Measurement Science
and Technology , 24(2):027001, 2012.
Florian Weigend. Accurate coulomb-fitting basis sets for h to rn. Physical chemistry chemical physics ,
8(9):1057–1065, 2006.
Florian Weigend and Reinhart Ahlrichs. Balanced basis sets of split valence, triple zeta valence and
quadruple zeta valence quality for h to rn: Design and assessment of accuracy. Physical Chemistry
Chemical Physics , 7(18):3297–3305, 2005.
J.H. Wilkinson. The Algebraic Eigenvalue Problem . Monographs on numerical analysis. Clarendon
Press, 1988. ISBN 9780198534181. URL https://books.google.com/books?id=
5wsK1OP7UFgC .
Ji Xiaohong, Wang Zhen, Gao Zhifeng, Zheng Hang, Zhang Linfeng, Ke Guolin, and E Weinan.
Uni-mol2: Exploring molecular pretraining model at scale. arXiv preprint arXiv:2406.14969 ,
2024.
Shi Yin, Xinyang Pan, Xudong Zhu, Tianyu Gao, Haochong Zhang, Feng Wu, and Lixin He.
Harmonizing so(3)-equivariance with neural expressiveness: a hybrid deep learning framework
oriented to the prediction of electronic structure hamiltonian. 2024.
Haiyang Yu, Meng Liu, Youzhi Luo, Alex Strasser, Xiofeng Qian, Xioning Qian, and Shuiwang Ji.
Qh9: A quantum hamiltonian prediction benchmark for qm9 molecules. In Advances in Neural
Information Processing Systems , 2023a. URL https://arxiv.org/pdf/2306.09549.
pdf.
Haiyang Yu, Zhao Xu, Xiaofeng Qian, Xiaoning Qian, and Shuiwang Ji. Efficient and equivariant
graph networks for predicting quantum hamiltonian. In International Conference on Machine
Learning , pp. 40412–40424. PMLR, 2023b.
He Zhang, Chang Liu, Zun Wang, Xinran Wei, Siyuan Liu, Nanning Zheng, Bin Shao, and Tie-Yan
Liu. Self-consistency training for hamiltonian prediction. 2024.
Yang Zhong, Hongyu Yu, Mao Su, Xingao Gong, and Hongjun Xiang. Transferable equivariant
graph neural networks for the hamiltonians of molecules and solids. npj Computational Materials ,
9(1):182, 2023.
Barret Zoph, Irwan Bello, Sameer Kumar, Nan Du, Yanping Huang, Jeff Dean, Noam Shazeer, and
William Fedus. St-moe: Designing stable and transferable sparse expert models. arXiv preprint
arXiv:2202.08906 , 2022.
16

Published as a conference paper at ICLR 2025
Appendix Table of Contents
A Notation Summary 18
B Additional Related Work 19
C Additional Experiments 20
C.1 Evaluation Metrics . . . . . . . . . . . . . . . . . . . . . . . . . . . . . . . . . . 20
C.2 SAD Phenomenon . . . . . . . . . . . . . . . . . . . . . . . . . . . . . . . . . . 20
C.3 Error Scaling Analysis . . . . . . . . . . . . . . . . . . . . . . . . . . . . . . . . 21
C.4 Overfitting Risk Evaluation . . . . . . . . . . . . . . . . . . . . . . . . . . . . . . 22
C.5 Energy Predictions for Elongated Alkanes . . . . . . . . . . . . . . . . . . . . . . 22
C.6 Additional Ablation Study . . . . . . . . . . . . . . . . . . . . . . . . . . . . . . 22
D eSCN convolution 23
E Additional Background 26
E.1 Example . . . . . . . . . . . . . . . . . . . . . . . . . . . . . . . . . . . . . . . . 29
F Dipole Moment and Electronic Spatial Extent 30
G Group Theory 31
H Exensive Details on PubChemQH 32
I Extensive Details on WANet 32
J Additional Theory 37
J.1 Proof of Theorem . . . . . . . . . . . . . . . . . . . . . . . . . . . . . . . . . . . 37
J.2 Perturbation Analysis and Growth of Eigenvalues . . . . . . . . . . . . . . . . . . 39
J.3 Scaling Law of the Smallest Eigenvalue of Overlap Matrix . . . . . . . . . . . . . 41
J.3.1 k-Nearest-Neighbor Overlaps . . . . . . . . . . . . . . . . . . . . . . . . 41
K Training 42
L Discussion 44
L.1 Discussion on System Energy Error . . . . . . . . . . . . . . . . . . . . . . . . . 44
L.2 Additional Conformational Energy Analysis . . . . . . . . . . . . . . . . . . . . . 44
L.3 Discussion on SCF acceleration ratio . . . . . . . . . . . . . . . . . . . . . . . . . 44
17

Published as a conference paper at ICLR 2025
A N OTATION SUMMARY
Table 5: Summary of main notation used in the paper.
Notation Description
M:=\{Z,R\} Molecular system defined by nuclear charges Zand positions R
r∈R3Spatial coordinate of an electron
ψi(r) Single-electron orbital/wavefunction ias a function of electron spatial coordinate r
ρ(r) Electron density as a function of spatial coordinate r
B Number of basis functions in the basis set used to represent orbitals
ϕα(r)B
α=1Basis set consisting of Bbasis functions ϕα(r)
C∈RB×NCoefficient matrix where each column contains the basis function coefficients for an orbital
H∈RB×BHamiltonian matrix in the chosen basis representation
S∈RB×BOverlap matrix with elements Sαβ:=R
ϕ†
α(r)ϕβ(r)dr
ϵ∈RN×NDiagonal matrix containing orbital energies (eigenvalues of the Hamiltonian)
ˆHθ(M) Machine learning model parameterized by θfor predicting the Hamiltonian Hfrom a molecular structure M
LWA Wavefunction Alignment Loss function
ˆϵ,ϵ Predicted and ground truth eigenvalues (orbital energies) of ˆHandH
ˆC,C Predicted and ground truth eigenvectors (basis coefficients) of ˆHandH
H⋆Ground truth Hamiltonian matrix
ˆM Predicted transformed Hamiltonian matrix in an orthogonal basis
G Matrix obtained from the eigen decomposition of the overlap matrix S
k Number of occupied orbitals
I Set of indices corresponding to the k+ 1lowest eigenvalues
ρ Hyperparameter controlling the weight of occupied and unoccupied orbitals in the loss function
n Number of samples in the dataset / the n-th experts in MoE nodel
D Dataset used for training the machine learning model
N Number of electrons in the system / total number of experts in MoE model
H(0)Initial guess for the Hamiltonian matrix
H(k)Hamiltonian matrix at the k-th SCF iteration
C(k)Coefficient matrix at the k-th SCF iteration
ϵ(k)Diagonal matrix of orbital energies at the k-th SCF iteration
δ Convergence threshold for the SCF procedure
ℓ Degree of the irreducible representation (irrep) in SO(3) equivariant networks
D(ℓ)(g) Wigner D-matrix representation of group element g∈SO(3) of degree ℓ
Y(ℓ)(ˆr) Real spherical harmonics of degree ℓevaluated at unit vector ˆr
C(ℓ3,m3)
(ℓ1,m1),(ℓ2,m2)Clebsch-Gordan coefficients coupling irreps of degree ℓ1andℓ2into irrep of degree ℓ3
rts Vector pointing from node sto node t
ˆrts Unit vector pointing from node sto node t
Ltotal Total loss function combining Lalignand mean squared error (MSE) loss
λ1, λ2, λ3 Hyperparameters controlling the weights of different loss terms in Ltotal
18

Published as a conference paper at ICLR 2025
B A DDITIONAL RELATED WORK
Predicting Kohn-Sham Hamiltonians Early work on predicting Kohn-Sham Hamiltonians used
kernel ridge regression (Hegde \& Bowen, 2017), while newer approaches use neural networks (NNs).
Some NNs predict the wavefunction itself (Schütt et al., 2019; Gastegger et al., 2020; Hermann
et al., 2020), while others use equivariant (Unke et al., 2021a; Kochkov et al., 2021; Yu et al.,
2023b; Li et al., 2022; Zhong et al., 2023) or hybrid architectures (Yin et al., 2024) to predict the
molecular Hamiltonian. A novel training method has been proposed to address the scarcity of labeled
data (Zhang et al., 2024), and two benchmark datasets aim to standardize evaluation of molecular
Hamiltonian prediction (Khrabrov et al., 2022; Yu et al., 2023a; Khrabrov et al., 2024).
Equivariant Graph Neural Networks (EGNNs) It is often desired that machine learning (ML)
models exhibit equivariance to rotations, translations, or reflections, which guarantee that they respect
certain physical symmetries. Foundational work introduced group equivariant convolutional neural
networks (ECNNs) (Cohen \& Welling, 2016), whose importance was underscored by a proof that
equivariance and convolutional structure are equivalent given certain ordinary constraints (Kondor \&
Trivedi, 2018). One appealing way to implement convolution is with geometric graph neural networks
(geometric GNNs), which apply naturally to atomic systems by encoding them as graphs embedded in
R3. Two important families of geometric GNNs are invariant GNNs and Cartesian equivariant GNNs.
Over the past several years, invariant GNNs have achieved state-of-the-art results in predicting proper-
ties of molecules, crystals, and other materials (Schütt et al., 2018) (Sanyal et al., 2018) (Chen et al.,
2019) (Gasteiger et al., 2020) (Liu et al., 2022) (Gasteiger et al., 2021) (Wang et al., 2022), as well as
in predicting the folding structure of proteins (Jumper et al., 2021). Cartesian equivariant GNNs have
seen success in similar areas, benefiting from the greater flexibility of their representations (Jing et al.,
2020) (Satorras et al., 2021) (Du et al., 2022) (Simeon \& De Fabritiis, 2024). Cartesian equivariant
GNNs have also seen recent innovation in Cartesian equivariant transformer layers (Frank et al.,
2022). A third significant family of geometric GNNs is spherical equivariant GNNs, which use
spherical tensors rather than Cartesian tensors. As as a result, spherical equivariant GNNs behave
more naturally under rotations and avail themselves of many results of the representation theory of
SO(3) . They have shown dexterity in tasks in geometry, physics, and chemistry (Thomas et al., 2018);
modeled dynamic molecular systems (Anderson et al., 2019); enabled SO(3) - and SE(3) -equivariant
transformer layers (Fuchs et al., 2020) (Liao \& Smidt, 2022); accuarately and efficiently calculated
interatomic potentials (Batzner et al., 2022) (Batatia et al., 2022) (Musaelian et al., 2023); enabled
E(3)-equivariant fluid mechanical modeling (Toshev et al., 2023); and improved efficiency by
reducing certain convolution computations in SO(3) to equivalent ones in SO(2) (Passaro \& Zitnick,
2023). Li et al. (2025) propose to use Wiger -6jConvolution to shift computation from edges to
nodes.
19

Published as a conference paper at ICLR 2025
C A DDITIONAL EXPERIMENTS
C.1 E VALUATION METRICS
In this section, we provide a detailed description of the metrics in the main-text:
MAE for Hamiltonian The MAE for Hamiltonian assesses the accuracy of the predicted Hamilto-
nian matrices. This metric is crucial for evaluating the quality of the predicted electronic structure
and its components, which are foundational in quantum chemistry calculations.
MAE for ϵHOMO The MAE for the Highest Occupied Molecular Orbital (HOMO) energy ( ϵHOMO )
evaluates the precision of the predicted HOMO levels. Accurate HOMO predictions are essential as
they influence a molecule’s chemical reactivity and stability.
MAE for ϵLUMO The MAE for the Lowest Unoccupied Molecular Orbital (LUMO) energy
(ϵLUMO ) measures the accuracy of LUMO level predictions. Accurate LUMO predictions are critical
for understanding a molecule’s electron affinity and chemical behavior.
MAE for ϵ∆ The MAE for the energy gap between HOMO and LUMO ( ϵ∆) assesses the precision
of this important property, which determines the electronic properties and conductivity of materials.
MAE for ϵocc The MAE for occupied orbital energies ( ϵocc) assesses the accuracy of predicted
energies for all occupied molecular orbitals, providing a comprehensive measure of how well the
model captures the electronic structure.
MAE for ϵorb The MAE for all orbital energies ( ϵorb) measures the discrepancies in predicted
energies for both occupied and unoccupied orbitals. This metric evaluates the overall accuracy of the
model in predicting the entire spectrum of orbital energies.
MAE on Total Energy The MAE on total energy assesses the accuracy of the predicted total
energies derived from Hamiltonian matrices using pyscf . This metric is crucial for validating
the model’s accuracy in predicting the overall energy of the system, which is fundamental for
understanding molecular stability and reactions.
Cosine Similarity for Wavefunction/Eigenvectors (C) The cosine similarity for wavefunc-
tion/eigenvectors (C) measures the similarity between the predicted and actual wavefunctions or
eigenvectors of the system. High cosine similarity indicates that the predicted wavefunction distribu-
tion closely matches the actual distribution, which is important for accurately modeling electronic
properties.
SCF Iteration The SCF (Self-Consistent Field) iteration count evaluates the number of iterations
required to achieve convergence in DFT (Density Functional Theory) calculations using the predicted
Hamiltonian matrices when comparing with the inital guess. Mathmatically, it is defined as:
σ=Predicted Hamiltonian Iteractions
Inital Hamiltonian Iteractions.
This metric assesses the efficiency of the predicted matrices in expediting the DFT calculations.
C.2 SAD P HENOMENON
We present an additional graphical illustration of the SAD phenomenon discussed in the main
text, depicting the learning curve of the QHNet model using only elementwise loss on a subset of
thePubChemQH dataset4. Figure 5 shows that as the MAE decreases, the system energy exhibits
significant fluctuations. Notably, when the Hamiltonian’s MAE is around 0.3, the system energy MAE
reaches 30,000 kcal/mol. This highlights the non-monotonic relationship between the Hamiltonian
MAE and the resulting system energy. This extreme instability in derived properties, despite a
seemingly small Hamiltonian MAE, is a key characteristic of the SAD phenomenon, hindering the
applicability of the predicted Hamiltonian.
4This differs from the dataset used in the maintexts.
20

Published as a conference paper at ICLR 2025
0 250 500 750 1000 1250 1500 1750
Step0.300.350.400.450.500.550.60Hami-MAE
Hami-MAE vs Step
0 250 500 750 1000 1250 1500 1750
Step010000200003000040000System Energy
System Energy vs Step
Figure 5: An additional graphical illustration of the SAD phenomenon discussed in the main text,
depicting the learning curve of the QHNet model using only elementwise loss. The y-axis represents
the metric, and the x-axis represents the steps. The figure shows that as the MAE decreases, the
system energy exhibits significant fluctuations.
PubChemQH QH9
102
103
104
105Value (1/e) (L og Scale)Distribution of the Smallest Eigenval for S 
PubchemQH
QH9
Figure 6: the distribution ofκ(S)
∥S∥on QH9 and PubChemQH dataset
C.3 E RROR SCALING ANALYSIS
In Figure 4 and Table 6, we analyzed the error trend scaling for "intensive" properties such as ϵHOMO
andϵLUMO with an increasing number of electrons. This challenge, widely recognized in the field (Yu
et al., 2023a; Zhang et al., 2024), highlights a difficult out-of-distribution (OOD) generalization
scenario: models trained on smaller molecules often struggle to generalize to larger ones, limiting
their applicability. Previous state-of-the-art models have exhibited significant scaling errors when
extrapolating to larger systems (Yu et al., 2023a; Zhang et al., 2024).
We conducted an error scaling analysis for saturated carbon chains to assess the scalability of our
model. The error scaling coefficients for the HOMO and LUMO energies, as well as the energy gap,
are presented in Table 6. Our results show significantly lower error scaling compared to baseline
models for saturated carbon chains. Our model substantially outperforms the one without WALoss, as
reflected by the scaling coefficients in Table R2, marking a considerable improvement in extrapolation
capabilities within the field.
We further analyze the smallest eigenvalue distribution which is now included in Figure 6.
21

Published as a conference paper at ICLR 2025
Table 6: Scaling Coefficients of HOMO, LUMO, and Energy Gap with Carbon Atom Count
Model HOMO Scaling Coefficients LUMO Scaling Coefficients Gap Scaling Coefficients
WANet (with WALoss) 0.4211 0.4003 0.0206
Initial Guess 0.4221 0.4006 0.0214
Without WALoss 1.3639 2.7646 1.4007
C.4 O VERFITTING RISKEVALUATION
We evaluated the risk of overfitting to the PubChemQH dataset. As shown in Table 7, the evaluation
metrics on unseen data are comparable to those on the training set, indicating that our model
generalizes well and does not overfit.
Table 7: Overfitting Risk Evaluation
PubChem Hamiltonian MAE ↓ϵHOMO MAE↓ϵLUMO MAE↓ ϵ∆ ϵoccMAE↓ϵorbMAE↓ C↑ System Energy MAE ↓
Training MAE 0.4736 6.073 4.414 3.753 17.850 6.159 48.01\% 46.036
Test MAE 0.4744 0.7122 0.730 1.327 18.835 7.330 48.03\% 47.193
C.5 E NERGY PREDICTIONS FOR ELONGATED ALKANES
We have incorporated additional plots (Figure 7) that illustrate the predicted and actual values for
both LUMO and HOMO energies, as well as the HOMO-LUMO gap. As the number of carbon atoms
increases, both HOMO and LUMO energies exhibit an initial rise followed by a plateau with slight
increases. Our model, utilizing the WALoss method, effectively captures this trend, outperforming
baseline methods. Notably, Equiformer V2’s HOMO predictions fail to generalize to saturated
carbon systems, as these systems were not included in the training set. This observation underscores
the advantage of Hamiltonian-based models in capturing physical principles and achieving better
generalization compared to property regression models.
For isolated molecular systems, such as those studied here, computational chemists are primarily
concerned with per-electron energy levels. To demonstrate this, we plot the energy levels within a
reasonable window centered around the HOMO level and compare these with ground truth values
obtained from DFT calculations. As shown in Figure 7 A, our model closely replicates the ground
truth energy levels, outperforming baseline models. These results highlight the effectiveness of our
approach in accurately predicting the electronic structure of molecules.
C.6 A DDITIONAL ABLATION STUDY
In contrast to prior research, our study addresses the substantial challenges presented by the scale of
the PubChemQC t-zvp dataset, particularly in providing the model with a strong numeric starting point.
To overcome this difficulty, we shifted our focus to a more tractable objective: making predictions
based on an easily obtainable initial guess. Our ablation study shows that while predictions based on
the initial guess significantly improve performance in Hamiltonian prediction, they struggle when
applied to the prediction of physical properties.
Table 8: Performance Comparison of Baseline and Initial Guess Models on PubChemQH Dataset.
The best models are bolded.
Model Hamiltonian MAE ↓ϵHOMO MAE↓ϵLUMO MAE↓ϵ∆MAE↓ϵoccMAE↓ϵorbMAE↓CSimilarity ↑System Energy MAE ↓relative SCF Iterations ↓
WANet 0.6274 60.14 62.35 4.723 734.258 502.43 3.13\% 63579.233 334\%
WANet w/ Initial Guess 0.0379 54.107 56.628 3.618 695.418 481.513 4.42\% 60078.633 306\%
WANet w/ Initial Guess \& WALoss 0.4744 0.7122 0.730 1.327 18.835 7.330 48.03\% 47.193 82\%
To evaluate the contributions of each component in the WANet model, we conducted ablation studies.
The results are summarized in the Table 9. The ablation studies confirm the importance of each
architectural component in the WANet model.
22

Published as a conference paper at ICLR 2025
WANet Ground T ruth
Model−25−20−15−10−50Ener gy V alue (Aligned with HOMO)205210215220
200
w/o WA LossA B
C
Figure R1 . (A) The energy level for a saturated hydrocarbon system (C20H42). The energy is centered at 
HOMO for comparison. ( B) The predicted energy for differnet models on the satured hydrocarbon system.
The y-axis indicates the energy in kcal/mol. The x-axis denotes the number of the carbon atoms in the elo-
ngated carbon-chains. v2 indicates Equiformer V2. ‘wanet’ indicates WANet with WALoss. ‘groundtruth in-
dicates DFT calculations. ’init’ indicates a Fock matrix initialization algorithm using minao. ‘without’ indica-
tes WANet without WALoss. ( C) The HOMO-LUMO gap prediction for different models. 
Figure 7: ( A) The energy level for a saturated hydrocarbon system ( C20H42). The energy is centered
at HOMO for comparison. ( B) The predicted energy for differnet models on the satured hydrocarbon
system. The y-axis indicates the energy in kcal/mol. The x-axis denotes the number of the carbon
atoms in the elongated carbon-chains. v2 indicates Equiformer V2. ‘wanet’ indicates WANet with
WALoss. ‘groundtruth’ indicates DFT calculations. ‘init’ indicates a Fock matrix initialization
algorithm using minao. ‘without’ indicates WANet without WALoss. ( C) The HOMO-LUMO gap
prediction for different models.
Table 9: Ablation study results for the WANet model. The table shows the impact of different
architectural components.
w/ SO(2) w/ LSR-MoE w/ Many Body Inference Speed (it/s) GPU Memory Hamiltonian MAE ϵHOMO MAE↓ϵLUMO MAE↓ϵ∆MAE↓
✓ ✓ 1.34 13.87 0.5895 15.39 27.50 4.651
✓ ✓ 1.13 12.80 0.4883 2.27 4.01 2.906
✓ ✓ 0.51 24.47 0.4792 0.75 0.73 1.594
✓ ✓ ✓ 1.09 15.86 0.4744 0.71 0.73 1.327
D ESCN CONVOLUTION
The equivariant Spherical Channel Network (eSCN) is a graph neural network whose approach to
SO(3) -equivariant convolutions significantly reduces their computational burden. To see this, let’s
brush up on the notation used in the formula for the ℓo-th degree of the message mtsfrom source
node sto target node tin an SO(3) convolution: Wℓi,ℓf,ℓois a learnable weight; x(ℓi)
sis the ℓi-th
degree of the irrep feature of node s, the source node; ⊗is the tensor product; Y(ℓf)(ˆrts)is the ℓf-th
degree spherical harmonic projection of ˆrts; andˆrtsis the (normalized) relative position vector. Then,
the message is given by
m(ℓo)
ts=X
ℓi,ℓfWℓi,ℓf,ℓo
x(ℓi)
s⊗Y(ℓf)(ˆrts)(ℓo)
.
23

Published as a conference paper at ICLR 2025
Traditionally, this involves an ordinary SO(3) tensor product, which is decomposed into its irreps
using the Clebsch-Gordon coefficients, C(ℓo,mo)
(ℓi,mi),(ℓf,mf):
m(ℓo)
ts=X
ℓi,ℓfWℓi,ℓf,ℓoM
moX
mi,mf
x(ℓi)
s
miC(ℓo,mo)
(ℓi,mi),(ℓf,mf)
Y(ℓf)(ˆrts)
mf.
The tensor product, ⊗, has been exchanged for the proper Clebsch-Gordon coefficient. The direct
sum,L, is included because the space of tensors of degree ℓohas a basis indexed by mo.
However, the full SO(3) tensor product is compute-intensive, O(L6), where Lis the tensor degree.
In practice, this means that only tensors up to degree 2 or 3 are used, which is especially unfortunate
because higher-order tensors allow more precise representation of angular information. To lessen this
computational burden, eSCN rotates the irrep features, x(ℓi)
s, of a node xs, along with the relative
position vector, ˆrts, by a rotation chosen so that the relative position vector aligns with the y-axis.
Thus, it may be interpreted that eSCN changes the irreps into a more convenient basis, computes
the convolution, and changes back into the original basis. This requires multiplication by a change-
of-basis Wigner D-matrix before and its inverse after, but it is worthwhile because it reduces the
convolution to O(L3)by sparsifying the Clebsch-Gordon coefficients. In particular, eSCN guarantees
that the output coefficient is nonzero only if both mi=±moandmf= 0hold. The output tensors
of some order, therefore, are linear combinations of input tensors of that order. It is not necessary to
perform tensor multiplication and decomposition. The non-zero Clebsch-Gordon coefficients can
then be denoted C(ℓo,m)
(ℓi,m),(ℓf,0), with eSCN further guaranteeing that C(ℓo,m)
(ℓi,m),(ℓf,0)=C(ℓo,−m)
(ℓi,−m),(ℓf,0),
andC(ℓo,−m)
(ℓi,m),(ℓf,0)=−C(ℓo,m)
(ℓi,−m),(ℓf,0). Thus, eSCN guarantees that all but a few Clebsch-Gordon
coefficients will be zero, and provides simple formulas involving those that remain.
More formally, let Rbe a rotation matrix chosen so that R·ˆrts= (0,1,0), and let D(ℓ)be a Wigner
D-matrix representation of Rof degree ℓ(which would more properly read D(ℓ)(R), but which is
truncated for readability). Then, the message can be written
m(ℓo)
ts=
D(ℓo)−1X
ℓi,ℓfWℓi,ℓf,ℓo
D(ℓi)x(ℓi)
s⊗Y(ℓf)(R·ˆrts)(ℓo)
.
Rewriting the tensor products using the Clebsch-Gordon coefficients yields
m(ℓo)
ts=
D(ℓo)−1X
ℓi,ℓfWℓi,ℓf,ℓoM
moX
mi,mf
D(ℓi)x(ℓi)
s
miC(ℓo,mo)
(ℓi,mi),(ℓf,mf)
Y(ℓf)(R·ˆrts)
mf.
Since the Clebsch-Gordon coefficients are non-zero only when mf= 0, there is no need to sum over
mf:
m(ℓo)
ts=
D(ℓo)−1X
ℓi,ℓfWℓi,ℓf,ℓoM
moX
mi
D(ℓi)x(ℓi)
s
miC(ℓo,mo)
(ℓi,mi),(ℓf,0)
Y(ℓf)(R·ˆrts)
0.
The rotation Rhas been chosen so that R·ˆrtsyields a simple result, so the spherical harmonic term
drops out:
m(ℓo)
ts=
D(ℓo)−1X
ℓi,ℓfWℓi,ℓf,ℓoM
moX
mi
D(ℓi)x(ℓi)
s
miC(ℓo,mo)
(ℓi,mi),(ℓf,0).
Now, since one of the rotations on the righthand side has already dropped out, to make things simpler,
define ˜x(ℓi)
s=D(ℓi)x(ℓi)
s, yielding
m(ℓo)
ts=
D(ℓo)−1X
ℓi,ℓfWℓi,ℓf,ℓoM
moX
mi
˜x(ℓi)
s
miC(ℓo,mo)
(ℓi,mi),(ℓf,0).
Since the Clebsch-Gordon coefficients are non-zero only when mi=±mo, the second summation
can be omitted:
m(ℓo)
ts=
D(ℓo)−1X
ℓi,ℓfWℓi,ℓf,ℓoM
mo
˜x(ℓi)
s
moC(ℓo,mo)
(ℓi,mo),(ℓf,0)+
˜x(ℓi)
s
−moC(ℓo,mo)
(ℓi,−mo),(ℓf,0)
.
24

Published as a conference paper at ICLR 2025
Now, to avoid the need to sum over ℓf, rather than learning parameters Wℓi,ℓf,ℓo, eSCN learns
parameters ˜W(ℓi,ℓo)
m , defined for m≥0as
˜W(ℓi,ℓo)
m =X
ℓfWℓi,ℓf,ℓoC(ℓo,m)
(ℓi,m),(ℓf,0)=X
ℓfWℓi,ℓf,ℓoC(ℓo,−m)
(ℓi,−m),(ℓf,0),
and for m < 0as
˜W(ℓi,ℓo)
m =X
ℓfWℓi,ℓf,ℓoC(ℓo,m)
(ℓi,−m),(ℓf,0)=−X
ℓfWℓi,ℓf,ℓoC(ℓo,−m)
(ℓi,m),(ℓf,0).
There exists a linear bijection between Wand˜W, so this parameterization loses no information.
Finally, defining

yℓi,ℓo
ts
mo=˜W(ℓi,ℓo)
mo
˜xℓi
s
mo−˜W(ℓi,ℓo)
−mo
˜xℓi
s
−mo, m > 0;

y(ℓi,ℓo)
ts
mo=˜W(ℓi,ℓo)
mo
˜xℓi
s
−mo+˜W(ℓi,ℓo)
−mo
˜xℓi
s
mo, m < 0;

y(ℓi,ℓo)
ts
mo=˜W(ℓi,ℓo)
mo
˜xℓi
s
mo, m = 0,
the message equation can very concisely be written as
mℓo
ts=
Dℓo−1X
ℓiM
mo
yℓi,ℓo
ts
mo.
There is another way to interpret this that is perhaps more intuitive. Fixing the direction of the relative
position vector, ˆrstleaves a single rotational degree of freedom: the roll rotation about this axis. Thus,
eSCN reduces SO(3) convolution to SO(2) convolution. More formally, define the colatitude angle
θ∈[0, π]and longitudinal angle ϕ∈[0,2π]. Now, using Legendre polynomials P(ℓ)
m(θ), which
depend only on the colatitude angle, θ, the real spherical harmonic basis functions can be written
Y(ℓ)
m(θ, ϕ) =P(ℓ)
m(θ)eimϕ.
Aligning the relative position vector with the y-axis fixes θ, leaving behind basis functions of
the form eimϕ, which are the circular harmonic basis functions, used in SO(2) convolution. A
convolution about ϕcan take advantage of the Convolution Theorem, reducing convolution to
point-wise multiplication, further harvesting efficiency gains.
25

Published as a conference paper at ICLR 2025
E A DDITIONAL BACKGROUND
Quantum mechanics is most often approached as the study of the Schrödinger equation, a linear
partial differential equation whose solutions are called wavefunctions. In this paper, as is typical in
molecular modeling, the time-independent Schrödinger equation will be used:
Hψ=Eψ,
where H, the Hamiltonian operator, corresponds to the total energy of the system, Eis an eigenvalue
ofHcorresponding to the energy of ψ, and ψis an eigenfunction of H, also called a wavefunction,
or a solution to the Schrödinger equation.
As alluded to, the Hamiltonian operator contains all the information regarding the kinetic and potential
energies for all particles of a system. Thus,
H=−1
2NX
i=1∇2
i−1
2MX
A=11
MA∇2
A−NX
i=1MX
A=1Za
riA
+NX
i=1X
j>11
rij+MX
A=1X
B>AZAZB
RAB,
where the first summation is due to the kinetic energy of the electrons, the second summation is
due to the kinetic energy of the nuclei, the first double summation is due to the potential of the
attraction between electrons and nuclei, the second double summation is due to the potential of the
repulsion between electrons and electrons, and the third double summation is due to the potential of
the repulsion between nuclei.
This formula is unwieldy, however, and it can be simplified significantly without discarding much
information that would be useful in chemical prediction. The mass difference between electrons
and protons, which is the minimum mass difference between electrons and nuclei, is more than 3
orders of magnitude. Thus, given the same kinetic energy, electrons will be traveling several times
faster than nuclei. From the perspective of the electrons, the nuclei are nearly fixed, and from the
perspective of the nuclei, the electrons change their position instantaneously. Therefore, it suffices to
consider the positions of the nuclei fixed, setting
−1
2MX
A=11
MA∇2
A,
nuclear kinetic energy, to 0, and
MX
A=1X
B>AZAZB
RAB,
nuclear repulsive potential, to a constant. This is called the Born-Oppenheimer Approximation, and
it leaves behind the so-called electronic Hamiltonian,
Helec=−1
2NX
i=1∇2
i−NX
i=1MX
A=1Za
riA+NX
i=1X
j>11
rij,
which can more succinctly be written
Helec=T+VNe+Vee.
The electronic Hamiltonian is still solved for its eigenfunctions, giving
HelecΨelec(r1, . . . ,rN) =EelecΨelec(r1, . . . ,rN),
ignoring for simplicity’s sake electron spin to write Ψelecas a function of the positions of the electrons
only. For readability, the subscripts of Helec,Eelec, and Ψelecare henceforth omitted.
For a system of Nelectrons, Ψis a function of 3Narguments, one for each spatial dimension of
each electron. The Hartree-Fock Approximation, also called the Hartree Product, further simplifies
26

Published as a conference paper at ICLR 2025
this equation by decomposing the wavefunction into the product of Nwave functions, each of three
arguments, corresponding to each electron individually:
Ψ(r1, . . . ,rn)≈ψ1(r1)···ψN(rN).
It is impossible to observe the wavefunction itself, but the wavefunction can be used to derive the
probability of observing the system’s electrons anywhere in space. In particular, the probability of
observing the electrons at positions r1through rNis the square of the amplitude of the wavefunction
withr1through rNas input:
|Ψ(r1, . . . ,rN)|2= Ψ†(r1, . . . ,rN)Ψ(r1, . . . ,rN).
Given the Hartree-Fock Approximation, this equation can be converted into an electron density
function that sums over the wavefunction of each electron individually,
n(r) = 2NX
i=1ψ†
i(r)ψi(r),
multiplying by 2 to account for both spin up and spin down, which were neglected previously. This
means that n(r)gives the density of electrons at a point in space, r. This is a function of only 3
inputs, but it contains much of the information that is observable from the full wavefunction, which,
recall, is a function of 3Ninputs.
Density Functional Theory (DFT), which allows the electron density function, n(r), to be exploited,
rests on two fundamental theorems. First, the ground-state energy, E0, which is the smallest
eigenvalue of the Hamiltonian and corresponds to the energy of the lowest-energy wavefunction, is a
unique functional of the electron density function. Second, the electron density that minimizes the
energy of this functional is the true electron density, which means that it is the electron density that
corresponds to the full solution of the Schrödinger equation. Therefore, after the problem has been
reduced from one of 3Ninputs to one of 3inputs, significant information about the original problem
can still be found.
It is worth examining the first statement in more detail. Recall that a functional is a function that
maps functions to scalars. For example, F, defined by
F[f(x)] =Z1
−1f(x)dx,
is a functional that takes in an arbitrary real-valued function and gives out its integral from −1to1.
Thus, for example, if f(x) =x2+ 1, then F[f(x)] =8
3. The first statement, then, holds that there
exists a functional that uniquely determines the ground state energy of a particular electron density
function. It promises that no more information is needed. Written as an equation, E0=F(n(r)).
However, this set-up still relies on some means of finding the individual-electron wave function. This
is provided by the Kohn-Sham Equation:
h1
2∇2+V(r) +VH(r) +VXC(r)i
ψi(r) =ϵiψi(r).
The meaning of each term is as follows. The value1
2∇2is kinetic energy. The value V(r)is the
potential due to the interaction between the electron and the nuclei. The value VH(r), called the
Hartree potential, is defined
VH(r) =Zn(r′)
|r−r′|d3r′,
and it is due to the Coulomb repulsion between the electron and the electron density function,
defined by all electrons in the system. This means that the Hartree potential includes the Coulomb
repulsion between the electron and itself, since the electron itself is included in the electron density
function, n(r). This is an unphysical result, and it is one of several effects accounted for in the
exchange-correlation potential, VXC(r), which is defined
VXC(r) =δEXC(r)
δn(r).
27

Published as a conference paper at ICLR 2025
This is the functional derivative of the exchange-correlation energy with respect to electron density.
Exchange-correlation potential is due to quantum-chemical effects, and its form is not known because
the exact form of EXC(r)is not known. The Kohn-Sham Equation can be summarized as
HKSψi(r) =ϵiψi(r).
It is evident from the definitions of the terms in the Kohn-Sham Equation that the present approach
is circular. The Kohn-Sham Equation count on the Hartree potential, VH(r). The Hartree potential
counts on the electron density function, n(r). The electron density function counts on the single-
electron wavefunctions, ψi(r). And the single-electron wavefunctions count on the Kohn-Sham
Equation.
This is no problem. The following process leverages this circularity to check the soundness of a
particular n(r):
1. Define an initial trial electron density function, ntrial(r).
2.Solve the Kohn-Sham Equation using the trial electron density function, ntrial(r), to find
the single-electron wavefunctions, ψi(r).
3.Calculate the electron density function, nKS(r), implied by these single-electron wavefunc-
tions, by nKS(r) = 2PN
i=1ψ†
i(r)ψi(r).
4.Compare the calculated electron density, nKS(r), with the trial electron density, ntrial(r). If
the two densities are the same, or nearly so, then this electron density is accepted as correct,
and it can be used to find the ground-state energy, E0. If not, the trial electron density is
updated somehow, and the process repeats.
To make this process simpler and more efficient, it is common to represent the single-electron
wavefunctions as linear combinations of some predefined basis set \{ϕα(r)\}B
α=1, where Bis defined
as the cardinality of the basis. Often, the basis set is composed of atomic orbitals, especially
Gassian-type orbitals, which are particularly convenient in calculation. The expansion coefficients
of these wavefunctions can be organized in a matrix C∈RB×N, where each column icontains the
coefficients of wavefunction i. Each wave function can then be recovered as
ψi(r) =BX
α=1Cαiϕα(r).
This permits the Kohn-Sham Equation as a whole to be written in matrix form. Recall the form
HKSψi(r) =ϵiψi(r). Now, it is possible to rewrite ψi(r)according to its decomposition in the basis:
HKSBX
α=1Cαiϕα(r) =ϵiBX
α=1Cαiϕα(r).
Now, for any ϕβ(r)in the basis, left-multiplying both sides by ϕ†
β(r)yields
ϕ†
β(r)HKSBX
α=1Cαiϕα(r) =ϕ†
β(r)ϵiBX
α=1Cαiϕα(r),
and integrating with respect to ryields
Z
ϕ†
β(r)HKSBX
α=1Cαiϕα(r)dr=Z
ϕ†
β(r)ϵiBX
α=1Cαiϕα(r)dr.
More succinctly, defining (HKS)βα=R
ϕ†
βHKSϕα=⟨ϕβ|HKS|ϕα⟩andSβα=R
ϕ†
βϕα=
⟨ϕβ|ϕα⟩, this is
BX
α=1(HKS)βαCαi=ϵiBX
α=1SβαCαi.
This allows the construction of the matrix H, with elements Hβαdefined by Hβα= (HKS)βα=
⟨ϕβ|HKS|ϕα⟩, and the matrix S, with elements Sβα=⟨ϕβ|ϕα⟩. The matrix Hrepresents the
28

Published as a conference paper at ICLR 2025
Hamiltonian matrix, which depends on the coefficient matrix Cand is calculated using a method
called Density-Fitting, whose time complexity is O(B3). The matrix Sis the overlap matrix, which
depends on the basis set \{ϕα(r)\}B
α=1and accounts for its non-orthogonality. This means that if the
basis set is orthonormal, Sis the identity. The Kohn-Sham Equation in matrix form is therefore
HC=SCϵ,
or
H(C)C=SCϵ,
making clearer that Hdepends on C, where ϵis a diagonal matrix containing the orbital energies.
This forms a generalized eigenvalue problem, the principal difficulty of which is the dependence of
HonC.
As a result, traditional DFT uses the Self-Consistent Field method (SCF), which iteratively refines
the coefficient matrix by approximating the Hamiltonian. Using superscripts to denote the iteration to
which each matrix belongs, at each iteration, the Kohn-Sham Equation is
H(k)
C(k−1)
C(k)=SC(k)ϵ(k).
Therefore, H(k)is computed using Ck−1, and the generalized eigenvalue problem is solved for C(k)
andϵ(k). This process continues until convergence, which is formalized as ∥H(k+1)−H(k)∥ ≤δ.
The objective of Hamiltonian prediction is to eliminate the need for this computationally expensive
SCF iteration by directly estimating the target Hamiltonian, H⋆, for a given molecular structure, M.
To predict the Hamiltonian, a machine learning model ˆHθ(M)is parameterized by θ, guided by an
optimization process defined as
θ⋆= arg min
θ1
|D|X
(M,H⋆
M)∈Ddist( ˆHθ(M),H⋆
M),
where Dis the dataset, |D|is its cardinality, and dist(·,·)is a predefined metric. Machine learning
models hold the potential to drastically improve the efficiency of DFT calculations without sacrificing
accuracy.
E.1 E XAMPLE
In the example section, we explore the molecular structure and basis set details of water ( H2O),
which consists of one oxygen and two hydrogen atoms. We expand each electron’s wavefunctions
using a basis set, which, in practice, means each atom is represented using a predefined basis set that
collectively describes a set of electrons. Specifically, we use the STO-3G minimal basis set, where
oxygen is described by five basis functions ( 1s,2s,2px,2py,2pz), and each hydrogen atom has two
basis functions (1s for each). Altogether, this totals seven basis functions.
O
H H1s 2s
2px2py
2pz
1s 1s1sO2sO2pxO2pyO2pzO1sH11sH2
1sOF11F12F13F14F15F16F17
2sOF21F22F23F24F25F26F27
2pxF31F32F33F34F35F36F37
2pyF41F42F43F44F45F46F47
2pzF51F52F53F54F55F56F57
1sH1F61F62F63F64F65F66F67
1sH2F71F72F73F74F75F76F77
Figure 8: Molecular structure and basis functions for water (H 2O).
When considering the effect of rotations on the Hamiltonian matrix, it is important to understand how
the matrix elements transform under such operations.
29

Published as a conference paper at ICLR 2025
For instance, if we consider a rotation that affects only the 2px,2py, and 2pzorbitals of the oxygen
atom, the corresponding submatrix of the Hamiltonian Hpwithin the 2porbital space would transform
as:
H′
p=RpHpR⊤
p
where Hpis the submatrix containing the interactions between the 2px,2py, and 2pzorbitals. The
rotation matrix Rpis specific to the rotation in the 2porbital space.
1sO2sO2px2py2pz1sH11sH2
1sOF11F12F13F14F15F16F17
2sOF21F22F23F24F25F26F27
2pxF31F32F33F34F35F36F37
2pyF41F42F43F44F45F46F47
2pzF51F52F53F54F55F56F57
1sH1F61F62F63F64F65F66F67
1sH2F71F72F73F74F75F76F77
2px2py2pz
2pxF′
33F′
34F′
35
2pyF′
43F′
44F′
45
2pzF′
53F′
54F′
55
This transformation ensures that the physical properties described by the Hamiltonian remain con-
sistent under rotational operations, a fundamental requirement for accurately modeling molecular
systems.
F D IPOLE MOMENT AND ELECTRONIC SPATIAL EXTENT
In this section, we describe how to compute the dipole moment and the electronic spatial extent using
the Kohn-Sham orbitals derived from the Kohn-Sham Hamiltonian.
The Kohn-Sham equations are defined as

−ℏ2
2m∇2+Veff(r)
ψi(r) =ϵiψi(r), (4)
where ψi(r)denotes the Kohn-Sham orbitals, ϵiare the orbital energies, and Veff(r)is the effective
potential. The electron density is then expressed as
ρ(r) =occX
i|ψi(r)|2. (5)
The dipole moment dis computed as the sum of electronic and nuclear contributions. The electronic
contribution is given by
delec=−eZ
rρ(r)dr, (6)
and the nuclear contribution is
dnuc=−eX
AZARA, (7)
where erepresents the elementary charge, ZAdenotes the atomic number of nucleus A, andRAis the
position vector of nucleus A. The total dipole moment is therefore the sum of these two components:
30

Published as a conference paper at ICLR 2025
d=delec+dnuc. (8)
Next, we define the electronic spatial extent, denoted by ⟨r2⟩, which provides a measure of the spatial
distribution of the electron density. It is calculated as
⟨r2⟩=Z
r2ρ(r)dr. (9)
To compute these quantities in practice, one often works in a basis set representation. The electron
density matrix Pµνis defined as
Pµν= 2occX
iCiµCiν, (10)
where Ciµare the coefficients of the molecular orbitals in terms of the basis functions ϕµ. The dipole
integrals µµν
αfor a Cartesian direction αare expressed as
µµν
α=Z
ϕµ(r)rαϕν(r)dr. (11)
The electronic contribution to the dipole moment in the basis set representation is then given by
delec
α=−eX
µνPµνµµν
α. (12)
Similarly, the electronic spatial extent in the basis set representation is computed using the integral
⟨r2⟩µν=Z
ϕµ(r)r2ϕν(r)dr, (13)
leading to the final expression
⟨r2⟩=X
µνPµν⟨r2⟩µν. (14)
G G ROUP THEORY
If a function is equivariant to the action of a group, it does not matter whether the group acts on the
function’s input or output. More formally, for vector spaces VandWequipped with arbitrary group
representations DV(g)andDW(g), a function f:V→Wis equivariant to Gif
f(DV(g)v) =DW(g)f(v)
for all g∈Gand for all v∈V. In the case of SE(3) , a function fis equivariant if the output is the
same whether the input is slid and rotated, then put through f, or whether the input is put through f,
then slid and rotated in the same way. A function fis invariant to a group Gif
DW(g) =e,
the identity element in W, for all g∈G. In the case of SE(3) , a function fis equivariant if it gives
the same output no matter how its input is slid and rotated. Equivariance is a fundamental notion
in the modeling of physical systems. In the context of this paper, it is desirable that the function
that predicts the Hamiltonian matrix be SE(3) -equivariant, reflecting that the molecule’s energetic
properties are equivalent if the molecule is rotated or translated.
31

Published as a conference paper at ICLR 2025
Group representations are an instance of the more general notion of group homomorphisms. Given a
group Gwith group operation ◦and another group Hwith group operation ∗, a group homomorphism
is a map ρ:G→Hsuch that
ρ(g1◦g2) =ρ(g1)∗ρ(g2).
A group homomorphism, then, must preserve the structure of the group, which means that it does not
matter whether the group operation is performed in GorH. Note, however, that a homomorphism
might respect the group structure only trivially. For example, ρ:G→H, defined by ρ(g) =e, is
a trivial group homomorphism. A group representation is simply a group homomorphism where
H=Vis some vector space. In plainer language, this means that a group representation is a group
written as a set of matrices, whose group operation is matrix multiplication. It is always the case that
V⊆GL, since any matrix that is degenerate or non-square lacks an inverse and therefore fails to
satisfy the inverse axiom.
Two group representations DandD′are equivalent if there is a fixed matrix Psuch that D(g) =
P−1D′(g)Pfor all g∈G. In this case, DandD′can be interpreted as the same representation
defined with respect to different bases. A representation Dis reducible if it acts on independent
subspaces of V; otherwise, it is irreducible. An irreducible representation is called an irrep. More
formally, Dis reducible if
D(g) =P−1
D(ℓ0)(g)
D(ℓ1)(g)
...
P=P−1M
iD(ℓi)(g)
P,
which means that it Dis block-diagonal with respect to some basis. It is convenient to decompose
group representations into irreducible representations because this reduces the group operation
calculation to several smaller independent calculations. Irreps are the atoms of group representations
in the sense that arbitrary representations can be composed with the direct sum of irreps. The irreps
ofSO(3) are called Wigner D-matrices, with D(ℓ)(g)denoting a Wigner D-matrix representation of
gof degree ℓ. Wigner D-matrices are of size (2ℓ+ 1)×(2ℓ+ 1) , with higher-degree representations
allowing for more precise handling of angular information.
H E XENSIVE DETAILS ON PUBCHEM QH
Here, we present a detailed comparison between the PubChemQH dataset and the curated QH9
dataset. This comparison aims to highlight the key differences and similarities. Additionally, we
provide a comprehensive analysis of the atom number distribution within the PubChemQH dataset,
supported by a Figure 9.
Table 10: Comparison of PubChemQH and QH9
Feature PubChemQH QH9
Source PubChem Database QM9
Number of Molecules 50,321 130,831 (QH9-stable)
Functional B3LYP B3LYP
Basis Set Def2TZVP Def2SVP
SCF Convergence Tolerance 10−810−13
SCF Gradient threshold: 10−43.16×10−4
Grid Density Level 3 3
Mean of Node Number 61.85 18
Mean of Hamiltonian Size 1025 141
I E XTENSIVE DETAILS ON WAN ET
We here provide extensive details on the motivations of the WANet architecture.The WANet architec-
ture builds upon well-established equivariant neural network designs, which have been extensively
studied and validated in the field.
32

Published as a conference paper at ICLR 2025
Figure 9: Node atom distribution for PubChemQH Dataset
We provide a visual representation illustrating WANet’s architecture and its key components, which
is given in Figure10 . This visual aid will help readers better understand how WANet enhances the
scalability of Hamiltonian prediction for large molecular systems. Each architectural element was
carefully chosen to address specific computational and physical challenges at scale:
First, the motivation behind WANet’s architecture stems from a critical limitation in existing Hamil-
tonian prediction methods—their inability to scale to larger molecular systems. When using larger
basis sets, such as Def2-TZVP, higher-order irreducible representations are required to accurately
capture angular dependencies in molecular orbitals. This poses a severe computational challenge for
traditional SE(3)-equivariant methods, including QHNet, which become extremely expensive due to
their computational complexity. WANet addresses this bottleneck by introducing SO(2) convolutions,
which reduce computational complexity from O(L3)toO(L6). This improvement enables WANet
to process high tensor degrees efficiently. Without SO(2) convolutions, handling the scale of our
PubChemQH dataset and larger systems would be significantly more challenging and computationally
expensive.
Second, large molecular systems exhibit fundamentally different physics at various distance scales.
As molecular size increases, long-range interactions become more prominent. WANet’s Mixture-of-
Experts architecture is designed to model this complex physics efficiently. It employs specialized
experts for different interaction ranges, capturing both short-range effects (like covalent bonding)
and long-range phenomena (such as electrostatics). By sparsifying these experts, WANet achieves a
rich representation of molecular interactions while maintaining computational efficiency, making it
particularly well-suited for large-scale systems.
Third, WANet’s architecture is designed to accurately capture the intrinsic properties of molecular
systems, particularly for large molecules. The Hamiltonian matrix, which fully characterizes the
quantum state and electron distribution, presents unique challenges in prediction due to complex
electron correlation effects as the system size grows. To address this, WANet incorporates the MACE
architecture’s density trick, enabling efficient computation of many-body interactions without explicit
calculation of all terms. This approach is crucial for maintaining accuracy as molecular size increases
and electron correlation effects become more pronounced, ensuring WANet’s scalability and precision
in Hamiltonian prediction for large systems.
Pair Construction Layer The objective of the pair construction layer is to extend the model’s
capacity to consider non-diagonal node pairs by introducing a tensor product filter. This filter
modulates the projection of irreducible representations onto the space of node pair irreducible
representations, denoted by fts. It is important to note that, in contrast to the Node Convolution
Layer, which performs graph convolution on a radius graph or KNN graph, the pair construction
33

Published as a conference paper at ICLR 2025
Figure 10: Scalable architecture for predicting the Hamiltonian matrix ( ˆH). The framework
addresses three key scalability challenges in quantum many-body systems: (1) efficient convolutions,
(2) many-body interactions, and (3) long-range interactions. (1) Efficient convolutions are achieved
through an SO(3)-equivariant convolution reduced to SO(2) for computational efficiency. (2) Many-
body interactions are captured using a density trick, which implicitly includes three-body terms by
combining pairwise interactions quadratically. (3) Long-range interactions are modeled with a mixture
of Local-Structure Representation (LSR) experts, where a softmax-based gating mechanism allocates
pairwise inputs to top-performing experts. Finally, the diagonal and non-diagonal components are
assembled into the predicted Hamiltonian matrix, providing a scalable and accurate representation of
the system dynamics.
34

Published as a conference paper at ICLR 2025
Gates
Expert 2Expert 
n-1Expert n+Mixture of LSR Expertsx x
Distribution 
over Experts
OutputTrained via
BackpropPick TopK
Experts...
...Expert 1
Short Range 
(Bond Length)
Van Der Waals
RBF(r\_ij)
Softmax
Figure 11: illustration of the Mixture of the Long-Short-Range Pair experts
layer considers all possible node interactions by operating on a complete graph. The mathematical
formulation of this layer is given as follows:
fℓo
ts=X
li,ljWli,lj,lo
xli
s⊗xlj
tlo
, (15)
where xlisandxlj
tare the li-th and lj-th irreducible representations of source node sand target node
t, respectively, and Wli,lj,loare the learned weights that couple these representations into the output
representation ℓo.
To improve the efficiency of the tensor product, we employed the channel-grouped tensor product,
where the channels of the first and second tensors are tied to collectively construct the output channel
path5. Additionally, to accommodate the symmetry inherent in the Hamiltonian matrix, we implement
a symmetric structure in the pair representations. Specifically, the representations for node pairs
(s, t)and(t, s)must be identical, reflecting the symmetrical nature of physical interactions. This is
formulated as:
fℓo
ts′=fℓo
st′=1
2(fℓo
ts+fℓo
st),
where fℓo
tsandfℓo
stdenote the initial, unsymmetrized tensor products for the node pairs sandtbefore
the application of symmetry. The primed notations fℓo
ts′andfℓo
st′represent the final, symmetrized
outputs.
Graphical Illustration of MoE Here we provide an additional illustration of the Mixture of the
Long-Short-Range Pair experts described in the maintext, which is shown in Figure 11.
Expansion Block In the construction of final Hamiltonian blocks that encompass full orbital
information using pair irreducible representations and many-body irreducible representation, a tensor
expansion operation is employed alongside the filtering process. This expansion is defined by the
following relation:
5This is also called “uuw” tensor product in e3nn implementation.
35

Published as a conference paper at ICLR 2025

⊗ℓofℓo(mi,mj)
(ℓi,ℓj)=ℓoX
mo=−ℓoC(ℓo,mo)
(ℓi,mi),(ℓj,mj)fℓo
mo, (16)
where Cdenotes the Clebsch-Gordan coefficients, and ⊗symbolizes the tensor expansion which is
the converse operation of tensor product. uℓi⊗vℓjcan be expressed as a sum over tensor expansions:
uℓi⊗vℓj=X
ℓ3Wli,lj,lo⊗fℓo, (17)
subject to the coupling constraints |ℓi−ℓj| ≤ℓo≤ℓi+ℓj.
Loss Weighting. We determined the loss weights through a principled ablation study on a validation
set. While keeping λ1andλ2fixed at 1, we varied λ3from 0.5 to 3 to understand the impact of
WALoss weighting. We found the method to be robust across a reasonable range of values, with
λ1=λ2= 1, andλ3= 2.5providing consistently strong performance. The table below summarizes
the performance metrics for different values of λ1, λ2andλ3.
Table 11: Performance metrics for with different λ1,λ2, and λ3values.
λ1λ2λ3Hamiltonian MAE ↓ϵHOMO MAE↓ϵLUMO MAE↓ ϵ∆ ϵoccMAE↓ϵorbMAE↓ C↑ System Energy MAE ↓
1 1 3 0.5258 0.8688 1.9116 2.0407 21.92 8.18 48.90\% 54.77
1 1 2.5 0.4744 0.7122 0.73 1.327 18.84 7.33 48.03\% 47.193
1 1 2 0.4918 0.7847 1.24981 1.46951 21.15 7.7 47.72\% 59.79
1 1 1.5 0.4586 1.50825 3.29805 3.1123 23.57 8.56 47.29\% 64.857
1 1 1 0.4807 1.1596 3.45767 3.2068 23.16 9.2 47.15\% 62.61719
1 1 0.5 0.4223 2.28836 6.83532 5.69888 28.96 11.78 45.30\% 72.9
36

Published as a conference paper at ICLR 2025
J A DDITIONAL THEORY
J.1 P ROOF OF THEOREM
Theorem 2. LetH,ˆH∈Rn×nbe symmetric matrices, and let S∈Rn×nbe a symmetric positive
definite matrix. Consider the generalized eigenvalue problems:
HC=SCϵ,ˆHˆC=SˆCˆϵ,
where ϵandˆϵare diagonal matrices of eigenvalues, and CandˆCare the corresponding eigenvector
matrices. Define ∆H=ˆH−H, and let δbe the minimum distance between the eigenvalue of interest
and the rest of the spectrum of S−1H. Then, the following bounds hold:
1.Eigenvalue Differences:
λi(ˆH,S)−λi(H,S)≤κ(S)
∥S∥2∥∆H∥F≤κ(S)
∥S∥2∥∆H∥1,1,
where κ(S) =∥S∥2∥S−1∥2is the condition number of Swith respect to the spectral norm,
∥ · ∥ Fdenotes the Frobenius norm, and ∥ · ∥ 1,1denotes the element-wise l1norm.
2.Eigenspace Angle:
sinθ≤κ(S)
∥S∥2·∥∆H∥F
δ≤κ(S)
∥S∥2·∥∆H∥1,1
δ,
where θis the angle between the eigenspaces corresponding to λi(H,S)andλi(ˆH,S).
Proof. Consider the generalized eigenvalue problems for HandˆHwith respect to S:
HC=SCϵ,ˆHˆC=SˆCˆϵ.
Since Sis symmetric positive definite, it is invertible. We can transform the generalized eigenvalue
problems into standard eigenvalue problems by multiplying both sides by S−1:
A=S−1H,ˆA=S−1ˆH=A+E,
where E=S−1∆H.
Weyl’s Perturbation Theorem
Theorem 3 (Weyl’s Perturbation Theorem) .LetA,ˆA∈Rn×nbe symmetric matrices with eigenval-
uesλ1≤λ2≤ ··· ≤ λnandˆλ1≤ˆλ2≤ ··· ≤ ˆλn, respectively. Then, for all i= 1, . . . , n ,
ˆλi−λi≤ ∥ˆA−A∥2.
Applying Weyl’s Theorem to AandˆA, we have:
λi(ˆA)−λi(A)≤ ∥E∥2=∥S−1∆H∥2.
Bounding the Eigenvalue Differences
Using the sub-multiplicative property of the spectral norm:
∥S−1∆H∥2≤ ∥S−1∥2∥∆H∥2.
Since the spectral norm is bounded by the Frobenius norm:
∥∆H∥2≤ ∥∆H∥F.
Combining these inequalities:
∥S−1∆H∥2≤ ∥S−1∥2∥∆H∥F.
37

Published as a conference paper at ICLR 2025
The condition number κ(S)is defined as:
κ(S) =∥S∥2∥S−1∥2=⇒ ∥S−1∥2=κ(S)
∥S∥2.
Substituting back:
∥S−1∆H∥2≤κ(S)
∥S∥2∥∆H∥F.
Therefore:λi(ˆA)−λi(A)≤κ(S)
∥S∥2∥∆H∥F.
Since the eigenvalues of AandˆAcorrespond to the generalized eigenvalues of (H,S)and(ˆH,S),
respectively, we have:
λi(ˆH,S)−λi(H,S)≤κ(S)
∥S∥2∥∆H∥F.
Since∥∆H∥F≤ ∥∆H∥1,1, we can further bound:
λi(ˆH,S)−λi(H,S)≤κ(S)
∥S∥2∥∆H∥1,1.
Davis-Kahan sinθTheorem
Theorem 4 (Davis-Kahan sinθTheorem) .LetA,ˆA∈Rn×nbe symmetric matrices, and let U
andˆUbe the invariant subspaces of AandˆAcorresponding to eigenvalues in intervals IandˆI,
respectively. If δ= dist( I,ˆIc)>0, then
∥sin Θ∥2≤∥ˆA−A∥2
δ,
where Θis the matrix of principal angles between UandˆU.
Bounding the Eigenspace Angle
Applying the Davis-Kahan Theorem to our case:
sinθ≤∥E∥2
δ=∥S−1∆H∥2
δ.
Using the previously established bound:
∥S−1∆H∥2≤κ(S)
∥S∥2∥∆H∥F,
we obtain:
sinθ≤κ(S)
∥S∥2·∥∆H∥F
δ.
Similarly, since ∥∆H∥F≤ ∥∆H∥1,1:
sinθ≤κ(S)
∥S∥2·∥∆H∥1,1
δ.
This completes the proof.
38

Published as a conference paper at ICLR 2025
J.2 P ERTURBATION ANALYSIS AND GROWTH OF EIGENVALUES
Theorem 5 (Generalized Bai-Yin’s law) .LetAbe an n×nrandom matrix with independent entries
having mean µand variance σ2. Then, with high probability, the spectral norm of Ais bounded by:
∥A∥2≤ |µ|n+ 2σ√n.
Proof. Consider the matrix Adecomposed into its mean and fluctuation components:
A=µJ+B,
where Jis the matrix of all ones and Bis a random matrix with zero-mean entries and variance σ2.
First, we bound the spectral norm of the mean component µJ. The matrix Jis a rank-1 matrix with
all entries equal to 1. Its largest singular value is n, so:
∥µJ∥2=|µ| ·n.
Next, we bound the spectral norm of the fluctuation component B. Since Bhas i.i.d. entries with
zero mean and variance σ2, using Bai-Yin’s law, we get:
∥B∥2≤2σ√n,
with high probability.
Combining these bounds using the triangle inequality, we have:
∥A∥2≤ ∥µJ∥2+∥B∥2≤ |µ|n+ 2σ√n.
Thus, with high probability, the spectral norm of Ais bounded by:
∥A∥2≤ |µ|n+ 2σ√n.
Theorem 6 (Perturbation Sensitivity Scaling) .Consider the generalized eigenvalue problem HC=
SCϵ, where H∈RB×Bis a symmetric Hamiltonian matrix, S∈RB×Bis a positive definite overlap
matrix, C∈RB×kis the eigenvector matrix, and ϵ∈Rk×kis a diagonal matrix of eigenvalues
λi(H,S). Let ˆH=H+ ∆Hbe the perturbed Hamiltonian, with ∆H∈RB×Ba perturbation
matrix whose entries have mean µand variance σ2. Suppose the smallest eigenvalue of Ssatisfies:
λmin(S) =c+A
1 +
B
N0α,
where c >0,A >0,α >0, and N0>0are fixed constants. Then, for each eigenvalue, there exists
a constant K > 0such that:
λi(ˆH,S)−λi(H,S)≤K
σB1/2+B|µ|
,
for all sufficiently large B.
Proof. Consider the unperturbed generalized eigenvalue problem Hci=λiSci, where HandS
are symmetric, Sis positive definite, and cT
iSci= 1. The perturbed system is ˆHˆci=ˆλiSˆci, with
ˆH=H+ ∆H,ˆci=ci+δci, and ˆλi=λi+δλi. Substituting and expanding yields:
(H+ ∆H)(ci+δci) = (λi+δλi)S(ci+δci).
Subtracting the unperturbed equation and projecting onto cT
i, noting cT
i(H−λiS) = 0 , we obtain:
cT
i∆Hci+cT
i∆Hδci=δλi+δλicT
iSδci.
Solving for the perturbation:
δλi=cT
i∆Hci+cT
i∆Hδci
1 +cT
iSδci.
39

Published as a conference paper at ICLR 2025
0 5 10 15 20 25 30
Number of Carbon Atoms05000100001500020000250001 / Minimum Eigenvalue
P-value: 4.58e-09
R²: 0.73
Pearson Coefficient: 0.85Reciprocal of Minimum Eigenvalue vs. Number of Carbon Atoms (0 to 29)
Data
Regression Line
Figure 12: The scaling of the smallest eigenvalue of the overlap matrix in satured hydrocarbons.
For small ∆H,δciisO(∥∆H∥2), socT
i∆HδciandcT
iSδciare higher-order terms. Here, we
only consider the first-order term arises from evaluating the perturbation with the unperturbed
eigenvector. Higher-order terms account for eigenvector shifts. We claim that analyzing the higher-
order terms gives rise to unnecessary complication and does not alter the nature of catastrophic
scaling. Specifically:
δλi≈cT
i∆Hci.
Bounding this term:
|δλi| ≤ |cT
i∆Hci| ≤ ∥ci∥2
2∥∆H∥2.
Since cT
iSci= 1andS⪰λmin(S)I, we have:
∥ci∥2
2≤1
λmin(S)=1
c+A
1+
B
N0α=1 +
B
N0α
ch
1 +
B
N0αi
+A.
For∆Has a random matrix, apply a spectral norm bound (by Bai-Yin law):
∥∆H∥2≤C1(σB1/2+B|µ|),
with constant C1, holding with high probability. Combining:
|δλi| ≤1 +
B
N0α
ch
1 +
B
N0αi
+AC1(σB1/2+B|µ|).
Thus, the leading-order bound is:
λi(ˆH,S)−λi(H,S)=O
1 +
B
N0α
ch
1 +
B
N0αi
+A
σB1/2+B|µ|
=O
σB1/2+B|µ|
.
Remark 1. i y If higher-order terms like cT
i∆Hδciand the denominator correction are included,
the perturbation includes an additional term of order O 
1+
B
N0α
ch
1+
B
N0αi
+A2
(σB1/2+B|µ|)2!
,
reflecting the quadratic contribution of the perturbation strength.
40

Published as a conference paper at ICLR 2025
J.3 S CALING LAW OF THE SMALLEST EIGENVALUE OF OVERLAP MATRIX
Figure 12 shows the relationship between the number of carbon atoms in saturated hydrocarbons and
the reciprocal of the smallest eigenvalue of the overlap matrix. The x-axis represents the number of
carbon atoms, ranging from 0 to 29. The y-axis represents the reciprocal of the smallest eigenvalue,
ranging up to 25,000. Specifically, the regression analysis yields a p-value of 4.58×10−9, anR2
value of 0.73, and a Pearson coefficient of 0.85, suggesting a strong linear relationship.
We also compare another scaling law suggested by Høyvik (2020). As shown in Figure 13, we
investigated the relationship between1
λminand the number of basis functions for various molecules.
Our results indicate a clear power-law scaling, which can be expressed as y=a·xb. The fitted
parameters for each molecule are as follows: Eicosane (C20H42): y= 0.000107 ·x2.43, Graphene
(C24): y= 0.00158 ·x2.83, and Diamond (C10): y= 0.0178·x4.16. This power-law behavior
emphasizes the significant impact of the number of basis functions on1
λmin, with different scaling
exponents for each molecule.
J.3.1 k-NEAREST -NEIGHBOR OVERLAPS
Here, we analyzed a system where each basis function overlaps with its k-nearest neighbors. The
overlap matrix S≻0in this case is a symmetric banded matrix with bandwidth 2k+ 1:
Sij=

1, ifi=j,
s|i−j|,if0<|i−j| ≤k,
0, if|i−j|> k.
To derive a scaling equation for the smallest eigenvalue, consider the case where k= 1, which
corresponds to the nearest-neighbor model. Other cases can be derived similarly using plane waves
and discrete Fourier Transform. The nearest-neighbor model holds physical significance in the
context of tight-binding models. For example, in chain-like polyenes, the Huckel model is based on a
nearest-neighbor analysis, providing a meaningful framework for discussing this case. The overlap
matrix Sis given as:
S=
1s0··· 0
s1s......
0s1...0
............s
0··· 0s1

The eigenvalues λnof this tridiagonal matrix can be derived using the known solutions for such
matrices:
λn= 1 + 2 scosnπ
B+ 1
, n = 1,2, . . . , B.
The smallest eigenvalue corresponds to n=B:
λmin= 1 + 2 scosBπ
B+ 1
.
Using the trigonometric identity cos(π−θ) =−cos(θ)and approximating for large B:
cosBπ
B+ 1
=−cosπ
B+ 1
≈ − 
1−1
2π
B+ 12!
.
Substituting back:
41

Published as a conference paper at ICLR 2025
2.9 3.0 3.1 3.2 3.3 3.4 3.5
log(Number of Basis Functions)3.03.23.43.63.84.04.24.4log(1 / λ\_min)Eicosane
eicosane data
eicosane linear fit
2.9 3.0 3.1 3.2 3.3 3.4
log(Number of Basis Functions)5.56.06.57.0log(1 / λ\_min)Graphene
graphene data
graphene linear fit
2.5 2.6 2.7 2.8 2.9 3.0
log(Number of Basis Functions)456789log(1 / λ\_min)Diamond
diamond data
diamond linear fit
Figure 13: The scaling of the smallest eigenvalue of the overlap matrix in three systems. Data adapted
from Høyvik (2020).
λmin≈1−2s+sπ
B2
.
Since S≻0, this yields 1−2s≥0. Here, we investigated the smallest eigenvalue λminasB→ ∞ ,
identifying two distinct cases:
1.Saturating Case: When s <0.5the smallest eigenvalue saturates at a finite value 1−2sas
B→ ∞ .
2.Non-Saturating Case: Fors= 0.5,λmindecreases indefinitely with increasing B.
K T RAINING
Predicting the Gap from Initial Guess In contrast to prior research, our study tackles the
significant challenges posed by the scale of the PubChemQC t-zvp dataset. It is very challenging
for the model to develop a good numeric starting point. To address this, we have shifted our focus
towards a more tractable objective: making predictions based on an initial guess, which is easy to
obtain. This adjustment is formalized in our objective function as follows:
θ⋆= arg min
θ1
|D|X
(M,H⋆
M)∈Ddist
ˆHθ(M),H⋆
M−H(0)
M
, (18)
where |D|denotes the cardinality of dataset D, and dist(·,·)is a predefined distance metric, H(0)
Mis the initial guess of the Hamiltonian. We conducted an ablation study, showing that although
predictions based on the initial guess significantly improve performance in Hamiltonian prediction,
they struggle with predicting physical properties.
Total Loss Function The total loss function used to train our model is a combination of the orbital
alignment loss and the mean squared error (MSE) loss between the predicted and true Hamiltonian
matrices:
Ltotal=λ1 
1
nnX
i=1∥ˆHi+H(0)
i−H⋆
i∥2
F!
+λ2 
1
nnX
i=1∥ˆHi+H(0)
i−H⋆
i∥1
F!
+λ3Lalign (19)
where λ1, λ2, λ3are hyperparameters that control the relative importance of each loss component.
Experimental Setting The experimental settings are presented in Table 12. For the PubChemQH
dataset, we used an 80/10/10 train/validation/test split, resulting in 40,257 training molecules, 5,032
validation molecules, and 5,032 test molecules. We trained all models of maximum 300,000 steps
with a batch size of 8, using early stopping with a patience of 1,000 steps. WANet converges at
278,391 steps, QHNet at 258,267 steps, and PhisNet at 123,170 steps. All models used the Adam
42

Published as a conference paper at ICLR 2025
optimizer with a learning rate of 0.001 for PubChemQH, along with a polynomial learning rate
scheduler with 1,000 warmup steps. We used gradient clipping of 1, and a radius cutoff of 5 Å. For
QHNet and PhisNet, we referred to the official implementation for these two models.
Table 12: Hyperparameter settings for the experimental study using the PubChemQH and QH9
datasets.
Hyperparameter PubChemQH (WANet) QH9 (WANet) PubChemQH (QHNet) QH9 (QHNet) PubChemQH (PhiSNet) Description
Learning Rate 0.001 5e-4 0.001 5e-4 0.001
Batch Size 8 32 8 32 8
Scheduler Polynomial Polynomial Polynomial Polynomial Polynomial
LR Warmup Steps 1,000 1,000 1,000 1,000 1,000 Number of steps to linearly increase the learning rate
Max Steps 300,000 300,000 300,000 300,000 300,000 Maximum number of training steps
Model Order 6 4 6 4 6 Maximum degree of the spherical harmonics
Embedding Dimension 128 128 128 128 128 Dimension of the node embedding
Bottle Hidden Size N/A N/A 32 32 N/A Size of the hidden layer in the bottleneck
Number of GNN Layers 5 5 5 5 5 Number of graph neural network layers
Max Radius 5 15 5 15 5 Maximum distance between nieghboring atoms
Sphere Channels 128 128 128 128 128 Number of channels in the spherical harmonics
Edge Channels 128 128 32 32 32 Number of channels for edge features
Drop Path Rate 0.1 0.1 N/A N/A N/A Probability of dropping a path in the network
Projection Drop 0.0 0.0 N/A N/A N/A Dropout rate for the projection layer
The training details for the HOMO, LUMO, and GAP predictions are presented in Table 13. For a
fair comparison, all regression models used identical dataset splits (40,257 training molecules, 5,032
validation molecules, and 5,032 test molecules). We used batch size of 32 and consistent optimizer
settings across all models: Adam optimizer with a learning rate of 0.001 and a polynomial learning
rate scheduler with 1,000 warmup steps. WANet followed the same training setup described in
Table 12. The complete regression model hyperparameters are detailed in Table 13. For Equiformer
V2, Uni-Mol+, and Uni-Mol2 models, we used their original implementations.
Table 13: The training details for the HOMO, LUMO, and GAP predictions.
Hyperparameter Uni-Mol+ (Lu et al., 2023) Uni-Mol2 (Xiaohong et al., 2024) Equiformer v2 Description
Learning Rate 0.001 0.001 0.001
Batch Size 32 32 32
Scheduler Polynomial Polynomial Polynomial
LR Warmup Steps 1,000 1,000 1,000 Number of steps to linearly increase the learning rate
Max Steps 300,000 300,000 300,000 Maximum number of training steps
Embedding Dimension 768 768 128 Dimension of the input embedding
FFN Embedding Dim 3072 3072 512 Embedding dimension for the feed-forward network
Number of Encoder Layers 6 6 5 Number of layers in the transformer encoder
Drop Path Rate 0.0 0.0 0.1 Probability of dropping a path in the network
Activation Function GELU GELU SiLU Activation function used in the model
43

Published as a conference paper at ICLR 2025
L D ISCUSSION
L.1 D ISCUSSION ON SYSTEM ENERGY ERROR
The mean absolute error (MAE) of 47 kcal/mol for system energy prediction is notably above
the threshold of chemical accuracy (typically 1 kcal/mol). This highlights the inherent challenges
of accurately predicting system energies for large molecular systems. To our knowledge, this
work is the first to evaluate Hamiltonian predictions on system energy, which presents a non-trivial
implementation challenge. Previous studies have typically focused on metrics such as cosine similarity
of eigenvectors or MAE of occupied energy levels. By directly assessing system energy, our approach
provides a more comprehensive and practical evaluation of Hamiltonian accuracy, which is critical
for large-scale molecular simulations.
Our model demonstrates a significant improvement over baseline models in predicting system energies
for these large systems, indicating enhanced prediction accuracy. The primary goal of this work
is to showcase the scalability and applicability of our approach to large molecular systems, where
achieving absolute precision is inherently difficult due to their complexity. Despite this, our results
represent a meaningful step toward more accurate and efficient predictions in this challenging domain.
The advances we have made underscore the potential of our approach to be further refined and applied
across a range of complex molecular systems.
Moreover, other critical molecular properties such as the highest occupied molecular orbital (HOMO),
the lowest unoccupied molecular orbital (LUMO), and dipole moments are predicted with reasonable
accuracy. These properties, often vital in quantum chemistry workflows, demonstrate the practical
utility of our model beyond system energy prediction. The ability to predict multiple important
molecular properties with competitive error rates reinforces the broader applicability of our model to
a variety of quantum chemistry and materials science applications
L.2 A DDITIONAL CONFORMATIONAL ENERGY ANALYSIS
In quantum chemistry applications, relative energy differences between conformational states are
typically more relevant than absolute energies. To better evaluate our method’s practical utility, we
conducted an additional conformational energy analysis. For each molecule in our test set, we:
1. Sampled 100 conformations.
2.Computed ∆Ebetween each conformation and a reference structure using both DFT and
WANet.
3. Calculated the MAE of these energy differences.
For sampling, we employed Gaussian perturbation to introduce noise in the atomic positions. Starting
from each molecule’s initial conformation from the test sets, we generated 100 perturbed geometries
by adding random Gaussian noise ( σ= 0.1Å) to the atomic coordinates. The potential energy
values for these perturbed conformations were computed using the B3LYP/def2-TZVP level of
theory. The resulting energy standard deviation between sampled conformations ranged from 10
to around 100 kcal/mol (with average std =49.84kcal/mol), depending on the molecule’s flexibility.
For baseline models (included in the Table 14), the PhisNet model withWALoss achieved 9.90
kcal/mol MAE, the QHNet model with WALoss achieved 10.98 kcal/mol MAE, and the WANet
model without WALoss achieved 48.92 kcal/mol MAE. QHNet without WALoss yielded MAEs of
50.56 kcal/mol. Interestingly, we observed that the model without WALoss produce meaningless
prediction, demonstrating the effectiveness of WALoss. It worth noting that while more complex
sampling strategy could be used, those methods are more computationally expensive as they require
DFT calculations.
L.3 D ISCUSSION ON SCF ACCELERATION RATIO
We report an 18\% reduction in SCF cycles, which is a notable improvement, especially considering
that our work targets significantly larger and more complex molecular systems compared to previous
studies such as QH9 and QHNet, where SCF reductions of 18–35\% were achieved. Given the
44

Published as a conference paper at ICLR 2025
Table 14: Relative energy differences between conformational states.
Model MAE of Energy Differences (kcal/mol)
WANet w/WALoss 1.12
PhisNet w/WALoss 9.90
QHNet w/WALoss 10.98
WANet 48.92
QHNet 50.56
increased complexity and size of our systems, this 18\% reduction represents a considerable improve-
ment, reflecting the effectiveness of our model in accelerating convergence for more challenging
molecular simulations. This result highlights the scalability of our approach in handling large-scale
quantum chemical calculations efficiently.
45

\end{document}
